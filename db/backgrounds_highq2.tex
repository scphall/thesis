\subsection{Backgrounds at high \qsq}

In data at stripping level, events where the \db candidate originates upstream of the \Bd vertex are observed,
particularly at high \qsq, as shown in \Fig{fig:negtau:withmc}.
The origin of this background is from a $B\Bbar$ pair, where one decays into a \mumu pair, and the
other into a \kpi.
The \chisqip cuts placed on the muons and hadrons make it so that the efficiency of reconstructing a 2-body decay is greater for higher 2-particle masses at short $B$ lifetimes (simply due to the extra kick given to the child particles).
Since we require the \kpi pair have a \Kstar mass, then in the high \qsq region the di-muon mass is much bigger than the \Kstar mass.  This means that the efficiency is higher at shorter distances from the PV to select the di-muon pair than the \Kstar; therefore, in the case where the \Kstar and di-muons come from different $B$ decays, it's more likely that the di-muon pair is closer to the PV and so appears to have a negative $\tau$.
The converse should be true when the \kpi mass is large and the di-muon mass is small and we see this effect in the data when requiring
the $K\pi$  mass be large and the di-muon mass
consistent with a \Kstar ($|m(\mumu)-m^{PDG}(\Kstar)|<100\mev$; as shown in
\Fig{fig:negtau:2d}.

This background does not concern us as it will populate the local sidebands (it does not peak in
di-muon mass) and so will be subtracted automatically by our procedure.  Furthermore, we do not
expect significant contributions in the positive lifetime  region.  This is shown in
\Fig{fig:res:mxvtau} in the $B$ mass sidebands where this background is seen at negative $\tau$ but
no evidence is seen fof it at positive $\tau$.

%At high \qsq in the data sample, we observed a tail in the negative direction in the distribution
%of $\tau(\db)$.
%The extent of this tail in different \qsq regions can be seen in \Fig{fig:negtau:withmc}.
%This background is a kind of combinatorial background, it can be seen in the data for
%$m^\mathrm{meas}(\Bd)-m^\mathrm{PDG}(\Bd)>100\mev$ in the $m(\db), \tau(\db)$ plane in
%\Fig{fig:negtau:2d}.

%This background comes from dark boson candidates that originate from behind the vertex of the \Bd,
%this happens because the $\chisq_\mathrm{IP}$ for high \qsq candidates is necessarily higher than
%that of a low \qsq candidate.
%Therefore these candidates are accepted.
%We tested this using the same decay with the mass criteria flipped, that is:
%\begin{itemize}
  %\item $m(\Kp\pim)$ is unrestrained (though interested in high mass),
  %\item $m(\mumu)\in[796,996]$ so as to mimic the \Kstar mass,
%\end{itemize}
%and looking at mass against lifetime the same effect is visible (though to a much lesser extent).




%\begin{figure}
  %\begin{center}
    %\subfloat[\label{fig:negtau:withmc:low}]{\includegraphics[width=0.48\textwidth]{anaNegTau_2_6}}
    %\subfloat[\label{fig:negtau:withmc:mid}]{\includegraphics[width=0.48\textwidth]{anaNegTau_6_15}}\\
    %\subfloat[\label{fig:negtau:withmc:high}]{\includegraphics[width=0.48\textwidth]{anaNegTau_15_20}}
  %\end{center}
  %\caption{\small
    %Lifetime distributions for $\tau(\db)<0$ for three \qsq regions:
    %\protect\subref{fig:negtau:withmc:low} $2<\qsq<6$,
    %\protect\subref{fig:negtau:withmc:mid} $6<\qsq<15$, and
    %\protect\subref{fig:negtau:withmc:high} $15<\qsq<20\gevgev$.
    %The double Gaussian PDF centred at zero is from a fit to a MC sample in the given \qsq range.
    %One can clearly see that a tail extending out towards negative lifetimes as \qsq increases.
  %}
  %\label{fig:negtau:withmc}
%\end{figure}


%\begin{figure}
  %\begin{center}
    %\includegraphics[width=0.48\textwidth]{anaNegTau_kpi_2d}
  %\end{center}
  %\caption{\small
    %Mass against its lifetime for \Kstar candidate.
    %A slight negative skew in the lifetime dimension is observed.
  %}
  %\label{fig:negtau:2d}
%\end{figure}



