\chapter{Conclusions}
\label{ch:conc}

%\bam{
%AIM FOR 3 PAGES
%REFERENCE THE SHIT OUT OF IT
%}

This thesis presents two analyses undertaken using data collected by the \lhcb experiment
and a third to be completed, all
with the objective of searching for evidence of \bsm physics.
Each analysis looks for evidence in slightly different ways, and use either an indirect or direct
search approach.
%using indirect and direct methods, to search for evidence of physics that is
%inconsistent with our understanding of the \sm of particle physics.
%These analyses found the first evidence for the decay \btodsphi, the first observations of the
%decays \btokpipimumu and \btophikmumu, and the search for the decay of a dark boson into a dimuon
%pair.

%%%%%%%%%%%%%%%%%%%%%%%%%%%%%%%%%%%%%%%%%%%%%%%%%%%%%%%%%%%%%%%%%%%%%%%%%%%%%%%%%%%%%%%%%%%%%%%%%%
% Dsphi
%%%%%%%%%%%%%%%%%%%%%%%%%%%%%%%%%%%%%%%%%%%%%%%%%%%%%%%%%%%%%%%%%%%%%%%%%%%%%%%%%%%%%%%%%%%%%%%%%%
An analysis of the decay \btodsphi was presented in \Chap{ch:dsphi}.
Measurements made in this analysis were sensitive to \np effects contributing at tree
level because it proceeds via an annihilation-type diagram, and is mediated by a virtual \Wp.
This gauge boson can be replaced by a range of charged bosons adding to the amplitude, leading to
an enhanced branching fraction.
Additional \np diagrams would also introduce a cross-term in the amplitude, and therefore an
observed non-zero \CP-asymmetry would be indicative of \np.

Branching fraction predictions for the decay \btodsphi are of order
$10^{-7}$~\cite{Zou:2009zza,Mohanta:2002wf,PhysRevD.76.057701,Lu:2001yz}.
However, this analysis resulted in a higher than expected branching fraction, but one that is
still compatible with the \sm since large uncertainties in the hadronic form factors make the
branching fraction very difficult to predict.
Of the six candidates that were nearest the \Bp mass, three were from a decaying \Bp, and three
were from a \Bm; meaning that there is zero \CP-asymmetry.
This value had to be corrected for known asymmetries from detector, and production effects, but
final value was still consistent with $\acp=0$.

The branching fraction calculation sheds no light on the discrepancies observed in measurements of
\V{ub}.
A recent measurement from the \lhcb experiment uses the baryonic decay \decay{\Lb}{p\mun\neumb} to
calculate a value of $\left|\V{ub}\right|$ to be $(3.27\pm0.23)\e{-3}$~\cite{Aaij:2015bfa}.
This is an exclusive measurement, and as such is in agreement with other exclusive measurements.
Reference~\cite{Crivellin:2014zpa} asserts that the
discrepancies cannot be explained by new physics, and is rather due to underestimated uncertainties
in either theory or experiment.

An updated analysis would take advantage of the additional data collected by the \lhcb experiment
in 2012, and potentially in Run-2 of the \lhc.
Using additional \Ds decay channels should also be considered, particularly \decay{\Ds}{\pip\pipi}
which has a branching fraction of $(1.09\pm0.05)\e{-2}$: a factor of about five less than
$\BF\big(\decay{\Ds}{\kkpi}\big)$.
However, this would introduce larger and more complex backgrounds.
%Other annihilation-type diagrams may be useful given larger statistics.


%%%%%%%%%%%%%%%%%%%%%%%%%%%%%%%%%%%%%%%%%%%%%%%%%%%%%%%%%%%%%%%%%%%%%%%%%%%%%%%%%%%%%%%%%%%%%%%%%%
% hhhmumu
%%%%%%%%%%%%%%%%%%%%%%%%%%%%%%%%%%%%%%%%%%%%%%%%%%%%%%%%%%%%%%%%%%%%%%%%%%%%%%%%%%%%%%%%%%%%%%%%%%

Chapter~\ref{ch:hhh} presents an analysis leading to the first observations of the two decays
\btokpipimumu and \btophikmumu.
Theoretical predictions to these inclusive decay are made difficult by the structure of the strange
resonance states that contribute to the \kpipi and \phik systems.

The decay \btokpipimumu is a high statistics channel and could be used for future analyses similar to
those of interest in \btokstrmumu, since both are sensitive to the same operators.
An angular analysis would help constrain the scalar, pseudoscalar, and tensor amplitudes of the
decay, all of which are vanishingly small in the \sm, but this is made difficult by the number of
contributing states to the \kpipi system,
but with mode statistics it would perhaps be possible to gain a better understanding of these
strange states.
It would also be possible to use the dimuon distribution from \btokpipimumu to search for a dark
boson using the same method outlined in \Chap{ch:db}.
Although the invariant mass of the \kpipi system is, on average, larger than the \Kstarz mass,
there is still plenty of phasespace for the dimuon spectrum from \btokpipimumu to be interesting.
This is because as soon as the mass of the hypothetical dark boson is heavier than twice the mass
of the $\tau$ the branching fractions (in non-leptophobic models) plummet.
Therefore, regardless of the decay, it would not be expected to see a new particle in the diumon
spectrum above $m_\db=2m_\tau\simeq3554\mev$.

Given larger statistics, the additional channels \decay{\Bp}{\Kp\Km\pip} and \decay{\Bp}{\pip\pipi}
may be observable, and give access to the ratio $\V{td}/\V{ts}$ and be complimentary to the current
measurements from $B$-meson oscillations.

%The difference in \kpipi distributions...
Improvements could trivially be made after another measurement of
$\BF\big(\btojpsiphik\big)$ with reduced uncertainties.
For this reason, \Ref{LHCb-PAPER-2014-030} quotes the ratio of branching fractions.

Future measurements of these decays would benefit from additional statistics.
Particularly in order to understand the complex resonance structure observed in the hadronic
systems.
If one could begin to understand the structure of the \kpipi system, not only would make for
interesting spectroscopy measurements, but also it would give a handle on to the \kpipi angular
distributions.
This, in turn, would make \btokpipimumu sensitive to the same operators to \btokstrmumu.

The observable $P_5^\prime$ is an angular observable in \btokpipimumu which has reduced theoretical
observables~\cite{LHCb-PAPER-2013-037} because it nearly free from reliance on form-factors.
A recent measurement from \lhcb indicates that it is $3.7\stdev$ away from standard model
predictions in the region $4.0<\qsq<8.0\gevgev$.



Increased data would also allow measurements other three-hadron, two-muon final states.
Measurements of the branching fractions of the decays \decay{\Bp}{\pip\pipi\mumu} and
\decay{\Bp}{\kk\pip\mumu} would give access to the ratio of \ckm matrix elements $\V{td}/\V{ts}$.

The decay \btokpipimumu has a relatively large branching fraction, and could be used in future
searches for \np particles in the dimuon distribution: in a similar way to the analysis described
in \Chap{ch:db}.


%First observations
%potential in the future to measure vtd/vts, also another background free channel to look for a \db
%doesn't matter that m(K(1270)) > m(K*), because there is still plenty of phasespace, nothing counts
%below m(tautau) anyway.
%%
%high statistics
%FCNC, observables?
%angular distrbutions difficult
%Strange resonance structures interesting

%Angular analsus constrain (pseudo)scalsr and texsor amplitudes, which ae canishinglu small in th
%sm, uet enhanced in many vsm scenarios., it is stated on lllll tensor



%%%%%%%%%%%%%%%%%%%%%%%%%%%%%%%%%%%%%%%%%%%%%%%%%%%%%%%%%%%%%%%%%%%%%%%%%%%%%%%%%%%%%%%%%%%%%%%%%%
% DARK BOSON
%%%%%%%%%%%%%%%%%%%%%%%%%%%%%%%%%%%%%%%%%%%%%%%%%%%%%%%%%%%%%%%%%%%%%%%%%%%%%%%%%%%%%%%%%%%%%%%%%%

Finally, a direct search for a dark boson that may belong to any number of \bsm physics models that
include a dark sector.
FILL THIS BIT IN, NOT FINISHED THAT CONCLUSION YET, DEPENDS WHAT RESULTS I SHOW.
\begin{itemize}
  \item \btokstrdb
  \item Sensitivity
  \item Novel statistical techniques make this possible
  \item Also selection
  \item Probe regions of parameter space for various models
  \item Use same techniqes on other mass distributions
  \item SHIP?
  \item Models uniquly acessible to \lhcb
\end{itemize}


%%%%%%%%%%%%%%%%%%%%%%%%%%%%%%%%%%%%%%%%%%%%%%%%%%%%%%%%%%%%%%%%%%%%%%%%%%%%%%%%%%%%%%%%%%%%%%%%%%
% CONCLUSIONS
%%%%%%%%%%%%%%%%%%%%%%%%%%%%%%%%%%%%%%%%%%%%%%%%%%%%%%%%%%%%%%%%%%%%%%%%%%%%%%%%%%%%%%%%%%%%%%%%%%

In conclusion, the \sm continues in its resilience, seeming to be in agreement to the limit of
accuracy that experimental high energy particle physics can reach.
There is a complementarity that exists between indirect and direct measurements.
Historically, it has often been the case that indications of future discoveries were first
anticipated by observations made by indirect experiments.
Precision measurements in the flavour sector will continue to play an important role in cornering
the nature of \np.
Another advantage of indirect measurements is their ability to access vastly higher energy scales
than direct measurements due to contributions from virtual particles.

Considering the current landscape, it is perhaps increasingly difficult to interpret
discrepancies observed between experiment and theory in various channels.
As of Run 1 of the \lhc, there has been no clear indication of where new physics may lie, there
have been no clear-cut indications of \np from rare tree- and loop-level decays, but there are a
number of discrepancies with significances greater than $3\stdev$.
%\begin{itemize}
  %\item LHCb has different sensitivity to atlas
  %\item C9
  %\item P5 primed
  %\item D*taunu
%\end{itemize}

%Beyond the interpretation of measurements, the ideal scenario would, of course, to have

Unfortunately, interpretation of precise measurements of $B$ physics observables that can be made
experimentally are often made difficult by the theoretical framework which must be adopted to make
predictions.
Firstly, the theoretical environment must account for multiple energy scales: the scale of \np, the
electroweak scale, the scale of \QCD, and of hadronic effects.
The problem can be simplified using \EFT to separate short and long distance effects by integrating
out the heavy fields and dealing only with the dynamics of lighter particles.
Further simplification to the \QCD effects can be made with \HQET.

The idea that nature is natural, is an attractive one.
As such, it is not unreasonable to expect --- or at least hope --- that \np lives just
around the corner.
It is tempting to push to higher energies to look for clear signatures indicative of \np.
However, there are swathes of parameter space yet to explore at the energies achieved during Run 1
of the \lhc.


%\begin{itemize}
  %\item complementarity between direct and indirect
  %\item LHCb has different sensitivity to atlas
  %\item C9
  %\item P5 primed
  %\item D*taunu
  %\item precision tests offered by indirect
%\end{itemize}

%Of course, there are theoretcial uncertainties, primarily introduced by \QCD effects.
%
%consistency with the \sm
%
%from Vdb measurements
%FCNC
%Further restructing parameter space for dark sector models
%
%Additiosnal meaasurements would be rreat...
%
%
%Nice plot of measurements from different experiments.



%presented ways of serching for direct evidence of np at lhcb, which is unusual
%direct and indirect are complimntray




In summary, the interplay between direct and indirect meansurements in the arena of high energy
physics has long been important, and will continue to be so.






