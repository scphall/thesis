\chapter{Conclusions}
\label{ch:conc}

\bam{
AIM FOR 3 PAGES
REFERENCE THE SHIT OUT OF IT
}

This thesis presents two analyses undertaken using data collected by the \lhcb experiment
and a third to be completed, all
with the objective of searching for evidence of \bsm physics.
Each analysis looks for evidence in slightly different ways, and use either an indirect or direct
search approach.
%using indirect and direct methods, to search for evidence of physics that is
%inconsistent with our understanding of the \sm of particle physics.
%These analyses found the first evidence for the decay \btodsphi, the first observations of the
%decays \btokpipimumu and \btophikmumu, and the search for the decay of a dark boson into a dimuon
%pair.

%%%%%%%%%%%%%%%%%%%%%%%%%%%%%%%%%%%%%%%%%%%%%%%%%%%%%%%%%%%%%%%%%%%%%%%%%%%%%%%%%%%%%%%%%%%%%%%%%%
% Dsphi
%%%%%%%%%%%%%%%%%%%%%%%%%%%%%%%%%%%%%%%%%%%%%%%%%%%%%%%%%%%%%%%%%%%%%%%%%%%%%%%%%%%%%%%%%%%%%%%%%%
An analysis of the decay \btodsphi was presented in \Chap{ch:dsphi}.
Measurements made in this analysis were sensitive to \np effects contributing at tree
level because it proceeds via an annihilation-type diagram, and is mediated by a virtual \Wp.
This gauge boson can be replaced by a range of charged bosons adding to the amplitude, leading to
an enhanced branching fraction.
Additional \np diagrams would also introduce a cross-term in the amplitude, and therefore an
observed non-zero \CP-asymmetry would be indicative of \np.

Branching fraction predictions for the decay \btodsphi are of order
$10^{-7}$~\cite{Zou:2009zza,Mohanta:2002wf,PhysRevD.76.057701,Lu:2001yz}.
However, this analysis resulted in a higher than expected branching fraction, but one that is
still compatible with the \sm since large uncertainties in the hadronic form factors make the
branching fraction very difficult to predict.
Of the six candidates that were nearest the \Bp mass, three were from a decaying \Bp, and three
were from a \Bm; meaning that there is zero \CP-asymmetry.
This value had to be corrected for known asymmetries from detector, and production effects, but
final value was still consistent with $\acp=0$.

%of the  and be observable above the \sm contribution because of the suppression of the decay from
%the factor of $\V{ub}$ in the amplitude.

%in which the branching
%fraction of the decay was measured.

%The first evidence of the decay \btodsphi was presented in \Chap{ch:dsphi}, which is also the
%first evidence of a fully hadronic annihilation-type decay.
%The measured branching fraction is somewhat higher than the theoretical
%predictions~\cite{Zou:2009zza,Mohanta:2002wf,PhysRevD.76.057701,Lu:2001yz} which are of
%order $10^{-7}$, but the large theoretical uncertainties mean that the experimental value is
%not incompatible with the \sm.

%ACP

%At the time that the search for the decay \btodsphi was embarked upon, tensions between the
%inclusive and exclusive measurements of \V{ub} were larger than they currently are.
%However,
%still important to

%The BF measured is in agreement with the sm predictions,
%More about Vub, recent results?
%ACP compatible with the \sm




%%%%%%%%%%%%%%%%%%%%%%%%%%%%%%%%%%%%%%%%%%%%%%%%%%%%%%%%%%%%%%%%%%%%%%%%%%%%%%%%%%%%%%%%%%%%%%%%%%
% hhhmumu
%%%%%%%%%%%%%%%%%%%%%%%%%%%%%%%%%%%%%%%%%%%%%%%%%%%%%%%%%%%%%%%%%%%%%%%%%%%%%%%%%%%%%%%%%%%%%%%%%%

Chapter~\ref{ch:hhh} presents an analysis leading to the first observations of the two decays
\btokpipimumu and \btophikmumu.




First observations
potential in the future to measure vtd/vts, also another background free channel to look for a \db
doesn't matter that m(K(1270)) > m(K*), because there is still plenty of phasespace, nothing counts
below m(tautau) anyway.

high statistics
FCNC, observables?
angular distrbutions difficult
Strange resonance structures interesting

Angular analsus constrain (pseudo)scalsr and texsor amplitudes, which ae canishinglu small in th
sm, uet enhanced in many vsm scenarios., it is stated on lllll tensor



\btokstrdb
Sensitivity
Novel statistical techniques make this possible
Also selection
Most sensitive...
Probe regions of parameter space for various models



In conclusion,
consistency with the \sm

from Vdb measurements
FCNC
Further restructing parameter space for dark sector models

Additiosnal meaasurements would be rreat...


Nice plot of measurements from different experiments.








