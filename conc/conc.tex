\chapter{Conclusions}
\label{ch:conc}

This thesis presents three published
analyses~\cite{LHCb-PAPER-2012-025,LHCb-PAPER-2014-030,LHCb-PAPER-2015-036}, all
undertaken using data collected by the \lhcb experiment~\cite{Alves:2008zz}.
The results of two of these analyses boast first observations.
Each has the objective of finding evidence of physics \bsm in the decays of $B$ mesons.
The analysis techniques employed are all slightly different, most obviously
Ref.~\cite{LHCb-PAPER-2012-025} and Ref.~\cite{LHCb-PAPER-2014-030} describe indirect
searches for \np, while Ref.~\cite{LHCb-PAPER-2015-036} is a direct search for a dark boson of
unknown mass and lifetime.

%%%%%%%%%%%%%%%%%%%%%%%%%%%%%%%%%%%%%%%%%%%%%%%%%%%%%%%%%%%%%%%%%%%%%%%%%%%%%%%%%%%%%%%%%%%%%%%%%%
% Dsphi
%%%%%%%%%%%%%%%%%%%%%%%%%%%%%%%%%%%%%%%%%%%%%%%%%%%%%%%%%%%%%%%%%%%%%%%%%%%%%%%%%%%%%%%%%%%%%%%%%%
An analysis of the decay \btodsphi is presented in \Chap{ch:dsphi}.
First evidence for the decay was seen with a statistical significance of greater than $3\sigma$,
this also constitutes the first evidence for a fully hadronic decay via an annihilation-type
diagram.
The branching fraction measurement made in this analysis is sensitive to \np effects contributing
to the decay \btodsphi can only propagate via one diagram at tree-level, and this is suppressed by
a factor $|\V{ub}|^2$.
The element \V{ub} has the largest uncertainties in the \ckm matrix and there are historic tensions
between values of \V{ub} made using inclusive an exclusive modes.

The branching fraction measurement
\begin{equation*}
  \BF\big(\btodsphi\big) =
  \big(1.87\,^{+1.25}_{-0.73}\stat\pm0.19\syst\pm0.32\normerr\big)\e{-6}
  %\label{eq:dsphi:result}
\end{equation*}
is somewhat higher than \sm predictions, which are
of order $10^{-7}$~\cite{Zou:2009zza,Mohanta:2002wf,PhysRevD.76.057701,Lu:2001yz}, but
not incompatible considering \added{the range of the theoretical predictions, and the large
statistical uncertainties inherent in the measurement}.
This value of $\BF(\btodsphi)$ sheds no light on the true value of \V{ub}.
Reference~\cite{Crivellin:2014zpa} asserts that these
discrepancies cannot be explained by physics \bsm, and is rather due to underestimated
uncertainties in either theory or experiment.

Since the decay \btodsphi is mediated by a \Wp boson, in \np scenarios another charged boson, such
as a $H^+$ from a \gls{twoHDM}, could contribute to the decay amplitude.
These additional processes could alter the branching fraction significantly, which is not observed,
or introduce extra phases into the decay, causing the \CP-asymmetry to deviate from $\acp=0$, as
expected in the \sm.
The value measured, after correcting for production and detection asymmetries, is
\begin{equation*}
  \acp\big(\btodsphi\big)=
  -\big(0.01\pm0.41\stat\pm0.03\syst\big),
\end{equation*}
which is consistent with \sm expectations.


%%%%%%%%%%%%%%%%%%%%%%%%%%%%%%%%%%%%%%%%%%%%%%%%%%%%%%%%%%%%%%%%%%%%%%%%%%%%%%%%%%%%%%%%%%%%%%%%%%
% hhhmumu
%%%%%%%%%%%%%%%%%%%%%%%%%%%%%%%%%%%%%%%%%%%%%%%%%%%%%%%%%%%%%%%%%%%%%%%%%%%%%%%%%%%%%%%%%%%%%%%%%%

Chapter~\ref{ch:hhh} presented an analysis with the first observations and a measurements of the
branching fractions of the decays \btokpipimumu and \btophikmumu, as well as the
differential branching fraction of \btokpipimumu in bins of \qsq.
The integrated branching fractions of these decays are
\begin{align*}
  \BF\big( \btokpipimumu \big)&=
  \big(4.36\,^{+0.29}_{-0.27}\stat\pm0.21\syst\pm0.18\normerr\big)\e{-7}, \\
  \BF\big(\btophikmumu\big)&=
  \big(0.82\,^{+0.19}_{-0.17}\stat\,^{+0.10}_{-0.04}\syst\pm0.27\normerr\big)\e{-7},
\end{align*}
and both have statistical significances of greater than $5\sigma$
The decay \btokpipimumu has a large branching fraction and could be used for future analyses
similar to those of interest in \btokstrmumu, since both are sensitive to the same operators.
An angular analysis would help constrain the scalar, pseudoscalar, and tensor amplitudes of the
decay, all of which are vanishingly small in the \sm.
This is made difficult by the number of contributing states to the \kpipi system,
but with more statistics it will be possible to gain a better understanding of
strange states that decay into kaons and pions.

Interpreting precise measurements of $B$ physics observables that can be made
experimentally are often made difficult by the form-factor parameterisation which must be adopted.
It is these \QCD effects, particularly the form-factors, that are the dominant sources of
theoretical uncertainty.
Difficulties in dealing with \QCD has been demonstrated by a lack of consistent predictions for
$\BF(\btodsphi)$, and the absence of any predictions for either $\BF(\btokpipimumu)$ and
$\BF(\btophikmumu)$.
Techniques such as lattice \QCD~\cite{Bouchard:2013pna} have recently been employed in this are
with encouraging results~\cite{RotheLattice}.
This bodes well for future measurements of this kind.

%general H. J. Rothe, Lattice Gauge Theories: An Introduction; 4th ed., World Scientific Lecture
%Notes in Physics, World Scientific, Singapore, 2012.
%kmumu http://arxiv.org/abs/1306.2384


Given larger statistics, the additional channels \decay{\Bp}{\Kp\Km\pip\mumu} and
\decay{\Bp}{\pip\pipi\mumu}
may be observable, and give access to the ratio of \ckm matrix elements $\V{td}/\V{ts}$.
These would be complimentary to the current
measurements from $B$-meson oscillations.

The large uncertainties in the measurement of $\BF(\btophikmumu)$ are primarily due to
uncertainties propagated from the branching fraction of the normalization channel \btojpsiphik.
The paper in \Ref{LHCb-PAPER-2014-030} quotes the ratio of branching fractions in order for
$\BF(\btophikmumu)$ to be calculated given an improved measurement.

%%%%%%%%%%%%%%%%%%%%%%%%%%%%%%%%%%%%%%%%%%%%%%%%%%%%%%%%%%%%%%%%%%%%%%%%%%%%%%%%%%%%%%%%%%%%%%%%%%
% DARK BOSON
%%%%%%%%%%%%%%%%%%%%%%%%%%%%%%%%%%%%%%%%%%%%%%%%%%%%%%%%%%%%%%%%%%%%%%%%%%%%%%%%%%%%%%%%%%%%%%%%%%

Finally, a direct search for a \np particle, \db, belonging to some dark sector is presented.
Using
\btokstrmumu candidates the dimuon invariant mass spectrum is searched for a signal indicative of a
dark boson decaying via \dbtomumu.
The selection is designed specifically not to bias any corner of the mass-lifetime space that the
\db might inhabit, this is done primarily with the aid of the uBoost algorithm.
Efficiencies and resolutions are parameterised using discrete simulated samples of \btokstrdb
and spline interpolation is used to understand selection and resolution effects for any value of
\mass{\db}.
A novel frequentist strategy was employed to perform the search.
The strategy involved a scan in regions of dimuon invariant mass.
For each region, the $p$-value for a signal excess was calculated.
Once the look elsewhere effect was accounted for the minimum local $p$-value is $0.48\sigma$ at
$m_{\mumu}=4285.0\mev$.
%which scans in mass for and, at each point, calculates
%the local $p$-value that the observed signal is consistent with the null hypothesis of zero signal.
%Once the look-elsewhere effect was accounted for, the significance of the minimum local $p$-value
%was equivalent to a significance of
%$0.48\sigma$ at $m_{\mumu}=4285.0\mev$.
This is consistent with with no new particle observed.
Further studies will push tested masses to the boundaries of vetoed regions.
The projected limits for an inflaton model indicate that much of the allowed parameter space will
be excluded for $m_\db<1000\mev$.
A similar approach could be used in any arbitrary mass spectrum to search for a multitude of
particles appearing above a smoothly varying background.


%%%%%%%%%%%%%%%%%%%%%%%%%%%%%%%%%%%%%%%%%%%%%%%%%%%%%%%%%%%%%%%%%%%%%%%%%%%%%%%%%%%%%%%%%%%%%%%%%%
% CONCLUSIONS
%%%%%%%%%%%%%%%%%%%%%%%%%%%%%%%%%%%%%%%%%%%%%%%%%%%%%%%%%%%%%%%%%%%%%%%%%%%%%%%%%%%%%%%%%%%%%%%%%%
\added{
In conclusion, the \sm continues in its resilience, seeming to be in agreement to the limit of
accuracy that experimental high energy particle physics can reach.
There is a complementarity that exists between indirect and direct measurements.
Historically, it has often been the case that indications of future discoveries were first
anticipated by observations made by indirect experiments.
%This has been the case because indirect measurements are sensitive to loop-level processes and
%therefore virtual particles.
This pattern looks set to endure,
as precision measurements in the flavour sector continue to play an important role in searching for
\np.
As of Run 1 of the \lhc, there has been no clear indication of where \np may lie.
There are no results indicating \np with a significance of $5\sigma$.
However, there are many flavour physics observables with significances greater than
$3\sigma$~\cite{LHCb-CONF-2015-002,LHCb-PAPER-2015-025,Lees:2013uzd,Lees:2012xj,Bozek:2010xy,LHCb-PAPER-2014-024,LHCb-PAPER-2015-023}
which makes the area very interesting.
%making the
%but, there are a number of discrepancies between \sm predictions and measurements of observables in
%the flavour sector with significances greater than $3\sigma$.
%Although none of these reach the $5\sigma$ mark, the
%interesting.
}
%\deleted{
%Considering the current landscape, it is perhaps increasingly difficult to interpret these
%discrepancies, and how they impact various \np models.
%}

\added{
%The idea that nature behaves in a physically natural way, is an attractive one.
It is not unreasonable to expect --- or at least hope --- that \np lives just
around the corner.
With the factor of two increase in statistics expected in LHCb's total dataset in Run 2 of the
\lhc, it will be very interesting to see how the flavour sector observables that differ from the
\sm expectation change.
It is clear that the important interplay between indirect and direct searches in the arena of
\gls{HEP} is set to continue.
}


%The idea that nature is natural, is an attractive one.
%As such, it is not unreasonable to expect --- or at least hope --- that \np lives just
%around the corner.
%Run 2 of the \lhc will collide protons with a centre-of-mass energy of about $14\tev$ with the aim
%to see signals indicative of \np.
%That being said, there are still areas of parameter space of various theoretical scenarios that are
%accessible at lower masses to probe.
%In the absence of direct evidence from the high energy frontier, and
%the plethora of dark sector models which contain weakly interacting messenger particles, the
%intensity frontier may be a good place to search.
%
%The interplay between direct and indirect searches in the arena of high energy physics
%has long been important, and will continue to be so.








\clearpage
