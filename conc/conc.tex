\chapter{Conclusions}
\label{ch:conc}

This thesis presents three analyses: two have been published, claiming first
observations~\cite{LHCb-PAPER-2012-025,LHCb-PAPER-2014-030}, and the third is currently under
review; all were
undertaken using data collected by the \lhcb experiment~\cite{Alves:2008zz}.
Each has the objective of finding evidence of physics \bsm in the decays of $B$ mesons.
The analysis techniques employed are all slightly different, most obviously the first two analyses
are indirect searches for \np, while the third is a direct search for an dark boson of unknown mass
and lifetime.

%%%%%%%%%%%%%%%%%%%%%%%%%%%%%%%%%%%%%%%%%%%%%%%%%%%%%%%%%%%%%%%%%%%%%%%%%%%%%%%%%%%%%%%%%%%%%%%%%%
% Dsphi
%%%%%%%%%%%%%%%%%%%%%%%%%%%%%%%%%%%%%%%%%%%%%%%%%%%%%%%%%%%%%%%%%%%%%%%%%%%%%%%%%%%%%%%%%%%%%%%%%%
An analysis of the decay \btodsphi is presented in \Chap{ch:dsphi}.
First evidence for the decay was seen with a statistical significance of greater than $3\stdev$,
this also constitutes the first evidence for a fully hadronic decay via an annihilation-type
diagram.
Measurements made in this analysis were sensitive to \np effects contributing at tree
level because it proceeds via an annihilation-type diagram, suppressed by a factor $|\V{ub}|^2$.
The element \V{ub} has the largest uncertainties in the \ckm matrix and there are historic tensions
between values of \V{ub} made using inclusive an exclusive modes.

%This means that the decay amplitude for \btodsphi is suppressed by a factor $|\V{ub}|^2$, making it
%a rare decay having potentially large contributions at tree-level from \np particles.

The branching fraction measurement
\begin{equation*}
  \BF\big(\btodsphi\big) =
  \big(1.87\,^{+1.25}_{-0.73}\stat\pm0.19\syst\pm0.32\normerr\big)\e{-6}
  %\label{eq:dsphi:result}
\end{equation*}
is somewhat higher than \sm predictions, which are
of order $10^{-7}$~\cite{Zou:2009zza,Mohanta:2002wf,PhysRevD.76.057701,Lu:2001yz}, but
not incompatible considering large theoretical uncertainties.
This value of $\BF(\btodsphi)$ sheds no light on the true value of \V{ub}.
Reference~\cite{Crivellin:2014zpa} asserts that these
discrepancies cannot be explained by new physics, and is rather due to underestimated uncertainties
in either theory or experiment.

Since the decay \btodsphi is mediated by a \Wp boson, in \np scenarios another charged boson, such
as a $H^+$ from a \gls{twoHDM}, could contribute to the decay amplitude.
These additional processes could alter the branching fraction significantly, which is not observed,
or introduce extra phases into the decay, causing the \CP-asymmetry to deviate from $\acp=0$, as
expected in the \sm.
The value measured, after correcting for production and detection asymmetries, is
\begin{equation*}
  \acp\big(\btodsphi\big)=
  -\big(0.01\pm0.41\stat\pm0.03\syst\big),
\end{equation*}
which is perfectly consistent with \sm expectations.

%sheds no light on the discrepancies observed in measurements of
%\V{ub}, and nor does the \V{ub} measurement using the decay \decay{\Lb}{p\mun\neumb} which is
%consistent with other exclusive measurements.
%Reference~\cite{Crivellin:2014zpa} asserts that these
%discrepancies cannot be explained by new physics, and is rather due to underestimated uncertainties
%in either theory or experiment.

%and is mediated by a virtual \Wp.
%This gauge boson can be replaced by a range of charged bosons adding to the amplitude, leading to
%an enhanced branching fraction.
%Additional \np diagrams would also introduce a cross-term in the amplitude, and therefore an
%observed non-zero \CP-asymmetry would be indicative of \np.

%This annihilation-type decay is mediated by a \Wp boson, which could be replaced by other charged
%bosons from \bsm models; for a example a charged Higgs from a \gls{twoHDM}.
%Were this to be the case the value of $\BF(\btodsphi)$ could be significatly different to \sm
%predictions.
%Branching fraction predictions for the decay \btodsphi are of order
%$10^{-7}$~\cite{Zou:2009zza,Mohanta:2002wf,PhysRevD.76.057701,Lu:2001yz}.
%This analysis resulted in a higher than expected branching fraction, but one that is
%still compatible with the \sm since large uncertainties in the hadronic form factors make the
%branching fraction very difficult to predict.

%Another observable in the decay \btodsphi is the \CP-asymmetry, which should be zero in the \sm,
%because there is only a single phase in the amplitude at leading order.
%Additional phases from \np could alter the value of \acp to being inconsistent with zero.
%However, of the six \btodsphi candidates falling within $2\stdev$ of the \Bp mass, three were from
%a decaying \Bp, and three were the charge conjugate decay from \Bm, giving a raw \acp of zero.
%After corrections to account for known production and detector asymmetries, the final value of \acp
%was still consistent with zero.

%An updated analysis would take advantage of the additional data collected by the \lhcb experiment
%in 2012, and potentially in data collected by \lhcb in Run-2, which begins in 2015.
%Using additional \Ds decay channels should also be considered, particularly \decay{\Ds}{\pip\pipi}
%which has a branching fraction of $(1.09\pm0.05)\e{-2}$: a factor of about five less than
%$\BF\big(\decay{\Ds}{\kkpi}\big)$.
%However, this would introduce larger and more complex backgrounds.
%Other annihilation-type diagrams may be useful given larger statistics.


%%%%%%%%%%%%%%%%%%%%%%%%%%%%%%%%%%%%%%%%%%%%%%%%%%%%%%%%%%%%%%%%%%%%%%%%%%%%%%%%%%%%%%%%%%%%%%%%%%
% hhhmumu
%%%%%%%%%%%%%%%%%%%%%%%%%%%%%%%%%%%%%%%%%%%%%%%%%%%%%%%%%%%%%%%%%%%%%%%%%%%%%%%%%%%%%%%%%%%%%%%%%%

Chapter~\ref{ch:hhh} presents an analysis leading to the first observations of the two decays
\btokpipimumu and \btophikmumu\, their branching fractions were measured, as was the differential
branching fraction of \btokpipimumu in bins of \qsq.
The integrated branching fractions of these decays are
\begin{align*}
  \BF\big( \btokpipimumu \big)&=
  \big(4.36\,^{+0.29}_{-0.27}\stat\pm0.21\syst\pm0.18\normerr\big)\e{-7}, \\
  \BF\big(\btophikmumu\big)&=
  \big(0.82\,^{+0.19}_{-0.17}\stat\,^{+0.10}_{-0.04}\syst\pm0.27\normerr\big)\e{-7},
\end{align*}
and both have statistical significances of greater than $5\stdev$
%Theoretical predictions to these inclusive decay are made difficult by the structure of the strange
%resonance states that contribute to the \kpipi and \phik systems.
The decay \btokpipimumu has a large branching fraction and could be used for future analyses
similar to those of interest in \btokstrmumu, since both are sensitive to the same operators.
An angular analysis would help constrain the scalar, pseudoscalar, and tensor amplitudes of the
decay, all of which are vanishingly small in the \sm.
This is made difficult by the number of contributing states to the \kpipi system,
but with more statistics it will be possible to gain a better understanding of
strange states that decay into kaons and pions.
%There doubtless interesting spectroscopy to be done regarding these states.

%It would also be possible to use the dimuon distribution from \btokpipimumu to search for a dark
%boson using the same method outlined in \Chap{ch:db}.
%Although the invariant mass of the \kpipi system is, on average, larger than the \Kstarz mass,
%there is still plenty of phasespace for the dimuon spectrum from \btokpipimumu to be interesting in
%the region $m_{\mumu} < 2m_{\tau}$.
%This is because as soon as the mass of the hypothetical dark boson is heavier than twice the mass
%of the $\tau$ the branching fractions (in non-leptophobic models) plummet.
%Therefore, regardless of the decay, it would not be expected to see a new particle in the diumon
%spectrum above $m_\db=2m_\tau\simeq3554\mev$.

%Future measurements of these decays would benefit from additional statistics.
%Particularly in order to understand the complex resonance structure observed in the hadronic
%systems.
%If one could begin to understand the structure of the \kpipi system, not only would make for
%interesting spectroscopy measurements, but also it would give a handle on to the \kpipi angular
%distributions.
%This, in turn, would make \btokpipimumu sensitive to the same operators to \btokstrmumu.

Given larger statistics, the additional channels \decay{\Bp}{\Kp\Km\pip\mumu} and
\decay{\Bp}{\pip\pipi\mumu}
may be observable, and give access to the ratio of \ckm matrix elements $\V{td}/\V{ts}$ and be
complimentary to the current
measurements from $B$-meson oscillations.

%The difference in \kpipi distributions...

The large uncertainties in the measurement of $\BF(\btophikmumu)$ are primarily due to
uncertainties propagated from the branching fraction of the normalization channel \btojpsiphik.
The paper in \Ref{LHCb-PAPER-2014-030} quotes the ratio of branching fractions in order for
$\BF(\btophikmumu)$ to be calculated given an improved measurement.

%Given increased statistics and further understanding of the \kpipi system, perhaps a complimentary
%measurement of $C_9$ could  be made, and the source of the deviation understood.
%Either additional massive vector particles are contributing, or the \jpsi is
%The angular observable $P_5^\prime$ is an angular observable in \btokpipimumu which has reduced theoret
%observables~\cite{LHCb-PAPER-2013-037} because it nearly free from reliance on form-factors.
%A measurement from \lhcb indicates that it is $3.7\stdev$ away from standard model
%predictions in the region $4.0<\qsq<8.0\gevgev$.

%Increased data would also allow measurements other three-hadron, two-muon final states.
%Measurements of the branching fractions of the decays \decay{\Bp}{\pip\pipi\mumu} and
%\decay{\Bp}{\kk\pip\mumu} would give access to the ratio of \ckm matrix elements $\V{td}/\V{ts}$.

%The decay \btokpipimumu has a relatively large branching fraction, and could be used in future
%searches for \np particles in the dimuon distribution: in a similar way to the analysis described
%in \Chap{ch:db}.

%First observations
%potential in the future to measure vtd/vts, also another background free channel to look for a \db
%doesn't matter that m(K(1270)) > m(K*), because there is still plenty of phasespace, nothing counts
%below m(tautau) anyway.
%%
%high statistics
%FCNC, observables?
%angular distrbutions difficult
%Strange resonance structures interesting

%Angular analsus constrain (pseudo)scalsr and texsor amplitudes, which ae canishinglu small in th
%sm, uet enhanced in many vsm scenarios., it is stated on lllll tensor


%%%%%%%%%%%%%%%%%%%%%%%%%%%%%%%%%%%%%%%%%%%%%%%%%%%%%%%%%%%%%%%%%%%%%%%%%%%%%%%%%%%%%%%%%%%%%%%%%%
% DARK BOSON
%%%%%%%%%%%%%%%%%%%%%%%%%%%%%%%%%%%%%%%%%%%%%%%%%%%%%%%%%%%%%%%%%%%%%%%%%%%%%%%%%%%%%%%%%%%%%%%%%%

%Finally, a direct search for a dark boson that may belong to any number of \bsm physics models that
%include a dark sector.
%\bam{PVALUE SENTENCE.}
%
%The projected sensitivity of this analysis probes regions that have, hitherto, been unexplored.
%Selection techniques, such as uBoost, and novel statistical techniques are set to make these
%analyses increasingly common; enabling searches for excesses in any invariant mass spectrum that
%may be interesting.
%
%\lhcb lies in a unique position to probe these parameter spaces.
%
%%It is possible that the \lhc will be the last energy frontier machine for many years, an
%
%There are a great many dark sector models, with particles of with arbitarty lifetimes implying
%small coupling, therefore intensity frontier.
%SHIP.


Finally, a direct search for a \np particle, \db, belonging to some dark sector is presented using
\btokstrmumu candidates, and searching for an signal in the dimuon invariant mass spectrum.
The selection is designed specifically not to bias any corner of the mass-lifetime space that the
\db might inhabit, this is done primarily with the aid of the uBoost algorithm.
Efficiencies and resolutions are parameterized using discrete simulated samples of \btokstrdb
and spline interpolation is used to understand selecting and dynamical effects for any value of
\mass{\db}.
A novel strategy was employed to extract the minimum local $p$-value, and so a global $p$-value,
which is $xxx$ and $m=$.
The projected limits for an inflaton model indicate that much of the allowed parameter space will
be excluded for $m_\db<1000\mev$.
A similar approach could be used in any arbitrary mass spectrum to search for a multitude of
particles appearing above a smoothly varying background.

%\begin{itemize}
  %\item \btokstrdb
  %\item Sensitivity
  %\item Novel statistical techniques make this possible
  %\item Also selection
  %\item Probe regions of parameter space for various models
  %\item Use same techniqes on other mass distributions
  %\item SHIP?
  %\item Models uniquly acessible to \lhcb
%\end{itemize}

%%%%%%%%%%%%%%%%%%%%%%%%%%%%%%%%%%%%%%%%%%%%%%%%%%%%%%%%%%%%%%%%%%%%%%%%%%%%%%%%%%%%%%%%%%%%%%%%%%
% CONCLUSIONS
%%%%%%%%%%%%%%%%%%%%%%%%%%%%%%%%%%%%%%%%%%%%%%%%%%%%%%%%%%%%%%%%%%%%%%%%%%%%%%%%%%%%%%%%%%%%%%%%%%
In conclusion, the \sm continues in its resilience, seeming to be in agreement to the limit of
accuracy that experimental high energy particle physics can reach.
There is a complementarity that exists between indirect and direct measurements.
Historically, it has often been the case that indications of future discoveries were first
anticipated by observations made by indirect experiments.
This looks set to continue,
as precision measurements in the flavour sector will continue to play an important role in
cornering the nature of \np.
Another advantage of indirect measurements is their ability to access vastly higher energy scales
than direct measurements due to contributions from virtual particles.

Considering the current landscape, it is perhaps increasingly difficult to interpret
discrepancies observed between experiment and theory in various channels.
As of Run 1 of the \lhc, there has been no clear indication of where new physics may lie, there
have been no clear-cut indications of \np from rare tree- and loop-level decays, but there are a
number of discrepancies with significances greater than $3\stdev$.

%\begin{itemize}
  %\item LHCb has different sensitivity to atlas
  %\item C9
  %\item P5 primed
  %\item D*taunu
%\end{itemize}
%Beyond the interpretation of measurements, the ideal scenario would, of course, to have

Unfortunately, interpretation of precise measurements of $B$ physics observables that can be made
experimentally are often made difficult by the form-factor parameterization which must be adopted.
Firstly, the theoretical environment must account for multiple energy scales.
The \np and electroweak scales are conveniently separated by
using \EFT to separate short and long distance effects by integrating
out the heavy fields and dealing only with the dynamics of lighter particles.
Lastly, \QCD calculations must be made, which is where dominant theoretical uncertainties arise.


The idea that nature is natural, is an attractive one.
As such, it is not unreasonable to expect --- or at least hope --- that \np lives just
around the corner.
Run-2 of the \lhc will collide protons with a centre-of-mass energy of $14\tev$ with the aim to see
signals indicative of \np.
That being said, there are still areas of parameter space of various theoretical scenarios that are
accessible at lower masses to probe.
%It is tempting to push to higher energies to look for clear signatures indicative of \np.
%However, there are swathes of parameter space yet to explore at the energies achieved during Run 1
%of the \lhc.
In the absence of direct evidence from the high energy frontier, it is worth also exploring the
intensity frontier.
With the plethora of dark sector models which contain weakly interacting messenger particles, the
intensity frontier may be a good place to search.

The interplay between direct and indirect searches in the arena of high energy physics
has long been important, and will continue to be so.

%Considering that the weaker the coupling, the longer lived the \db, an experiment which facilitates
%a longer decay volume would be an excellent addition to the current particle detectors.
%Such an experiment is \gls{SHIP}~\cite{Anelli:2015pba,Alekhin:2015byh}, which is a long baseline
%experiment proposed to be fed by the \sps.




%\begin{itemize}
  %\item complementarity between direct and indirect
  %\item LHCb has different sensitivity to atlas
  %\item C9
  %\item P5 primed
  %\item RK
  %\item D*taunu
  %\item precision tests offered by indirect
%\end{itemize}

%Of course, there are theoretcial uncertainties, primarily introduced by \QCD effects.
%
%consistency with the \sm
%
%from Vdb measurements
%FCNC
%Further restructing parameter space for dark sector models
%
%Additiosnal meaasurements would be rreat...
%
%
%Nice plot of measurements from different experiments.



%presented ways of serching for direct evidence of np at lhcb, which is unusual
%direct and indirect are complimntray





