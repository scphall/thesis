% This file contains all the default packages and modifications for
% LHCb formatting

% Show layouts
\usepackage{layout}
%\usepackage{showframe}
\footskip=25pt
\usepackage[a4paper,top=2.5cm,bottom=2.0cm,left=2.5cm,right=2.5cm,asymmetric]{geometry}
\headheight=14pt

%% Allow the page size to vary a bit ...
\raggedbottom
% To avoid Latex to be too fussy with line breaking ...
\sloppy


%% %%%%%%%%%%%%%%%%%%
%%  Packages
%% %%%%%%%%%%%%%%%%%%
\usepackage{booktabs}
\usepackage{lipsum}
\usepackage{parskip}
\usepackage{xspace} % To avoid problems with missing or double spaces
\usepackage{subfig}
\usepackage{microtype}
\usepackage{graphicx}
\usepackage{color}
\usepackage{colortbl}
\usepackage{verse}
%\usepackage{bm}

\usepackage{tikz}
\usepackage{pdftricks}
\begin{psinputs}
  \usepackage{pstricks}
  \usepackage{pst-node}
\end{psinputs}
\usetikzlibrary{positioning,arrows,patterns}
\usetikzlibrary{decorations.markings}
\usetikzlibrary{decorations.pathreplacing}
\usetikzlibrary{calc}

\usepackage{lineno} % for line numbering

%% Graphics
\graphicspath{{./figs/}
              {./Thesis/figs/}
              {./intro/figs/}
              {./theory/figs/}
              {./lhcb/figs/}
              {./dsphi/figs/}
              {./hhh/figs/}
              {./db/figs/}
              {./conc/figs/}}

%% Math
\usepackage{amsmath} % Adds a large collection of math symbols
\usepackage{amssymb}
\usepackage{amsfonts}
\usepackage{upgreek} % Adds in support for greek letters in roman typeset
\usepackage{rotating}
\usepackage{setspace}
\usepackage{calc}
\usepackage{multirow}

\usepackage{preamble/tocloft}
\renewcommand{\cftsecleader}{\cftdotfill{\cftdotsep}}
\makeatletter
%\renewcommand{\@dotsep}{5000}
%\renewcommand*\l@section{\@dottedtocline{1}{1.5em}{2.3em}}
\makeatother
\renewcommand{\cftdot}{{\color{black!35}.}}


\usepackage[justification=justified,singlelinecheck=false]{caption}
\usepackage[labelfont=bf,format=hang]{caption}

\renewcommand{\abstractname}{}


\onehalfspacing


%\newcommand\firstsections

% FONTS
%\usepackage[T1]{fontenc}
%\renewcommand*\rmdefault{ppl}
%\renewcommand*\rmdefault{lmr}
%\renewcommand*\rmdefault{stix}
%\usepackage{fourier}
%\usepackage[T1]{fontenc}
%\newfontfamily\semibold{Minion Pro Semibold}

%\fontseries{b}\selectfont
%\renewcommand{\bfdefault}{sb}
%bx bold extended (default)
%b bold
%sb semibold
%m medium
%c condensed

\usepackage{ifthen}
\newboolean{pdflatex}
\setboolean{pdflatex}{true} % False for eps figures

\newboolean{articletitles}
\setboolean{articletitles}{true} % False removes titles in references

\newboolean{uprightparticles}
\setboolean{uprightparticles}{false} %True for upright particle symbols

\usepackage{preamble/slashed}
%\usepackage[printonlyused,withpage]{acronym}
%\setacronymstyle{long-short}

%% fix to allow peaceful coexistence of line numbering and
%% mathematical objects
%% http://www.latex-community.org/forum/viewtopic.php?f=5&t=163
%%
\newcommand*\patchAmsMathEnvironmentForLineno[1]{%
\expandafter\let\csname old#1\expandafter\endcsname\csname #1\endcsname
\expandafter\let\csname oldend#1\expandafter\endcsname\csname
end#1\endcsname
 \renewenvironment{#1}%
   {\linenomath\csname old#1\endcsname}%
   {\csname oldend#1\endcsname\endlinenomath}%
}
\newcommand*\patchBothAmsMathEnvironmentsForLineno[1]{%
  \patchAmsMathEnvironmentForLineno{#1}%
  \patchAmsMathEnvironmentForLineno{#1*}%
}
\AtBeginDocument{%
\patchBothAmsMathEnvironmentsForLineno{equation}%
\patchBothAmsMathEnvironmentsForLineno{align}%
\patchBothAmsMathEnvironmentsForLineno{flalign}%
\patchBothAmsMathEnvironmentsForLineno{alignat}%
\patchBothAmsMathEnvironmentsForLineno{gather}%
\patchBothAmsMathEnvironmentsForLineno{multline}%
}



% Get hyperlinks to captions and in references.
% These do not work with revtex. Use "hypertext" as class option instead.
\usepackage{hyperref}    % Hyperlinks in references
\usepackage[all]{hypcap} % Internal hyperlinks to floats.  bookmarks

%\usepackage[acronym]{glossaries}
\usepackage[nopostdot,nogroupskip,xindy,nonumberlist]{glossaries}
%\setglossarystyle{altlist}
%\usepackage{glossary-mcols}
%\setglossarystyle{mcolindex}

\makeglossaries
\newacronym[longplural=European Centre for Nuclear Research's]{CERNLabel}{CERN}{European Centre for Nuclear Research}
\newacronym{BBDT}{BBDT}{Bonsai BDT}
\newacronym{BDT}{BDT}{Boosted Decision Tree}
\newacronym{BSM}{BSM}{Beyond the Standard Model}
\newacronym{CC}{CC}{Charged Current}
\newacronym{CKM}{CKM}{Cabibbo-Kobayashi-Maskawa}
\newacronym{DM}{DM}{Dark Matter}
\newacronym{DT}{DT}{Decision Tree}
\newacronym{ECALLabel}{ECAL}{Electromagnetic Calorimeter}
\newacronym{EFT}{EFT}{Effective Field Theory}
\newacronym{FCNC}{FCNC}{Flavour Changing Neutral Current}
\newacronym{IT}{IT}{Inner Tracker}
\newacronym{OT}{OT}{Outer Tracker}
\newacronym{ST}{ST}{Silicon Tracker}
\newacronym{HPD}{HPD}{Hybrid Photomultipliers}
\newacronym{HCALLabel}{HCAL}{Hadron Calorimeter}
\newacronym{HLTLabel}{HLT}{High Level Trigger}
\newacronym{HQET}{HQET}{Heavy Quark Effective Theory}
\newacronym{IP}{IP}{Impact Parameter}
\newacronym{LHCLabel}{LHC}{Large Hadron Collider}
\newacronym{LHCbLabel}{LHCb}{LHC beauty}
\newacronym{M125Label}{M1-5}{Muon stations}
\newacronym{MFV}{MFV}{Minimal Flavour Violation}
\newacronym{MVA}{MVA}{Multivariate Analysis}
\newacronym{NC}{NC}{Neutral Current}
\newacronym{NP}{NP}{New Physics}
\newacronym{OPE}{OPE}{Operator Product Expansion}
\newacronym{PIDLabel}{PID}{Particle Identification}
\newacronym{PID}{PID}{Particle Identification}
\newacronym{PSLabel}{PS}{Preshower detector}
\newacronym{PV}{PV}{Primary Vertex}
\newacronym{QCD}{QCD}{Quantum Chromodynamics}
\newacronym{QED}{QED}{Quantum Electrodynamics}
\newacronym{RICHLabel}{RICH}{Ring Imaging Cherenkov}
\newacronym{RPV}{RPV}{$R$-Parity Violation}
\newacronym{SM}{SM}{Standard Model}
\newacronym{SPDLabel}{SPD}{Scintialting Pad Detector}
\newacronym{SPSLabel}{SPS}{Super Proton Synchrotron}
\newacronym{SSB}{SSB}{Spontaneous Symmetry Breaking}
\newacronym{SUSY}{SUSY}{Supersymmetry}
\newacronym{T123Label}{T1-3}{Trigger stations}
\newacronym{TTLabel}{TT}{Tracker Turicensis}
\newacronym{UT}{UT}{Unitarity Triangle}
\newacronym{VELOLabel}{VELO}{Vertex Locator}
\newacronym{VEV}{VEV}{Vacuum Expectation Value}
\newacronym{twoHDM}{2HDM}{Two Higgs Doublet Model}
\newacronym{uBDT}{uBDT}{Uniform BDT}
\newacronym{CP}{\ensuremath{C\!P}}{Charge-Parity}
\newacronym{CPV}{\ensuremath{C\!PV}}{$C\!P$ Violation}
\newacronym{C}{\ensuremath{C}}{Charge}
\newacronym{P}{\ensuremath{P}}{Parity}
\newacronym{T}{\ensuremath{T}}{Time}
\newacronym{L0}{L0}{Level 0 Trigger}
\newacronym{HLT1}{HLT1}{High Level Trigger 1}
\newacronym{HLT2}{HLT2}{High Level Trigger 2}
\newacronym{TIS}{\ensuremath{\mathtt{TIS}}\xspace}{Trigger Independent of Signal}
\newacronym{TOS}{\ensuremath{\mathtt{TOS}}\xspace}{Trigger On of Signal}
\newacronym{BAU}{BAU}{Baryon Asymmetry of the Universe}
\newacronym{MWPC}{MWPC}{Multi-Wire Proportional Chambers}

%\newacronym[longplural={Flavour Changing Neutral Currents},shortplural={FCNCs}]{FCNC}{FCNC}{Flavour Changing Neutral Current}

\def\BAU {\gls{BAU}\xspace}
\def\CP {\gls{CP}\xspace}
\def\CPV {\gls{CPV}\xspace}
\def\BBDT {\gls{BBDT}\xspace}
\def\BDT {\gls{BDT}\xspace}
\def\BDTs {\glspl{BDT}\xspace}
\def\CC {\gls{CC}\xspace}
\def\DT {\gls{DT}\xspace}
\def\DTs {\glspl{DT}\xspace}
\def\EFT {\gls{EFT}\xspace}
\def\HQET {\gls{HQET}\xspace}
\def\MFV {\gls{MFV}\xspace}
\def\MVA {\gls{MVA}\xspace}
\def\NC {\gls{NC}\xspace}
\def\OPE {\gls{OPE}\xspace}
\def\QCD {\gls{QCD}\xspace}
\def\QED {\gls{QED}\xspace}
\def\SSB {\gls{SSB}\xspace}
\def\SUSY {\gls{SUSY}\xspace}
\def\VEV {\gls{VEV}\xspace}
\def\bsm {\gls{BSM}\xspace}
\def\ckm {\gls{CKM}\xspace}
\def\dm {\gls{DM}\xspace}
\def\fcnc {\gls{FCNC}\xspace}
\def\fcncs {\glspl{FCNC}\xspace}
\def\np {\gls{NP}\xspace}
\def\pid {\gls{PID}\xspace}
\def\pv {\gls{PV}\xspace}
\def\rpv {\gls{RPV}\xspace}
\def\sm {\gls{SM}\xspace}
\def\twoHDM {\gls{twoHDM}\xspace}
\def\uBDT {\gls{uBDT}\xspace}
\def\ut {\gls{UT}\xspace}

%\newglossaryentry{Linux}
%{
  %name=Linux,
  %description={is a generic term referring to the family of Unix-like
  %computer operating systems that use the Linux kernel},
  %plural=Linuces
%}
%\newacronym[longplural=Frames per Second]{fpsLabel}{FPS}{Frame per Second}



\newlength\maxlength
\newlength\thislength

%\newglossarystyle{mystyle}
%{%
  %\renewenvironment{theglossary}%
  %{% start of glossary
   %% Find maximum width of the first column:
    %\setlength{\maxlength}{0pt}%
    %\forglsentries[\currentglossary]{\thislabel}%
    %{%
       %\settowidth{\thislength}{\glsentryshort{\thislabel}}%
       %\ifdim\thislength>\maxlength
         %\setlength{\maxlength}{\thislength}%
       %\fi
    %}%
    %% Now calculate the width of the second column:
    %\settowidth{\thislength}{\hspace{1.5em}=\hspace{1em}}%
    %\setlength{\glsdescwidth}{\linewidth-\maxlength-\thislength-2\tabcolsep}%
    %% Start the tabular environment
    %%\begin{tabular}{l@{\hspace{1.5em}=\hspace{1em}}p{\glsdescwidth}}
    %%\begin{longtable}{lp{\glsdescwidth}@{\hspace{1.5em}\hspace{3em}}}
    %\begin{longtable}{p{0.1\textwidth}p{0.95\textwidth}}
    %%\flushleft
    %%\begin{longtable}{lp{\glsdescwidth}}
    %%\toprule
    %%\multicolumn{1}{l}{\textbf{Abk.}} &
    %%\multicolumn{1}{@{}l}{\textbf{Bedeutung}}\\%
    %%\midrule
  %}%
  %{% end of glossary
     %%\bottomrule
     %\end{longtable}%
     %%\end{tabular}%
  %}%
  %% Header has been incorporated into \begin{theglossary}
  %\renewcommand*{\glossaryheader}{}%
  %% Don't do anything between letter groups
  %\renewcommand*{\glsgroupheading}[1]{}%
  %\renewcommand*{\glsgroupskip}{}%
  %% Set display for each the acronym entry
  %\renewcommand{\glossentry}[2]{%
    %\glstarget{##1}{\glsentryshort{##1}}% short form
    %&
    %\glsentrylong{##1}% long form
    %\\% end of row
  %}%
  %% No sub-entries, so \subglossentry doesn't need redefining
%}
\newglossarystyle{mystyle}
{%
  \renewenvironment{theglossary}%
  {% start of glossary
   % Find maximum width of the first column:
    \setlength{\maxlength}{0pt}%
    \forglsentries[\currentglossary]{\thislabel}%
    {%
       \settowidth{\thislength}{\glsentryshort{\thislabel}}%
       \ifdim\thislength>\maxlength
         \setlength{\maxlength}{\thislength}%
       \fi
    }%
    % Now calculate the width of the second column:
    %\settowidth{\thislength}{\hspace{1.5em}=\hspace{1em}}%
    %\setlength{\glsdescwidth}{\linewidth-\maxlength-\thislength-2\tabcolsep}%
    % Start the tabular environment
    %\begin{tabular}{l@{\hspace{1.5em}=\hspace{1em}}p{\glsdescwidth}}
    %\begin{longtable}{lp{\glsdescwidth}@{\hspace{1.5em}\hspace{3em}}}
    %\begin{longtable}{p{0.1\textwidth}p{0.95\textwidth}}
    %\flushleft
    %\begin{longtable}{lp{\glsdescwidth}}
    %\toprule
    %\multicolumn{1}{l}{\textbf{Abk.}} &
    %\multicolumn{1}{@{}l}{\textbf{Bedeutung}}\\%
    %\midrule
  }%
  {% end of glossary
     %\bottomrule
     %\end{longtable}%
     %\end{tabular}%
  }%
  % Header has been incorporated into \begin{theglossary}
  \renewcommand*{\glossaryheader}{}%
  % Don't do anything between letter groups
  \renewcommand*{\glsgroupheading}[1]{}%
  \renewcommand*{\glsgroupskip}{}%
  % Set display for each the acronym entry
  \renewcommand{\glossentry}[2]{%
    \makebox[0.12\textwidth][l]{\glstarget{##1}{\glsentryshort{##1}}}% short form
    \glsentrylong{##1}% long form
    \\% end of row
  }%
  % No sub-entries, so \subglossentry doesn't need redefining
}

 % Add in the glossary

%%% $Id: lhcb-symbols-def.tex 49563 2014-02-26 14:13:00Z shall $
%%% ======================================================================
%%% Purpose: Standard LHCb aliases
%%% Author: Originally Ulrik Egede, adapted by Tomasz Skwarnicki for templates,
%%% rewritten by Chris Parkes
%%% Maintainer : Ulrik Egede (2010 - 2012)
%%% =======================================================================

%%% To use this file outside the normal LHCb document environment, the
%%% following should be added in a preamble (before \begin{document}
%%%


%%%%%%%%%%%%%
% Experiments
%%%%%%%%%%%%%
\def\lhcb {\mbox{LHCb}\xspace}
%\def\lhcb {\gls{LHCbLabel}\xspace}
\def\ux85 {\mbox{UX85}\xspace}
\def\cern {\mbox{CERN}\xspace}
%\def\lhc    {\mbox{LHC}\xspace}
\def\lhc    {\gls{LHCLabel}\xspace}
\def\sps    {\gls{SPSLabel}\xspace}
\def\atlas  {\mbox{ATLAS}\xspace}
\def\cms    {\mbox{CMS}\xspace}
\def\alice  {\mbox{ALICE}\xspace}
\def\babar  {\mbox{BaBar}\xspace}
\def\belle  {\mbox{Belle}\xspace}
\def\aleph  {\mbox{ALEPH}\xspace}
\def\delphi {\mbox{DELPHI}\xspace}
\def\opal   {\mbox{OPAL}\xspace}
\def\lthree {\mbox{L3}\xspace}
\def\lep    {\mbox{LEP}\xspace}
\def\cdf    {\mbox{CDF}\xspace}
\def\dzero  {\mbox{D0}\xspace}
\def\sld    {\mbox{SLD}\xspace}
\def\cleo   {\mbox{CLEO}\xspace}
\def\argus  {\mbox{ARGUS}\xspace}
\def\uaone  {\mbox{UA1}\xspace}
\def\uatwo  {\mbox{UA2}\xspace}
\def\tevatron {Tevatron\xspace}

%% LHCb sub-detectors and sub-systems

\def\pu     {PU\xspace}
%\def\velo   {VELO\xspace}
%\def\rich   {RICH\xspace}
\def\velo   {\gls{VELOLabel}\xspace}
\def\rich   {\gls{RICHLabel}\xspace}
\def\richone {RICH1\xspace}
\def\richtwo {RICH2\xspace}
%\def\ttracker {TT\xspace}
\def\ttracker {\gls{TTLabel}\xspace}
\def\intr   {IT\xspace}
\def\st     {ST\xspace}
\def\ot     {OT\xspace}
\def\Tone   {T1\xspace}
\def\Ttwo   {T2\xspace}
\def\Tthree {T3\xspace}
\def\Mone   {M1\xspace}
\def\Mtwo   {M2\xspace}
\def\Mthree {M3\xspace}
\def\Mfour  {M4\xspace}
\def\Mfive  {M5\xspace}
\def\Muons  {\gls{M125Label}\xspace}
\def\Tracks  {\gls{T123Label}\xspace}
%\def\ecal   {ECAL\xspace}
%\def\spd    {SPD\xspace}
%\def\presh  {PS\xspace}
%\def\hcal   {HCAL\xspace}
\def\ecal   {\gls{ECALLabel}\xspace}
\def\spd    {\gls{SPDLabel}\xspace}
\def\presh  {\gls{PSLabel}\xspace}
\def\hcal   {\gls{HCALLabel}\xspace}
\def\bcm    {BCM\xspace}

\def\ode    {ODE\xspace}
\def\daq    {DAQ\xspace}
\def\tfc    {TFC\xspace}
\def\ecs    {ECS\xspace}
\def\lone   {L0\xspace}
%\def\hlt    {HLT\xspace}
\def\hlt    {\gls{HLTLabel}\xspace}
\def\hltone {HLT1\xspace}
\def\hlttwo {HLT2\xspace}

%%% Upright (not slanted) Particles

\ifthenelse{\boolean{uprightparticles}}%
{\def\Palpha      {\ensuremath{\upalpha}\xspace}
 \def\Pbeta       {\ensuremath{\upbeta}\xspace}
 \def\Pgamma      {\ensuremath{\upgamma}\xspace}
 \def\Pdelta      {\ensuremath{\updelta}\xspace}
 \def\Pepsilon    {\ensuremath{\upepsilon}\xspace}
 \def\Pvarepsilon {\ensuremath{\upvarepsilon}\xspace}
 \def\Pzeta       {\ensuremath{\upzeta}\xspace}
 \def\Peta        {\ensuremath{\upeta}\xspace}
 \def\Ptheta      {\ensuremath{\uptheta}\xspace}
 \def\Pvartheta   {\ensuremath{\upvartheta}\xspace}
 \def\Piota       {\ensuremath{\upiota}\xspace}
 \def\Pkappa      {\ensuremath{\upkappa}\xspace}
 \def\Plambda     {\ensuremath{\uplambda}\xspace}
 \def\Pmu         {\ensuremath{\upmu}\xspace}
 \def\Pnu         {\ensuremath{\upnu}\xspace}
 \def\Pxi         {\ensuremath{\upxi}\xspace}
 \def\Ppi         {\ensuremath{\uppi}\xspace}
 \def\Pvarpi      {\ensuremath{\upvarpi}\xspace}
 \def\Prho        {\ensuremath{\uprho}\xspace}
 \def\Pvarrho     {\ensuremath{\upvarrho}\xspace}
 \def\Ptau        {\ensuremath{\uptau}\xspace}
 \def\Pupsilon    {\ensuremath{\upupsilon}\xspace}
 \def\Pphi        {\ensuremath{\upphi}\xspace}
 \def\Pvarphi     {\ensuremath{\upvarphi}\xspace}
 \def\Pchi        {\ensuremath{\upchi}\xspace}
 \def\Ppsi        {\ensuremath{\uppsi}\xspace}
 \def\Pomega      {\ensuremath{\upomega}\xspace}

 \def\PDelta      {\ensuremath{\Delta}\xspace}
 \def\PXi      {\ensuremath{\Xi}\xspace}
 \def\PLambda      {\ensuremath{\Lambda}\xspace}
 \def\PSigma      {\ensuremath{\Sigma}\xspace}
 \def\POmega      {\ensuremath{\Omega}\xspace}
 \def\PUpsilon      {\ensuremath{\Upsilon}\xspace}

 %\mathchardef\Deltares="7101
 %\mathchardef\Xi="7104
 %\mathchardef\Lambda="7103
 %\mathchardef\Sigma="7106
 %\mathchardef\Omega="710A


 \def\PA      {\ensuremath{\mathrm{A}}\xspace}
 \def\PB      {\ensuremath{\mathrm{B}}\xspace}
 \def\PC      {\ensuremath{\mathrm{C}}\xspace}
 \def\PD      {\ensuremath{\mathrm{D}}\xspace}
 \def\PE      {\ensuremath{\mathrm{E}}\xspace}
 \def\PF      {\ensuremath{\mathrm{F}}\xspace}
 \def\PG      {\ensuremath{\mathrm{G}}\xspace}
 \def\PH      {\ensuremath{\mathrm{H}}\xspace}
 \def\PI      {\ensuremath{\mathrm{I}}\xspace}
 \def\PJ      {\ensuremath{\mathrm{J}}\xspace}
 \def\PK      {\ensuremath{\mathrm{K}}\xspace}
 \def\PL      {\ensuremath{\mathrm{L}}\xspace}
 \def\PM      {\ensuremath{\mathrm{M}}\xspace}
 \def\PN      {\ensuremath{\mathrm{N}}\xspace}
 \def\PO      {\ensuremath{\mathrm{O}}\xspace}
 \def\PP      {\ensuremath{\mathrm{P}}\xspace}
 \def\PQ      {\ensuremath{\mathrm{Q}}\xspace}
 \def\PR      {\ensuremath{\mathrm{R}}\xspace}
 \def\PS      {\ensuremath{\mathrm{S}}\xspace}
 \def\PT      {\ensuremath{\mathrm{T}}\xspace}
 \def\PU      {\ensuremath{\mathrm{U}}\xspace}
 \def\PV      {\ensuremath{\mathrm{V}}\xspace}
 \def\PW      {\ensuremath{\mathrm{W}}\xspace}
 \def\PX      {\ensuremath{\mathrm{X}}\xspace}
 \def\PY      {\ensuremath{\mathrm{Y}}\xspace}
 \def\PZ      {\ensuremath{\mathrm{Z}}\xspace}
 \def\Pa      {\ensuremath{\mathrm{a}}\xspace}
 \def\Pb      {\ensuremath{\mathrm{b}}\xspace}
 \def\Pc      {\ensuremath{\mathrm{c}}\xspace}
 \def\Pd      {\ensuremath{\mathrm{d}}\xspace}
 \def\Pe      {\ensuremath{\mathrm{e}}\xspace}
 \def\Pf      {\ensuremath{\mathrm{f}}\xspace}
 \def\Pg      {\ensuremath{\mathrm{g}}\xspace}
 \def\Ph      {\ensuremath{\mathrm{h}}\xspace}
 \def\Pi      {\ensuremath{\mathrm{i}}\xspace}
 \def\Pj      {\ensuremath{\mathrm{j}}\xspace}
 \def\Pk      {\ensuremath{\mathrm{k}}\xspace}
 \def\Pl      {\ensuremath{\mathrm{l}}\xspace}
 \def\Pm      {\ensuremath{\mathrm{m}}\xspace}
 \def\Pn      {\ensuremath{\mathrm{n}}\xspace}
 \def\Po      {\ensuremath{\mathrm{o}}\xspace}
 \def\Pp      {\ensuremath{\mathrm{p}}\xspace}
 \def\Pq      {\ensuremath{\mathrm{q}}\xspace}
 \def\Pr      {\ensuremath{\mathrm{r}}\xspace}
 \def\Ps      {\ensuremath{\mathrm{s}}\xspace}
 \def\Pt      {\ensuremath{\mathrm{t}}\xspace}
 \def\Pu      {\ensuremath{\mathrm{u}}\xspace}
 \def\Pv      {\ensuremath{\mathrm{v}}\xspace}
 \def\Pw      {\ensuremath{\mathrm{w}}\xspace}
 \def\Px      {\ensuremath{\mathrm{x}}\xspace}
 \def\Py      {\ensuremath{\mathrm{y}}\xspace}
 \def\Pz      {\ensuremath{\mathrm{z}}\xspace}
}
{
  \def\Palpha      {\ensuremath{\alpha}\xspace}
 \def\Pbeta       {\ensuremath{\beta}\xspace}
 \def\Pgamma      {\ensuremath{\gamma}\xspace}
 \def\Pdelta      {\ensuremath{\delta}\xspace}
 \def\Pepsilon    {\ensuremath{\epsilon}\xspace}
 \def\Pvarepsilon {\ensuremath{\varepsilon}\xspace}
 \def\Pzeta       {\ensuremath{\zeta}\xspace}
 \def\Peta        {\ensuremath{\eta}\xspace}
 \def\Ptheta      {\ensuremath{\theta}\xspace}
 \def\Pvartheta   {\ensuremath{\vartheta}\xspace}
 \def\Piota       {\ensuremath{\iota}\xspace}
 \def\Pkappa      {\ensuremath{\kappa}\xspace}
 \def\Plambda     {\ensuremath{\lambda}\xspace}
 \def\Pmu         {\ensuremath{\mu}\xspace}
 \def\Pnu         {\ensuremath{\nu}\xspace}
 \def\Pxi         {\ensuremath{\xi}\xspace}
 \def\Ppi         {\ensuremath{\pi}\xspace}
 \def\Pvarpi      {\ensuremath{\varpi}\xspace}
 \def\Prho        {\ensuremath{\rho}\xspace}
 \def\Pvarrho     {\ensuremath{\varrho}\xspace}
 \def\Ptau        {\ensuremath{\tau}\xspace}
 \def\Pupsilon    {\ensuremath{\upsilon}\xspace}
 \def\Pphi        {\ensuremath{\phi}\xspace}
 \def\Pvarphi     {\ensuremath{\varphi}\xspace}
 \def\Pchi        {\ensuremath{\chi}\xspace}
 \def\Ppsi        {\ensuremath{\psi}\xspace}
 \def\Pomega      {\ensuremath{\omega}\xspace}
 \mathchardef\PDelta="7101
 \mathchardef\PXi="7104
 \mathchardef\PLambda="7103
 \mathchardef\PSigma="7106
 \mathchardef\POmega="710A
 \mathchardef\PUpsilon="7107
 \def\PA      {\ensuremath{A}\xspace}
 \def\PB      {\ensuremath{B}\xspace}
 \def\PC      {\ensuremath{C}\xspace}
 \def\PD      {\ensuremath{D}\xspace}
 \def\PE      {\ensuremath{E}\xspace}
 \def\PF      {\ensuremath{F}\xspace}
 \def\PG      {\ensuremath{G}\xspace}
 \def\PH      {\ensuremath{H}\xspace}
 \def\PI      {\ensuremath{I}\xspace}
 \def\PJ      {\ensuremath{J}\xspace}
 \def\PK      {\ensuremath{K}\xspace}
 \def\PL      {\ensuremath{L}\xspace}
 \def\PM      {\ensuremath{M}\xspace}
 \def\PN      {\ensuremath{N}\xspace}
 \def\PO      {\ensuremath{O}\xspace}
 \def\PP      {\ensuremath{P}\xspace}
 \def\PQ      {\ensuremath{Q}\xspace}
 \def\PR      {\ensuremath{R}\xspace}
 \def\PS      {\ensuremath{S}\xspace}
 \def\PT      {\ensuremath{T}\xspace}
 \def\PU      {\ensuremath{U}\xspace}
 \def\PV      {\ensuremath{V}\xspace}
 \def\PW      {\ensuremath{W}\xspace}
 \def\PX      {\ensuremath{X}\xspace}
 \def\PY      {\ensuremath{Y}\xspace}
 \def\PZ      {\ensuremath{Z}\xspace}
 \def\Pa      {\ensuremath{a}\xspace}
 \def\Pb      {\ensuremath{b}\xspace}
 \def\Pc      {\ensuremath{c}\xspace}
 \def\Pd      {\ensuremath{d}\xspace}
 \def\Pe      {\ensuremath{e}\xspace}
 \def\Pf      {\ensuremath{f}\xspace}
 \def\Pg      {\ensuremath{g}\xspace}
 \def\Ph      {\ensuremath{h}\xspace}
 \def\Pi      {\ensuremath{i}\xspace}
 \def\Pj      {\ensuremath{j}\xspace}
 \def\Pk      {\ensuremath{k}\xspace}
 \def\Pl      {\ensuremath{l}\xspace}
 \def\Pm      {\ensuremath{m}\xspace}
 \def\Pn      {\ensuremath{n}\xspace}
 \def\Po      {\ensuremath{o}\xspace}
 \def\Pp      {\ensuremath{p}\xspace}
 \def\Pq      {\ensuremath{q}\xspace}
 \def\Pr      {\ensuremath{r}\xspace}
 \def\Ps      {\ensuremath{s}\xspace}
 \def\Pt      {\ensuremath{t}\xspace}
 \def\Pu      {\ensuremath{u}\xspace}
 \def\Pv      {\ensuremath{v}\xspace}
 \def\Pw      {\ensuremath{w}\xspace}
 \def\Px      {\ensuremath{x}\xspace}
 \def\Py      {\ensuremath{y}\xspace}
 \def\Pz      {\ensuremath{z}\xspace}
}

%%%%%%%%%%%%%%%%%%%%%%%%%%%%%%%%%%%%%%%%%%%%%%%
% Particles

%% Leptons

\let\emi\en
\def\electron   {\ensuremath{\Pe}\xspace}
\def\en         {\ensuremath{\Pe^-}\xspace}   % electron negative (\em is taken)
\def\ep         {\ensuremath{\Pe^+}\xspace}
\def\epm        {\ensuremath{\Pe^\pm}\xspace}
\def\epem       {\ensuremath{\Pe^+\Pe^-}\xspace}
\def\ee         {\ensuremath{\Pe^-\Pe^-}\xspace}

\def\mmu        {\ensuremath{\Pmu}\xspace}
\def\mup        {\ensuremath{\Pmu^+}\xspace}
\def\mun        {\ensuremath{\Pmu^-}\xspace} % muon negative (\mum is taken)
\def\mumu       {\ensuremath{\Pmu^+\Pmu^-}\xspace}
\def\mtau       {\ensuremath{\Ptau}\xspace}

\def\taup       {\ensuremath{\Ptau^+}\xspace}
\def\taum       {\ensuremath{\Ptau^-}\xspace}
\def\tautau     {\ensuremath{\Ptau^+\Ptau^-}\xspace}

\def\ellm       {\ensuremath{\ell^-}\xspace}
\def\ellp       {\ensuremath{\ell^+}\xspace}
\def\ellell     {\ensuremath{\ell^+ \ell^-}\xspace}

\def\neu        {\ensuremath{\Pnu}\xspace}
\def\neub       {\ensuremath{\overline{\Pnu}}\xspace}
\def\nuenueb    {\ensuremath{\neu\neub}\xspace}
\def\neue       {\ensuremath{\neu_e}\xspace}
\def\neueb      {\ensuremath{\neub_e}\xspace}
\def\neueneueb  {\ensuremath{\neue\neueb}\xspace}
\def\neum       {\ensuremath{\neu_\mu}\xspace}
\def\neumb      {\ensuremath{\neub_\mu}\xspace}
\def\neumneumb  {\ensuremath{\neum\neumb}\xspace}
\def\neut       {\ensuremath{\neu_\tau}\xspace}
\def\neutb      {\ensuremath{\neub_\tau}\xspace}
\def\neutneutb  {\ensuremath{\neut\neutb}\xspace}
\def\neul       {\ensuremath{\neu_\ell}\xspace}
\def\neulb      {\ensuremath{\neub_\ell}\xspace}
\def\neulneulb  {\ensuremath{\neul\neulb}\xspace}

%% Gauge bosons and scalars

\def\g      {\ensuremath{\Pgamma}\xspace}
\def\H      {\ensuremath{\PH^0}\xspace}
\def\Hp     {\ensuremath{\PH^+}\xspace}
\def\Hm     {\ensuremath{\PH^-}\xspace}
\def\Hpm    {\ensuremath{\PH^\pm}\xspace}
\def\W      {\ensuremath{\PW}\xspace}
\def\Wp     {\ensuremath{\PW^+}\xspace}
\def\Wm     {\ensuremath{\PW^-}\xspace}
\def\Wpm    {\ensuremath{\PW^\pm}\xspace}
\def\Z      {\ensuremath{\PZ^0}\xspace}

%% Quarks
\newcommand\xbar[1]  {\kern 0.1em\overline{\kern -0.1em #1}{}\xspace}

\def\quark     {\ensuremath{\Pq}\xspace}
\def\quarkbar  {\ensuremath{\xbar \quark}\xspace}
\def\qqbar     {\ensuremath{\quark\quarkbar}\xspace}
\def\uquark    {\ensuremath{\Pu}\xspace}
\def\uquarkbar {\ensuremath{\xbar \uquark}\xspace}
\def\uubar     {\ensuremath{\uquark\uquarkbar}\xspace}
\def\dquark    {\ensuremath{\Pd}\xspace}
\def\dquarkbar {\ensuremath{\xbar \dquark}\xspace}
\def\ddbar     {\ensuremath{\dquark\dquarkbar}\xspace}
\def\squark    {\ensuremath{\Ps}\xspace}
\def\squarkbar {\ensuremath{\xbar \squark}\xspace}
\def\ssbar     {\ensuremath{\squark\squarkbar}\xspace}
\def\cquark    {\ensuremath{\Pc}\xspace}
\def\cquarkbar {\ensuremath{\xbar \cquark}\xspace}
\def\ccbar     {\ensuremath{\cquark\cquarkbar}\xspace}
\def\bquark    {\ensuremath{\Pb}\xspace}
\def\bquarkbar {\ensuremath{\xbar \bquark}\xspace}
\def\bbbar     {\ensuremath{\bquark\bquarkbar}\xspace}
\def\tquark    {\ensuremath{\Pt}\xspace}
\def\tquarkbar {\ensuremath{\xbar \tquark}\xspace}
\def\ttbar     {\ensuremath{\tquark\tquarkbar}\xspace}

%% Light mesons

\def\pion  {\ensuremath{\Ppi}\xspace}
\def\piz   {\ensuremath{\pion^0}\xspace}
\def\pizs  {\ensuremath{\pion^0\mbox\,\rm{s}}\xspace}
\def\ppz   {\ensuremath{\pion^0\pion^0}\xspace}
\def\pip   {\ensuremath{\pion^+}\xspace}
\def\pim   {\ensuremath{\pion^-}\xspace}
\def\pipi  {\ensuremath{\pion^+\pion^-}\xspace}
\def\pipm  {\ensuremath{\pion^\pm}\xspace}
\def\pimp  {\ensuremath{\pion^\mp}\xspace}

\def\kaon  {\ensuremath{\PK}\xspace}
%%% do NOT use ensuremath here
  \def\Kbar  {\kern 0.2em\overline{\kern -0.2em \PK}{}\xspace}
\def\Kb    {\ensuremath{\Kbar}\xspace}
\def\Kz    {\ensuremath{\kaon^0}\xspace}
\def\Kzb   {\ensuremath{\Kbar^0}\xspace}
\def\KzKzb {\ensuremath{\Kz \kern -0.16em \Kzb}\xspace}
\def\Kp    {\ensuremath{\kaon^+}\xspace}
\def\Km    {\ensuremath{\kaon^-}\xspace}
\def\Kpm   {\ensuremath{\kaon^\pm}\xspace}
\def\Kmp   {\ensuremath{\kaon^\mp}\xspace}
\def\KpKm  {\ensuremath{\Kp \kern -0.16em \Km}\xspace}
\def\KS    {\ensuremath{\kaon^0_{\rm\scriptscriptstyle S}}\xspace}
\def\KL    {\ensuremath{\kaon^0_{\rm\scriptscriptstyle L}}\xspace}
\def\Kstarz  {\ensuremath{\kaon^{*0}}\xspace}
\def\Kstarzb {\ensuremath{\Kbar^{*0}}\xspace}
\def\Kstar   {\ensuremath{\kaon^*}\xspace}
\def\Kstarb  {\ensuremath{\Kbar^*}\xspace}
\def\Kstarp  {\ensuremath{\kaon^{*+}}\xspace}
\def\Kstarm  {\ensuremath{\kaon^{*-}}\xspace}
\def\Kstarpm {\ensuremath{\kaon^{*\pm}}\xspace}
\def\Kstarmp {\ensuremath{\kaon^{*\mp}}\xspace}
\def\Kone {\ensuremath{\kaon_1(1270)^+}\xspace}
\def\Koneprime {\ensuremath{\kaon_1(1400)^+}\xspace}
\def\Kop {\ensuremath{\kaon_1^+}\xspace}
\def\Kom {\ensuremath{\kaon_1^-}\xspace}
\def\KK  {\ensuremath{\Kp\Km}\xspace}

\newcommand{\etapr}{\ensuremath{\Peta^{\prime}}\xspace}

%% Heavy mesons

%%% do NOT use ensuremath here
  \def\Dbar    {\kern 0.2em\overline{\kern -0.2em \PD}{}\xspace}
\def\D       {\ensuremath{\PD}\xspace}
\def\Db      {\ensuremath{\Dbar}\xspace}
\def\Dz      {\ensuremath{\D^0}\xspace}
\def\Dzb     {\ensuremath{\Dbar^0}\xspace}
\def\DzDzb   {\ensuremath{\Dz {\kern -0.16em \Dzb}}\xspace}
\def\Dp      {\ensuremath{\D^+}\xspace}
\def\Dm      {\ensuremath{\D^-}\xspace}
\def\Dpm     {\ensuremath{\D^\pm}\xspace}
\def\Dmp     {\ensuremath{\D^\mp}\xspace}
\def\DpDm    {\ensuremath{\Dp {\kern -0.16em \Dm}}\xspace}
\def\Dstar   {\ensuremath{\D^*}\xspace}
\def\Dstarb  {\ensuremath{\Dbar^*}\xspace}
\def\Dstarz  {\ensuremath{\D^{*0}}\xspace}
\def\Dstarzb {\ensuremath{\Dbar^{*0}}\xspace}
\def\Dstarp  {\ensuremath{\D^{*+}}\xspace}
\def\Dstarm  {\ensuremath{\D^{*-}}\xspace}
\def\Dstarpm {\ensuremath{\D^{*\pm}}\xspace}
\def\Dstarmp {\ensuremath{\D^{*\mp}}\xspace}
\def\Ds      {\ensuremath{\D^+_\squark}\xspace}
\def\Dsp     {\ensuremath{\D^+_\squark}\xspace}
\def\Dsm     {\ensuremath{\D^-_\squark}\xspace}
\def\Dspm    {\ensuremath{\D^{\pm}_\squark}\xspace}
\def\Dsmp    {\ensuremath{\D^{\mp}_\squark}\xspace}
\def\Dss     {\ensuremath{\D^{*+}_\squark}\xspace}
\def\Dssp    {\ensuremath{\D^{*+}_\squark}\xspace}
\def\Dssm    {\ensuremath{\D^{*-}_\squark}\xspace}
\def\Dsspm   {\ensuremath{\D^{*\pm}_\squark}\xspace}
\def\Dssmp   {\ensuremath{\D^{*\mp}_\squark}\xspace}

\def\B       {\ensuremath{\PB}\xspace}
%%% do NOT use ensuremath here
\def\Bbar    {\ensuremath{\kern 0.18em\overline{\kern -0.18em \PB}{}}\xspace}
\def\Bb      {\ensuremath{\Bbar}\xspace}
\def\BBbar   {\ensuremath{\B\Bbar}\xspace}
\def\Bz      {\ensuremath{\B^0}\xspace}
\def\Bzb     {\ensuremath{\Bbar^0}\xspace}
\def\Bu      {\ensuremath{\B^+}\xspace}
\def\Bub     {\ensuremath{\B^-}\xspace}
\def\Bp      {\ensuremath{\Bu}\xspace}
\def\Bm      {\ensuremath{\Bub}\xspace}
\def\Bpm     {\ensuremath{\B^\pm}\xspace}
\def\Bmp     {\ensuremath{\B^\mp}\xspace}
\def\Bd      {\ensuremath{\B^0}\xspace}
\def\Bs      {\ensuremath{\B^0_\squark}\xspace}
\def\Bsb     {\ensuremath{\Bbar^0_\squark}\xspace}
\def\Bdb     {\ensuremath{\Bbar^0}\xspace}
\def\Bc      {\ensuremath{\B_\cquark^+}\xspace}
\def\Bcp     {\ensuremath{\B_\cquark^+}\xspace}
\def\Bcm     {\ensuremath{\B_\cquark^-}\xspace}
\def\Bcpm    {\ensuremath{\B_\cquark^\pm}\xspace}

%% Onia

\def\jpsi     {\ensuremath{{\PJ\mskip -3mu/\mskip -2mu\Ppsi\mskip 2mu}}\xspace}
\def\psitwos  {\ensuremath{\Ppsi{(2S)}}\xspace}
\def\psiprpr  {\ensuremath{\Ppsi(3770)}\xspace}
\def\etac     {\ensuremath{\Peta_\cquark}\xspace}
\def\chiczero {\ensuremath{\Pchi_{\cquark 0}}\xspace}
\def\chicone  {\ensuremath{\Pchi_{\cquark 1}}\xspace}
\def\chictwo  {\ensuremath{\Pchi_{\cquark 2}}\xspace}
  %\mathchardef\Upsilon="7107
  \def\Y#1S{\ensuremath{\PUpsilon{(#1S)}}\xspace}% no space before {...}!
\def\OneS  {\Y1S}
\def\TwoS  {\Y2S}
\def\ThreeS{\Y3S}
\def\FourS {\Y4S}
\def\FiveS {\Y5S}

\def\chic  {\ensuremath{\Pchi_{c}}\xspace}

%% Baryons

\def\proton      {\ensuremath{\Pp}\xspace}
\def\antiproton  {\ensuremath{\overline \proton}\xspace}
\def\neutron     {\ensuremath{\Pn}\xspace}
\def\antineutron {\ensuremath{\overline \neutron}\xspace}

\def\Deltares {\ensuremath{\PDelta}\xspace}
\def\Deltaresbar{\ensuremath{\overline \Deltares}\xspace}
\def\Xires {\ensuremath{\PXi}\xspace}
\def\Xiresbar{\ensuremath{\overline \Xires}\xspace}
\def\L {\ensuremath{\PLambda}\xspace}
\def\Lbar {\ensuremath{\kern 0.1em\overline{\kern -0.1em\PLambda}}\xspace}
\def\Lambdares {\ensuremath{\PLambda}\xspace}
\def\Lambdaresbar{\ensuremath{\Lbar}\xspace}
\def\Sigmares {\ensuremath{\PSigma}\xspace}
\def\Sigmaresbar{\ensuremath{\overline \Sigmares}\xspace}
\def\Omegares {\ensuremath{\POmega}\xspace}
\def\Omegaresbar{\ensuremath{\overline \Omegares}\xspace}

%%% do NOT use ensuremath here
 % \def\Deltabar{\kern 0.25em\overline{\kern -0.25em \Deltares}{}\xspace}
 % \def\Sigbar{\kern 0.2em\overline{\kern -0.2em \Sigma}{}\xspace}
 % \def\Xibar{\kern 0.2em\overline{\kern -0.2em \Xi}{}\xspace}
 % \def\Obar{\kern 0.2em\overline{\kern -0.2em \Omega}{}\xspace}
 % \def\Nbar{\kern 0.2em\overline{\kern -0.2em N}{}\xspace}
 % \def\Xb{\kern 0.2em\overline{\kern -0.2em X}{}\xspace}

\def\Lb      {\ensuremath{\L^0_\bquark}\xspace}
\def\Lbbar   {\ensuremath{\Lbar^0_\bquark}\xspace}
\def\Lc      {\ensuremath{\L^+_\cquark}\xspace}
\def\Lcbar   {\ensuremath{\Lbar^-_\cquark}\xspace}

%%%%%%%%%%%%%%%%%%
% Physics symbols
%%%%%%%%%%%%%%%%%

%% Decays
\def\BF         {{\ensuremath{\cal B}\xspace}}
\def\BRvis      {{\ensuremath{\BR_{\rm{vis}}}}}
\def\BR         {\BF}
\newcommand{\decay}[2]{\ensuremath{#1\!\to #2}\xspace}         % {\Pa}{\Pb \Pc}
\def\ra                 {\ensuremath{\rightarrow}\xspace}
\def\to                 {\ensuremath{\rightarrow}\xspace}

%% Lifetimes
\newcommand{\tauBs}{\ensuremath{\tau_{\Bs}}\xspace}
\newcommand{\tauBd}{\ensuremath{\tau_{\Bd}}\xspace}
\newcommand{\tauBz}{\ensuremath{\tau_{\Bz}}\xspace}
\newcommand{\tauBu}{\ensuremath{\tau_{\Bp}}\xspace}
\newcommand{\tauDp}{\ensuremath{\tau_{\Dp}}\xspace}
\newcommand{\tauDz}{\ensuremath{\tau_{\Dz}}\xspace}
\newcommand{\tauL}{\ensuremath{\tau_{\rm L}}\xspace}
\newcommand{\tauH}{\ensuremath{\tau_{\rm H}}\xspace}

%% Masses
\newcommand{\mBd}{\ensuremath{m_{\Bd}}\xspace}
\newcommand{\mBp}{\ensuremath{m_{\Bp}}\xspace}
\newcommand{\mBs}{\ensuremath{m_{\Bs}}\xspace}
\newcommand{\mBc}{\ensuremath{m_{\Bc}}\xspace}
\newcommand{\mLb}{\ensuremath{m_{\Lb}}\xspace}

%% EW theory, groups
\def\grpsuthree {\ensuremath{\mathrm{SU}(3)}\xspace}
\def\grpsutw    {\ensuremath{\mathrm{SU}(2)}\xspace}
\def\grpuone    {\ensuremath{\mathrm{U}(1)}\xspace}

\def\ssqtw {\ensuremath{\sin^{2}\!\theta_{\mathrm{W}}}\xspace}
\def\csqtw {\ensuremath{\cos^{2}\!\theta_{\mathrm{W}}}\xspace}
\def\stw   {\ensuremath{\sin\theta_{\mathrm{W}}}\xspace}
\def\ctw   {\ensuremath{\cos\theta_{\mathrm{W}}}\xspace}
\def\ssqtwef {\ensuremath{{\sin}^{2}\theta_{\mathrm{W}}^{\mathrm{eff}}}\xspace}
\def\csqtwef {\ensuremath{{\cos}^{2}\theta_{\mathrm{W}}^{\mathrm{eff}}}\xspace}
\def\stwef {\ensuremath{\sin\theta_{\mathrm{W}}^{\mathrm{eff}}}\xspace}
\def\ctwef {\ensuremath{\cos\theta_{\mathrm{W}}^{\mathrm{eff}}}\xspace}
\def\gv    {\ensuremath{g_{\mbox{\tiny V}}}\xspace}
\def\ga    {\ensuremath{g_{\mbox{\tiny A}}}\xspace}

\def\order   {\ensuremath{\mathcal{O}}\xspace}
\def\ordalph {\ensuremath{\mathcal{O}(\alpha)}\xspace}
\def\ordalsq {\ensuremath{\mathcal{O}(\alpha^{2})}\xspace}
\def\ordalcb {\ensuremath{\mathcal{O}(\alpha^{3})}\xspace}

%% QCD parameters
\newcommand{\as}{\ensuremath{\alpha_{\scriptscriptstyle S}}\xspace}
\newcommand{\MSb}{\ensuremath{\overline{\mathrm{MS}}}\xspace}
\newcommand{\lqcd}{\ensuremath{\Lambda_{\mathrm{QCD}}}\xspace}
\def\qsq       {\ensuremath{q^2}\xspace}

%% CKM, CP violation

\def\eps   {\ensuremath{\varepsilon}\xspace}
\def\epsK  {\ensuremath{\varepsilon_K}\xspace}
\def\epsB  {\ensuremath{\varepsilon_B}\xspace}
\def\epsp  {\ensuremath{\varepsilon^\prime_K}\xspace}

\def\CP                {\ensuremath{C\!P}\xspace}
\def\CPT               {\ensuremath{C\!PT}\xspace}

\def\rhobar {\ensuremath{\overline \rho}\xspace}
\def\etabar {\ensuremath{\overline \eta}\xspace}

\def\Vud  {\ensuremath{|V_{\uquark\dquark}|}\xspace}
\def\Vcd  {\ensuremath{|V_{\cquark\dquark}|}\xspace}
\def\Vtd  {\ensuremath{|V_{\tquark\dquark}|}\xspace}
\def\Vus  {\ensuremath{|V_{\uquark\squark}|}\xspace}
\def\Vcs  {\ensuremath{|V_{\cquark\squark}|}\xspace}
\def\Vts  {\ensuremath{|V_{\tquark\squark}|}\xspace}
\def\Vub  {\ensuremath{|V_{\uquark\bquark}|}\xspace}
\def\Vcb  {\ensuremath{|V_{\cquark\bquark}|}\xspace}
\def\Vtb  {\ensuremath{|V_{\tquark\bquark}|}\xspace}

%% Oscillations

\newcommand{\dm}{\ensuremath{\Delta m}\xspace}
\newcommand{\dms}{\ensuremath{\Delta m_{\squark}}\xspace}
\newcommand{\dmd}{\ensuremath{\Delta m_{\dquark}}\xspace}
\newcommand{\DG}{\ensuremath{\Delta\Gamma}\xspace}
\newcommand{\DGs}{\ensuremath{\Delta\Gamma_{\squark}}\xspace}
\newcommand{\DGd}{\ensuremath{\Delta\Gamma_{\dquark}}\xspace}
\newcommand{\Gs}{\ensuremath{\Gamma_{\squark}}\xspace}
\newcommand{\Gd}{\ensuremath{\Gamma_{\dquark}}\xspace}

\newcommand{\MBq}{\ensuremath{M_{\B_\quark}}\xspace}
\newcommand{\DGq}{\ensuremath{\Delta\Gamma_{\quark}}\xspace}
\newcommand{\Gq}{\ensuremath{\Gamma_{\quark}}\xspace}
\newcommand{\dmq}{\ensuremath{\Delta m_{\quark}}\xspace}
\newcommand{\GL}{\ensuremath{\Gamma_{\rm L}}\xspace}
\newcommand{\GH}{\ensuremath{\Gamma_{\rm H}}\xspace}

\newcommand{\DGsGs}{\ensuremath{\Delta\Gamma_{\squark}/\Gamma_{\squark}}\xspace}
\newcommand{\Delm}{\mbox{$\Delta m $}\xspace}
\newcommand{\ACP}{\ensuremath{{\cal A}^{\CP}}\xspace}
\newcommand{\Adir}{\ensuremath{{\cal A}^{\rm dir}}\xspace}
\newcommand{\Amix}{\ensuremath{{\cal A}^{\rm mix}}\xspace}
\newcommand{\ADelta}{\ensuremath{{\cal A}^\Delta}\xspace}
\newcommand{\phid}{\ensuremath{\phi_{\dquark}}\xspace}
\newcommand{\sinphid}{\ensuremath{\sin\!\phid}\xspace}
\newcommand{\phis}{\ensuremath{\phi_{\squark}}\xspace}
\newcommand{\betas}{\ensuremath{\beta_{\squark}}\xspace}
\newcommand{\sbetas}{\ensuremath{\sigma(\beta_{\squark})}\xspace}
\newcommand{\stbetas}{\ensuremath{\sigma(2\beta_{\squark})}\xspace}
\newcommand{\stphis}{\ensuremath{\sigma(\phi_{\squark})}\xspace}
\newcommand{\sinphis}{\ensuremath{\sin\!\phis}\xspace}

%% Tagging
\newcommand{\edet}{{\ensuremath{\varepsilon_{\rm det}}}\xspace}
\newcommand{\erec}{{\ensuremath{\varepsilon_{\rm rec/det}}}\xspace}
\newcommand{\esel}{{\ensuremath{\varepsilon_{\rm sel/rec}}}\xspace}
\newcommand{\etrg}{{\ensuremath{\varepsilon_{\rm trg/sel}}}\xspace}
\newcommand{\etot}{{\ensuremath{\varepsilon_{\rm tot}}}\xspace}

\newcommand{\mistag}{\ensuremath{\omega}\xspace}
\newcommand{\wcomb}{\ensuremath{\omega^{\rm comb}}\xspace}
\newcommand{\etag}{{\ensuremath{\varepsilon_{\rm tag}}}\xspace}
\newcommand{\etagcomb}{{\ensuremath{\varepsilon_{\rm tag}^{\rm comb}}}\xspace}
\newcommand{\effeff}{\ensuremath{\varepsilon_{\rm eff}}\xspace}
\newcommand{\effeffcomb}{\ensuremath{\varepsilon_{\rm eff}^{\rm comb}}\xspace}
\newcommand{\efftag}{{\ensuremath{\etag(1-2\omega)^2}}\xspace}
\newcommand{\effD}{{\ensuremath{\etag D^2}}\xspace}

\newcommand{\etagprompt}{{\ensuremath{\varepsilon_{\rm tag}^{\rm Pr}}}\xspace}
\newcommand{\etagLL}{{\ensuremath{\varepsilon_{\rm tag}^{\rm LL}}}\xspace}

%% Key decay channels

\def\BdToKstmm    {\decay{\Bd}{\Kstarz\mup\mun}}
\def\BdbToKstmm   {\decay{\Bdb}{\Kstarzb\mup\mun}}

\def\BsToJPsiPhi  {\decay{\Bs}{\jpsi\phi}}
\def\BdToJPsiKst  {\decay{\Bd}{\jpsi\Kstarz}}
\def\BdbToJPsiKst {\decay{\Bdb}{\jpsi\Kstarzb}}

\def\BsPhiGam     {\decay{\Bs}{\phi \g}}
\def\BdKstGam     {\decay{\Bd}{\Kstarz \g}}

\def\BTohh        {\decay{\B}{\Ph^+ \Ph'^-}}
\def\BdTopipi     {\decay{\Bd}{\pip\pim}}
\def\BdToKpi      {\decay{\Bd}{\Kp\pim}}
\def\BsToKK       {\decay{\Bs}{\Kp\Km}}
\def\BsTopiK      {\decay{\Bs}{\pip\Km}}

%% Rare decays
\def\BdKstee  {\decay{\Bd}{\Kstarz\epem}}
\def\BdbKstee {\decay{\Bdb}{\Kstarzb\epem}}
\def\bsll     {\decay{\bquark}{\squark \ell^+ \ell^-}}
\def\AFB      {\ensuremath{A_{\mathrm{FB}}}\xspace}
\def\FL       {\ensuremath{F_{\mathrm{L}}}\xspace}
\def\AT#1     {\ensuremath{A_{\mathrm{T}}^{#1}}\xspace}           % 2
\def\btosgam  {\decay{\bquark}{\squark \g}}
\def\btodgam  {\decay{\bquark}{\dquark \g}}
\def\Bsmm     {\decay{\Bs}{\mup\mun}}
\def\Bdmm     {\decay{\Bd}{\mup\mun}}
\def\ctl       {\ensuremath{\cos{\theta_l}}\xspace}
\def\ctk       {\ensuremath{\cos{\theta_K}}\xspace}

%% Wilson coefficients and operators
\def\C#1      {\ensuremath{\mathcal{C}_{#1}}\xspace}                       % 9
\def\Cp#1     {\ensuremath{\mathcal{C}_{#1}^{'}}\xspace}                    % 7
\def\Ceff#1   {\ensuremath{\mathcal{C}_{#1}^{\mathrm{(eff)}}}\xspace}        % 9
\def\Cpeff#1  {\ensuremath{\mathcal{C}_{#1}^{'\mathrm{(eff)}}}\xspace}       % 7
\def\Ope#1    {\ensuremath{\mathcal{O}_{#1}}\xspace}                       % 2
\def\Opep#1   {\ensuremath{\mathcal{O}_{#1}^{'}}\xspace}                    % 7

%% Charm

\def\xprime     {\ensuremath{x^{\prime}}\xspace}
\def\yprime     {\ensuremath{y^{\prime}}\xspace}
\def\ycp        {\ensuremath{y_{\CP}}\xspace}
\def\agamma     {\ensuremath{A_{\Gamma}}\xspace}
\def\kpi        {\ensuremath{\Kp\pim}\xspace}
\def\kk         {\ensuremath{\Kp\Km}\xspace}
\def\dkpi       {\decay{\PD}{\PK\Ppi}}
\def\dkk        {\decay{\PD}{\PK\PK}}
\def\dkpicf     {\decay{\Dz}{\Km\pip}}

%% QM
\newcommand{\bra}[1]{\ensuremath{\langle #1|}}             % {a}
\newcommand{\ket}[1]{\ensuremath{|#1\rangle}}              % {b}
\newcommand{\braket}[2]{\ensuremath{\langle #1|#2\rangle}} % {a}{b}

%%%%%%%%%%%%%%%%%%%%%%%%%%%%%%%%%%%%%%%%%%%%%%%%%%
% Units
%%%%%%%%%%%%%%%%%%%%%%%%%%%%%%%%%%%%%%%%%%%%%%%%%%
\newcommand{\unit}[1]{\ensuremath{\rm\,#1}\xspace}          % {kg}

%% Energy and momentum
\newcommand{\tev}{\ensuremath{\mathrm{\,Te\kern -0.1em V}}\xspace}
\newcommand{\gev}{\ensuremath{\mathrm{\,Ge\kern -0.1em V}}\xspace}
\newcommand{\gevsq}{\ensuremath{\mathrm{\,Ge\kern -0.1em V}^2}\xspace}
\newcommand{\mev}{\ensuremath{\mathrm{\,Me\kern -0.1em V}}\xspace}
\newcommand{\kev}{\ensuremath{\mathrm{\,ke\kern -0.1em V}}\xspace}
\newcommand{\ev}{\ensuremath{\mathrm{\,e\kern -0.1em V}}\xspace}
\newcommand{\gevc}{\ensuremath{{\mathrm{\,Ge\kern -0.1em V\!/}c}}\xspace}
\newcommand{\mevc}{\ensuremath{{\mathrm{\,Me\kern -0.1em V\!/}c}}\xspace}
\newcommand{\gevcc}{\ensuremath{{\mathrm{\,Ge\kern -0.1em V\!/}c^2}}\xspace}
\newcommand{\pergevgevcccc}{\ensuremath{{\mathrm{\,Ge\kern -0.1em V^{-2}}c^4}}\xspace}
\newcommand{\gevgevcccc}{\ensuremath{{\mathrm{\,Ge\kern -0.1em V^2\!/}c^4}}\xspace}
\newcommand{\gevgev}{\ensuremath{{\mathrm{\,Ge\kern -0.1em V^2}}}\xspace}
\newcommand{\mevcc}{\ensuremath{{\mathrm{\,Me\kern -0.1em V\!/}c^2}}\xspace}

%% Distance and area
\def\km   {\ensuremath{\rm \,km}\xspace}
\def\m    {\ensuremath{\rm \,m}\xspace}
\def\cm   {\ensuremath{\rm \,cm}\xspace}
\def\cma  {\ensuremath{{\rm \,cm}^2}\xspace}
\def\mm   {\ensuremath{\rm \,mm}\xspace}
\def\mma  {\ensuremath{{\rm \,mm}^2}\xspace}
\def\mum  {\ensuremath{\,\upmu\rm m}\xspace}
\def\muma {\ensuremath{\,\upmu\rm m^2}\xspace}
\def\nm   {\ensuremath{\rm \,nm}\xspace}
\def\fm   {\ensuremath{\rm \,fm}\xspace}
\def\barn{\ensuremath{\rm \,b}\xspace}
\def\barnhyph{\ensuremath{\rm -b}\xspace}
\def\mbarn{\ensuremath{\rm \,mb}\xspace}
\def\mub{\ensuremath{\rm \,\upmu b}\xspace}
\def\mbarnhyph{\ensuremath{\rm -mb}\xspace}
\def\nb {\ensuremath{\rm \,nb}\xspace}
\def\invnb {\ensuremath{\mbox{\,nb}^{-1}}\xspace}
\def\pb {\ensuremath{\rm \,pb}\xspace}
\def\invpb {\ensuremath{\mbox{\,pb}^{-1}}\xspace}
\def\fb   {\ensuremath{\mbox{\,fb}}\xspace}
\def\invfb   {\ensuremath{\mbox{\,fb}^{-1}}\xspace}
\def\ifb   {\ensuremath{\mbox{\,fb}^{-1}}\xspace}

%% Time
\def\sec  {\ensuremath{\rm {\,s}}\xspace}
\def\ms   {\ensuremath{{\rm \,ms}}\xspace}
\def\mus  {\ensuremath{\,\upmu{\rm s}}\xspace}
\def\ns   {\ensuremath{{\rm \,ns}}\xspace}
\def\ps   {\ensuremath{{\rm \,ps}}\xspace}
\def\fs   {\ensuremath{\rm \,fs}\xspace}

\def\mhz  {\ensuremath{{\rm \,MHz}}\xspace}
\def\khz  {\ensuremath{{\rm \,kHz}}\xspace}
\def\hz   {\ensuremath{{\rm \,Hz}}\xspace}

\def\invps{\ensuremath{{\rm \,ps^{-1}}}\xspace}

\def\yr   {\ensuremath{\rm \,yr}\xspace}
\def\hr   {\ensuremath{\rm \,hr}\xspace}

%% Temperature
\def\degc {\ensuremath{^\circ}{C}\xspace}
\def\degk {\ensuremath {\rm K}\xspace}

%% Material lengths, radiation
\def\Xrad {\ensuremath{X_0}\xspace}
\def\NIL{\ensuremath{\lambda_{int}}\xspace}
\def\mip {MIP\xspace}
\def\neutroneq {\ensuremath{\rm \,n_{eq}}\xspace}
\def\neqcmcm {\ensuremath{\rm \,n_{eq} / cm^2}\xspace}
\def\kRad {\ensuremath{\rm \,kRad}\xspace}
\def\MRad {\ensuremath{\rm \,MRad}\xspace}
\def\ci {\ensuremath{\rm \,Ci}\xspace}
\def\mci {\ensuremath{\rm \,mCi}\xspace}

%% Uncertainties
\def\sx    {\ensuremath{\sigma_x}\xspace}
\def\sy    {\ensuremath{\sigma_y}\xspace}
\def\sz    {\ensuremath{\sigma_z}\xspace}

\newcommand{\stat}{\ensuremath{\mathrm{(stat)}}\xspace}
\newcommand{\syst}{\ensuremath{\mathrm{(syst)}}\xspace}
\newcommand{\normerr}{\ensuremath{\mathrm{(norm)}}\xspace}

%% Maths

\def\order{{\ensuremath{\cal O}}\xspace}
\newcommand{\chisq}{\ensuremath{\chi^2}\xspace}

\def\deriv {\ensuremath{\mathrm{d}}}

\def\gsim{{~\raise.15em\hbox{$>$}\kern-.85em
          \lower.35em\hbox{$\sim$}~}\xspace}
\def\lsim{{~\raise.15em\hbox{$<$}\kern-.85em
          \lower.35em\hbox{$\sim$}~}\xspace}

\newcommand{\mean}[1]{\ensuremath{\left\langle #1 \right\rangle}} % {x}
\newcommand{\abs}[1]{\ensuremath{\left\|#1\right\|}} % {x}
\newcommand{\Real}{\ensuremath{\mathcal{R}e}\xspace}
\newcommand{\Imag}{\ensuremath{\mathcal{I}m}\xspace}

\def\PDF {PDF\xspace}

\def\sPlot{\mbox{\em sPlot}}
\def\sWeight{\mbox{\em sWeight}}
%%%%%%%%%%%%%%%%%%%%%%%%%%%%%%%%%%%%%%%%%%%%%%%%%%
% Kinematics
%%%%%%%%%%%%%%%%%%%%%%%%%%%%%%%%%%%%%%%%%%%%%%%%%%

%% Energy, Momenta
\def\Ebeam {\ensuremath{E_{\mbox{\tiny BEAM}}}\xspace}
\def\sqs   {\ensuremath{\protect\sqrt{s}}\xspace}

\def\ptot       {\mbox{$p$}\xspace}
\def\pt         {\mbox{$p_{\rm T}$}\xspace}
\def\et         {\mbox{$E_{\rm T}$}\xspace}
\def\dpp        {\ensuremath{\mathrm{d}\hspace{-0.1em}p/p}\xspace}

\newcommand{\dedx}{\ensuremath{\mathrm{d}\hspace{-0.1em}E/\mathrm{d}x}\xspace}

%% PID

\def\dllxpi     {\ensuremath{\mathrm{DLL}_{X\pion}}\xspace}
\def\dllkpi     {\ensuremath{\mathrm{DLL}_{\kaon\pion}}\xspace}
\def\dllppi     {\ensuremath{\mathrm{DLL}_{\proton\pion}}\xspace}
\def\dllepi     {\ensuremath{\mathrm{DLL}_{\electron\pion}}\xspace}
\def\dllmupi    {\ensuremath{\mathrm{DLL}_{\mmu\pi}}\xspace}
\def\dllkp    {\ensuremath{\mathrm{DLL}_{\kaon\proton}}\xspace}

%% Geometry
\def\mphi       {\mbox{$\phi$}\xspace}
\def\mtheta     {\mbox{$\theta$}\xspace}
\def\ctheta     {\mbox{$\cos\theta$}\xspace}
\def\stheta     {\mbox{$\sin\theta$}\xspace}
\def\ttheta     {\mbox{$\tan\theta$}\xspace}

\def\degrees{\ensuremath{^{\circ}}\xspace}
\def\krad {\ensuremath{\rm \,krad}\xspace}
\def\mrad{\ensuremath{\rm \,mrad}\xspace}
\def\rad{\ensuremath{\rm \,rad}\xspace}

%% Accelerator
\def\betastar {\ensuremath{\beta^*}}
\newcommand{\lum} {\ensuremath{\mathcal{L}}\xspace}
\newcommand{\intlum}[1]{\ensuremath{\int\lum=#1\xspace}}  % {2 \,\invfb}

%%%%%%%%%%%%%%%%%%%%%%%%%%%%%%%%%%%%%%%%%%%%%%%%%%%%%%%%%%%%%%%%%%%%
% Software
%%%%%%%%%%%%%%%%%%%%%%%%%%%%%%%%%%%%%%%%%%%%%%%%%%%%%%%%%%%%%%%%%%%%

%% Programs
\def\evtgen     {\mbox{\textsc{EvtGen}}\xspace}
\def\pythia     {\mbox{\textsc{Pythia}}\xspace}
\def\fluka      {\mbox{\textsc{Fluka}}\xspace}
\def\tosca      {\mbox{\textsc{Tosca}}\xspace}
\def\ansys      {\mbox{\textsc{Ansys}}\xspace}
\def\spice      {\mbox{\textsc{Spice}}\xspace}
\def\garfield   {\mbox{\textsc{Garfield}}\xspace}
\def\geant      {\mbox{\textsc{Geant4}}\xspace}
\def\hepmc      {\mbox{\textsc{HepMC}}\xspace}
\def\gauss      {\mbox{\textsc{Gauss}}\xspace}
\def\gaudi      {\mbox{\textsc{Gaudi}}\xspace}
\def\boole      {\mbox{\textsc{Boole}}\xspace}
\def\brunel     {\mbox{\textsc{Brunel}}\xspace}
\def\davinci    {\mbox{\textsc{DaVinci}}\xspace}
\def\erasmus    {\mbox{\textsc{Erasmus}}\xspace}
\def\moore      {\mbox{\textsc{Moore}}\xspace}
\def\ganga      {\mbox{\textsc{Ganga}}\xspace}
\def\dirac      {\mbox{\textsc{Dirac}}\xspace}
\def\root       {\mbox{\textsc{Root}}\xspace}
\def\roofit     {\mbox{\textsc{RooFit}}\xspace}
\def\pyroot     {\mbox{\textsc{PyRoot}}\xspace}
\def\photos     {\mbox{\textsc{Photos}}\xspace}

%% Languages
\def\cpp        {\mbox{\textsc{C\raisebox{0.1em}{{\footnotesize{++}}}}}\xspace}
\def\python     {\mbox{\textsc{Python}}\xspace}
\def\ruby       {\mbox{\textsc{Ruby}}\xspace}
\def\fortran    {\mbox{\textsc{Fortran}}\xspace}
\def\svn        {\mbox{\textsc{SVN}}\xspace}

%% Data processing
\def\kbytes     {\ensuremath{{\rm \,kbytes}}\xspace}
\def\kbsps      {\ensuremath{{\rm \,kbytes/s}}\xspace}
\def\kbits      {\ensuremath{{\rm \,kbits}}\xspace}
\def\kbsps      {\ensuremath{{\rm \,kbits/s}}\xspace}
\def\mbsps      {\ensuremath{{\rm \,Mbits/s}}\xspace}
\def\mbytes     {\ensuremath{{\rm \,Mbytes}}\xspace}
\def\mbps       {\ensuremath{{\rm \,Mbyte/s}}\xspace}
\def\mbsps      {\ensuremath{{\rm \,Mbytes/s}}\xspace}
\def\gbsps      {\ensuremath{{\rm \,Gbits/s}}\xspace}
\def\gbytes     {\ensuremath{{\rm \,Gbytes}}\xspace}
\def\gbsps      {\ensuremath{{\rm \,Gbytes/s}}\xspace}
\def\tbytes     {\ensuremath{{\rm \,Tbytes}}\xspace}
\def\tbpy       {\ensuremath{{\rm \,Tbytes/yr}}\xspace}

\def\dst        {DST\xspace}

%%%%%%%%%%%%%%%%%%%%%%%%%%%
% Detector related
%%%%%%%%%%%%%%%%%%%%%%%%%%%

%% Detector technologies
\def\nonn {\ensuremath{\rm {\it{n^+}}\mbox{-}on\mbox{-}{\it{n}}}\xspace}
\def\ponn {\ensuremath{\rm {\it{p^+}}\mbox{-}on\mbox{-}{\it{n}}}\xspace}
\def\nonp {\ensuremath{\rm {\it{n^+}}\mbox{-}on\mbox{-}{\it{p}}}\xspace}
\def\cvd  {CVD\xspace}
\def\mwpc {MWPC\xspace}
\def\gem  {GEM\xspace}

%% Detector components, electronics
\def\tell1  {TELL1\xspace}
\def\ukl1   {UKL1\xspace}
\def\beetle {Beetle\xspace}
\def\otis   {OTIS\xspace}
\def\croc   {CROC\xspace}
\def\carioca {CARIOCA\xspace}
\def\dialog {DIALOG\xspace}
\def\sync   {SYNC\xspace}
\def\cardiac {CARDIAC\xspace}
\def\gol    {GOL\xspace}
\def\vcsel  {VCSEL\xspace}
\def\ttc    {TTC\xspace}
\def\ttcrx  {TTCrx\xspace}
\def\hpd    {HPD\xspace}
\def\pmt    {PMT\xspace}
\def\specs  {SPECS\xspace}
\def\elmb   {ELMB\xspace}
\def\fpga   {FPGA\xspace}
\def\plc    {PLC\xspace}
\def\rasnik {RASNIK\xspace}
\def\elmb   {ELMB\xspace}
\def\can    {CAN\xspace}
\def\lvds   {LVDS\xspace}
\def\ntc    {NTC\xspace}
\def\adc    {ADC\xspace}
\def\led    {LED\xspace}
\def\ccd    {CCD\xspace}
\def\hv     {HV\xspace}
\def\lv     {LV\xspace}
\def\pvss   {PVSS\xspace}
\def\cmos   {CMOS\xspace}
\def\fifo   {FIFO\xspace}
\def\ccpc   {CCPC\xspace}

%% Chemical symbols
\def\cfourften     {\ensuremath{\rm C_4 F_{10}}\xspace}
\def\cffour        {\ensuremath{\rm CF_4}\xspace}
\def\cotwo         {\ensuremath{\rm CO_2}\xspace}
\def\csixffouteen  {\ensuremath{\rm C_6 F_{14}}\xspace}
\def\mgftwo     {\ensuremath{\rm Mg F_2}\xspace}
\def\siotwo     {\ensuremath{\rm SiO_2}\xspace}

%%%%%%%%%%%%%%%
% Special Text
%%%%%%%%%%%%%%%
\newcommand{\eg}{\mbox{\itshape e.g.}\xspace}
\newcommand{\ie}{\mbox{\itshape i.e.}}
\newcommand{\etal}{{\slshape et al.}\xspace}
\newcommand{\etc}{\mbox{\itshape etc.}\xspace}
\newcommand{\cf}{\mbox{\itshape cf.}\xspace}
\newcommand{\ffp}{\mbox{\itshape ff.}\xspace}
\newcommand{\vs}{\mbox{\itshape vs.}\xspace}



 % Add in the predefined LHCb symbols

%%%%%%%%%%%%%%%
% My Additions
%%%%%%%%%%%%%%%

%% My Decays
%--------------------------------

\newcommand{\e}[1]{\ensuremath{\times 10^{#1}}}
\newcommand{\commentt}[1]{}

\newif\ifstartedinmathmode
\newcommand*{\bam}[1]{%
  \relax\ifmmode\startedinmathmodetrue\else\startedinmathmodefalse\fi
  {\ifstartedinmathmode{{\color{red}\boldsymbol{#1}}}\else{{\color{red}\bf #1}}\fi}
}

\newenvironment{bullets}{
\begin{itemize}
    \setlength{\itemsep}{0pt}
    \setlength{\parskip}{0pt}
    \setlength{\parsep}{0pt}}
{\end{itemize}}
\newenvironment{enumeratesp}{
\begin{enumerate}
    \setlength{\leftmargin}{0pt}
    \setlength{\itemsep}{0pt}
    \setlength{\parskip}{0pt}
    \setlength{\parsep}{0pt}}
{\end{enumerate}}

\def\pz{\phantom{0}}
\def\pzz{\phantom{00}}
\def\xxx{\ensuremath{xxx}\xspace}


% B -> Ds phi
\def\btodsphi {\decay{\Bp}{\Dsp\phi}}
\def\btodsstrphi {\decay{\Bp}{\Dssp\phi}\xspace}
\def\bctodsphi {\decay{\Bc}{\Dsp\phi}\xspace}
\def\btodskstar   {\decay{\Bp}{\Dsp\Kstarbz}}
\def\btodkstar    {\decay{\Bp}{\Dp\Kstarbz}}
\def\bstodskstrk  {\decay{\Bsb}{\Dsp\Kstarz\Km}}
\def\bstodsstrkstrk {\decay{\Bsb}{\Dssp\Kstarz\Km}}


% B -> hhhmumu
\def\Kone{\ensuremath{K_{1}}\xspace}
\def\Koneb{{\kern 0.2em\overline{\kern -0.2em K}{}}_{1}\xspace} % no ensuremath
\def\Konep{\ensuremath{\Kone^{+}}\xspace}
\def\Konem{\ensuremath{\Kone^{-}}\xspace}

\def\thetakone{\ensuremath{\theta_{\Kone}}\xspace}
\def\sinthetakone{\ensuremath{\sin\theta_{\Kone}}\xspace}
\def\costhetakone{\ensuremath{\cos\theta_{\Kone}}\xspace}
\def\thetahel{\ensuremath{\theta_\mathrm{hel}}\xspace}
\def\thetadir{\ensuremath{\theta_\mathrm{dir}}\xspace}

\newcommand\regionX[1]{{\ensuremath{\mathbf{#1}}}\xspace}
\def\rA{\regionX{A}}
\def\rB{\regionX{B}}
\def\rC{\regionX{C}}
\def\rD{\regionX{D}}

\def\pipi{\ensuremath{\pip\pim}\xspace}
\def\kpipi{{\ensuremath{\Kp\pip\pim}}\xspace}
\def\kkpi{{\ensuremath{\Kp\Km\pip}}\xspace}
\def\phik{{\ensuremath{\phi\Kp}}\xspace}
\def\kphi{{\ensuremath{\phi\Kp}}\xspace}
\def\pdg{{\ensuremath{\mathrm{PDG}}}\xspace}
\def\bdt{{\ensuremath{\mathrm{PDG}}}\xspace}


\def\btokpipimumu{\mbox{\decay{\Bp}{\Kp\pip\pim\mumu}}\xspace}
\def\btokpipijpsi{\mbox{\decay{\Bp}{\jpsi\Kp\pip\pim}}\xspace}
\def\btokpipipsitwos{\mbox{\decay{\Bp}{\psitwos\Kp\pip\pim}}\xspace}
\def\btojpsikpipi{\mbox{\decay{\Bp}{\jpsi\Kp\pip\pim}}\xspace}
\def\btopsitwosk{\mbox{\decay{\Bp}{\psitwos\Kp}}\xspace}
\def\btophikmumu{\decay{\Bp}{\phi\Kp\mumu}\xspace}
\def\btokphimumu{\btophikmumu}
\def\btophikjpsi{\mbox{\decay{\Bp}{\jpsi\phi\Kp}}\xspace}
\def\btokphijpsi{\mbox{\decay{\Bp}{\jpsi\phi\Kp}}\xspace}
\def\btojpsiphik{\mbox{\decay{\Bp}{\jpsi\phi\Kp}}\xspace}
\def\psitwostojpsipipi{\mbox{\decay{\psitwos}{\jpsi\pip\pim}}\xspace}
\def\btokonemumu{\mbox{\decay{\Bp}{\Konep(1270)\mumu}}\xspace}
\def\btokstmumu{\mbox{\decay{\Bd}{\Kstarz\mumu}}\xspace}
\def\jpsitomumu{\mbox{\decay{\jpsi}{\mumu}}\xspace}
\def\kstartokpi{\mbox{\decay{\Kstarz}{\kpi}}\xspace}
\def\bstodspi{\mbox{\decay{\Bs}{\Dsm\pip}}\xspace}
\def\bstojpsiphi{\mbox{\decay{\Bs}{\jpsi\phi}}\xspace}
\def\phitokk{\mbox{\decay{\phi}{\kk}}\xspace}
\def\dstokkpi{\mbox{\decay{\Ds}{\kkpi}}\xspace}
\def\btodsd{\mbox{\decay{\Bp}{\Ds\Dzb}}\xspace}

\newcommand\kone[1]{\ensuremath{K_1(#1)^+}\xspace}

\def\Kstarent{\ensuremath{K^*(892)^0}\xspace}
\def\db{\ensuremath{\chi}\xspace}
\def\dbtomumu{\ensuremath{\decay{\db}{\mumu}}\xspace}

\def\Tm  {\ensuremath{{\,\mathrm{Tm}}}\xspace}

\def\d{\ensuremath{{\rm d}}\xspace}
\def\dBF{\ensuremath{{\rm d}\BF}\xspace}
\def\dqsq{\ensuremath{{\rm d}\qsq}\xspace}
\def\max{\ensuremath{\mathrm{max}}\xspace}
\def\min{\ensuremath{\mathrm{min}}\xspace}
\def\sig{\ensuremath{\mathrm{sig}}\xspace}
\def\norm{\ensuremath{\mathrm{norm}}\xspace}

\newcommand{\chisqfd}{\ensuremath{\chi^2_{\rm FD}}\xspace}
\newcommand{\chisqip}{\ensuremath{\chi^2_{\rm IP}}\xspace}
\newcommand{\chisqvtx}{\ensuremath{\chi^2_{\rm vtx}}\xspace}
\newcommand{\chisqtrk}{\ensuremath{\chi^2_{\rm trk}}\xspace}
\newcommand{\chisqvs}{\ensuremath{\chi^2_{\rm VS}}\xspace}
\def\ndof{\ensuremath{\mathrm{DOF}}\xspace}
\def\photon{\ensuremath{\gamma}\xspace}

\newcommand\V[1]{\ensuremath{V_{#1}^{}}\xspace}
\newcommand\Vtxt[1]{\ensuremath{V_{#1}}\xspace}
\newcommand\Vconj[1]{\ensuremath{V_{#1}^{*}}\xspace}
\def\VCKM{\ensuremath{\mathbf{V}}\xspace}
\def\VCKMconj{\ensuremath{\mathbf{V}^\dagger}\xspace}
\newcommand\Lag[1]{\ensuremath{\mathcal{L}_\mathrm{#1}}\xspace}
\newcommand\Lagm[1]{\ensuremath{\mathcal{L}_{#1}}\xspace}
\newcommand\Ham[1]{\ensuremath{\mathcal{H}_\mathrm{#1}}\xspace}
\newcommand\eff[1]{\ensuremath{\varepsilon_\mathrm{#1}}\xspace}
\newcommand\num[1]{\ensuremath{N_\mathrm{#1}}\xspace}
\def\matho{\ensuremath{\mathcal{O}}\xspace}
\def\pid{\gls{PIDLabel}\xspace}
\newcommand\Op[1]{\ensuremath{\mathcal{O}_{#1}}\xspace}

\def\stdev{\ensuremath{\,\sigma}\xspace}
\def\pc{\ensuremath{\,\%}\xspace}

\newcommand\Eq[1]{Eq.~\ref{#1}\xspace}
\newcommand\Fig[1]{Fig.~\ref{#1}\xspace}
\newcommand\Tab[1]{Table~\ref{#1}\xspace}
\newcommand\Table[1]{\Tab{#1}\xspace}
\newcommand\Sec[1]{Sec.~\ref{#1}\xspace}
\newcommand\Sect[1]{\Sec{#1}}
\newcommand\Chap[1]{Chap.~\ref{#1}\xspace}
\newcommand\Appendix[1]{Appendix~\ref{#1}\xspace}
\newcommand\App[1]{\Appendix{#1}}
\newcommand\Ref[1]{Ref.~\cite{#1}}

\newcommand{\cellc}[1]{\multicolumn{1}{c}{#1}}
\newcommand{\cellr}[1]{\multicolumn{1}{r}{#1}}
\newcommand{\celll}[1]{\multicolumn{1}{l}{#1}}
\newcommand{\tmath}[1]{\ensuremath{\boldsymbol{#1}}\xspace}

\newcommand{\Xbar}[1]{\ensuremath{\kern 0.18em\overline{\kern -0.18em {#1}}{}}\xspace}

\newcommand{\cxx}[1]{\ensuremath{c_{#1}}}
\newcommand{\sxx}[1]{\ensuremath{s_{#1}}}

\def\acp{\ensuremath{\mathcal{A}_{\CP}}\xspace}
\def\phii{\mbox{\ensuremath{\phi}}\xspace}

\newcommand\twocols[1]{\multicolumn{2}{c}{#1}}

% Particles
\def\infl{\ensuremath{\chi}\xspace}
\def\db{\ensuremath{\chi}\xspace}

\def\btokstrinfl{\decay{\Bd}{\Kstarz\infl}}
\def\btokstrdb{\decay{\Bd}{\Kstarz\db}}
\def\btokstrmumu{\decay{\Bd}{\Kstarz\mumu}}

\def\mdb{\ensuremath{m_{\db}}\xspace}
\def\tdb{\ensuremath{\tau_{\db}}\xspace}
\def\xtsvty{\ensuremath{X(1070)}\xspace}

%\makeatletter
%\newcommand{\cmathbox}[2][x]{%
  %\settowidth{\@tempdima}{#1}%
  %\makebox[\@tempdima][c]{#2}%
%}
%\makeatother

\def\littlerule{\midrule[0.01em]}
\def\tis{\ensuremath{\mathtt{TIS}}\xspace}
\def\tos{\ensuremath{\mathtt{TOS}}\xspace}
\def\tistos{\ensuremath{\mathtt{TISTOS}}\xspace}
\def\approx{\ensuremath{\sim\!}\xspace}
\def\nlto{\raisebox{3pt}{$\drsh$}\xspace}
\newcommand\plane[2]{$#1\,#2$\nobreakdash-plane}

 % Add in the predefined LHCb symbols

\usepackage{fancyhdr}
\pagestyle{fancy}
\cfoot{} % kill central lage number
\fancyhf{}
\renewcommand{\chaptermark}[1]{ \markboth{{\bf #1}}{} }
\renewcommand{\sectionmark}[1]{ \markright{{\bf #1}}{} }
\fancyfoot[C]{\thepage}

\fancyfoot[C]{{\bf \thepage}}
\renewcommand{\headrulewidth}{0.5pt}

% Redefine the plain page style
\fancypagestyle{plain}{%
  \fancyhf{}%
  \fancyfoot[C]{{\bf \thepage}}%
  \renewcommand{\headrulewidth}{0pt}% Line at the header invisible
  \renewcommand{\footrulewidth}{0pt}% Line at the footer visible
}

% Redefine the plain page style
\fancypagestyle{chapters}{%
  \fancyhf{}%
  \fancyfoot[C]{{\bf \thepage}}%
  \fancyhead[RE,LO]{\leftmark}
  \renewcommand{\headrulewidth}{0.5pt}
}

%\lhead{\leftmark}
%\rhead{\rightmark}
%% using leftmark and rightmark requires redefinition of the following to look proper
%\makeatletter
%\renewcommand\tableofcontents{%
  %\section*{\contentsname
    %\@mkboth{\MakeUppercase\contentsname}}
    %%\MakeUppercase\contentsname}{}}
    %%}{}}
  %\@starttoc{toc}%
%}
%\makeatother
\makeatletter
%% patch \listoffigures not to issue \@mkboth but \markboth
%% and to issue \addcontentsline
\patchcmd{\listoffigures}
{\@mkboth{\MakeUppercase\listfigurename}{\MakeUppercase\listfigurename}}
{\addcontentsline{toc}{chapter}{\listfigurename}\markboth{\listfigurename}{}}
{}{}
\patchcmd{\listoftables}
{\@mkboth{\MakeUppercase\listtablename}{\MakeUppercase\listtablename}}
{\addcontentsline{toc}{chapter}{\listtablename}\markboth{\listtablename}{}}
{}{}
\makeatother



%\newcommand{\shortdate}{\ifnum\number\day<10 0\fi \number\day \space%
  %\ifcase \month \or January\or February\or March\or April\or May%
  %\or June\or July\or August\or September\or October\or November\or December\fi,%
%\number \year}
\newcommand{\shortdate}{%\ifnum\number\day<10 0\fi \number\day \space%
  \ifcase \month \or January\or February\or March\or April\or May%
  \or June\or July\or August\or September\or October\or November\or December\fi,%
\space\number \year}




% Make this the last packages you include before the \begin{document}
\usepackage{cite} % Allows for ranges in citations
\usepackage{mciteplus}

