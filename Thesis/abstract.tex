%\setcounter{page}{2}
%\chapter*{\centering Declaration}
\begin{abstract}
  %\newgeometry{left=4.5cm,right=4.5cm}
  \chapter*{\centering Abstract}
  %\thispagestyle{fancy}

  \begin{center}
  {\setlength{\currentparskip}{\parskip}% save the value
  \begin{minipage}{\abstractpagewidth}
    \setlength{\parskip}{\currentparskip}% restore the value

    It is known that the Standard Model of particle physics describes an incomplete picture of the
    nature at a fundamental level.
    For example, the Standard Model cannot incorporate dark matter or explain the matter dominated
    nature of the Universe.
    This thesis presents three analyses, each of which are undertaken using different methodology,
    while sharing the goal of searching for evidence of new physics that might go
    towards explaining phenomena unexplained by the Standard Model.

    The first analysis shows evidence for the rare decay \btodsphi with greater than $3\stdev$
    significance; this also constitutes the first evidence for a fully-hadronic annihilation-type
    decay of a \Bp meson.
    Measurements of the branching fraction and $C\!P$-asymmetry of the decay are made.
    The branching fraction is seen to be greater than Standard Model predictions, but still in
    agreement with them, while the $C\!P$-asymmetry is perfectly in line with the Standard Model.

    The second analysis claims the first observations of the decays \btokpipimumu and \btophikmumu,
    which are both flavour changing neutral currents, forbidden at leading order in the Standard
    Model.
    Branching fractions of both these decays are measured, and for the high statistics channel
    \btokpipimumu the differential branching fraction as a function of the invariant mass squared
    of the dimuon system is also presented.

    These first two analyses both constitute indirect searches for physics beyond the scope of the
    Standard Model, where the observables are sensitive to contributions from new physics entering
    at loop-level.
    In contrast, the third analysis presented in this thesis involves the direct search for a new
    dark boson, \db, which is a messenger particle between a dark sector and the Standard Model
    particles.
    Using a fully frequentist technique, the dimuon component of candidates of the decay
    \btokstrmumu for an excess consistent with \dbtomumu.

    %Direct searches look for the decay products of a particle that form a peak in an invariant mass
    %spectrum, and therefore observe the

    %new physics in a different
    %way,

    %The SM fails, eg, incomplete
    %therefore need np
    %search for np in variety of ways: direct and indirect

    %seach for dphi, first evidence etc
    %search for hhh, strange states
    %finally look for direct evidence of bsm paritcle

    %data size

    \end{minipage}}
  \end{center}
%\chapter*{\centering Declaration}
%\restoregeometry

\end{abstract}
%\setcounter{page}{3}


