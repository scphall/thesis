\section{The uBoost algorithm}
\label{sec:bdt:uboost}
The uniform Boosting, uBoost, algorithm\footnote{
  This section is based on \Ref{Stevens:2013dya}.} is designed to give a uniform
response in the signal efficiency of some variables $\zeta_i$.
The procedure of creating a uniform \BDT, \uBDT, builds from the weighting technique used in the
AdaBoost algorithm, but additional weight is applied to events that lie in a region of parameter
space which is not performing with the desired efficiency.
Consider a \BDT whose signal efficiency in the variables $\zeta_i$ is required to be $\epsilon$.
The weighting of each event is modified from that used in the AdaBoost algorithm in \Eq{eq:ada:wt}
to
\begin{equation}
  w_i^t = u_i^t\times c_i^t \times w_i^{t-1},
\end{equation}
where $u$ denotes a weighting proportional to the distance away from $\epsilon$ in the local
region:
\begin{equation}
  u_i^t = \exp\big(\beta_t(\bar\epsilon-\epsilon_i^t)\big).
\end{equation}
The boosting parameter $\beta$ is calculated as
\begin{equation}
  \beta_t = \frac12\ln\left(\frac{1-e_t}{e_t}\right)
\end{equation}
and
\begin{equation}
  e_t = \sum_i w_i^{t-1}c_i^t\left|\bar\epsilon-\epsilon_i^t\right|.
\end{equation}

This leads to a single \BDT whose response is analogous to \Eq{eq:ada:fullbdt} but with the addition
of the target efficiency $\bar\epsilon$
\begin{equation}
  T(x_i,\bar\epsilon) = \sum_{t=1}^{n_\mathrm{trees}} \alpha_tT_t(x_i,\bar\epsilon).
  \label{eq:ada:fullbdt}
\end{equation}
Therefore, a single of these \BDTs is associated with a given target efficiency, where
fraction of $T(x_i,\epsilon)>\Xbar{T}(\bar\epsilon)$ is $\bar\epsilon$.
An arbitrary number of \BDTs, $N$, can be concatenated, each with a different target efficiency, and
the total response is
\begin{equation}
  \mathcal{T}(x_i) =
  \frac1N\sum_{e=1}^{n}\Theta\left(T(x_i, \bar\epsilon) - \Xbar{T}(\bar\epsilon)\right).
\end{equation}
For $\mathcal{T}(x_i)$ to be a continuous distribution, $N\!\to\infty$, however in practice
$N\sim100$ is all that is needed for analysis.

Using of the uBoost algorithm is well motivated in \Chap{ch:db}, which is a search for a particle
of unknown mass and lifetime.
The uBoost technique allows a \BDT to be trained, while ensuring that the response does not bias
the selection towards given lifetimes or masses of the unknown particle.













