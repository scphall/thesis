\section{Summary}
Multivariate selection techniques are a vital part of \gls{HEP} analyses, and help rare
processes to be separated from underlying combinatorial backgrounds.
There are a number of algorithms available to the analyst, for many analyses the choice can be a
matter of taste or convenience.
However, there are circumstances where the use of a particular algorithm is very important.
For example, the \glspl{BBDT} outlined in \Sec{sec:lhcb:trig} is must be fast; and the uBoost
algorithm implemented for the analysis in \Chap{ch:db} is required to prevent the selection being
bias towards a particular region of mass and lifetime.
Without these various machine learning techniques the field would be restricted to cut based
analyses and the sensitivity of many searches, especially those for very rare decays, would be
considerably less.

The variables and training samples used to train the \BDTs will be discussed in the relevant
chapters.
