%\chapter{BDTs}

The analyses detailed in \Chap{ch:hhh}, \Chap{ch:dsphi} and \Chap{ch:db} of this thesis, make
prodigious use of multivariate techniques for the removal of combinatorial backgrounds.
Combinatorial background is formed from a random combination of tracks that happen to form a well
defined vertex and pass selection criteria.
To remove these backgrounds multivariate analysis (MVA) techniques can be employed.
A multivariate discriminator exploits correlations between weakly discriminating variables to
produce a single, more separating, classifier.

An MVA algorithm takes a sample of the signal and background candidates to separate, and a set of
variables to be used.
Samples of events are split in two, some are used for training the MVA, and the remaining are used
for testing it.
The input, or training, variables define an $n$-dimensional space populated by the input samples.
The algorithm then classifies regions in this $n$-dimensional space as signal- or background-like;
such that an arbitrary event placed somewhere in the space would also be classified based on the
point it inhabits.
The Boosted Decision Tree (BDT) algorithm is used throughout this thesis because it can handle
a weighted training sample, including negative weights, and can exploit non-linear correlations
between variables~\cite{Breiman,Roe}.

A BDT is composed of a combination of numerous Decision Trees (DTs), each of which is a classifier
in its own right, but can only distinguish between high density populations of signal and
background candidates.
The objective is to use DTs to best separate signal and background.
Training a DT begins with a single node, a decision is chosen which best separates the signal from
background based on the value of the $\mathrm{G}_{ini}$ index, which is defined as
\begin{equation}
  \mathrm{G}_{ini} = 2(1-\eff{sig})(1-\eff{bkg}) = 2\cdot\frac{b}{s+b}\cdot\frac{s}{s+b}
  = \frac{2sb}{(s+b)^2},
\end{equation}
where $s$ and $b$ are the weighted sum of signal and background candidates, respectively, at the
node of interest.
The efficiency of signal and background are denoted by \eff{sig} and \eff{bkg}, and the purity of a
node is defined as $1-\varepsilon$.
The decision, or cut, which results in the largest reduction in $\mathrm{G}_{ini}$ is the one that
is chosen, since for a pure child node the aim is for a pure sample of signal or background, that
is an efficiency \eff{sig} or \eff{bkg} to be zero.
This process of splitting is repeated until there is no possible improvement in the purity of child
nodes, at which point the node is dubbed a leaf and is not split.
Each leaf maps out an area in $n$-dimensions, and is classified as a signal or background leaf
depending on the purity of the training sample enclosed by that area.
Candidates from the other samples can then be traced through the DT and assigned as being
consistent with either signal or background.
However, some candidates in the training sample will be misclassified.

Decision Trees have the advantage of clarity over other machine learning algorithms such as neural
nets, and can be written down exactly.
They are also insensitive to variables with very little separation power because the
$\mathrm{G}_{ini}$ index never identifies a cut on them as being profitable.
However, DTs are sensitive to statistical fluctuations in the training sample.

To negate this problem DTs can be \emph{Boosted} using one of a number of algorithms.
The procedure of boosting removes the power of that statistical fluctuations has over the final
Boosted Decision Tree (BDT).

A different approach to training BDTs is taken in each of the analyses in this thesis.
Each technique is outlined in the following sections.


\section{Bagging}
\label{sec:bdt:bag}
Bootstrap aggregating, or bagging~\cite{Bagging}, is a method of boosting whereby the effects of
statistical fluctuations are lessened by making many independent DTs and using the average
response.
Assume a true distribution, $f(x_i)$, where $x_i$ are the training variables.
The hypothesised value, from the output of a single DT, is $h_\alpha(x_i)$, for $1<\alpha<n$, where
$n$ is the number of DTs and each DT has a weight, $w_\alpha$ assiciated with it.
Using these definitions, it is possible to define three errors, namely:
square of the error of a single estimator
\begin{equation}
  \epsilon_\alpha(x_i) = \left(f(x_i)-h_\alpha(x_i)\right)^2;
  \label{eq:bdt:bag1}
\end{equation}
the weighted average of individual errors
\begin{equation}
  \bar\epsilon(x_i) = \sum_{\alpha=1}^nw_\alpha\epsilon_\alpha(x_i)
  \label{eq:bdt:bag2}
\end{equation}
and the error of an ensemble of DTs
\begin{equation}
  e(x_i) = \left(f(x_i)-\bar h(x_i)\right)^2.
  \label{eq:bdt:bag3}
\end{equation}
The weighted variance of response of the estimators $h_\alpha$ around a weighted mean is defined as
\begin{equation}
  V(x_i) = \sum_{\alpha=1}^nw_\alpha\left(h_\alpha(x_i) - \bar h(x_i)\right)^2.
  \label{eq:bdt:bag4}
\end{equation}
By inserting $f(x_i)-f(x_i)$ into \Eq{eq:bdt:bag4}, manipulating the algebra and remembering that
$\sum_\alpha=1$, the relationship
\begin{equation}
  e(x_i) = \bar\epsilon(x_i) - V(x_i)
  \label{eq:bdt:bag5}
\end{equation}
can be derived.
The means that the error squared of the ensemble of DTs is equal to the average error squared of an
individual estimator, minus the weighted variance; therefore the process of bagging reduces the
effect of statistical fluctuations in the training samples~\cite{Krogh95neuralnetwork}.

A bagged BDT is trained by randomly selecting events, with replacement, to train a single DT.
Hundreds of DTs can then be trained, and the weighted response from all DTs is the result of the
classifier, in the range zero to one.


%Random selection of
%samples (with replacement) features (without replacement)
%During bootstrap approximately 1 − 1/e ≈ 2/3 samples are retained and 1/e ≈ 1/3 samples left out



\section{AdaBoost}
\label{sec:bdt:ada}
The Adaptive Boost, AdaBoost, algorithm~\cite{AdaBoost} negates the effect of statistical
fluctuations in a data set by increaging the weights of misclassified events.
The algorithm begins by training a DT as described above, where each event has unit weight.
For subsequent DTs, the weight for each event, $i$, is modified for a tree $t$, to be
\begin{equation}
  w_i^t = c_i^t \times w_i^{t-1},
  \label{eq:ada:wt}
\end{equation}
where the $c$ is determined to be
\begin{equation}
  c_i^t = e^{\alpha_t\gamma_i^t}.
\end{equation}
Here, $\gamma_i^t$ is unity if event $i$ was classified incorrectly in tree $t-1$, otherwise it is
zero.
The value of $\alpha_t$ is the weight that the DT carries, and is given by
\begin{equation}
  \alpha_t = \frac12\ln\left(\frac{1-\epsilon_t}{\epsilon_t}\right)
\end{equation}
where $\epsilon$ is the weighted error rate.
Weights are then renormalised such that they sum to unity.
Multiple DTs are made in this fashion, forming a forest; where the response of the BDT classifier
is a combination of responces from all DTs in the forest.
The total response of a BDT, $T$, for an event characterized by $x_i$, is
\begin{equation}
  T(x_i) = \sum_{t=1}^{N_\mathrm{trees}} \alpha_tT_t(x_i)
  \label{eq:ada:fullbdt}
\end{equation}
where $T_t(x_i)$ is the response of tree $t$, which returns one if it classifies $x_i$ as being
signal-like, and negative one if it is background-like.

This reweighting proceedure articficially fluctuates the training sample which is used to train
each DT.


%For a training sample, described by the varibles $x_i$ with true responses $f(x_i)\in{-1,1}$,
%a BDT produced with the AdaBoost algorithm gives the output of
%\begin{equation}
  %T_m = \sum_{j=1}^{m}\alpha_jh_j(x_i),
%\end{equation}
%where $h_j(x_i)$ is the responce of a single DT.
%The coefficent $\alpha_j$, is the weight given to DT $j$, which is calculated as
%\begin{equation}
  %\alpha_j = \frac12\ln\left(\frac{1-\epsilon_j}{\epsilon_j}\right)
%\end{equation}
%where $\epsilon_j$ is the weighted error for a single DT
%\begin{equation}
  %%\epsilon_j = \frac{\sum_{y_i\neq h(x_i)}w_i}{\sum_{i=1} h(x_i)w_i}.
  %\epsilon_j = \frac{W_\mathrm{misclassified}}{W_\mathrm{total}}
%\end{equation}
%Each event in the training sample is given a weight which changes as the BDT is trained.
%Initially each event has unit weight, $w_i^{(1)}=1$, but this changes to
%\begin{equation}
  %w_i^{(m)} = e^{-y_iC_m(x_i)},
%\end{equation}
%for subsequent DTs, which boosts, thus boosting the importance of misclassified events.
%Decision tree number $m+1$ is then calculated using the above equations, and added to the total
%BDT.


\section{uBoost}
\label{sec:bdt:uboost}
The uniform Boosting, uBoost, alorithm~\cite{Stevens:2013dya} is designed to give a uniform
response in the signal efficiency of some variables $\zeta_i$.
The proceedure of creating a uniform BDT, uBDT, builds from the weigting technique used in the
AdaBoost algorithm, but additional weight is applied to events that lie in a region of parameter
space which is not prefoming with the desired efficiency.
Consider a BDT whose signal efficiency in the variables $\zeta_i$ is required to be $\epsilon$.
The weighting of each event is modified from that used in the AdaBoost algorithm in \Eq{eq:ada:wt}
to
\begin{equation}
  w_i^t = u_i^t\times c_i^t \times w_i^{t-1},
\end{equation}
where $u$ deontes a weighting proportional to the distance away from $\epsilon$ in the local
region:
\begin{equation}
  u_i^t = e^{\beta_t(\bar\epsilon-\epsilon_i^t}.
\end{equation}
The boosting parameter $\beta$ is calculated as
\begin{equation}
  \beta_t = \frac12\ln\left(\frac{1-e_t}{e_t}\right)
\end{equation}
and
\begin{equation}
  e_t = \sum_i w_i^{t-1}c_i^t\left|\bar\epsilon-\epsilon_i^t\right|.
\end{equation}

This leads to a single BDT whose response is analagous to \Eq{eq:ada:fullbdt} but with the addition
of the target efficiency $\bar\epsilon$
\begin{equation}
  T(x_i,\bar\epsilon) = \sum_{t=1}^{N_\mathrm{trees}} \alpha_tT_t(x_i,\bar\epsilon).
  \label{eq:ada:fullbdt}
\end{equation}
Therefore, a single of these BDTs is assiciated with a given target efficiency, where
fraction of $T(x_i,\epsilon)>\bar T(\bar\epsilon)$ is $\bar\epsilon$.
An arbiary number of BDTs can be concatinated, each with a different target efficiency, and the
total response is
\begin{equation}
  \mathcal{T}(x_i) =
  \frac1N\sum_{e=1}^{N}\Theta\left(T(x_i, \bar\epsilon) - \bar T(\bar\epsilon)\right).
\end{equation}






%builds on the AdaBoost algorithm;
%with additional weight given to events in the training sample

%This resulting split leaves two sub-nodes, one classifing predominantly background- and the other
%predominantly signal-like events.
%This is known as splitting.
%Each split

%The process of spilitting can continue
%
%the other




%Each of the analyses in this thesis uses the Boosted Decision Tree (BDT) algorithm because it can
%handle




%A Decision Tree (DT) is a tree-like graph of decisions where, in this case, the edges correspond to
%cuts on training variables and the final nodes are deemed to result in signal- of
%background-like events.

%A single DT is formed using a sample of signal and background candidates and a list of training
%variables which might weakly separate the two.
%A cut on a training variable is chosen to best separate the signal from the backgtround sample.
%Each candidate is then defined as being most signal- or background-like depending on this cut.
%This process can continue, splitting the sample down further, until...

%\section{Bagging and boosting}
%This algorithm will produce a DT whose output is weakly correlated to the true distribution of
%candidates.
%In order to increse the correlation, the technique of boosting is applied.
%The process of boosting with reference to a DT is to weight events that are misclassified by the
%DT, and then retrain the DT taking them into account.
%There are are many varieties of boosting,






%The boosted decision tree (BDT) is a machine
%\cite{AdaBoost}
