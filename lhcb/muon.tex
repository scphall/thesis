\subsection{Muon systems}
Finally, there is the muon system.
Each station uses Multi-Wire Projection Chambers exclusively, except for the centre of M1, where
the expected flux would age this technology too quickly; in this area Gas Electron Multiplier
detectors are used.
Like the calorimeters, the muon stations have increased cell density near the beam; however in
contrast, the cell density is greater in the bending plane than the non-bending plane in order to
increase momentum resolution.
Only muon stations M2-5 are used for particle identification; theses are interleaved with plates of
$80\cm$ thick lead plates, so only very energetic muons reach M5.
For this reason M4 and M5 are used to identify these penetrating muons, while M2 and M3 have
increased single hit resolution for momentum measurements.
In many \lhcb analyses a muon is identified based on hits within the muon system, the criteria is
known as {\tt isMuon}.
It is defined as follows: particles with momenta $3<p<6\gev$ must be associated with hits in M2 and M3; if
$6<p<10\gev$ there must be hits in M2, M3 and either M4 or M5; else if $p>10\gev$ there must be
associated hits in all M2-5.


