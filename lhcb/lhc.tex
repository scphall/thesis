\section{The LHC}
The \lhc is a superconducting synchrotron which can simultaneously accelerate beams of proton bunches
in opposite directions.
Physically, the \lhc is located at \cern, near Geneva, Switzerland; it is 27\km in
circumference and spans the Franco-Swiss border at a depth of about 100\m.
Protons are supplied to the \lhc from the \sps with an energy of 450\gev, they can then
be accelerated and collided with a centre-of-mass energy of up to 14\tev.
In the years 2011 and 2012 the centre-of-mass energy was 7 and $8\tev$ respectively.
Once the desired energy is reached the beams are collided at four interaction points.
The \lhcb detector is
situated at one of them~\cite{Alves:2008zz}.
Collisions of proton bunches occur every 50\ns reaching luminosities of up to
$7\e{32}\,\mathrm{cm}^{2}\mathrm{s}^{-1}$, however the beams entering \lhcb must be luminosity
levelled, to $3(4)\e{32}\,\mathrm{cm}^{2}\mathrm{s}^{-1}$ in 2011(2012), in order to reduce
detector occupancy.
These high energy collisions produce a vast number of \bbbar pairs which subsequently hadronize
into \bquark hadrons.
It is the prospect of studying these bound states of \bquark (and other heavy
flavour) quarks that has motivated the design of the \lhcb detector.


