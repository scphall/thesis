\section{Physics beyond the Standard Model}
%flavour structire of yukawa scouplings is not constrained by gauge invariance, which in turn
%implies that quark masses and their mixing are free parpameterrrs of the theoruy,

Apart from the significant deficit in the amount of CPV supplied by the SM with respect to the
amount required for a matter dominated Universe; there are a host of experimental reasons to
suspect that the SM is incomplete.
These include: the existence of gravity; the fact that the luminous matter totals less than $5\,\%$
of the mass in the Universe; and that neutrinos have mass.

Theoretical motivations for physics beyond the SM primarily revolve around the concept of
naturalness.

Naturalness is a concept whereby all physical couplings strengths are of $\mathcal{O}(1)$.





These reasons can be categorised as either experimental or theoretical.


\subsection{FCNCs}
Since the SM is such a good description of physics at energy scales that have been probed to date,
it is resonable to assume that this diverges at some cut off energy scale, $\Lambda$.
This scale can be set based on the solution of the hierarchy problem, which indicates that
$\Lambda$ should be less than a few TeV.
A bound can also be set on $\Lambda$ by considering processes which are absent from SM processes at
tree level, such as flavour changing neutral currnents (FCNCs).
This results in a value of $\Lambda$ which far exceeds the few TeV from solving the hierarchy
problem.
The conflict between these two determinations is named the \emph{Flavour Problem}.

The most pessimistic solution to the flavour problem is \emph{Minimal Flavour Violation} (MFV)
which simply assumes that beyond SM physics follows a Yukawa coupling like structure in the flavour
sector, this would lead to no discernable new physics in the flavour sector.


%The \decay{b}{s} FCNC is forbidden at tree level in the SM and are only allowed in higher-order
%electroweak processes.
%New particles from extensions to the SM can enter these loops and significantly alter

PAGE 9 PAtricks thesis






%Despite its countless successes,
%there are many experimental and theoretical arguments indicating that the SM is an incomplete
%picture of particle physics.
%Many theoretical problems arise from the idea of naturalness, that is that...
%
%%Experimental observations that are it is well known that the SM is incomplete; arguments for this
%%come from both the experimental and theoretical
%%leaves some experimentally observed phenomena unexplained.
%Experimentally, there are observed phenomena which are left unexplained by the SM.
%Neutrinos are treated as massless in the SM, but they are seen to oscillate in flavour space
%indicating that they must, in fact, have mass.
%Flat rotation curves of galaxies and gravitational lensing indicate the existence of Dark Matter,
%which is entriely unaccounted for by the SM.
%%Shortcomings include the
%% credence
%%For example the SM does not explain: gravity, dark matter, dark energy, and neutrino masses.
%
%Another problem is that the SM cannot reconcile the matter-antimatter asymmetry observable in the
%Universe today.
%The hypothesized process which caused this asymmetry is known as baryogenesis.
%%Baryogenesis is the term for a hypothesized process which resulted in the matter-dominated nature of
%%the Universe.
%Whatever this process may be, it must satisfy:
%\begin{itemize}
  %\item at least one baryon number (B) violating process,
  %\item Charge and Charge-Parity (CP) violation,
  %\item interactions out of thermal equilibrium.
%\end{itemize}
%These are the Sakharov conditions~\cite{1991SvPhU..34..392S}, and outline the minimum requirements
%of baryogenesis.
%The first of these criteria is an obvious one: at the time of the Big Bang $B=0$, whereas today
%$B\gg0$; hence $B$ must not be conserved in some process.
%If a process conserves charge then
%\begin{equation}
  %\Gamma(X\to Y+B)=\Gamma(\bar X \to \bar Y+\bar B),
%\end{equation}
%so $B$ will be conserved over time.
%However, this condition is insufficient.
%Consider a process $X\to q_Lq_L$ which has a CP-conjugate process $\bar X\to \bar q_R\bar q_R$;
%then
%\begin{equation}
  %\Gamma(X\to q_Lq_L) + \Gamma(X\to q_Rq_R)
  %=
  %\Gamma(\bar X\to \bar q_R\bar q_R) + \Gamma(\bar X\to \bar q_L\bar q_L)
%\end{equation}
%would still result in $B$ conservation even if C is violated.
%Thus, the process must be CP violating.
%The final criteria ensures that baryogenesis occurs at a higher rate than anti-baryogenesis.
%%Clearly the process must result in the violation of baryon number, and it must happen out of
%%thermal equilibrium otherwise the process would occur equally as often in each direction.
%%The third condition, CP violation (CPV) means that
%%$\Gamma(A+B\to C)\neq\Gamma(\bar A + \bar B \to \bar C)$, so the annihilation of the products of
%%the interaction cannot washout the asymmetry.
%
%As discussed, the flavour sector is the only source of CPV in the SM, but comes up short by around 10 orders of
%magnitude when explaining the matter dominated nature of the Universe~\cite{Cline:2006ts,Huet:1994jb}.
%%There is therefore a powerful reason to believe that NP enters the flavour sector.
%%The following chapter will elucidate as to how the flavour sector is the only source of CPV in the
%%SM.
%
%
%
%

