%&spell
\section{Physics beyond the Standard Model}
\label{sec:bsm}

For a long time the SM was waiting for the discovery of the Higgs boson to be completed.
In 2012, the Higgs boson was observed in 2012~\cite{Chatrchyan:2012ufa,Aad:2012tfa} by the
\cms~\cite{Chatrchyan:2012ufa} and \atlas~\cite{Aad:2012tfa} collaborations with a mass of
$\sim125\gev$.
This final piece of the picture has made the SM a remarkably robust theory with no predictions
deviating significantly from experimental observations.
Indeed, the fine structure constant, $\alpha$, which characterizes the coupling strength of
electromagnetic interactions
% is one of the most accurately predicted physical values.  The value of $\alpha^{-1}$
is measured to be:
\begin{equation}
  \alpha^{-1} = 137.035\,999\,074\,(44),
\end{equation}
with a relative uncertainty of 0.32 parts per billion~\cite{PDG2012}.
While the current best theoretical prediction, which accounts for up to tenth order Feynman
diagrams, is:
\begin{equation}
  \alpha^{-1} = 137.035\,999\,073\,(35),
\end{equation}
to an accuracy of 0.25 parts per billion~\cite{Aoyama:2012wj}.
Agreement between such precisely measured values, makes the theory of Quantum Electrodynamics,
which describes interactions of photons and charged particles in the SM, one of the most accurate
theories yet constructed.

Despite its countless successes, there are a plethora of indications --- both
experimental and theoretical --- that additional physics exists, beyond the SM (BSM).



\subsection{Failures and inconsistencies of the Standard Model}
\label{sec:bsm:fail}
%%%%%%%%%%%%%%%%%%%%%%%%%%%%%%%%%%%%%%%%%%%%%%%%%%%%%%%%%%%%%%%%%%%%%%%%%%%%%%%%%%%%%%%%%%%%%%%%%%%
% Experimental
%%%%%%%%%%%%%%%%%%%%%%%%%%%%%%%%%%%%%%%%%%%%%%%%%%%%%%%%%%%%%%%%%%%%%%%%%%%%%%%%%%%%%%%%%%%%%%%%%%%
One problem is that the SM cannot reconcile the matter-antimatter asymmetry observable in the
Universe today.
The hypothesized process which caused this asymmetry is known as baryogenesis.
Whatever this process may be it must satisfy the three Sakharov
conditions~\cite{1991SvPhU..34..392S}; which outline minimum requirements for baryogenesis.
The first, most obvious, criteria is that baryogenesis must violate baryon number.
The second Sakharov condition is that both Charge (C) and Charge-Parity (CP) are violated.
Lastly, baryogenesis must occur out of thermal equilibrium.
While the SM does contain CPV, it is approximately ten orders of
magnitude~\cite{Cline:2006ts,Huet:1994jb} too small to explain the baryon asymmetry of the
Universe.
%$B$; because at the time the Big Bang, $B=0$, whereas today $B\gg0$.
%Whatever this process may be, it must include:
%\begin{itemize}
  %\item at least one baryon number (B) violating process,
  %\item Charge and Charge-Parity (CP) violation,
  %\item interactions out of thermal equilibrium.
%\end{itemize}
%\comment{
%If C were conserved, then for a particle $X$ decaying into a particle $Y$ and $q$, where $q$ is
%baryonic:
%\begin{equation}
  %\Gamma\big(X\to Y+q\big)=\Gamma\big(\Xbar{X} \to \Xbar{Y}+\bar q\big),
%\end{equation}
%which would lead to $B$ being conserved over time.
%Violation of C alone is insufficient.
%Consider a process $X\to q_Lq_L$ which has a CP-conjugate process $\bar X\to \bar q_R\bar q_R$;
%then
%\begin{equation}
  %\Gamma\big(X\to q_Lq_L\big) + \Gamma\big(X\to q_Rq_R)
  %=
  %\Gamma\big(\Xbar{X}\to \bar q_R\bar q_R\big) + \Gamma\big(\Xbar{X}\to \bar q_L\bar q_L\big)
%\end{equation}
%would still result in $B$ conservation even if C is violated.
%Thus, the process must be CP violating.
%The final criteria ensures that baryogenesis occurs at a higher rate than anti-baryogenesis.
%}

Neither is Dark Matter nor Dark Energy is accounted for.
The most recent results show that the total amount of luminous matter in the Universe is about
$4.9\%$~\cite{Adam:2015rua,PDG2014}, the remaining being composed of Dark Matter and Dark Energy.
Dark Matter was first evidenced by H.~W.~Babcock in 1939 by flat rotation curves of
galaxies~\cite{1970ApJ...159..379R,1980ApJ...238..471R}.
Later, gravitational lensing around massive objects, such as the Bullet
Cluster~\cite{Markevitch:2003at}, provided further evidence.
Dark Matter constitutes about $26.8\,\%$ of the mass in the Universe, and the remainder is Dark
Energy.
Furthermore, the SM cannot explain massive neutrinos, which are necessary to explain neutrino
oscillations, which were first observed unambiguously at
Super-Kameokande~\cite{PhysRevLett.81.1562}.



The largest tension in the \ut is in  the measurements of \V{ub}.
A determination of \V{ub} can be made using inclusive and exclusive measurements of
$\decay{B}{X_u\ell\bar\nu_\ell}$ decays.
Inclusive measurements are made difficult from large
$\decay{B}{X_c\ell\bar\nu_\ell}$ backgrounds, while exclusive semi-leptonic modes suffer from
uncertainties introduced by form factors.
A value of $\left|\V{ub}\right|$ can also be obtained from the annihilation decay
$\decay{\Bp}{\taup\nu_\tau}$ which has low statistics.
Determinations of \V{ub} from these sources are:
\begin{align}
  |\V{ub}| &= \big(4.41\,^{+0.21}_{-0.23}\big)\e{-3}, & \quad\big(\mathrm{exclusive}\big)\text{\cite{PDG2014}}& \nonumber\\
  |\V{ub}| &= \big(3.28\pm{0.29}\big)\e{-3},  & \big(\mathrm{inclusive}\big)\text{\cite{PDG2014}}&\nonumber\\
  |\V{ub}| &= \big(4.22\pm{0.42}\big)\e{-3},  &
  \bam{Update-to-lhcb} \big(\decay{\Bp}{\taup\nu_\tau}\big)\text{\cite{PDG2012}}&\nonumber\\
  %|\V{ub}| &= \big(4.22\pm{0.42}\big)\e{-3}  & \big(\decay{\Bp}{\taup\nu_\tau}\big)\text{\cite{{PhysRevD.88.031102}&\nonumber
  \label{eq:th:vub}
\end{align}
limits on \ut measurements are shown in \Fig{fig:th:ut}.
There are obvious tensions between the inclusive and exclusive modes, which could hint at some new
process.
A more accurate value of \V{ub} from the decay $\decay{\Bp}{\taup\nu_\tau}$ might shed light on the
situation.
Global measurements of \V{ub} from the semi-leptonic and $\decay{\Bp}{\taup\nu_\tau}$
modes are shown alongside one another in \Fig{fig:th:ut}.

Some evidence for a NP particle comes from the Hyper-CP experiment~\cite{Burnstein:2004uk}, which
observes three $\decay{\Sigma^+}{p\mumu}$ events which survive a stringent selection.
These three events also peak in the invariant mass of the dimuon pair.
The narrowness of the peak in $m_{\mumu}$ it is consistent with a decay $\decay{\Sigma^+}{P^0p}$
where $\decay{P^0}{\mumu}$ where $m_{P^0}=214.3\pm0.5\mev$~\cite{Park:2005eka}.

\begin{itemize}
  \item $P_5^\prime$
\end{itemize}




%%%%%%%%%%%%%%%%%%%%%%%%%%%%%%%%%%%%%%%%%%%%%%%%%%%%%%%%%%%%%%%%%%%%%%%%%%%%%%%%%%%%%%%%%%%%%%%%%%%
% Theoretical
%%%%%%%%%%%%%%%%%%%%%%%%%%%%%%%%%%%%%%%%%%%%%%%%%%%%%%%%%%%%%%%%%%%%%%%%%%%%%%%%%%%%%%%%%%%%%%%%%%%
Theoretical reasons suggesting BSM physics
are often rather subjective, and revolve around the idea of natrualness.
Naturalness is a concept whereby a theory is deemed to be natural, or more plausible, if it has few
free parameters, all of which are of order one.
The SM is not a natural theory.
For example: there are a total of 18 free parameters in the SM, 13 of which reside in the flavour
sector;
The CKM matrix in $\Lag{\it q}^\mathrm{CC}$ exhibits a strongly hierarchic structure, favouring
flavour conserving weak interactions;
and the range of quark masses vary by four orders of magnitude.

Unnatural models with parameters which differ wildly in magnitude tend to result in fine-tuning.
This is when some parameters, or process, must cause a cancellation to extremely high precision in
order to agree with experimental observations.
In the SM, quantum loop corrections to the Higgs mass are of the order $10^{19}$
for $m_H\simeq125\gev$~\cite{Chatrchyan:2012ufa,Aad:2012tfa}.
This means that the cancellations required to result in a Hiigs mass comparable to the masses of
the weak vector bosons must be exact to 17 orders of magnitude.
This is known as the \emph{hierarchy problem}.
Additional particles from NP would contribute to these loops, and redduce of fine tuning required,
making the model more natural.

Fine tuning also appears in QCD.
A gauge invariant term that can be added to \Lag{QCD} is
\begin{equation}
  \Lag{QCD}^\theta = \theta\frac{g^2}{32\pi^2}
  F_{\mu\nu}^\alpha\widetilde F^{\mu\nu}_\alpha,
  \label{eq:strongcp}
\end{equation}
where $\theta$ and $g$ are constants, and $\alpha$ indicates a sum over colours.
The operator $F_{\mu\nu}$ is the gluon field strength tensor, and
\begin{equation}
  \widetilde F^{\mu\nu}_\alpha = \frac12\varepsilon_{\mu\nu\rho\sigma}F^{\rho\sigma}_\alpha.
\end{equation}
An interaction such $\Lag{QCD}^\theta$ would conserve charge symmetry, but violate parity and time
conjugation~\cite{Peccei:2006as}.
Such symmetry violations are in contradiction with the observed properties of the strong
force.
Bounds placed on the value of the neutron dipole moment, $|d_n| <2.9\e{-26}\,\mathrm{ecm}$
(at 90\% CL)~\cite{Baker:2006ts} require $\theta$ to be very small,
$\theta<10^{-19}$~\cite{Crewther:PQref9}, when \emph{a priori} it could be in the range
$0<\theta<2\pi$.
This occurrence of fine tuning is referred to as the \emph{strong CP problem}.
%A solution to this problem is to introduce an additional chiral symmetry, such that $\theta$
%becomes a field, the quanta of which are called axions.
%The axion model is discussed in more detail in \Sect{sec:db:theory}.

%\bam{Paragraph about the fact we don't know where and how NP enters.}

Despite the evidence for BSM physics, it is not known how it manifests itself.
There are numerous theories which seek to explain away various problems.

New physics models accommodate Dark Matter in different ways.
Some models have a \emph{dark} or \emph{hidden} sector which, apart from gravity, only
communicates with the visible sector feebly via messenger particles.
These messenger particles could potentially be observed after they decay into SM particles after
mixing with a $H$ or $Z$.
A solution to the strong CP problem is to introduce an additional chiral symmetry, such that
$\theta$, in \Eq{eq:strongcp}, becomes a field: the quanta of which are called axions.
This axion could be the messenger particle between the dark and visible
sectors~\cite{Peccei:2006as}.
%, as could the inflaton or dark $Z$; models including these
%particles are further detailed in \Sec{sec:db:intro}.

%%%%%%%%%%%%%%%%%%%%%%%%%%%%%%%%%%%%%%%%%%%%%%%%%%%%%%%%%%%%%%%%%%%%%%%%%%%%%%%%%%%%%%%%%%%%%%%%%%
%%%%%%%%%%%%%%%%%%%%%%%%%%%%%%%%%%%%%%%%%%%%%%%%%%%%%%%%%%%%%%%%%%%%%%%%%%%%%%%%%%%%%%%%%%%%%%%%%%
%%%%%%%%%%%%%%%%%%%%%%%%%%%%%%%%%%%%%%%%%%%%%%%%%%%%%%%%%%%%%%%%%%%%%%%%%%%%%%%%%%%%%%%%%%%%%%%%%%

%\subsection{Flavour problem}
%%The most pessimistic solution to the flavour problem is Minimal Flavour Violation (MFV)
%%which simply assumes that beyond SM physics follows a Yukawa coupling like structure in the flavour
%%sector, this would lead to no discernible new physics in the flavour sector.
%The most general way to parameterize NP is with an effective Lagrangian describing generic
%interactions at an energy scale $\mu$, in which long and short distance effects are separated.
%Long distance (equivalently low energy) effects are described by coefficients, $c$, which can be
%calculated using perturbative methods.
%Short distance (or high energy) effects are characterized by terms of operators, $\mathcal{O}$,
%which must be calculated non-peturbatively because they contain QCD interactions.
%The resulting effective Lagrangian includes a sum over all processes, $i$, which contribute at a
%given dimension, $d$:
%\begin{equation}
  %\Lag{eff}
  %=
  %\Lag{SM} + \sum_d\frac1{\Lambda^{d-4}}
  %\sum_ic_i^{(d)}\mathcal{O}_i^{(d)}.
  %\label{eq:th:lageff}
%\end{equation}
%Processes that cause FCNCs contribute in $d=6$, and can be written as
%\begin{equation}
  %\Delta\mathcal{L}^\mathrm{FCNC}
  %=
  %\sum_{i\neq j}\frac{c_{ij}}{\Lambda^2}
  %\left(\Xbar{\mathcal{O}}_{Li}\gamma^\mu\mathcal{O}_{Lj}\right)^2,
%\end{equation}
%where $c_{ij}$ are dimensionless FCNC couplings, where $i$ and $j$ are different quark generations.
%Bounds set on the energy scale $\Lambda$ by the analysis in \Ref{Isidori:2010kg} are:
%\begin{equation}
  %\Lambda > \frac{|c_{ij}|^\frac12}{|\Vconj{ti}\V{tj}|}\times4.4\tev
  %\sim
  %\left\{
    %\begin{array}{l}
      %|c_{sd}|^\frac12\times 1.3\e{4}\tev \\
      %|c_{bd}|^\frac12\times 5.1\e{2}\tev \\
      %|c_{bs}|^\frac12\times 1.1\e{2}\tev \\
    %\end{array}\right.
%\end{equation}
%These values are calculated under the assumption that NP has a natural flavour structure, where
%$c_{ij}\sim\mathcal{O}(1)$.
%So, either couplings are of order unity and NP begins to contribute at over $100\tev$; or couplings
%strengths are $\mathrm{O}(10^{-5})$ and NP contributes at the $1\tev$ level;
%it is expected for NP to appear at the $1\tev$ scale in order to solve the hierarchy problem.
%Either way, there is a conflict between the most natural coupling and energy scale; this is known
%as the \emph{flavour problem}.
%This leads to a host of contradictions, leading to the conclusion that NP has a highly non-generic
%flavour structure.
%
%The most pessimistic solution to the flavour problem is Minimal Flavour Violation (MFV)
%which simply assumes that beyond SM physics follows a Yukawa coupling like structure in the flavour
%sector, this would lead to no discernible new physics in the flavour sector.
%Assuming that nature has not chosen MFV, then contradictions from flavour problem indicate that NP
%searches should be made for both: particles
%with high mass, and particles which have small coupling strengths.
%The \lhcb experiment can probe the mass scale, since precision measurements of tree and loop
%diagrams are sensitive to virtual particles contributing at all orders, whose on-shell mass could
%be many TeV.
%The relatively high luminosity of interactions supplied by the \lhc mean that \lhcb is also
%sensitive to low coupling strengths, such as messenger particles from the dark sector.
%The following chapters explore a variety of different searches for beyond Standard Model physics.






%%%%%%%%%%%%%%%%%%%%%%%%%%%%%%%%%%%%%%%%%%%%%%%%%%%%%%%%%%%%%%%%%%%%%%%%%%%%%%%%%%%%%%%%%%%%%%%%%%
%%%%%%%%%%%%%%%%%%%%%%%%%%%%%%%%%%%%%%%%%%%%%%%%%%%%%%%%%%%%%%%%%%%%%%%%%%%%%%%%%%%%%%%%%%%%%%%%%%
%%%%%%%%%%%%%%%%%%%%%%%%%%%%%%%%%%%%%%%%%%%%%%%%%%%%%%%%%%%%%%%%%%%%%%%%%%%%%%%%%%%%%%%%%%%%%%%%%%





%Dark Matter is the lightest supersymmetric particle which is stable and messenger particle is super
%goldstino
Supersymmetry (SUSY)
supplies a Dark Matter candidate in the shape of the lightest supersymmetric
particle (LSP), which could communicate with the visible sector via a super-golstino.
The LSP is stable becuase of R-parity conservation.
Supersymmetry is a theory which introduces an additional super-particle for each SM fermion and
gauge boson, whose spin is different by a half integer.
The Higgs sector in SUSY comprises four Higgs doublets; two are spin-0 and two are spin-$\tfrac12$,
and then there are two each for $Y=\pm\tfrac12$.
After SUSY is broken there are five Higgs physical scalar particles, two are CP-even ($h^0$,
$H^0$) one is CP-odd scalar ($A^0$) and two charged are charged ($H^\pm$).
It also immediately solves the hierarchy problem.
Unfortunately, masses of the super-particles are unconstrained, and could be anywhere between a few
TeV and the Planck scale.

Some searches look directly for evidence of NP, this is the case for the analysis detailed in
\Chap{ch:db}, where a new particle, \db, is searched for in the dimuon invariant mass spectrum of
\decay{\Bd}{\Kstarent\mumu} consistent with \decay{\db}{\mumu}.
This is sensitive to a range of models which predict a light particle with a mass in the rtange
$2m_\mu\lesssim m_\db\lesssim4000\mev$, such as the axion model.
It is also sensitive to the $P^0$ that was potentially seen by the Hyper-CP experiment.

Instead of counting on NP to behave in an expected way, it is possible to search in a model
independent way by exploring general physics couplings.j
To do this it is useful to introduce the Operator Product Expansion (OPE).


