%&spell
\section{Physics beyond the Standard Model}

For a long time the SM was waiting for the discovery of the Higgs boson to be completed.
In 2012, the Higgs boson was observed in 2012~\cite{Chatrchyan:2012ufa,Aad:2012tfa} by the
\cms~\cite{Chatrchyan:2012ufa} and \atlas~\cite{Aad:2012tfa} collaborations with a mass of
$\sim125\gev$.
This final piece of the picture has made the SM a remarkably robust theory with no predictions
deviating significantly from experimental observations.
Indeed, the fine structure constant, $\alpha$, which characterizes the coupling strength of
electromagnetic interactions
% is one of the most accurately predicted physical values.  The value of $\alpha^{-1}$
is measured to be:
\begin{equation}
  \alpha^{-1} = 137.035\,999\,074\,(44),
\end{equation}
with a relative uncertainty of 0.32 parts per billion~\cite{PDG2012}.
While the current best theoretical prediction, which accounts for up to tenth order Feynman
diagrams, is:
\begin{equation}
  \alpha^{-1} = 137.035\,999\,073\,(35),
\end{equation}
to an accuracy of 0.25 parts per billion~\cite{Aoyama:2012wj}.
Agreement between such precisely measured values, makes the theory of Quantum Electrodynamics,
which describes interactions of photons and charged particles in the SM, one of the most accurate
theories yet constructed.

Despite its countless successes, there are a plethora of indications --- both
experimental and theoretical --- that additional physics exists, beyond the SM (BSM).



\subsection{Failures and inconsistencies of the Standard Model}
%%%%%%%%%%%%%%%%%%%%%%%%%%%%%%%%%%%%%%%%%%%%%%%%%%%%%%%%%%%%%%%%%%%%%%%%%%%%%%%%%%%%%%%%%%%%%%%%%%%
% Experimental
%%%%%%%%%%%%%%%%%%%%%%%%%%%%%%%%%%%%%%%%%%%%%%%%%%%%%%%%%%%%%%%%%%%%%%%%%%%%%%%%%%%%%%%%%%%%%%%%%%%
One problem is that the SM cannot reconcile the matter-antimatter asymmetry observable in the
Universe today.
The hypothesized process which caused this asymmetry is known as baryogenesis.
Whatever this process may be it must satisfy the three Sakharov
conditions~\cite{1991SvPhU..34..392S}; which outline minimum requirements for baryogenesis.
The first, most obvious, criteria is that baryogenesis must violate baryon number.
The second Sakharov condition is that both Charge (C) and Charge-Parity (CP) are violated.
Lastly, baryogenesis must occur out of thermal equilibrium.
While the SM does contain CPV, it is approximately ten orders of
magnitude~\cite{Cline:2006ts,Huet:1994jb} too small to explain the baryon asymmetry of the
Universe.
%$B$; because at the time the Big Bang, $B=0$, whereas today $B\gg0$.
%Whatever this process may be, it must include:
%\begin{itemize}
  %\item at least one baryon number (B) violating process,
  %\item Charge and Charge-Parity (CP) violation,
  %\item interactions out of thermal equilibrium.
%\end{itemize}
%\comment{
%If C were conserved, then for a particle $X$ decaying into a particle $Y$ and $q$, where $q$ is
%baryonic:
%\begin{equation}
  %\Gamma\big(X\to Y+q\big)=\Gamma\big(\Xbar{X} \to \Xbar{Y}+\bar q\big),
%\end{equation}
%which would lead to $B$ being conserved over time.
%Violation of C alone is insufficient.
%Consider a process $X\to q_Lq_L$ which has a CP-conjugate process $\bar X\to \bar q_R\bar q_R$;
%then
%\begin{equation}
  %\Gamma\big(X\to q_Lq_L\big) + \Gamma\big(X\to q_Rq_R)
  %=
  %\Gamma\big(\Xbar{X}\to \bar q_R\bar q_R\big) + \Gamma\big(\Xbar{X}\to \bar q_L\bar q_L\big)
%\end{equation}
%would still result in $B$ conservation even if C is violated.
%Thus, the process must be CP violating.
%The final criteria ensures that baryogenesis occurs at a higher rate than anti-baryogenesis.
%}

Neither is Dark Matter nor Dark Energy is accounted for.
The most recent results show that the total amount of luminous matter in the Universe is about
$4.9\%$~\cite{Ade:2015srua,PDG2014}, the remaining being composed of Dark Matter and Dark Energy.
Dark Matter was first evidenced by H.~W.~Babcock in 1939 by flat rotation curves of
galaxies~\cite{1970ApJ...159..379R,1980ApJ...238..471R}.
Later, gravitational lensing around massive objects, such as the Bullet
Cluster~\cite{Markevitch:2003at}, provided further evidence.
Dark Matter constitutes about $26.8\,\%$ of the mass in the Universe, and the remainder is Dark
Energy.
Furthermore, the SM cannot explain massive neutrinos, which are necessary to explain neutrino
oscillations, which were first observed unambiguously at
Super-Kameokande~\cite{PhysRevLett.81.1562}.



The largest tension in the \ut is in  the measurements of \V{ub}.
A determination of \V{ub} can be made using inclusive and exclusive measurements of
$\decay{B}{X_u\ell\bar\nu_\ell}$ decays.
Inclusive measurements are made difficult from large
$\decay{B}{X_c\ell\bar\nu_\ell}$ backgrounds, while exclusive semi-leptonic modes suffer from
uncertainties introduced by form factors.
A value of $\left|\V{ub}\right|$ can also be obtained from the annihilation decay
$\decay{\Bp}{\taup\nu_\tau}$ which has low statistics.
Determinations of \V{ub} from these sources are:
\begin{align}
  |\V{ub}| &= \big(4.41\,^{+0.21}_{-0.23}\big)\e{-3}, & \quad\big(\mathrm{exclusive}\big)\text{\cite{PDG2014}}& \nonumber\\
  |\V{ub}| &= \big(3.28\pm{0.29}\big)\e{-3},  & \big(\mathrm{inclusive}\big)\text{\cite{PDG2014}}&\nonumber\\
  |\V{ub}| &= \big(4.22\pm{0.42}\big)\e{-3},  &
  \big(\decay{\Bp}{\taup\nu_\tau}\big)\text{\cite{PDG2012}}\bam{UPDATE to \lhcb}&\nonumber\\
  %|\V{ub}| &= \big(4.22\pm{0.42}\big)\e{-3}  & \big(\decay{\Bp}{\taup\nu_\tau}\big)\text{\cite{{PhysRevD.88.031102}&\nonumber
  \label{eq:th:vub}
\end{align}
limits on \ut measurements are shown in \Fig{fig:th:ut}.
There are obvious tensions between the inclusive and exclusive modes, which could hint at some new
process.
A more accurate value of \V{ub} from the decay $\decay{\Bp}{\taup\nu_\tau}$ might shed light on the
situation.

Some evidence for a NP particle comes from the Hyper-CP experiment~\cite{Burnstein:2004uk}, which
observes three $\decay{\Sigma^+}{p\mumu}$ events which survive a stringent selection.
These three events also peak in the invariant mass of the dimuon pair.
The narrowness of the peak in $m_{\mumu}$ it is consistent with a decay $\decay{\Sigma^+}{P^0p}$
where $\decay{P^0}{\mumu}$ where $m_{P^0}=214.3\pm0.5\mev$~\cite{Park:2005eka}.

\begin{itemize}
  \item $P_5^\prime$
\end{itemize}




%%%%%%%%%%%%%%%%%%%%%%%%%%%%%%%%%%%%%%%%%%%%%%%%%%%%%%%%%%%%%%%%%%%%%%%%%%%%%%%%%%%%%%%%%%%%%%%%%%%
% Theoretical
%%%%%%%%%%%%%%%%%%%%%%%%%%%%%%%%%%%%%%%%%%%%%%%%%%%%%%%%%%%%%%%%%%%%%%%%%%%%%%%%%%%%%%%%%%%%%%%%%%%
Theoretical reasons suggesting BSM physics
are often rather subjective, and revolve around the idea of natrualness.
Naturalness is a concept whereby a theory is deemed to be natural, or more plausible, if it has few
free parameters, all of which are of order one.
The SM is not a natural theory.
For example: there are a total of 18 free parameters in the SM, 13 of which reside in the flavour
sector;
The CKM matrix in $\Lag{\it q}^\mathrm{CC}$ exhibits a strongly hierarchic structure, favouring
flavour conserving weak interactions;
and the range of quark masses vary by four orders of magnitude.

Unnatural models with parameters which differ wildly in magnitude tend to result in fine-tuning.
This is when some parameters, or process, must cause a cancellation to extremely high precision in
order to agree with experimental observations.
In the SM, quantum loop corrections to the Higgs mass are of the order $10^{19}$
for $m_H\simeq125\gev$~\cite{Chatrchyan:2012ufa,Aad:2012tfa}.
This means that the cancellations required to result in a Hiigs mass comparable to the masses of
the weak vector bosons must be exact to 17 orders of magnitude.
This is known as the \emph{hierarchy problem}.
Additional particles from NP would contribute to these loops, and redduce of fine tuning required,
making the model more natural.

Fine tuning also appears in QCD.
A gauge invariant term that can be added to \Lag{QCD} is
\begin{equation}
  \Lag{QCD}^\theta = \theta\frac{g^2}{32\pi^2}
  F_{\mu\nu}^\alpha\widetilde F^{\mu\nu}_\alpha,
\end{equation}
where $\theta$ and $g$ are constants, and $\alpha$ indicates a sum over colours.
The operator $F_{\mu\nu}$ is the gluon field strength tensor, and
\begin{equation}
  \widetilde F^{\mu\nu}_\alpha = \frac12\varepsilon_{\mu\nu\rho\sigma}F^{\rho\sigma}_\alpha.
\end{equation}
An interaction such $\Lag{QCD}^\theta$ would conserve charge symmetry, but violate parity and time
conjugation~\cite{Peccei:2006as}.
Such symmetry violations are in contradiction with the observed properties of the strong
force.
Bounds placed on the value of the neutron dipole moment, $|d_n| <2.9\e{-26}\,\mathrm{ecm}$
(at 90\% CL)~\cite{Baker:2006ts} require $\theta$ to be very small,
$\theta<10^{-19}$~\cite{Crewther:PQref9}, when \emph{a priori} it could be in the range
$0<\theta<2\pi$.
This occurrence of fine tuning is referred to as the \emph{strong CP problem}.
A solution to this problem is to introduce an additional chiral symmetry, such that $\theta$
becomes a field, the quanta of which are called axions.
%The axion model is discussed in more detail in \Sect{sec:db:theory}.

\bam{Paragraph about the fact we don't know where and how NP enters.}
To explore general physics coupings it is useful to introduce the Operator Product Expansion (OPE).







