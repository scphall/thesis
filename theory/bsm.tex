\section{Physics beyond the Standard Model}
\label{sec:bsm}

For a long time the completion of the \sm was reliant on the discovery of the Higgs boson.
Finally, in 2012, the \cms~\cite{Chatrchyan:2008aa} and \atlas~\cite{Aad:2008zzm} collaborations
observed a Higgs-like boson with a mass of $m_H\simeq125\gev$~\cite{Chatrchyan:2012ufa,Aad:2012tfa}.
%\replaced{
This final piece of the picture has made the \sm a remarkably robust theory in which, aside from
the major failings described in the next section, there are currently no predictions deviating
significantly from experimental observations.
%}{
  %This final piece of the picture has made the \sm a remarkably robust theory with no predictions
  %deviating significantly from experimental observations.
%}
Indeed, the theory of \QED~--- which describes interactions between
photons and charged particles in the \sm~--- is one of the most accurate theories yet constructed.
The coupling constant in \QED is the fine structure constant, $\alpha$, which has been measured
experimentally to be~\cite{PDG2012}
\begin{align}
  \alpha^{-1}_\mathrm{exp} &= 137.035\,999\,074\,(44), \nonumber\\
  \intertext{and predicted theoretically to be~\cite{Aoyama:2012wj}}
  \alpha^{-1}_\mathrm{th} &= 137.035\,999\,073\,(35). \nonumber
\end{align}
These measurements have precisions which are better than one part per billion, and the extent to
which they agree is testament to our understanding of \QED interactions.

Despite the countless successes of the \sm, there are a plethora of indications --- both
experimental and theoretical --- that additional physics exists, \bsm.


\subsection{Failures and inconsistencies of the Standard Model}
\label{sec:bsm:fail}
%%%%%%%%%%%%%%%%%%%%%%%%%%%%%%%%%%%%%%%%%%%%%%%%%%%%%%%%%%%%%%%%%%%%%%%%%%%%%%%%%%%%%%%%%%%%%%%%%%%
% Experimental
%%%%%%%%%%%%%%%%%%%%%%%%%%%%%%%%%%%%%%%%%%%%%%%%%%%%%%%%%%%%%%%%%%%%%%%%%%%%%%%%%%%%%%%%%%%%%%%%%%%
There are some phenomena that have been observed experimentally which cannot be explained by the
\sm.
The flavour-change of neutrinos in flavour space mean that they must have
mass; this is not accounted for the \sm framework.
Neither are the observations of the \BAU, or \dm.

The \sm cannot reconcile the matter-antimatter asymmetry observed in the
Universe today.
The hypothesised process which caused this asymmetry is known as baryogenesis, the minimum
requirements for this process are outlined in the
%Whatever this process may be it must satisfy the
three Sakharov
conditions~\cite{1991SvPhU..34..392S}.
%which outline the minimum requirements for baryogenesis.
The first, most obvious, criteria is that baryogenesis must violate baryon number.
The second Sakharov condition is that both \gls{C} and \CP are violated.
Lastly, baryogenesis must occur out of thermal equilibrium.
While the \sm does contain  some \CPV, it is approximately ten orders of
magnitude~\cite{Cline:2006ts,Huet:1994jb} too small to explain the \BAU.
In \Chap{ch:dsphi} a measurement of the \CP-asymmetry of the decay \btodsphi is made in an effort
to find \CPV processes not predicted by the \sm that would go towards explaining the \BAU.


% DARK MATTER
It is well known that the vast majority of mass in the Universe is unaccounted for.
Luminous matter totals only \approx$4.9\pc$ of the Universe~\cite{Adam:2015rua,PDG2014}, and the rest
is known only as \dm (\approx$26.8\pc$) and dark energy (\approx$68.3\pc$).
Dark Matter is an old and well motivated concept with the first evidence found in 1939 by H.~W.~Babcock
in the form of flat galactic rotation curves~\cite{1970ApJ...159..379R,1980ApJ...238..471R}.
Since then, further credence to the existence of \dm has come from corroborating evidence supplied
by, for example, gravitational lensing around the Bullet
Cluster~\cite{Markevitch:2003at} and the Cosmic Microwave Background~\cite{1990ApJ...349L...1T}.


Observations of \dm are used to motivate \np models which include \emph{dark sectors}.
A dark sector is a name for a particle, or group of particles, which is gauged under a
different gauge group to the \sm particles and therefore cannot interact with them directly.
There are a plethora of such models, but generally dark particles can only interact with the \sm
via weakly interacting messenger particles, which could be either vector or scalar.
In generality, these are known as \emph{Dark Bosons}.
This thesis documents a search for a dark boson in the dimuon spectrum of \btokstrmumu in
\Chap{ch:db}.

Some excitement was caused by a hint of a dark sector messenger particle from the Hyper-\CP
experiment~\cite{Burnstein:2004uk}, which observes three $\decay{\Sigma^+}{p\mumu}$ events which
survive a stringent selection.
These three events also peak in the invariant mass of the dimuon pair.
The narrowness of this peak is indicative of a two body decay, consistent with
$\decay{\Sigma^+}{pP^0}$ and the subsequent decay of the \np particle via $\decay{P^0}{\mumu}$,
where $m_{P^0}=214.3\pm0.5\mev$~\cite{Park:2005eka}.
The $P^0$ could be a supersymmetric Goldstone boson, or a dark boson from many other theories.



\SUSY is a theory which imposes a symmetry relating fermions to bosons, and naturally supplies a
\dm candidate in the shape of the lightest supersymmetric
particle.
The lightest \SUSY particle is stable, in most models, because the symmetry of $R$-parity
is assumed to be conserved.
The definition of $R$-parity involves the baryon number, lepton number, and spin of a particle;
the upshot is that for SM(\SUSY) particles $R$-parity is 1(-1).
The Higgs sector in \SUSY comprises four Higgs doublets; two are spin-0 and two are spin-$\tfrac12$,
and then there are two each for $Y=\pm\tfrac12$.
After \SUSY is broken there are five Higgs physical scalar particle; two are \CP-even ($h^0$,
$H^0$); one is a \CP-odd scalar ($A^0$) and two are charged ($H^\pm$).
Unfortunately, masses of the super-particles are unconstrained, and could be anywhere between a few
TeV and the Planck scale.


Particle dynamics can be affected by massive \np particles, like those in \SUSY, in lower order
processes because at this level virtual particles can contribute.
\glspl{FCNC} are heavily suppressed in the \sm.
Firstly, they are forbidden at tree-level; secondly, loop-level diagrams are suppressed by factors
coming from the \ckm matrix.
These rare, and background-suppressed processes provide ideal environments in which to search for \bsm
physics, since new massive off-shell particles can contribute to the loops and cause significant
deviations from \sm expectations.
Chapter~\ref{ch:hhh} details an observation of a high statistics \fcnc decay,
which could be used for future \np searches.



%%%%%%%%%%%%%%%%%%%%%%%%%%%%%%%%%%%%%%%%%%%%%%%%%%%%%%%%%%%%%%%%%%%%%%%%%%%%%%%%%%%%%%%%%%%%%%%%%%%
% Theoretical
%%%%%%%%%%%%%%%%%%%%%%%%%%%%%%%%%%%%%%%%%%%%%%%%%%%%%%%%%%%%%%%%%%%%%%%%%%%%%%%%%%%%%%%%%%%%%%%%%%%
Theoretical shortcomings of the \sm include: its inability to incorporate gravity at the quantum
scale and the existence of dark energy.
But theoretical arguments are often less tangible, and
rather subjective, revolving around the idea of \emph{naturalness}.
Naturalness is a concept whereby a theory is deemed to be natural, or more plausible, if it has few
free parameters, all of which have a magnitude $\mathcal{O}(1)$.
The \sm is not a natural theory: having
a total of 18 free parameters, 13 of which reside in the flavour
sector.
Other unnatural features of the flavour sector of the \sm are that the \ckm matrix is strongly
hierarchic, and quark masses vary by four orders of magnitude.

Of all the fundamental parameters in the \ckm matrix, \V{ub} is known to the lowest precision, and
it is therefore important to accurately measure it.
%\replaced{
This parameter is particularly interesting because drives the largest discrepancies in global fits to the \ut~\cite{Charles:2015gya}.
%}{
%This parameter is particularly interesting because it is the source of the largest
%tension in the \ut.
%}
A determination of \V{ub} can be made using inclusive and exclusive measurements of semi-leptonic
$\decay{B}{X_u\ell\bar\nu_\ell}$ decays; where $X_u$ is some meson containing a \uquark quark.
Inclusive measurements are made difficult by large
$\decay{B}{X_c\ell\bar\nu_\ell}$ backgrounds, while exclusive semi-leptonic modes suffer from
theoretical uncertainties.
Both these methods are well established, and both rely on non-perturbative \QCD calculations.
Current inclusive and exclusive measurements of \V{ub} are~\cite{PDG2014,Amhis:2014hma}:
\begin{align}
  \left|\V{ub}\right|_\mathrm{excl}
  &= \big(3.28\pm{0.29}\big)\e{-3} \nonumber\\
  \left|\V{ub}\right|_{\makebox[\widthof{$_\mathrm{excl}$}][l]{$_\mathrm{incl}$}}
  &= \big(4.41\,^{+0.21}_{-0.23}\big)\e{-3}. \nonumber
  \label{eq:th:vub}
\end{align}
Currently, there is no explanation for this discrepancy between inclusive and exclusive
measurements.
%\replaced{
An exclusive measurement of $\left|\V{ub}\right| = (3.27\pm0.23)\e{-3}$~\cite{Aaij:2015bfa} has
also been made by the \lhcb collaboration using the baryonic decay \decay{\Lb}{p\mun\neumb}.
%}{
%A measurement from the \lhcb experiment uses the baryonic decay \decay{\Lb}{p\mun\neumb}
%calculated a value of $\left|\V{ub}\right|$ to be $(3.27\pm0.23)\e{-3}$~\cite{Aaij:2015bfa}.
%This is an exclusive measurement, and is in agreement with other exclusive measurements.
%}

Another method to access the \ckm matrix parameter $\left|\V{ub}\right|$ is via the
annihilation-type decay $\decay{\Bp}{\taup\nu_\tau}$.
Measurements from both the \babar and \belle experiments of
$\BF\big(\decay{\Bp}{\taup\nu_\tau}\big)$~\cite{Lees:2012ju,Abdesselam:2014hkd} suffer from small
statistics, but are found to be in better agreement with values of $\left|\V{ub}\right|$
determined using inclusive measurements than exclusive.
The lack of statistics for this channel is because with the missing energy from neutrinos, and the
finite lifetime of the $\taup$, it is very unlikely to form a vertex of good quality.
Searching for the decay $\decay{\Bp}{\taup\nu_\tau}$ is not viable at \lhcb; instead, decays of the
same topologies can be searched for.
The decay \btodsphi is also an annihilation-type decay in which \V{ub} appears in the amplitude;
an analysis of this decay is described in \Chap{ch:dsphi}.

Current measurements of angles and side lengths of the \ut, from \Ref{Charles:2015gya}, are shown
in \Fig{fig:th:ckmfitter}.
This figure also shows
global \V{ub} measurements from the semi-leptonic and $\decay{\Bp}{\taup\nu_\tau}$
modes alongside one another.


\begin{figure}
  \begin{center}
      \includegraphics[width=0.80\textwidth]{rhoeta_small_Vub}
  \end{center}
  \caption[Unitarity triangle and current constraints]
  {
    Diagram of the \ut with coloured bands indicating various constraints on
    side lengths, angles and position of the apex, which is taken from the CKMfitter group in
    Ref.~\protect\cite{Charles:2015gya}.
    The constraints on \V{ub} from the combination of inclusive and exclusive modes
    ($\left|\V{ub}\right|_\mathrm{SL}$) is given separately to a value obtained using
    $\BF\left(\decay{\Bp}{\taup\nu_\tau}\right)$, ($\left|\V{ub}\right|_{\tau\nu}$).
  }
  \label{fig:th:ckmfitter}
\end{figure}

Unnatural \np models with parameters that differ wildly in magnitude tend to
lead to parameters or processes that must cancel to absurdly
high precision in order to agree with experimental observations.
These precise cancellations are known as \emph{fine tuning}.
In the \sm, quantum loop corrections to the Higgs mass are of the order $10^{19}$
for $m_H\simeq125\gev$~\cite{Chatrchyan:2012ufa,Aad:2012tfa}.
This means that the cancellations required to result in a Higgs mass comparable to the masses of
the weak vector bosons must be exact to 17 orders of magnitude.
This instance of fine tuning is known as the \emph{hierarchy problem}.
A solution for the hierarchy problem is to introduce \np particles, whose contributions to
loop level processes reduce the magnitude of fine tuning required to a level deemed
acceptable.
\SUSY immediately solves the hierarchy problem because for every \sm particle that
contributes to the Higgs mass, a \SUSY particle also contributes, but with the opposite sign.
It should be noted, however, that while \SUSY does solve a number of problems, it is not natural
since it has far more free parameters than the \sm.

Fine tuning also appears in \QCD.
A gauge invariant term that can be added to \Lag{QCD} is
\begin{equation}
  \Lag{QCD}^\theta = \theta\frac{g^2}{32\pi^2}
  G_{\mu\nu}^a\widetilde G^{\mu\nu}_a,
  \label{eq:strongcp}
\end{equation}
%where \replaced{
where $\theta$ is a phase, and $g$ is a constant~\cite{Peccei:2006as}.
%}{
%$\theta$ and $g$ are constants}
%~\cite{Peccei:2006as}.
The operator $G_{\mu\nu}$ is the gluon field strength tensor, and
\begin{equation}
  \widetilde G^{\mu\nu}_a = \frac12\varepsilon_{\mu\nu\rho\sigma}G^{\rho\sigma}_a.
\end{equation}
Interactions in $\Lag{QCD}^\theta$ would conserve \gls{C} symmetry, but violate both \gls{P} and
\gls{T} conjugation~\cite{Peccei:2006as}.
Such symmetry violations contradict the observed properties of the strong
force, so $\Lag{QCD}$ must either be absent, or heavily suppressed.
Bounds placed on the value of the neutron dipole moment, $|d_n| <2.9\e{-26}\,\mathrm{ecm}$
(at 90\% CL)~\cite{Baker:2006ts} leads to the constraint that
$\theta<10^{-19}$~\cite{Crewther:PQref9}, when \emph{a priori} it could be in the range
$0<\theta<2\pi$.
This occurrence of fine tuning is referred to as the \emph{strong \CP problem}.

Despite the evidence for \bsm physics and the list of problems that must be solved, its precise
manifestation is unknown.
There are numerous theories concerning NP scenarios which seek to solve various problems.

A solution to the strong \CP problem is to introduce an additional chiral $U(1)$ symmetry,
which is known as a \gls{PQ} symmetry~\cite{Peccei:2006as}.
Breaking the \gls{PQ} symmetry leads to $\theta$, in \Eq{eq:strongcp}, becoming a
field with quanta known as \emph{axions}.
These axions could be the messenger particle between a dark and visible
sector~\cite{Peccei:2006as}.


%%%%%%%%%%%%%%%%%%%%%%%%%%%%%%%%%%%%%%%%%%%%%%%%%%%%%%%%%%%%%%%%%%%%%%%%%%%%%%%%%%%%%%%%%%%%%%%%%%
%%%%%%%%%%%%%%%%%%%%%%%%%%%%%%%%%%%%%%%%%%%%%%%%%%%%%%%%%%%%%%%%%%%%%%%%%%%%%%%%%%%%%%%%%%%%%%%%%%
%%%%%%%%%%%%%%%%%%%%%%%%%%%%%%%%%%%%%%%%%%%%%%%%%%%%%%%%%%%%%%%%%%%%%%%%%%%%%%%%%%%%%%%%%%%%%%%%%%


Some searches look directly for evidence of \np; this is the case for the analysis detailed in
\Chap{ch:db}, where a new particle, \db, is searched for in the dimuon invariant mass spectrum of
\decay{\Bd}{\Kstarent\mumu} consistent with \decay{\db}{\mumu}.
This is sensitive to a range of models which predict a light particle with a mass in the range
$2m_\mu\lesssim m_\db\lesssim4000\mev$, such as the axion model.
It is also sensitive to the $P^0$ that was hinted at by the Hyper-\CP
experiment~\cite{Park:2005eka}.

Instead of counting on NP to behave in an expected way, it is possible to search in a model
independent manner by exploring general physics couplings.
To do this it is useful to introduce the \OPE~\cite{PhysRev.179.1499}.


