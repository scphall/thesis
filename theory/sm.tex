\section{The Standard Model}
The current formulation of the SM of particle physics was concocted in the 1970s, when the Higgs
mechanism was incorporated into Glashow's electroweak theory by Salam and Weinberg.
The theory prescribes a treatment
as to how fundamental particles interact via three of the four
fundamental forces, namely: the strong, weak and electromagnetic forces.

%Mathematically, the SM is a locally gauge invariant quantum field theory, where excitations of
%various fields manifest themselves as particles.
%There is a global Poincare symmetry as required by special relativity; and a local
%$SU(3)\times SU(2)\times U(1)$ symmetry which encapsulates the SM Lagrangian:
%\begin{equation}
  %\Lag{SM} = \Lag{EW} + \Lag{Strong} + \Lag{Higgs}.
%\end{equation}
%Where the components describe the electroweak, strong and Higgs interactions.
%Each generator of this local gauge group is associated with a gauge boson which mediates
%interactions between other bosons and fermions.

Mathematically, the SM is a locally gauge invariant quantum field theory.
It inhabits a space-time with a global Poincar\'e symmetry that obeys a local
$SU(3)\times SU(2)\times U(1)$ symmetry.
Each generator in this group corresponds to a gauge-boson; so the strong force ($SU(3)$ group) has
eight gluons that mediate the interaction, and the electroweak force ($SU(2)\times U(1)$ group) has
$3+1$ gauge bosons, which are the weak gauge bosons ($Z$, $W^\pm$) and the photon ($\gamma$).
These are all vector fields.
Fermions are described by spinor fields, $\psi$, which obey the Dirac equation:
\begin{equation}
  (i\hbar\gamma^\mu\partial_\mu - mc)\psi = 0.
  \label{th:eq:dirac}
\end{equation}
The fermions of the SM constitute six leptons (electron, electron neutrino, muon, muon neutrino,
tau and tau neutrino) and six quarks (up, down, charm, strange, top and bottom), which are
organized into pairs forming three generations.
For each fermion there is a corresponding antiparticle with the same mass and opposite charge ---
charge being the conserved quantity resulting from the global gauge symmetry (by Noether's
theorem).
There is also a single scalar field in the SM, that of the Higgs boson.
%The only additional particle is the scalar Higgs boson.

The SM Lagrangian can be expressed as a sum of components:
\begin{equation}
  \Lag{SM} = \Lag{Strong} + \Lag{V} + \Lag{\ell} + \Lag{\it q} + \Lag{Higgs} + \Lag{Yuk}.
  \label{eq:th:lag}
\end{equation}
The first three terms describe: interactions of the strong force between colour carrying particles,
weak vector boson self-interactions, and the electroweak behavior of leptons.
The remaining terms describe the electroweak behavior of quarks, the Higgs interaction and Yukawa
couplings, respectively.
These latter terms are of fundamental importance as to how the flavour changing currents and CPV
occur in the SM, and will be be discussed in detail.
%the SM.
%Each of these shall be discussed in turn.

The structure of CP and flavour violation emerges as a direct consequence of the Higgs mechanism
breaking the local electroweak symmetry.
The Lagrangian of the scalar Higgs field is:
\begin{align}
  \Lag{Higgs}
  &= \left(D_\mu\Phi\right)^\dagger\left(D^\mu\Phi\right) - V(\Phi) \\
  &= \left(D_\mu\Phi\right)^\dagger\left(D^\mu\Phi\right) - \mu^2\left(\Phi^\dagger\Phi\right) +
  \lambda\left(\Phi^\dagger\Phi\right),
  \label{eq:th:laghiggs1}
\end{align}
where $\mu$ and $\lambda$ are constants, $D_\mu$ is the covariant derivative, and $\Phi$ is the
Higgs doublet, defined by:
\begin{equation}
  \Phi = \frac{1}{\sqrt{2}}
  \begin{pmatrix}
    \phi_1 + i\phi_2 \\
    \phi_3 + i\phi_4 \\
  \end{pmatrix}.
  \label{eq:th:phi}
\end{equation}
Taking $\mu^2<0$ and $\lambda>0$ moves the minimum of the potential $V(\Phi)$ away from zero to a distance $v$:
\begin{equation}
  v = \sqrt{\frac{\mu^2}{\lambda}}.
\end{equation}
At this point the Higgs field gets a vacuum expectation value (VEV)
%$\braket{\phi} = \tfrac{1}{\sqrt{2}}v$.
of $\langle\phi\rangle = \tfrac{1}{\sqrt{2}}v$.
The direction of the VEV from the origin is arbitrary, but the choice of:
\begin{align}
  \bra{0}\phi_1\ket{0} =
  \bra{0}\phi_2\ket{0} =
  \bra{0}\phi_4\ket{0} = 0  &&
  \bra{0}\phi_3\ket{0} = v,
\end{align}
is convenient, and changes Eq.~\ref{eq:th:phi} to:
\begin{equation}
  \Phi = \frac{1}{\sqrt{2}}
  \begin{pmatrix}
    \eta_1 + i\eta_2 \\
    v + i\eta_4 \\
  \end{pmatrix}.
  \label{eq:th:eta}
\end{equation}
Here, $\eta_1$, $\eta_2$ and $\eta_4$, are Goldstone bosons which, by choosing an appropriate
gauge, become the longitudinal components of the weak bosons.
This choice of gauge simplifies $\Phi$ to:
\begin{equation}
  \Phi =
  \begin{pmatrix} 0 \\ v+H
  \end{pmatrix},
  \label{eq:th:phi2}
\end{equation}
where $H$ is the physical Higgs boson.
%Using this in Eq.~\ref{eq:th:laghiggs1} gives:
Inserting Eq.~\ref{eq:th:phi2} into Eq.~\ref{eq:th:laghiggs1} gives:
%\begin{multline}
  %\Lag{Higgs} =
  %\frac12\left(\partial_\mu H\right)\left(\partial^\mu H\right)
  %+\frac14g^2\left(v^2 + 2vH + H^2\right)W_\mu^+W^{-\mu}  \\
  %+\frac18\left(g^2 + g^{\prime2}\right)\left(v^2 + 2vH + H^2\right)Z_\mu Z^\mu
  %+ \mu^2H^2 + \frac14\lambda\left(H^4+4vH^3\right) + \cdots,
  %\label{eq:th:laghiggs2}
%\end{multline}
\begin{equation}
  \Lag{Higgs} =
  \frac12\left(\partial_\mu H\right)\left(\partial^\mu H\right)
  +\mu^2H^2
  +\left(m_W^2W_\mu^+W^{-\mu} + \frac{m_Z^2}{2}Z_\mu Z^\mu\right)
  \cdot
  \left(1 + \frac{H}{v}\right)^2
  \label{eq:th:laghiggs2}
\end{equation}
where $g$ and $g^\prime$ are coupling constants and other terms are three- and four-point
interactions of the Higgs with itself and weak gauge bosons.
Thus, the $U(1)$ local gauge symmetry is broken, weak gauge boson acquire a mass while photons remain
massless; as is consistent with observations.


%It is not possible to directly insert mass terms for fermions, because the required terms
%($m(\bar\psi_L\psi_R+\bar\psi_R\psi_L$) are not allowed \bam{ELABORATE}.
All fermions (except neutrinos) also get a mass after the spontaneous symmetry breaking (SSB) of
the $U(1)$ symmetry.
The Dirac mass term for a chiral field should be of the form:
\begin{equation}
  \Lag{mass} = -m_\psi\left(\bar\psi_R\psi_L + \bar\psi_L\psi_R\right),
\end{equation}
but the left- and right-handed fields have different $U(1)$ charges %Y hypercharge
and so transform differently under local gauge transformations and so cannot be added to \Lag{SM}.
However, masses can be generated through the Yukawa couplings (\Lag{Yuk} in \Eq{eq:th:lag}), which
describe integrations between all fermionic fields and the Higgs doublet, and can be written:
\begin{equation}
  \Lag{Yuk} = \sum_{\substack{\ell=\\e,\mu,\tau}}\left(\Lag{Yuk}^\ell\right) + \Lag{Yuk}^q,
  \label{eq:th:yukking}
\end{equation}
where $\ell$ and $q$ denote the lepton and quark sectors respectively.
First considering the lepton term, after SSB:
\begin{align}
  \Lag{Yuk}^\ell
  &= - g_\ell\left(\bar\chi_L\Phi \ell_R + \bar \ell_R\Phi^\dagger\chi_L\right) \\
  &= - \frac{g_\ell v}{\sqrt{2}}\left(\bar\ell_L\ell_R + \bar\ell_R\ell_L\right)\cdot
  \left(1 + \frac{H}{v}\right)
     %- \frac{g_\ell}{\sqrt{2}}H\left(\bar\ell_L\ell_R + \bar\ell_R\ell_L\right),
\end{align}
where $g_\ell$ is a coupling constant, and
\begin{align}
  \chi_L = \begin{pmatrix}\nu_L \\ \ell_L \end{pmatrix}.
\end{align}
Thus the leptons get a mass of $m_\ell = \tfrac{1}{\sqrt{2}}g_\ell v$, and interact with the Higgs
field.

The story for $\Lag{Yuk}^q$ is a bit more involved.
Before SSB:
\begin{align}
  \Lag{Yuk}^q &= - y_{ij}^u\bar Q_L^i\Phi u_R^j
  - y_{ij}^d\bar Q_L^i\tilde\Phi d_R^j + \mathrm{h.c.},
  \label{eq:th:lagyukq}
\end{align}
where there is an implicit sum over all generations $i$ and $j$, $y^{u,d}$ is a $3\times3$ matrix
characterizing the Yukawa couplings between quark generations,
\begin{align}
  \tilde\Phi_i &= \varepsilon_{ij}\Phi_j,
    \qquad \mathrm{and} \quad Q_L = \begin{pmatrix}u_L \\ d_L \end{pmatrix}.
\end{align}
After SSB, and the Higgs acquires a VEV, \Eq{eq:th:lagyukq} becomes:
\begin{equation}
  \Lag{Yuk}^q =
  - \frac{v}{\sqrt{2}}
  \left(
  y_{ij}^u\bar u_L^iu_{R,j}
  + y_{ij}^d\bar d_L^id_{R,j},
  + \mathrm{h.c.}
  \right)
  \cdot
  \left(1 + \frac{H}{v}\right),
  \label{eq:th:lagyuk2}
\end{equation}
where the mass of the quarks is $m_q = \frac{v}{\sqrt{2}}y_{ij}^q$.
However, it is more convenient to change a basis in which the matrix $m^q$ is diagonal such that
$m_{ij}^\mathrm{diag} = V_{Lik}m_{kl}(V_R^\dagger)_{lj}$.
This is exactly equivalent to transforming the chiral quark fields for up- and down-type quarks
accordingly:
\begin{align}
  q_L^\alpha = \left(V_L^q\right)_{\alpha i}q_L &&
  q_R^\alpha = \left(V_R^q\right)_{\alpha i}q_R,
\end{align}
where the index of the original basis is identified with $i$ and the mass basis uses $\alpha$.

The rotations of the basis of the chiral quark fields leave much of \Lag{SM} unchanged since
$V_{qL}^\dagger V_{qL} = V_{qR}^\dagger V_{qR} = \mathbb{1}$.
However, this is not the case in the charged current (CC) part of \Lag{\it q}; which transforms as:
\begin{align}
  \Lag{\it q}^\mathrm{CC}
  &= \frac{g}{2}i\gamma^\mu
  \left[\bar u_Ld_LW^+_\mu + \bar d_Lu_LW^-_\mu
  \right]  \\
  &= \frac{g}{2}i\gamma^\mu
  \left[
    \bar u_L\left(V_{uL}V_{dL}^\dagger\right)d_LW^+_\mu +
    \bar d_L\left(V_{dL}V_{uL}^\dagger\right)u_LW^-_\mu
  \right]  \\
  &= \frac{g}{2}i\gamma^\mu
  \left[
    V\bar u_Ld_LW^+_\mu +
    V^\dagger\bar d_Lu_LW^-_\mu
  \right].
\end{align}
The matrix $V$ is defined is known as the Cabibbo-Kobayashi-Maskawa (CKM) matrix
and parameterizes the couplings between up- and down-type quarks in charged weak currents.




