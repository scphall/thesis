\section{The Standard Model}
The current formulation of the SM of particle physics was concocted in the 1970s, when the Higgs
mechanism was incorporated into Glashow's electroweak theory by Salam and Weinberg.
The theory prescribes a treatment
as to how fundamental particles interact via three of the four
fundamental forces, namely: the strong, weak and electromagnetic forces.

%Mathematically, the SM is a locally gauge invariant quantum field theory, where excitations of
%various fields manifest themselves as particles.
%There is a global Poincare symmetry as required by special relativity; and a local
%$SU(3)\times SU(2)\times U(1)$ symmetry which encapsulates the SM Lagrangian:
%\begin{equation}
  %\Lag{SM} = \Lag{EW} + \Lag{Strong} + \Lag{Higgs}.
%\end{equation}
%Where the components describe the electroweak, strong and Higgs interactions.
%Each generator of this local gauge group is associated with a gauge boson which mediates
%interactions between other bosons and fermions.

Mathematically, the SM is a locally gauge invariant quantum field theory.
It inhabits a space-time with a global Poincar\'e symmetry that obeys a local
$SU(3)\times SU(2)\times U(1)$ symmetry.
Each generator in this group corresponds to a gauge-boson; so the strong force ($SU(3)$ group) has
eight gluons that mediate the interaction, and the electroweak force ($SU(2)\times U(1)$ group) has
$3+1$ gauge bosons, which are the weak gauge bosons ($Z$, $W^\pm$) and the photon ($\gamma$).
These are all vector fields.
Fermions are described by spinor fields, $\psi$, which obey the Dirac equation:
\begin{equation}
  (i\hbar\gamma^\mu\partial_\mu - mc)\psi = 0.
  \label{th:eq:dirac}
\end{equation}
The fermions of the SM constitute six leptons (electron, electron neutrino, muon, muon neutrino,
tau and tau neutrino) and six quarks (up, down, charm, strange, top and bottom), which are
organized into pairs forming three generations.
For each fermion there is a corresponding antiparticle with the same mass and opposite charge ---
charge being the conserved quantity resulting from the global gauge symmetry (by Noether's
theorem).
There is also a single scalar field in the SM, that of the Higgs boson.
%The only additional particle is the scalar Higgs boson.

The SM Lagrangian can be expressed as a sum of components:
\begin{equation}
  \Lag{SM} = \Lag{QCD} + \Lag{V} + \Lag{\ell} + \Lag{\it q} + \Lag{Higgs} + \Lag{Yuk}.
  \label{eq:th:lag}
\end{equation}
The first term describes interactions of the strong force between colour carrying particles in the
theory of Quantum Chromodynamics (QCD).
Descriptions of weak vector boson self-interactions, and the electroweak behavior of
leptons are encapsulated in the next two terms.
Finally, the remaining terms describe the electroweak behavior of quarks, the Higgs interaction and
Yukawa couplings, respectively.
These latter terms are of fundamental importance as to how flavour changing currents and CPV
occur in the SM, and will be be discussed in detail.
%the SM.
%Each of these shall be discussed in turn.

%%\subsection{Uncertainties in theoretical predictions}
\section{Dealing with QCD}

For the branching fractions discussed in this thesis, theoretical uncertainties from \QCD make
predictions difficult\footnote{
  The following section is based on Refs.~\cite{Pich:1998xt}.
}.
Quantum Chromodynamics describes the interactions of colour charged particles (quarks and
gluons).
%interactions between quarks via the strong force which is mediated by gluons.
The behaviour of \QCD exhibits two peculiarities: confinement, and asymptotic freedom.
Confinement describes the strong force over long distances ($\approx1\fm$)
where, unlike QED, the interaction strength does not lessen with distance.
This means that as a quark is separated from others, there is enough energy in the gluon field to
create a new quark-antiquark pair.
The resulting bound states always have net zero colour charge.
Therefore, free quarks are never observed over macroscopic distances
and are observed as mesons (quark-antiquark bound states), baryons (bound states of
three quarks), or tetra-quarks~\cite{LHCb-PAPER-2014-014}.
%but are seen to behave as free
%particles in deep inelastic scattering.
Asymptotic freedom means that forces between quarks become asymptotically weaker as the energy of
the system increases --- and the distance decreases.

Predictions of $b$-hadron processes involving \QCD can also be made using effective field theory.
Despite the large mass of the \bquark quark with respect to $\Lambda_\mathrm{QCD}\simeq200\mev$,
the system can be treated perturbatively since $\alpha_\mathrm{QCD}(m_\bquark)$ is sufficiently
small.
This is known as a Heavy Quark Effective Theory (HQET).
In contrast to an EFT where the weak fields have been integrated out, in a HQET
it is not possible to remove heavy quark contributions entirely because the \bquark quark
cannot decay without violating flavour number.
This essentially treats a $b$-hadron like a hydrogen atom, where the \bquark quark takes
the place of the nucleus, allowing for a highly simplified theoretical treatment, with corrections
of order $m_\bquark^{-1}$.

Despite the use of HQET, the fact is that hadrons are inherently non-perturbative objects, and so
it is useful to make further assumptions.
An important supposition is that of \emph{factorization} which assumes that the short-distance,
process dependent, \QCD effects are separable from hadronization, the long distance effects.
Hadronization is very difficult to calculate with \QCD, for this reason form-factors are used to
empirically encapsulate the process.
Form-factors must be measured experimentally and are the dominant source of uncertainty in the
calculation of $B$ mesons decaying into hadronic (or semi-hadronic) final states.





%The mass of the \bquark quark is sufficiently high that QCD calculations in $B$ decays can be made
%using peturbation theory.
%Furthermore, initial conditions can modelled using Heavy Quark Effective Theory (HQET), which
%essentially models a bound state of a heavy and light quark like a hydrogen atom, with the \bquark
%taking the role of the nucleus.
%However, this latter approximation breaks down at low energies where the hadron has an energy
%comparable to the mass of the \bquark quark.

%Another important assumption for QCD predictions is that of factorizability.
%A decay is factorizable if one can separate the initial, partonic, state from the hadronization of
%the final state quarks.
%Hadronization is very difficult to model, and therefore empirical models, encoded into
%form-factors, are used.
%It is these form-factors which are the dominant source of theoretical uncertainty.


%These oddities mean that QCD must be dealt with in different ways depending on the energy regime of
%interest.
%For high momentum interations the coupling strength, $\alpha_\mathrm{QCD}$ is small and the system
%can be dealt with using peturbation theory.
%But, for low momentum interactions $\alpha_\mathrm{QCD}$ increases because of the
%\emph{running} of the coupling.
%In the latter regime the system cannot be modelled with peturbation theory because it is not
%infrared safe and rather Lattice QCD must be used.
%Another inadequacy of peturbation theory is that it considers asymptotic states of quarks and
%gluons as free states, where in actuallity the physical states which are observed are hadrons.
%
%Difficulties with calculating QCD interactions leads to the necessity of form factors, which are
%empirical functions with parameters measured experimentally.
%These are
%
%
%\begin{itemize}
  %\item Heavy quark effective field theory simpolify calculations
    %interactions of the heacy wuark are soft cmopared tot the large mass of the $B$, and partonic
    %process is expanded in terms of $\Lambda_{QCD}/m_B$
  %\item QCD factorization setarates partonic process from the hadronisation of the $sq$ pair
  %\item Hadronisation is encoded into hadronic form-factors which is the dominant source of
    %uncertainty for theoretical predictions
%\end{itemize}








%The following chapter will elucidate as to how the flavour sector is the only source of CPV in the
%SM.
The structure of CP and flavour violation emerges as a direct consequence of the Higgs mechanism
breaking the local electroweak symmetry.
The Lagrangian of the scalar Higgs field is:
\begin{align}
  \Lag{Higgs}
  &= \left(D_\mu\Phi\right)^\dagger\left(D^\mu\Phi\right) - V(\Phi) \\
  &= \left(D_\mu\Phi\right)^\dagger\left(D^\mu\Phi\right) - \mu^2\left(\Phi^\dagger\Phi\right) +
  \lambda\left(\Phi^\dagger\Phi\right),
  \label{eq:th:laghiggs1}
\end{align}
where $\mu$ and $\lambda$ are constants, $D_\mu$ is the covariant derivative, and $\Phi$ is the
Higgs doublet, defined by:
\begin{equation}
  \Phi = \frac{1}{\sqrt{2}}
  \begin{pmatrix}
    \phi_1 + i\phi_2 \\
    \phi_3 + i\phi_4 \\
  \end{pmatrix}.
  \label{eq:th:phi}
\end{equation}
Taking $\mu^2<0$ and $\lambda>0$ moves the minimum of the potential $V(\Phi)$ away from zero to a distance $v$:
\begin{equation}
  v = \sqrt{\frac{\mu^2}{\lambda}}.
\end{equation}
At this point the Higgs field gets a vacuum expectation value (VEV)
%$\braket{\phi} = \tfrac{1}{\sqrt{2}}v$.
of $\langle\phi\rangle = \tfrac{1}{\sqrt{2}}v$.
The direction of the VEV from the origin is arbitrary, but the choice of:
\begin{align}
  \bra{0}\phi_1\ket{0} =
  \bra{0}\phi_2\ket{0} =
  \bra{0}\phi_4\ket{0} = 0  &&
  \bra{0}\phi_3\ket{0} = v,
\end{align}
is convenient, and changes Eq.~\ref{eq:th:phi} to:
\begin{equation}
  \Phi = \frac{1}{\sqrt{2}}
  \begin{pmatrix}
    \eta_1 + i\eta_2 \\
    v + i\eta_4 \\
  \end{pmatrix}.
  \label{eq:th:eta}
\end{equation}
Here, $\eta_1$, $\eta_2$ and $\eta_4$, are Goldstone bosons which, by choosing an appropriate
gauge, become the longitudinal components of the weak bosons.
This choice of gauge simplifies $\Phi$ to:
\begin{equation}
  \Phi =
  \begin{pmatrix} 0 \\ v+H
  \end{pmatrix},
  \label{eq:th:phi2}
\end{equation}
where $H$ is the physical Higgs boson.
%Using this in Eq.~\ref{eq:th:laghiggs1} gives:
Inserting Eq.~\ref{eq:th:phi2} into Eq.~\ref{eq:th:laghiggs1} gives:
%\begin{multline}
  %\Lag{Higgs} =
  %\frac12\left(\partial_\mu H\right)\left(\partial^\mu H\right)
  %+\frac14g^2\left(v^2 + 2vH + H^2\right)W_\mu^+W^{-\mu}  \\
  %+\frac18\left(g^2 + g^{\prime2}\right)\left(v^2 + 2vH + H^2\right)Z_\mu Z^\mu
  %+ \mu^2H^2 + \frac14\lambda\left(H^4+4vH^3\right) + \cdots,
  %\label{eq:th:laghiggs2}
%\end{multline}
\begin{equation}
  \Lag{Higgs} =
  \frac12\left(\partial_\mu H\right)\left(\partial^\mu H\right)
  +\mu^2H^2
  +\left(m_W^2W_\mu^+W^{-\mu} + \frac{m_Z^2}{2}Z_\mu Z^\mu\right)
  \cdot
  \left(1 + \frac{H}{v}\right)^2
  \label{eq:th:laghiggs2}
\end{equation}
where $g$ and $g^\prime$ are coupling constants and other terms are three- and four-point
interactions of the Higgs with itself and weak gauge bosons.
Thus, the $U(1)$ local gauge symmetry is broken, weak gauge boson acquire a mass while photons remain
massless; as is consistent with observations.


%It is not possible to directly insert mass terms for fermions, because the required terms
%($m(\bar\psi_L\psi_R+\bar\psi_R\psi_L$) are not allowed \bam{ELABORATE}.
All fermions (except neutrinos) also get a mass after the spontaneous symmetry breaking (SSB) of
the $U(1)$ symmetry.
The Dirac mass term for a chiral field should be of the form:
\begin{equation}
  \Lag{mass} = -m_\psi\left(\bar\psi_R\psi_L + \bar\psi_L\psi_R\right),
\end{equation}
but the left- and right-handed fields have different $U(1)$ charges %Y hypercharge
and so transform differently under local gauge transformations and so cannot be added to \Lag{SM}.
However, masses can be generated through the Yukawa couplings (\Lag{Yuk} in \Eq{eq:th:lag}), which
describe integrations between all fermionic fields and the Higgs doublet, and can be written:
\begin{equation}
  \Lag{Yuk} = \sum_{\substack{\ell=\\e,\mu,\tau}}\left(\Lag{Yuk}^\ell\right) + \Lag{Yuk}^q,
  \label{eq:th:yukking}
\end{equation}
where $\ell$ and $q$ denote the lepton and quark sectors respectively.
First considering the lepton term, after SSB:
\begin{align}
  \Lag{Yuk}^\ell
  &= - g_\ell\left(\bar\chi_L\Phi \ell_R + \bar \ell_R\Phi^\dagger\chi_L\right) \\
  &= - \frac{g_\ell v}{\sqrt{2}}\left(\bar\ell_L\ell_R + \bar\ell_R\ell_L\right)\cdot
  \left(1 + \frac{H}{v}\right)
     %- \frac{g_\ell}{\sqrt{2}}H\left(\bar\ell_L\ell_R + \bar\ell_R\ell_L\right),
\end{align}
where $g_\ell$ is a coupling constant, and
\begin{align}
  \chi_L = \begin{pmatrix}\nu_L \\ \ell_L \end{pmatrix}.
\end{align}
Thus the leptons get a mass of $m_\ell = \tfrac{1}{\sqrt{2}}g_\ell v$, and interact with the Higgs
field.

The story for $\Lag{Yuk}^q$ is a bit more involved.
Before SSB:
\begin{align}
  \Lag{Yuk}^q &= - y_{ij}^u\bar Q_L^i\Phi u_R^j
  - y_{ij}^d\bar Q_L^i\tilde\Phi d_R^j + \mathrm{h.c.},
  \label{eq:th:lagyukq}
\end{align}
where there is an implicit sum over all generations $i$ and $j$, $y^{u,d}$ is a $3\times3$ matrix
characterizing the Yukawa couplings between quark generations,
\begin{align}
  \tilde\Phi_i &= \varepsilon_{ij}\Phi_j,
    \qquad \mathrm{and} \quad Q_L = \begin{pmatrix}u_L \\ d_L \end{pmatrix}.
\end{align}
After SSB, and the Higgs acquires a VEV, \Eq{eq:th:lagyukq} becomes:
\begin{equation}
  \Lag{Yuk}^q =
  - \frac{v}{\sqrt{2}}
  \left(
  y_{ij}^u\bar u_L^iu_{R,j}
  + y_{ij}^d\bar d_L^id_{R,j},
  + \mathrm{h.c.}
  \right)
  \cdot
  \left(1 + \frac{H}{v}\right),
  \label{eq:th:lagyuk2}
\end{equation}
where the mass of the quarks is $m_q = \frac{v}{\sqrt{2}}y_{ij}^q$.
However, it is more convenient to change a basis in which the matrix $m^q$ is diagonal such that
$m_{ij}^\mathrm{diag} = V_{Lik}m_{kl}(V_R^\dagger)_{lj}$.
This is exactly equivalent to transforming the chiral quark fields for up- and down-type quarks
accordingly:
\begin{align}
  q_L^\alpha = \left(V_L^q\right)_{\alpha i}q_L &&
  q_R^\alpha = \left(V_R^q\right)_{\alpha i}q_R,
\end{align}
where the index of the original basis is identified with $i$ and the mass basis uses $\alpha$.

The rotations of the basis of the chiral quark fields leave much of \Lag{SM} unchanged since
$V_{qL}^\dagger V_{qL} = V_{qR}^\dagger V_{qR} = \mathbb{1}$.
However, this is not the case in the charged current (CC) part of \Lag{\it q}; which transforms as:
\begin{align}
  \Lag{\it q}^\mathrm{CC}
  &= \frac{g}{2}i\gamma^\mu
  \left[\bar u_Ld_LW^+_\mu + \bar d_Lu_LW^-_\mu
  \right]  \\
  &= \frac{g}{2}i\gamma^\mu
  \left[
    \bar u_L\left(V_{uL}V_{dL}^\dagger\right)d_LW^+_\mu +
    \bar d_L\left(V_{dL}V_{uL}^\dagger\right)u_LW^-_\mu
  \right]  \\
  &= \frac{g}{2}i\gamma^\mu
  \left[
    V\bar u_Ld_LW^+_\mu +
    V^\dagger\bar d_Lu_LW^-_\mu
  \right].
\end{align}
The matrix $V$ is defined is known as the Cabibbo-Kobayashi-Maskawa (CKM) matrix
and parameterizes the couplings between up- and down-type quarks in charged weak currents.




