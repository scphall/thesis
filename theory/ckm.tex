\section{The CKM matrix and Unitarity Triangle}
\label{sec:ckm}

The CKM matrix is defined as:
\begin{equation}
  \VCKM = \left(\V{uL}V_{dL}^\dagger\right) =
  \begin{pmatrix}
    \V{ud} & \V{us} & \V{ub} \\
    \V{cd} & \V{cs} & \V{cb} \\
    \V{td} & \V{ts} & \V{tb} \\
  \end{pmatrix},
\end{equation}
where each $|V_{ij}|$ parameterizes the probability of an up-type quark, $i$, transitioning to
down-type quark $j$ in a weak interaction.
In the SM, it is assumed that the total charged current couplings of up- to down-type quarks is the
same as down- to up-type.
This means that the CKM matrix is unitary, $V^\dagger V = \mathbb{1}$, and therefore it contains
four physical parameters: three angles ($\theta_{12}$, $\theta_{13}$ and $\theta_{23}$) and one
complex phase ($\delta$).
In fact, the observation of CPV in kaon mixing led to the prediction of a third generation before
its discovery because a $3\times3$ matrix is the smallest necessary for a phase to enter a unitary
matrix.

%preferentially
%hierarchichal
%arbiatry

%The CKM matrix is the source of all flavour violation in the SM.
%In the SM, it is assumed that the total charged current couplings of up- to down-type quarks is the
%same as down- to up-type.
%This means that the CKM matrix is unitary, $V^\dagger V = \mathbb{1}$, and therefore it contains
%four physical parameters: three angles ($\theta_{12}$, $\theta_{13}$ and $\theta_{23}$) and one
%complex phase ($\delta$).

There are many ways of representing the CKM matrix, one way is as a product of three rotation
matrices, one of which contains the complex phase, this is the known as the \emph{standard}
parameterization.
The \emph{Wolfenstein} parameterization is obtained by defining
\begin{align}
  \sin\theta_{12}&=\lambda, \nonumber\\
  \sin\theta_{23}&=A\lambda^2, \nonumber\\
  e^{-i\delta}\sin\theta_{13} &= A\lambda^3(\rho-i\eta),
\end{align}
%\begin{align}
  %\sin\theta_{12}&=\lambda,
  %& \sin\theta_{23}&=A\lambda^2,
  %& \mathrm{and}&
  %& e^{-i\delta}\sin\theta_{13} &= A\lambda^3(\rho-i\eta),
%\end{align}
%Because this matrix can be written as a product of three rotation
%matrices (with angles $\theta_{12}$, $\theta_{13}$ and $\theta_{23}$) and the CP violating phase
%($\delta$), it can be parameterized in terms of $\lambda$, $A$, $\rho$ and $\eta$.
%Where: $\sin\theta_{12}=\lambda$,
%$\sin\theta_{23}=A\lambda^2$ and $e^{-i\delta}\sin\theta_{13} = A\lambda^3(\rho-i\eta)$.
%and Taylor expanding products of the three rotation matrices, it can be shown that:
which results in
\begin{align}
  \VCKM\simeq
    \begin{pmatrix}
      %1-\tfrac12\lambda & \lambda & A\lambda^3(\rho-i\eta+\tfrac{i}2\eta\lambda^2) \\
      %-\lambda & 1-\lambda^2-i\eta A^2\lambda^4 & A\lambda^2(1+i\eta\lambda^2) \\
      %A\lambda^3(1-\rho-i\eta) & -A\lambda^2 & 1 \\
      1-\tfrac12\lambda & \lambda & A\lambda^3\big(\rho-i\eta\big) \\
      -\lambda & 1-\lambda^2 & A\lambda^2 \\
      A\lambda^3\big(1-\rho-i\eta\big) & -A\lambda^2 & 1 \\
    \end{pmatrix}.
  \label{eq:th:wolfenstein}
\end{align}
%where
%\begin{align}
  %\lambda &= 0.22537\pm0.00061, & A =
%\end{align}
Since $A\neq0$ and $\lambda\neq0$, it is clear that \VCKM is not diagonal, and therefore flavour
changing currents are allowed in the SM.
However, the diagonal elements are close to unity and the CKM matrix exhibits a strong
hierarchical structure, such that it is most probable that weak currents do not violate flavour.
%for which there is no explaination in the SM.



The unitarity condition can be expressed as
$V_{\alpha\beta}^{\phantom{\dagger}}V_{\beta\gamma}^* = \delta_{\alpha\gamma}$
%\begin{equation}
  %V_{\alpha\beta}^{\phantom{\dagger}}V_{\beta\gamma}^* = \delta_{\alpha\gamma},
  %\boldsymbol{V}_{ij}\boldsymbol{V}_{jk}^\dagger = \delta_{ik}.
  %\sum_{i=1}^3\left|V_{ij}\right|^2 = 1
  %\sum_{i=1}^3\left(V_{ij}^*V_{jk}\right) = \mathbb{1},
  %\label{eq:th:unitarity}
%\end{equation}
which, when $\alpha\neq\gamma$, gives six equations of the form:
\begin{align}
  %\sum_{i=1}^3V_{ij}^*V_{ki} &= 0 && \sum_{i=1}^3V_{ji}^*V_{ik} =0, & j&\neq k;
  \phantom{\beta\neq\gamma}&&\sum_{\alpha=1}^3V_{\alpha\beta}^*V_{\gamma\alpha}^{\phantom{*}} &= 0
  && \sum_{\alpha=1}^3V_{\beta\alpha}^*V_{\alpha\gamma}^{\phantom{*}}=0, & \beta&\neq\gamma.
  \label{eq:th:offdiag}
\end{align}
These equations map closed triangles on the complex plane.
%Two of these triangles have all sides of similar length
One of these triangles, which has sides of similar length,
$\big(\mathcal{O}(\lambda^3)\big)$,
is known as \emph{the} Unitarity Triangle (UT) and is described by
%Taking the equations of the triangles in Eq.~\ref{eq:th:unitarity} where all sides have length of
%$\mathcal{O}(\lambda^3)$ leaves two triangles, one of which is:
\begin{equation}
  %\V{ud}\Vconj{ub} + \V{cd}\Vconj{cb} + \V{td}\Vconj{tb} = 0.
  1 + \frac{\V{ud}\Vconj{ub}}{\V{cd}\Vconj{cb}} + \frac{\V{td}\Vconj{tb}}{\V{cd}\Vconj{cb}} = 0.
  \label{eq:th:ut}
\end{equation}
The UT is depicted in \Fig{fig:th:ut}, it has a base of unit length, an apex at
\begin{align}
  %\bar\rho+i\bar\eta = (1-\tfrac12\lambda^2)(\rho+i\eta)
  \bar\rho+i\bar\eta &= (1-\tfrac12\lambda^2)(\rho+i\eta) \nonumber\\
  &=\frac{\V{ud}\Vconj{ub}}{\V{cd}\Vconj{cb}}
\end{align}
%If divided through by $\V{cd}\Vconj{cb}$, then Eq.~\ref{eq:th:ut} can be mapped onto the complex
%plane, where the apex is at $\bar\rho+i\bar\eta = (1-\tfrac12\lambda^2)(\rho+i\eta)$.
%and the angles are
and forms the angles
\begin{align}
  \alpha &=    \arg\left(-\frac{\V{td}\Vconj{tb}}{\V{ud}\Vconj{ub}}\right), \nonumber\\
  \beta  &=\pi-\arg\left( \frac{\V{td}\Vconj{tb}}{\V{cd}\Vconj{cb}}\right), \nonumber\\
  %& &\mathrm{and}
  \gamma &=    \arg\left( \frac{\V{ud}\Vconj{ub}}{\V{cd}\Vconj{cb}}\right).
\end{align}
which define phase differences between edges.
%are phases between CKM matrix elements.
%This triangle is depicted in \Fig{fig:th:ut}.
%, is simply a graphical representation of the CKM matrix.
%Measurements of CKM matrix elements constrain the angles, side lengths and apex of the UT, these
%constraints are also shown in Fig.~\ref{fig:th:ut}.

\begin{figure}
  %\subfloat[\label{fig:th:ut:sch}] {
  %\subfloat[\label{fig:th:ut:plane}] {
  \begin{center}
      \includegraphics[scale=1]{diagram_ut}
      %\includegraphics[width=0.48\textwidth]{diagram_ut}
      \includegraphics[width=0.47\textwidth]{rhoeta_small_Vub}
  \end{center}
  \caption[Unitarity triangle]{\small
    Schematic diagram of the Unitarity triangle given in Eq.~\ref{eq:th:ut} on the complex plane,
    where the base has been normalized to unit length.
    Alongside is shown the same triangle with coloured bands indicating various constraints on
    side lengths, angles and position of the apex.
  }
  \label{fig:th:ut}
\end{figure}

Precise determination of the CKM matrix elements is important, for  they are each fundamental
parameters of the SM.
They also contain all the information about flavour violation and CPV that is allowed within the
framework of the SM.
This information is represented by the UT, a current fit of measurement of angles and side lengths
is shown in \Fig{fig:th:ut}.

%\bam{This is to be moved:
%If additional CP-violating phases were to exist beyond the SM, their effect would be seen in global
%fits of the UT.}
%
%\bam{
%Each CKM matrix element is ususally measured in a particular way.
%For example, the elements \Vtxt{td} and \Vtxt{ts} are measured using $B-\Bbar$ oscillations since tree
%level determination using \tquark quarks is difficult to do with precision.
%However, the nature of NP is unknown and could effect different processes in different ways.
%A good example of this is \Vtxt{ub}, which is the CKM element known to the lowest degree of accuracy.
%}







%A determination of \V{ub}
%can be made using inclusive and exculsive measurements of
%\decay{B}{X_u\ell\bar\nu_\ell} decays.
%Inclusive measurements are made difficult from large
%\decay{B}{X_c\ell\bar\nu_\ell} backgrounds, while exclusive semi-leptonic modes suffer from
%uncertainties introduced by form factors.
%A combination of these results leads to a value of
%$\left|\V{ub}\right|_\mathrm{SL}=\left(4.13\pm0.49\right)\e{-3}$ \cite{PDG2012}.
%But, \V{ub} can also be obtained using the tree level decay of
%%A value of $\left|\V{ub}\right|$ can also be obtained from the annihilation decay
%\decay{\Bp}{\taup\nu_\tau}:
%$\left|\V{ub}\right|_{\tau\nu}=\left(4.22\pm0.42\right)\e{-3}$ \cite{PDG2012}.
%The branching fraction which is used to calculate this latter value~\cite{Amhis:2012bh} is
%somewhatStrong CP problem...
%higher than the SM prediction and particularly sensitive to NP models which include a charged
%Higgs.
%Figure \ref{fig:th:ut} shows the length of the side $|\V{ud}\Vconj{ub}|/|\V{cd}\Vconj{cb}|$
%calculated by both values of \V{ub}.
%The decay \btodsphi has a very similar topology to \decay{\Bp}{\taup\nu_\tau} in that it proceeds
%via the annihilation of the \Bp meson constituent quarks and the resulting $W^+$ decays into quarks
%that form the final state.
%This decay is discussed in \Sec{sec:dsphi}.

%The other side is sensitive to the value of \V{tb} and \V{td}.
%The value of \V{tb} is determined from decays of \tquark quarks using the ratio
%$\BF(\decay{t}{Wb})/\BF(\decay{t}{Wq})$, where $q=d,b,s$.
%Oscillation frequencies of \Bd and \Bs mesons ($\Delta m_d$ and $\Delta m_s$ respectively) are used
%to measure the \V{td} and \V{ts}.



