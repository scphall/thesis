%\subsection{Uncertainties in theoretical predictions}
\section{Dealing with QCD}

For the branching fraction measurements discussed in this thesis, theoretical uncertainties from \QCD make
predictions difficult\footnote{
  The following section is based on Ref.~\cite{Pich:1998xt}.
}.
\QCD describes the interactions of colour charged particles (quarks and
gluons),
and exhibits two peculiarities: confinement, and asymptotic freedom.
%interactions between quarks via the strong force which is mediated by gluons.
Confinement means that over long distances (\approx$1\fm$)
the interaction strength of the strong force does not weaken --- unlike all other known
forces.
This means that as a quark is separated from others, there is enough energy in the gluon field to
create new quark-antiquark pairs, where the resulting bound states always have net zero colour
charge.
Free quarks cannot be seen over macroscopic distances,
and are instead observed as mesons, baryons, tetra-quarks~\cite{LHCb-PAPER-2014-014} or
even penta-quarks~\cite{LHCb-PAPER-2015-029}.
%but are seen to behave as free
%particles in deep inelastic scattering.
Asymptotic freedom means that forces between quarks become asymptotically weaker as the energy of
the system increases, and the distance decreases.

Predictions of $b$-hadron processes involving \QCD can also be made using an \EFT.
Despite the large mass of the \bquark quark with respect to $\Lambda_\mathrm{QCD}\simeq200\mev$,
the system can be treated perturbatively since $\alpha_\mathrm{QCD}(m_\bquark)$ is sufficiently
small.
This is known as a \HQET.
In contrast to an \EFT where the weak fields have been integrated out, in a \HQET
it is not possible to remove heavy quark contributions entirely because the \bquark quark
cannot decay without violating flavour number.
Essentially the $b$-hadron system is treated akin to a hydrogen atom, where the \bquark quark takes
the place of the nucleus, allowing for a highly simplified theoretical treatment, with corrections
of order $m_\bquark^{-1}$.

Despite the use of \gls{HQET}, the fact is that hadrons are inherently non-perturbative objects,
and so it is useful to make further assumptions.
An important supposition is that of \emph{factorisation}, which assumes that the short-distance,
process dependent, \QCD effects are separable from hadronization, the long distance effects.
Hadronization is very difficult to calculate with \QCD; for this reason \emph{form factors} are
used to empirically encapsulate the process.
Form factors must be measured experimentally and are the dominant source of uncertainty in hadronic
$B$ decays.
%calculation of $B$ mesons decaying into final states containing hadrons.

%The mass of the \bquark quark is sufficiently high that QCD calculations in $B$ decays can be made
%using peturbation theory.
%Furthermore, initial conditions can modelled using Heavy Quark Effective Theory (HQET), which
%essentially models a bound state of a heavy and light quark like a hydrogen atom, with the \bquark
%taking the role of the nucleus.
%However, this latter approximation breaks down at low energies where the hadron has an energy
%comparable to the mass of the \bquark quark.

%Another important assumption for QCD predictions is that of factorizability.
%A decay is factorizable if one can separate the initial, partonic, state from the hadronization of
%the final state quarks.
%Hadronization is very difficult to model, and therefore empirical models, encoded into
%form-factors, are used.
%It is these form-factors which are the dominant source of theoretical uncertainty.


%These oddities mean that QCD must be dealt with in different ways depending on the energy regime of
%interest.
%For high momentum interations the coupling strength, $\alpha_\mathrm{QCD}$ is small and the system
%can be dealt with using peturbation theory.
%But, for low momentum interactions $\alpha_\mathrm{QCD}$ increases because of the
%\emph{running} of the coupling.
%In the latter regime the system cannot be modelled with peturbation theory because it is not
%infrared safe and rather Lattice QCD must be used.
%Another inadequacy of peturbation theory is that it considers asymptotic states of quarks and
%gluons as free states, where in actuallity the physical states which are observed are hadrons.
%
%Difficulties with calculating QCD interactions leads to the necessity of form factors, which are
%empirical functions with parameters measured experimentally.
%These are
%
%
%\begin{itemize}
  %\item Heavy quark effective field theory simpolify calculations
    %interactions of the heacy wuark are soft cmopared tot the large mass of the $B$, and partonic
    %process is expanded in terms of $\Lambda_{QCD}/m_B$
  %\item QCD factorization setarates partonic process from the hadronisation of the $sq$ pair
  %\item Hadronisation is encoded into hadronic form-factors which is the dominant source of
    %uncertainty for theoretical predictions
%\end{itemize}






