\subsection{Uncertainties in theoretical predictions}


For the branching bractions discussed in this thesis, theoretical uncertaities from QCD make
predictions difficult.
Quantum Chromodynamics (QCD) describes the interactions of colour charged particles; specifically
interactions between quarks via the strong force which is mediated by gluons.
The behavior of QCD is exhibits two peculararites: confinemanet and asymptotic freedom.
Confinement describes the behavior of the strong force over long distances ($\sim1\fm$)
where, unlike QED, the interaction strength does not lessen with distance.
This means that as a quark is separated from others, there is enough energy in the gluon field to
create a new quark antiquark pair and the resulting bound states have net zero colour charge.
Therefore free quarks are never observed over mactoscopic distances, but are seen to behave as free
particles in deep inelastic scattering.
Asymptotic freedom means that forces betweeen quarks become asymptoticly weaker as the energy of
the system increases --- and the distace decreases.


The mass of the \bquark quark is sufficiently high that QCD calculations in $B$ decays can be made
using peturbation theory.
Furthermore, initial conditions can modelled using heavy quark effective field theory, which
essentially models a bound state of a heavy and light quark as an atom.
However, this latter approximation breaks down at low energies where the hadron has an energy
comparable to the mass of the \bquark quark.

Another important assumption for QCD predictions is that of factorizability.
A decay is factorizable if one can separate the initial, partonic, state from the hadronization of
the final state quarks.
Hadronization is very difficult to model, and therefore empirical models, encoded into
form-factors, are used.
It is these form-factors which are the dominant source of theoretical uncertainty.


%These oddities mean that QCD must be dealt with in different ways depending on the energy regime of
%interest.
%For high momentum interations the coupling strength, $\alpha_\mathrm{QCD}$ is small and the system
%can be dealt with using peturbation theory.
%But, for low momentum interactions $\alpha_\mathrm{QCD}$ increases because of the
%\emph{running} of the coupling.
%In the latter regime the system cannot be modelled with peturbation theory because it is not
%infrared safe and rather Lattice QCD must be used.
%Another inadequacy of peturbation theory is that it considers asymptotic states of quarks and
%gluons as free states, where in actuallity the physical states which are observed are hadrons.
%
%Difficulties with calculating QCD interactions leads to the necessity of form factors, which are
%empirical functions with parameters measured experimentally.
%These are
%
%
%\begin{itemize}
  %\item Heavy quark effective field theory simpolify calculations
    %interactions of the heacy wuark are soft cmopared tot the large mass of the $B$, and partonic
    %process is expanded in terms of $\Lambda_{QCD}/m_B$
  %\item QCD factorization setarates partonic process from the hadronisation of the $sq$ pair
  %\item Hadronisation is encoded into hadronic form-factors which is the dominant source of
    %uncertainty for theoretical predictions
%\end{itemize}






