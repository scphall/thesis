\section{The Flavour Problem and Minimal Flavour Violation}

The most general way to parameterize NP is with an effective Lagrangian describing generic
interactions at an energy scale $\mu$, in which long and shory distance effects are separated.
Long distance (equivalently low energy) effects are described by coefficients, $c$, which can be
calculated using perturbative methods.
Short distance (high energy) effects are parameterized in terms of operators, $\mathcal{O}$,
which must be calculated non-peturbativly because they contain QCD interactions.
The resulting effective Lagrangian includes a sum over all processes, $i$, which contribute at a
given dimension, $d$:
\begin{equation}
  \Lag{eff}
  =
  \Lag{SM} + \sum_d\frac1{\Lambda^{d-4}}
  + \sum_ic_i^{(d)}\mathcal{O}_i^{(d)}.
\end{equation}
Processes that cause FCNCs contribute in $d=6$, and can be wiritten as
\begin{equation}
  \Delta\mathcal{L}^\mathrm{FCNC}
  =
  \sum_{i\neq j}\frac{c_{ij}}{\Lambda^2}
  \left(\Xbar{\mathcal{O}}_{Li}\gamma^\mu\mathcal{O}_{Lj}\right)^2,
\end{equation}
where $c_{ij}$ are dimensionless FCNC couplings, where $i$ and $j$ are different quark generations.
Bounds set on the energy scale $\Lambda$ by the analysis in \Ref{Isidori:2010kg} are:
\begin{equation}
  \Lambda > \frac{|c_{ij}|^\frac12}{|\Vconj{ti}\V{tj}|}\times4.4\tev
  \sim
  \left\{
    \begin{array}{l}
      %1.3\e{4}\tev\times |c_{sd}|^\frac12 \\
      %5.1\e{2}\tev\times |c_{bd}|^\frac12 \\
      %1.1\e{2}\tev\times |c_{bs}|^\frac12 \\
      |c_{sd}|^\frac12\times 1.3\e{4}\tev \\
      |c_{bd}|^\frac12\times 5.1\e{2}\tev \\
      |c_{bs}|^\frac12\times 1.1\e{2}\tev \\
    \end{array}\right.
\end{equation}
These values are calculated under the assumption that NP has a natural flavour structure, where
$c_{ij}\approx\mathcal{O}(1)$.
So, either couplings are of order unity and NP begins to contribute at over $100\tev$; or couplings
strengths are $\mathrm{O}(10^{-5})$ and NP contributes at the $1\tev$ level;
it is expected for NP to appear at the $1\tev$ scale in order to solve the hierarcy problem.
Either way, there is a conflict between the most natural coupling and energy scale; this is known
as the flavour problem.
%So must conclude that NP has a highly non-generic flacour structure.

A solution to the flavour problem is the Minimal Flavour Violation (MFV) hypoethesis.
This is the, somewhat pessimistic, view that the flavour structure of NP mirrors that of the Yukawa
couplings; leading to NP flavour changing transitions to be hidden by those seen in the SM.



