\section{The flavour problem}

Historically, indirect evidence for \np has pointed theorists towards predicting the structure of
the \sm before observations could be made directly.
For example, \CPV observed in the kaon sector in 1964 led to the prediction of a third generation of
quarks by Kobayashi and Maskawa in 1973.
The \bquark quark was not discovered until 1977.


%\begin{itemize}
  %\item Access to higher energy scales in virtual particles in loops
  %\item Direct searches, see particles
  %\item high mass needs energy, small couplings need lumi
%\end{itemize}





Additional \np particles introduce terms defining their mass, and the way in which they couple
to other particles, must be added to the Lagrangian.
These additional free parameters are entirely unknown, and by precise measurements the physics
community places bounds on \np particle masses and parameters.
However, it is impossible to do this without making assumptions, albeit reasonable ones.
Typically, natural models are favoured, with generic couplings of order unity and the
scale at which \np enters is $\mathcal{O}(1\tev)$.

Just as in \Eq{eq:th:opehamnorm}, one can write an effective Lagrangian for $\Delta F=2$ down-type
\fcncs containing \np:
\begin{equation}
  \Delta\mathcal{L}^{\Delta F=2} =
  \sum_{i\neq j}\frac{c_{ij}}{\Lambda^2}
  \big(\Xbar{Q}_{Li}\gamma^\mu Q_{Lj}\big)^2.
\end{equation}
Constraints on the coupling constants for $c_{sd}$, $c_{bd}$, and $c_{bs}$ are taken from
values of $\Kz-\Kzb$, $\Bd-\Bdb$, and $\Bs-\Bsb$ mixing measurements, respectively.
An analysis in \Ref{Isidori:2010kg} gives constraints on the energy scale from these couplings:
\begin{equation}
  \Lambda > \frac{|c_{ij}|^\frac12}{|\Vconj{ti}\V{tj}|}\times4.4\tev
  \sim
  \left\{
    \begin{array}{l}
      |c_{sd}|^\frac12\times 1.3\e{4}\tev \\
      |c_{bd}|^\frac12\times 5.1\e{2}\tev \\
      |c_{bs}|^\frac12\times 1.1\e{2}\tev \\
    \end{array}\right..
\end{equation}
This introduces a conflict in the quest for naturalness, between generic couplings and \np entering
at the TeV scale.
%These values are calculated under the assumption that NP has a natural flavour structure, where
%$c_{ij}\approx\mathcal{O}(1)$.
So, either couplings are of order unity and NP begins to contribute at over $100\tev$; or couplings
strengths are $\mathrm{O}(10^{-5})$ and NP contributes at the $1\tev$ level;
it is expected for NP to appear at the $1\tev$ scale in order to solve the hierarchy problem.
%Either way, there is a conflict between the most natural coupling and energy scale; this is known
%as the flavour problem.
This contradiction is known as the \emph{flavour problem}.

A solution to the flavour problem is the Minimal Flavour Violation (MFV) hypothesis.
This is the, somewhat pessimistic, view that the flavour structure of NP mirrors that of the Yukawa
couplings; leading to NP flavour changing transitions to be hidden by those seen in the SM.


The contradiction of the flavour problem introduces a decision as to how to search for beyond the
\sm physics.
If \np particles were to be of high mass one would search at the \emph{energy frontier},
however if the coupling strengths of \np are weak then one would search at the
\emph{luminosity frontier}.
The \lhcb detector offers an excellent environment to search at both or these frontiers.
Firstly, the \lhc supplies \lhcb with (upto) $14\tev$, although it does not have full $4\pi$
coverage.
Secondly, the luminosity frontier is, arguably, accessible because the precision tracking and \pid
capabilities of the \lhcb detector.
This means that \lhcb can observe decays such as \decay{\Bs}{\mumu} which has a branching fraction
of $\big(2.8\,^{+0.7}_{-0.6}\e{-9}\big)$~\cite{LHCb-PAPER-2014-049}.






%The \lhcb detector is ideally placed to search for \np at the
%luminosity frontier, because although the lumin
%the luminosity is not as high as the $B$-factories


%\begin{itemize}
  %\item Conflict of scales
  %\item MFV
  %\item Energy and luminosity frontiers
  %\item Scales from K mixing and B mixing
%\end{itemize}

%So must conclude that NP has a highly non-generic flacour structure.
