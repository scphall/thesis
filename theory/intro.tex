
\begin{minipage}{\textwidth}
  {\it ``Before beginning a Hunt, it is wise to ask someone what you are looking for before you
  begin looking for it.''}

  {\hfill Winnie the Pooh}
\end{minipage}

\vspace{2em}

This thesis contains the work undertaken in three analyses; each of which concerns a different area
of interest in high energy physics.
The following chapter aims to motivate each analysis in turn after introducing the Standard Model
of particle physics.

Firstly, the formulation of the \sm will be outlined, with particular detail paid to the flavour
sector.
Various successes of the \sm will then be discussed before going on to identify its shortcomings
using arguments from both experiment and theory.
These shortcomings will then be used to motivate the three analyses:
a search for \btodsphi, which contains \V{ub} (\Chap{ch:dsphi});
a search for \btokpipimumu and \btophikmumu, which are flavour changing neutral currents
(\Chap{ch:hhh});
and seaching for dark sector particles in \btokstmumu (\Chap{ch:db}).
Theory relating specifically to each of these decays will be detailed in the relevant chapter.

%The convension of natural units, $c=\hbar=1$, is assumed throughout.
%Indices for four vectors are labelled by $\mu$ and $\nu$, and indices for the $SU_C(3)$ and
%$SU_L(2)$ generators are

%The following chapter will elucidate as to how the flavour sector is the only source of \CPV in the
%SM.
%It will then go on to motivate study of flavour physics at \lhcb by exploring some problems with
%the SM as a theory.
