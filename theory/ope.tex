\section{Operator Product Expansion}

When describing macroscopic physical systems, it is frequently necessary to simplify the situation
by making assumptions about the distance scales involved.
One would not, for example, dream of using quantum mechanics to model a ball colliding with a wall
despite the treatment being far more proper than a Newtonian approach.
An \EFT works in an equivalent way in particle physics
by decoupling the short- and long-range interactions and treating them separately.
Contributions from particles with very high mass, much greater than some pre-defined energy scale,
are suppressed.
%An equally valid interpretation is that massive, unstable particles have short lifetimes and can
%only interact over very small distances.
These simplifications are advantageous as they allow processes to be modelled at a scale
relevant to the particles involved without complications from other scales.
%A canonical example of this is Fermi's effective theory of weak decays which predates electroweak
%theory; in it, a process (such as $\beta$-decay) is collapsed into a four point interaction.

Creating an effective field theory for particle physics begins by defining an energy scale,
$\Lambda$, which separates the long and short range interactions.
For the case of a process involving a decaying \bquark quark, with initial state $\ket{i}$ and
final state $\ket{f}$,
the energies are of order $m_\bquark$.
In the full treatment of the \sm, contributions from the \tquark quark and weak bosons --- which all
have masses $\mathcal{O}(100\gev)$ --- must be accounted for.
Therefore, an appropriate choice for $\Lambda$ is \approx$m_W$.
Heavy fields above $\Lambda$ are then integrated out and are parameterised by complex numbers,
known as Wilson coefficients, $C_i$.
The remaining physics is encapsulated in the long distances operators, $\mathcal{O}_i$, each having
its own gauge group defining a particular type of process.
Transition matrix elements for the interaction in the effective Hamiltonian are
\begin{equation}
  \bra{f}\Ham{eff}\ket{i} =
  \sum_j C_j(\Lambda)\bra{f}\Op{j}^{(d)}\ket{i}\Big|_{\Lambda},
  \label{eq:th:opeham}
\end{equation}
which is simply a weighted sum over the long distance matrix elements $\bra{f}\Op{j}\ket{i}$ which
operates in dimension $d$.

The sum in \Eq{q:th:opeham} runs over an infinite number of operators --- this is clearly
impractical, and matters are simplified by
extracting factors of $\Lambda^{-1}$ from the Wilson coefficients
making each one.
This modifies the effective Hamiltonian to be a sum over dimensions:
\begin{equation}
  \bra{f}\Ham{eff}\ket{i} =
  \sum_{d}\frac{1}{\Lambda^{d}}
  \sum_j c_j^{(d)}\bra{f}\Op{j}^{(d)}\ket{i}\Big|_{\Lambda}.
  \label{eq:th:opehamnorm}
\end{equation}
This is entirely general, and one can calculate the Wilson coefficients to a high degree of
accuracy --- using perturbative methods --- in the \sm and many \bsm extensions.
Calculating the long distance operators is more challenging, however, because they contain \QCD
processes.

This approach leads to an effective Hamiltonian, with Wilson coefficients which are entirely
independent of the underlying physical processes and can be calculated to a good degree of accuracy
in a range of physics models.
In this way very different observables can be used to make measurements of Wilson coefficients and
compared independently of the actual process.
%In this way, measurements from different decays all have access to multiple processes and therefore
%different Wilson coefficients, so they can be compared.
Measurements can be used for predictions of processes, and to check the validity of \np models,
enabling experiments to favour or disfavour entire classes of physics \bsm.

