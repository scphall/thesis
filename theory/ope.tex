\subsection{Operator Product Expansion}

%The most pessimistic solution to the flavour problem is Minimal Flavour Violation (MFV)
%which simply assumes that beyond SM physics follows a Yukawa coupling like structure in the flavour
%sector, this would lead to no discernible new physics in the flavour sector.


When describing macroscopic physical systems, it is freqyentkly necessary to simplilfy the situation by makeing
certaian assumptions about the distance scales involoved.
Effective field theories do the same thing in particle physics:
By decoupling the short range from the long range interactions they can be treated separately.
Contributions from particles with very high mass, much greater than $\mu$ are suppressed.
An equally valid interpretation is that massive, unstable particles have short lifetimes and can
only interact over very small distances.
These simplifications are greatly advantageous, allowing processes to be modelled at a scale
relecant to the paricles invloved.
A canonical example of this is Fermi's effective theory of weak decays which predates electroweak
theory; in it, a process (such as $\beta$-decay) is collapsed into a four point interaction.

Creating an effective field theory for particle physics begins by defining an energy scale,
$\Lambda$, above which the long and short range interactions are separated.
Fior the case of a \bquark hadron decay, with initial state $\bra{i}$ and final state $\ket{f}$,
the energies are of order $m_\bquark$.
In the fill treatment of the SN, contributions from the \tquark quark and weak bosons --- which all
have masses $\mathcal{O}(100\gev)$ --- must be accounted for.
Therefore, an approproate choice for the $\Lambda$ is $\sim m_W$.
Heavy fields above $\Lambda$ are then integrated out and are parameterized by Wilson coefficients,
$C$, which are complex numbers.
The remaining physics is encapsulated in the long distances operators $\mathcal{O}$, each having
its own gauge group defining a particular type of process.
Transitionn matrix elements for the interac iotn in the effective Hamiltonian are
\begin{equation}
  \bra{f}\Ham{eff}\ket{i} =
  \left.
  \sum_j C_j(\Lambda)\bra{f}\Op{j}^{(d)}\ket{i}\right|_{\Lambda}.
\end{equation}
Which is simply a weighted sum over the long distance matrix elements $\bra{f}\Op{j}\ket{i}$.

The above sum runs over an infinite number of operators, which is clearly impractical.
To simplify matters one can extract factors of $\Lambda^{-1}$ from the Wilson coefficients which
makes the coefficients dimensionless.
This modifies the effective Hamiltonian to be a sum over dimensions, $d$:
\begin{equation}
  \bra{f}\Ham{eff}\ket{i} =
  \left.\sum_{d}\frac{1}{\Lambda^{d}}
  \sum_i c_j^{(d)}\bra{f}\Op{j}^{(d)}\ket{i}\right|_{\Lambda}.
\end{equation}
This is entirely general, and one can calculate the Wilson coefficients to a high degree of
accuracy --- using perturbative methods --- in the SM and many BSM extensions.
Calculating the long distance operators is more challenging, however, because they contain QCD
processes.

This approach leads to an effective Hamiltonian, with Wilson coefficients which are entirely
independent of the underlying physical processes and can be calculated to a good degree of accuracy
in a range of physics models.
In this way, measurements from different decays all have access to multiple processes, and can be
compared.
Resulting measurements can then be used to favour or disfavour entire classes of NP.



%Effective field theories are based on the premise that they processes are defined by the typical
%energy scale, $\mu$, at which they operate\footnote{Section taken from \cite{Mannel:2004ce}}.
%Contributions from particles with very high mass, much greater than $\mu$ are suppressed.
%An equally valid interpretation is that massive, unstable particles have short lifetimes and can
%only interact over very small distances.
%Processes operating at different energy scales are separated both spatially and temporally; and are
%therefore decoupled.

%Effective field theories allow processes to be modelled at a scale relevant to the particles
%involved.
%This is often applied to weak interactions of the \bquark quark.
%Instead of calculating \bquark decays using the full range of processes outlined by the SM, the
%particles who contribute at scales above $m_\bquark$ are integrated out, above a scale $\Lambda$,
%leaving the physics that is most relevant at the scale $\mu\simeq m_\bquark$.
%Here, processes are dominated by QCD, but higher scale effects still contribute.
%A canonical example of this is Fermi's effective theory of weak decays which predates electroweak
%theory; in it, a process (such as $\beta$-decay) is collapsed into a four point interaction.


%The most general way to parameterize the full effective theory is with an effective Hamiltonian
%describing generic interactions at an energy scale $\mu$.
%Long distance effects are encapsulated in operators $\mathcal{O}$, each defining a particular gauge
%structure.
%The high energy (short distance) effects that have been integrated out are simply described by
%Wilson coefficients $c$, which are complex numbers.
%The resulting effective Hamiltonian includes a sum over all processes, $i$, which contribute at a
%given dimension, $d$:
%\begin{equation}
  %\bra{f}\Ham{full}\ket{i} =
  %\left.\sum_{d}\frac{1}{\Lambda^{d-4}}
  %\sum_i c_i^{(d)}\bra{f}\Op{i}^{(d)}\ket{i}\right|_{\Lambda}.
%\end{equation}
%This is entirely general, and one can calculate the Wilson coefficients to a high degree of
%accuracy --- using perturbative methods --- in the SM and many BSM extensions.
%Long distance (equivalently low energy) effects are described by coefficients, $c$, which can be
%Short distance (or high energy) effects are characterized by terms of operators, $\mathcal{O}$,
%which must be calculated non-peturbatively because they contain QCD interactions.
%Values of Wilson coefficients are independent to the precise mediating process,
%and are sensitive to the underlying physics model.
%This means that Wilson coefficients can be compared between different measurements, and used to
%exclude or favour entire classes of BSM physics.



\subsection{OPE in  dsphi and kpipimumu}


The effective Hamiltonian for the FCNC $\decay{b}{s\ell^+\ell^-}$ is given by:
\begin{equation}
  \Ham{eff} = -4\frac{G_F}{\sqrt{2}}\Vconj{ts}\V{tb}\frac{e^2}{16\pi^2}
  \sum_{i}\big(c_i(\mu)\mathcal{O}_i(\mu)+c_i(\mu)^\prime\mathcal{O}_i^\prime(\mu)\big)
\end{equation}
similar to \Eq{eq:th:lageff}, where the coefficients are known as Wilson coefficients, which
correspond to the Wilson operators $\mathcal{O}_{1-10}$.
The operators $\mathcal{O}_{1-6}$ are sensitive to long distance contributions such as \ccbar
loops.
Operators which are particularly sensitive to NP contributions in \decay{b}{s\mumu} transitions are
\begin{align}
  \Op{7\pz} &= \frac{m_b}{e}\big(\bar s \sigma_{\mu\nu}P_Rb\big)F^{\mu\nu}
  %$\bra{f}\Op}\ket{i}$.
  &\Op{7\pz}^\prime &= \frac{m_b}{e}\big(\bar s \sigma_{\mu\nu}P_Lb\big)F^{\mu\nu}
  \nonumber\\
  %\mathcal{O}_8 &= g\frac{m_b}{e^2}\big(\bar s \sigma_{\mu\nu}T^aP_Rb\big)G^{\mu\nu a}
  %&\mathcal{O}_8^\prime &= g\frac{m_b}{e^2}\big(\bar s \sigma_{\mu\nu}T^aP_Rb\big)G^{\mu\nu a}
  %\\
  \Op{9\pz} &= \big(\bar s\gamma_\mu P_Lb\big)\big(\bar\ell\gamma^\mu\ell\big)
  &\Op{9\pz}^\prime &= \big(\bar s\gamma_\mu P_Rb\big)\big(\bar\ell\gamma^\mu\ell\big)
  \nonumber\\
  \Op{10} &= \big(\bar s\gamma_\mu P_Lb\big)\big(\bar\ell\gamma^\mu\gamma_5\ell\big)
  &\Op{10}^\prime &= \big(\bar s\gamma_\mu P_Rb\big)\big(\bar\ell\gamma^\mu\gamma_5\ell\big)
  \phantom{\frac{1}{1}}
  %\\
  %\mathcal{O}_{S} &= \frac{m_b}{m_{B_s}}\big(\bar s\gamma_\mu P_Rb\big)\big(\bar\ell\ell\big)
  %\\
  %\mathcal{O}_{P} &= \frac{m_b}{m_{B_s}}\big(\bar s\gamma_\mu P_Rb\big)\big(\bar\ell\gamma_5\ell\big)
\end{align}
where $P_{L,R}$ are the left and right projection operators.
The operators \Op{7} and \Op{9} describe the emission of a photon or $Z$ from a penguin loop,
and \Op{10} corresponds to a box type diagram with \Wp; these are shown in \Fig{fig:hhh:loops}.
Primed operators are the suppressed helicity, whose contributions are vanishingly small in the SM.




\begin{figure}
  \begin{center}
    \includegraphics[scale=1]{feynman_btosmumu_penguin}
    \includegraphics[scale=1]{feynman_btosmumu_box}
    \caption[Schematic Feynman diagrams for loop and box diagrams]
    {\small
      Schematic Feynman diagrams for the
      (left) penguin loop diagrams corresponding to the operators \Op{7} and \Op{9} depending on
      whether a $\gamma$ or $Z$ is emitted from the loop;
      (right) \Op{10} box diagram mediated by \Wp bosons.
    }
    \label{fig:hhh:loops}
  \end{center}
\end{figure}

As well as in loops, virtual particles can also contribute in some tree level diagrams.
The small value of $|\V{ub}|$ means that annihilation decays of \Bp mesons are heavily
suppressed in the SM.
%They are mediated by
%Tree level diagrams in the SM are typically high statistics modes, however annihilation type decays
%are heavily suppressed.
These rare modes are propagated by a \Wp in the SM; but this could be exchanged for any charged
boson, such as an $H^+$ from SUSY, this could alter the branching fraction or
introduce significant CPV.
%The decay \btodsphi is an annihilation decay of the \Bp.




















%New physics models must be able to accommodate Dark Matter.
%Some models have a \emph{dark} or \emph{hidden} sector which, apart from gravity, only
%communicates with the visible sector feebly via messenger particles.
%These messenger particles could potentially be observed after they decay into SM particles after
%mixing with a $H$ or $Z$.
%The axion could be such a particle, as could the inflaton or dark $Z$; models including these
%particles are further detailed in \Sec{sec:db:intro}.
%
%%Dark Matter is the lightest supersymmetric particle which is stable and messenger particle is super
%%goldstino
%Supersymmetry (SUSY) is a theory which introduces an additional super-particle for each SM fermion and
%gauge boson, whose spin is different by a half integer.
%The Higgs sector in SUSY comprises four Higgs doublets; two are spin-0 and two are spin-$\tfrac12$,
%and then there are two each for $Y=\pm\tfrac12$.
%After SUSY is broken there are five Higgs physical scalar particles, two are CP-even ($h^0$,
%$H^0$) one is CP-odd scalar ($A^0$) and two charged are charged ($H^\pm$).
%Supersymmetry supplies a Dark Matter candidate in the shape of the lightest supersymmetric
%particle, which could communicate with the visible sector via a super-golstino.
%It also immediately solves the hierarchy problem.
%The masses of the super-particles are unconstrained, and could be anywhere between a few TeV and
%the Planck scale.
%%The theory of SUSY immediately solves the hierarcy problem and provide a candidate for Dark Matter
%%in the shape of the lightest supersymmetric particle.
%%Super-goldstino
%%The messenger particles are super goldtinos that arise from the breaking of the symmetry,
%%However, the scale of the masses of the new particles are undefined, meaning
%%they could appear anywhere from a few TeV up to the Planck scale.





%Looking for NP in precision measurements from tree and loop diagrams is often referred to as
%\emph{indirect} seraches, since the source of NP is inferred.
%The other option is to search for \emph{direct} evidence of NP, such as in the invariant mass spectrum of
%a combination of particles.




\bam{The analysis...}







%\begin{figure}
  %\begin{center}
    %\includegraphics[scale=1]{feynman_btosmumu_penguin}
    %\includegraphics[scale=1]{feynman_btosmumu_box}
    %\caption[Schematic Feynman diagrams for loop and box diagrams]
    %{\small
      %Schematic Feynman diagrams for the
      %(left) penguin loop diagrams corresponding to the operators \Op{7} and \Op{9} depending on
      %whether a $\gamma$ or $Z$ is emitted from the loop;
      %(right) \Op{10} box diagram mediated by \Wp bosons.
    %}
    %\label{fig:hhh:loops}
  %\end{center}
%\end{figure}

%As well as in loops, virtual particles can also contribute in some tree level diagrams.
%The small value of $|\V{ub}|$ means that annihilation decays of \Bp mesons are heavily
%suppressed in the SM.
%%They are mediated by
%%Tree level diagrams in the SM are typically high statistics modes, however annihilation type decays
%%are heavily suppressed.
%These rare modes are propagated by a \Wp in the SM; but this could be exchanged for any charged
%boson, such as an $H^+$ from SUSY, this could alter the branching fraction or
%introduce significant CPV.
%%The decay \btodsphi is an annihilation decay of the \Bp.

%New physics models must be able to accommodate Dark Matter.
%Some models have a \emph{dark} or \emph{hidden} sector which, apart from gravity, only
%communicates with the visible sector feebly via messenger particles.
%These messenger particles could potentially be observed after they decay into SM particles after
%mixing with a $H$ or $Z$.
%The axion could be such a particle, as could the inflaton or dark $Z$; models including these
%particles are further detailed in \Sec{sec:db:intro}.
%
%%Dark Matter is the lightest supersymmetric particle which is stable and messenger particle is super
%%goldstino
%Supersymmetry (SUSY) is a theory which introduces an additional super-particle for each SM fermion and
%gauge boson, whose spin is different by a half integer.
%The Higgs sector in SUSY comprises four Higgs doublets; two are spin-0 and two are spin-$\tfrac12$,
%and then there are two each for $Y=\pm\tfrac12$.
%After SUSY is broken there are five Higgs physical scalar particles, two are CP-even ($h^0$,
%$H^0$) one is CP-odd scalar ($A^0$) and two charged are charged ($H^\pm$).
%Supersymmetry supplies a Dark Matter candidate in the shape of the lightest supersymmetric
%particle, which could communicate with the visible sector via a super-golstino.
%It also immediately solves the hierarchy problem.
%The masses of the super-particles are unconstrained, and could be anywhere between a few TeV and
%the Planck scale.
%%The theory of SUSY immediately solves the hierarcy problem and provide a candidate for Dark Matter
%%in the shape of the lightest supersymmetric particle.
%%Super-goldstino
%%The messenger particles are super goldtinos that arise from the breaking of the symmetry,
%%However, the scale of the masses of the new particles are undefined, meaning
%%they could appear anywhere from a few TeV up to the Planck scale.
%
%
%The most pessimistic solution to the flavour problem is Minimal Flavour Violation (MFV)
%which simply assumes that beyond SM physics follows a Yukawa coupling like structure in the flavour
%sector, this would lead to no discernible new physics in the flavour sector.
%Assuming that nature has not chosen MFV, then contradictions from flavour problem indicate that NP
%searches should be made for both: particles
%with high mass, and particles which have small coupling strengths.
%The \lhcb experiment can probe the mass scale, since precision measurements of tree and loop
%diagrams are sensitive to virtual particles contributing at all orders, whose on-shell mass could
%be many TeV.
%The relatively high luminosity of interactions supplied by the \lhc mean that \lhcb is also
%sensitive to low coupling strengths, such as messenger particles from the dark sector.
%The following chapters explore a variety of different searches for beyond Standard Model physics.
%


