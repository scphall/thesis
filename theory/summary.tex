\section{Summary}

The following paragraphs will give a brief case for each of the analyses covered in this thesis.

The decay \btodsphi proceeds via the tree-level annihilation of the constituent quarks of the \Bp
meson into a \Wp meson.
This then decays into a $c\bar s$ pair, and an \ssbar is \emph{popped} from the vacuum.
In the SM, the decay contains the two CKM matrix elements \V{cs} (which is approximately unity) and
\V{ub}, which is a source of some contention from the decay $\decay{\Bp}{\taup\nu_\tau}$, which
is also an annihilation diagram.
The branching fraction $\BF(\btodsphi)$ is sensitive to NP, because any charged boson can mediate
the decay, such as a Supersymmetric charged Higgs.
Furthermore, in the SM there should be no CP asymmetry observed, however additional diagrams could
interfere and cause a significant deviation from zero.
A study of this decay is given in \Chap{ch:dsphi}.

The FCNC transition \decay{b}{s\mumu} is forbidden at tree level in the SM, and instead their
leading order processes are loop level penguin and box diagrams.
Such transitions are also CKM suppressed in the SM (by \V{tb} and \V{ts}).
Just as in the fine tuning of the Higgs mass, virtual BSM particles can contribute to the decay and
alter it significantly.
The decays \btokpipimumu and \btophikmumu are both \decay{b}{s\mumu} FCNCs and could offer
additional avenues for measuring, for example, $|\V{td}|/|\V{ts}|$.
There are also an array of strange states that contribute to the \kpipi spectrum, which is unknown,
the same can be said of the \phik spectrum.
This analysis is outlined in \Chap{ch:hhh}.

Finally, \Chap{ch:db} describes the direct search for a dark boson, \decay{\db}{\mumu} in the
decay \decay{\Bd}{\Kstarent\mumu}.
The dark boson could be a scalar or vector, and could be consistent with a plethora of BSM
scenarios.












