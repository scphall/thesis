%&spell
\section{Physics beyond the Standard Model}
\label{sec:bsm}

For a long time the completion of the \sm was reliant on the discovery of the Hiigs boson.
Finally, in 2012, the \cms~\cite{Chatrchyan:2008aa} and \atlas~\cite{Aad:2008zzm} detectors
observed a Higgs boson with a mass of $m_H\simeq125\gev$~\cite{Chatrchyan:2012ufa,Aad:2012tfa}.
This final piece of the picture has made the \sm a remarkably robust theory with no predictions
deviating significantly from experimental observations.
Indeed, the theory of \QED --- which describe interactions between
photons and charged particles in the \sm --- is one of the most accurate theories yet constructed.
The coupling constant in \QED is the fine structure constant, $\alpha$, which has been measured
experimentally to be~\cite{PDG2012}
\begin{align}
  \alpha^{-1}_\mathrm{exp} &= 137.035\,999\,074\,(44), \nonumber\\
  \intertext{and predicted theoretically to be~\cite{Aoyama:2012wj}}
  \alpha^{-1}_\mathrm{th} &= 137.035\,999\,073\,(35), \nonumber
\end{align}
%\begin{align}
  %\phantom{.} &&\alpha^{-1}_\mathrm{exp} &= 137.035\,999\,074\,(44), && \text{\cite{PDG2012}} \nonumber\\
  %\intertext{cats are the best}
  %\phantom{.} &&\alpha^{-1}_\mathrm{th} &= 137.035\,999\,073\,(35), && \text{\cite{Aoyama:2012wj}}
%\end{align}
with precisions better than one part per billion.


%Indeed, the fine structure constant, $\alpha$, which characterizes the coupling strength of
%electromagnetic interactions
%% is one of the most accurately predicted physical values.  The value of $\alpha^{-1}$
%is measured to be:
%\begin{equation}
  %\alpha^{-1} = 137.035\,999\,074\,(44),
%\end{equation}
%with a relative uncertainty of 0.32 parts per billion~\cite{PDG2012}.
%While the current best theoretical prediction, which accounts for up to tenth order Feynman
%diagrams, is:
%\begin{equation}
  %\alpha^{-1} = 137.035\,999\,073\,(35),
%\end{equation}
%to an accuracy of 0.25 parts per billion~\cite{Aoyama:2012wj}.
%Agreement between such precisely measured values, makes the theory of Quantum Electrodynamics,
%which describes interactions of photons and charged particles in the \sm, one of the most accurate
%theories yet constructed.

Despite its countless successes, there are a plethora of indications --- both
experimental and theoretical --- that additional physics exists, \bsm.



\subsection{Failures and inconsistencies of the Standard Model}
\label{sec:bsm:fail}
%%%%%%%%%%%%%%%%%%%%%%%%%%%%%%%%%%%%%%%%%%%%%%%%%%%%%%%%%%%%%%%%%%%%%%%%%%%%%%%%%%%%%%%%%%%%%%%%%%%
% Experimental
%%%%%%%%%%%%%%%%%%%%%%%%%%%%%%%%%%%%%%%%%%%%%%%%%%%%%%%%%%%%%%%%%%%%%%%%%%%%%%%%%%%%%%%%%%%%%%%%%%%
One problem is that the \sm cannot reconcile the matter-antimatter asymmetry observed in the
Universe today.
The hypothesized process which caused this asymmetry is known as baryogenesis.
Whatever this process may be it must satisfy the three Sakharov
conditions~\cite{1991SvPhU..34..392S} which outline minimum requirements for baryogenesis.
The first, most obvious, criteria is that baryogenesis must violate baryon number.
The second Sakharov condition is that both \gls{C} and \CP are violated.
Lastly, baryogenesis must occur out of thermal equilibrium.
While the \sm does contain \CPV, it is approximately ten orders of
magnitude~\cite{Cline:2006ts,Huet:1994jb} too small to explain the baryon asymmetry of the
Universe.
Other pressing problems in the \sm are that it cannot reconcile massive neutrinos, gravity, or the
amount of mass that is not accounted for in the Universe.
%$B$; because at the time the Big Bang, $B=0$, whereas today $B\gg0$.
%Whatever this process may be, it must include:
%\begin{itemize}
  %\item at least one baryon number (B) violating process,
  %\item Charge and Charge-Parity (\CP) violation,
  %\item interactions out of thermal equilibrium.
%\end{itemize}
%\comment{
%If C were conserved, then for a particle $X$ decaying into a particle $Y$ and $q$, where $q$ is
%baryonic:
%\begin{equation}
  %\Gamma\big(X\to Y+q\big)=\Gamma\big(\Xbar{X} \to \Xbar{Y}+\bar q\big),
%\end{equation}
%which would lead to $B$ being conserved over time.
%Violation of C alone is insufficient.
%Consider a process $X\to q_Lq_L$ which has a \CP-conjugate process $\bar X\to \bar q_R\bar q_R$;
%then
%\begin{equation}
  %\Gamma\big(X\to q_Lq_L\big) + \Gamma\big(X\to q_Rq_R)
  %=
  %\Gamma\big(\Xbar{X}\to \bar q_R\bar q_R\big) + \Gamma\big(\Xbar{X}\to \bar q_L\bar q_L\big)
%\end{equation}
%would still result in $B$ conservation even if C is violated.
%Thus, the process must be \CP violating.
%The final criteria ensures that baryogenesis occurs at a higher rate than anti-baryogenesis.
%}


It is well known that the vast majority of mass in the Univsere is unaccounted for.
Luminous matter totals only \approx$4.9\pc$ of the Universe~\cite{Adam:2015rua,PDG2014}, and the rest
is known only as \dm (\approx$26.8\pc$) and dark energy (\approx$68.3\pc$).
Dark Matter is an old and well motivated concept with the first evidence found in 1939 by H.~W.~Babcock
in the form of flat galactic rotation curves~\cite{1970ApJ...159..379R,1980ApJ...238..471R}.
Since then, corroborating evidence from, for example, gravitational lensing around the Bullet
cluster~\cite{Markevitch:2003at} has given further credence to its existence.

%Neither is \dm nor Dark Energy is accounted for.
%The most recent results show that the total amount of luminous matter in the Universe is about
%$4.9\%$~\cite{Adam:2015rua,PDG2014}, the remaining being composed of \dm and Dark Energy.
%\dm was first evidenced by H.~W.~Babcock in 1939 by flat rotation curves of
%galaxies~\cite{1970ApJ...159..379R,1980ApJ...238..471R}.
%Later, gravitational lensing around massive objects, such as the Bullet
%Cluster~\cite{Markevitch:2003at}, provided further evidence.
%\dm constitutes about $26.8\,\%$ of the mass in the Universe, and the remainder is Dark
%Energy.
%Furthermore, the \sm cannot explain massive neutrinos, which are necessary to explain neutrino
%oscillations, which were first observed unambiguously at
%Super-Kameokande~\cite{PhysRevLett.81.1562}.




So far, observations which insist that the \sm is not the complete picture have been given, but
there are also some conflicts in more recent measurements that could point to \np.

The largest tension in the \ut is in  the measurements of \V{ub}.
A determination of \V{ub} can be made using inclusive and exclusive measurements of semi-leptonic
$\decay{B}{X_u\ell\bar\nu_\ell}$ decays.
Inclusive measurements are made difficult by large
$\decay{B}{X_c\ell\bar\nu_\ell}$ backgrounds, while exclusive semi-leptonic modes suffer from
uncertainties introduced by form factors.
A value of $\left|\V{ub}\right|$ can also be obtained from the annihilation decay
$\decay{\Bp}{\taup\nu_\tau}$, but this suffers from low statistics.
Determinations of \V{ub} from these sources are:
\begin{align}
  &&\left|\V{ub}\right|_\mathrm{exc}
  &= \big(4.41\,^{+0.21}_{-0.23}\big)\e{-3}
  & \text{\cite{PDG2014}}& \nonumber\\
  &&\left|\V{ub}\right|_{\makebox[\widthof{$_\mathrm{exc}$}][l]{$_\mathrm{inc}$}}
  &= \big(3.28\pm{0.29}\big)\e{-3}
  & \text{\cite{PDG2014}}&\nonumber\\
  &&\left|\V{ub}\right|_{\makebox[\widthof{$_\mathrm{exc}$}][l]{$_{\tau\nu}$}}
  &= \big(4.22\pm{0.42}\big)\e{-3}  &
  \bam{Update} \text{\cite{PDG2012}}.&
  %%|\V{ub}| &= \big(4.22\pm{0.42}\big)\e{-3}  & \big(\decay{\Bp}{\taup\nu_\tau}\big)\text{\cite{{PhysRevD.88.031102}&\nonumber
  \label{eq:th:vub}
\end{align}
There are obvious tensions between the inclusive and exclusive modes, which could hint at some new
process.
A more accurate value of \V{ub} from the decay $\decay{\Bp}{\taup\nu_\tau}$ might shed light on the
situation.
Limits on \ut measurements are shown in \Fig{fig:th:ut}, and
global \V{ub} measurements from the semi-leptonic and $\decay{\Bp}{\taup\nu_\tau}$
modes are shown alongside one another.

A hint at evidence for a \np particle comes from the Hyper-\CP experiment~\cite{Burnstein:2004uk}, which
observes three $\decay{\Sigma^+}{p\mumu}$ events which survive a stringent selection.
These three events also peak in the invariant mass of the dimuon pair.
The narrowness of this peak is indicative of a two body decay, consistent with  $\decay{\Sigma^+}{pP^0}$
and the subsequent decay of the \np particle via $\decay{P^0}{\mumu}$, where
$m_{P^0}=214.3\pm0.5\mev$~\cite{Park:2005eka}.


\begin{itemize}
  \item $P_5^\prime$ \& $R_K$??
\end{itemize}


%%%%%%%%%%%%%%%%%%%%%%%%%%%%%%%%%%%%%%%%%%%%%%%%%%%%%%%%%%%%%%%%%%%%%%%%%%%%%%%%%%%%%%%%%%%%%%%%%%%
% Theoretical
%%%%%%%%%%%%%%%%%%%%%%%%%%%%%%%%%%%%%%%%%%%%%%%%%%%%%%%%%%%%%%%%%%%%%%%%%%%%%%%%%%%%%%%%%%%%%%%%%%%
Theoretical reasons suggesting \bsm physics
are often rather subjective, and revolve around the idea of \emph{naturalness}.
Naturalness is a concept whereby a theory is deemed to be natural, or more plausible, if it has few
free parameters, all of which have a magnitude $\mathcal{O}(1)$.
The \sm is not a natural theory.
For example: there are a total of 18 free parameters in the \sm, 13 of which reside in the flavour
sector;
The \ckm matrix is strongly hierarchic
%--- favouring flavour conserving weak interactions ---
and the quark masses vary by four orders of magnitude.

Unnatural models with parameters which differ wildly in magnitude tend to
lead to parameters or processes that must cancel to absurdly
high precision in order to agree with experimental observations.
These precise cancellations are known as \emph{fine tuning}.
%result in fine-tuning.
%This is when some parameters, or process, must cause a cancellation to extremely high precision in
%order to agree with experimental observations.
In the \sm, quantum loop corrections to the Higgs mass are of the order $10^{19}$
for $m_H\simeq125\gev$~\cite{Chatrchyan:2012ufa,Aad:2012tfa}.
This means that the cancellations required to result in a Higgs mass comparable to the masses of
the weak vector bosons must be exact to 17 orders of magnitude.
This instance of fine tuning is known as the \emph{hierarchy problem}.
%Additional particles from NP would contribute to these loops, and reduce of fine tuning required,
%making the model more natural.
A solution for the hierarchy problem would be to introduce NP particles, whose contributions to
loop level processes reduce the magnitude of fine tuning required to a level that might be deemed
expectable.

Fine tuning also appears in \QCD.
A gauge invariant term that can be added to \Lag{QCD} is
\begin{equation}
  \Lag{QCD}^\theta = \theta\frac{g^2}{32\pi^2}
  G_{\mu\nu}^\alpha\widetilde G^{\mu\nu}_\alpha,
  \label{eq:strongcp}
\end{equation}
where $\theta$ and $g$ are constants, and $\alpha$ indicates a sum over colours.
The operator $G_{\mu\nu}$ is the gluon field strength tensor, and
\begin{equation}
  \widetilde G^{\mu\nu}_\alpha = \frac12\varepsilon_{\mu\nu\rho\sigma}G^{\rho\sigma}_\alpha.
\end{equation}
Interactions in $\Lag{QCD}^\theta$ would conserve \gls{C} symmetry, but violate both \gls{P} and
\gls{T} conjugation~\cite{Peccei:2006as}.
Such symmetry violations are in contradiction with the observed properties of the strong
force.
Bounds placed on the value of the neutron dipole moment, $|d_n| <2.9\e{-26}\,\mathrm{ecm}$
(at 90\% CL)~\cite{Baker:2006ts} require $\theta$ to be very small,
$\theta<10^{-19}$~\cite{Crewther:PQref9}, when \emph{a priori} it could be in the range
$0<\theta<2\pi$.
This occurrence of fine tuning is referred to as the \emph{strong \CP problem}.
%A solution to this problem is to introduce an additional chiral symmetry, such that $\theta$
%becomes a field, the quanta of which are called axions.
%The axion model is discussed in more detail in \Sect{sec:db:theory}.

%\bam{Paragraph about the fact we don't know where and how NP enters.}

Despite the evidence for \bsm physics and the list of problems that must be solved, its precise
manifestation is unknown.
There are numerous theories concerning NP scenarios which seek to solve various problems.

New physics models accommodate \dm in different ways.
Some models have a \emph{dark} or \emph{hidden} sector which, apart from gravity, only
communicates with the visible sector feebly via messenger particles.
These messenger particles could potentially be observed after they decay into \sm particles after
mixing with a $H$ or $Z$.
A solution to the strong \CP problem is to introduce an additional chiral symmetry, such that
$\theta$, in \Eq{eq:strongcp}, becomes a field: the quanta of which are called \emph{axions}.
These axions could be the messenger particle between the dark and visible
sectors~\cite{Peccei:2006as}.
%, as could the inflaton or dark $Z$; models including these
%particles are further detailed in \Sec{sec:db:intro}.

%%%%%%%%%%%%%%%%%%%%%%%%%%%%%%%%%%%%%%%%%%%%%%%%%%%%%%%%%%%%%%%%%%%%%%%%%%%%%%%%%%%%%%%%%%%%%%%%%%
%%%%%%%%%%%%%%%%%%%%%%%%%%%%%%%%%%%%%%%%%%%%%%%%%%%%%%%%%%%%%%%%%%%%%%%%%%%%%%%%%%%%%%%%%%%%%%%%%%
%%%%%%%%%%%%%%%%%%%%%%%%%%%%%%%%%%%%%%%%%%%%%%%%%%%%%%%%%%%%%%%%%%%%%%%%%%%%%%%%%%%%%%%%%%%%%%%%%%




%%%%%%%%%%%%%%%%%%%%%%%%%%%%%%%%%%%%%%%%%%%%%%%%%%%%%%%%%%%%%%%%%%%%%%%%%%%%%%%%%%%%%%%%%%%%%%%%%%
%%%%%%%%%%%%%%%%%%%%%%%%%%%%%%%%%%%%%%%%%%%%%%%%%%%%%%%%%%%%%%%%%%%%%%%%%%%%%%%%%%%%%%%%%%%%%%%%%%
%%%%%%%%%%%%%%%%%%%%%%%%%%%%%%%%%%%%%%%%%%%%%%%%%%%%%%%%%%%%%%%%%%%%%%%%%%%%%%%%%%%%%%%%%%%%%%%%%%





\SUSY supplies a \dm candidate in the shape of the lightest supersymmetric particle,
which is stable because symmetry of $R$-parity is assumed to be conserved.
%which could communicate with the visible sector via a super-golstino.
\SUSY is a theory which introduces an additional super-particle for each \sm fermion and
gauge boson, whose spin is different by a half integer.
The Higgs sector in \SUSY comprises four Higgs doublets; two are spin-0 and two are spin-$\tfrac12$,
and then there are two each for $Y=\pm\tfrac12$.
After \SUSY is broken there are five Higgs physical scalar particles, two are \CP-even ($h^0$,
$H^0$) one is \CP-odd scalar ($A^0$) and two are charged charged ($H^\pm$).
It also immediately solves the hierarchy problem because for every \sm particle that contributes to
the Higgs mass, a \SUSY particle also contributes, but with the opposite sign.
Unfortunately, masses of the super-particles are unconstrained, and could be anywhere between a few
TeV and the Planck scale.

Some searches look directly for evidence of NP, this is the case for the analysis detailed in
\Chap{ch:db}, where a new particle, \db, is searched for in the dimuon invariant mass spectrum of
\decay{\Bd}{\Kstarent\mumu} consistent with \decay{\db}{\mumu}.
This is sensitive to a range of models which predict a light particle with a mass in the range
$2m_\mu\lesssim m_\db\lesssim4000\mev$, such as the axion model.
It is also sensitive to the $P^0$ that was hinted at by the Hyper-\CP experiment.

Instead of counting on NP to behave in an expected way, it is possible to search in a model
independent manner by exploring general physics couplings.
To do this it is useful to introduce the \OPE.


