\section{Physics beyond the Standard Model}
%flavour structire of yukawa scouplings is not constrained by gauge invariance, which in turn
%implies that quark masses and their mixing are free parpameterrrs of the theoruy,

While the SM does contain a source of CPV in the CKM matrix, it is not sufficient in explain the
matter dominated nature of the Universe, coming up short by approximately ten orders of magnitude
~\cite{Cline:2006ts,Huet:1994jb},
This deficit is an indication of physics beyond the SM, and any discrepancies in UT measurements
could be indicative of where this NP lies.
One particular discrepancy in UT measurements comes from the CKM matrix element \V{ub}.

A determination of \V{ub}
can be made using inclusive and exculsive measurements of
$\decay{B}{X_u\ell\bar\nu_\ell}$ decays.
Inclusive measurements are made difficult from large
$\decay{B}{X_c\ell\bar\nu_\ell}$ backgrounds, while exclusive semi-leptonic modes suffer from
uncertainties introduced by form factors.
A combination of these results leads to a value of
$\left|\V{ub}\right|_\mathrm{SL}=\left(4.13\pm0.49\right)\e{-3}$ \cite{PDG2012}.
But, \V{ub} can also be obtained using the tree level decay of
A value of $\left|\V{ub}\right|$ can also be obtained from the annihilation decay
$\decay{\Bp}{\taup\nu_\tau}:$
$\left|\V{ub}\right|_{\tau\nu}=\left(4.22\pm0.42\right)\e{-3}$ \cite{PDG2012}.
The branching fraction which is used to calculate this latter value~\cite{Amhis:2012bh} is
higher than the SM prediction and particularly sensitive to NP models which include a charged
%somewhatStrong CP problem...
%Higgs.
Figure \ref{fig:th:ut} shows the length of the side $|\V{ud}\Vconj{ub}|/|\V{cd}\Vconj{cb}|$
calculated by both values of \V{ub}.
%The decay \btodsphi has a very similar topology to \decay{\Bp}{\taup\nu_\tau} in that it proceeds
%via the annihilation of the \Bp meson constituent quarks and the resulting $W^+$ decays into quarks
%that form the final state.
%This decay is discussed in \Sec{sec:dsphi}.

%While the flavour sector is the only source of CPV in the SM, it is approximately ten orders of
%magnitude too small to explain the matter dominated nature of the
%Universe~\cite{Cline:2006ts,Huet:1994jb}.
Aside from the insufficient amount of CPV in the SM,
there are many other experimental phenomena which point to the existence of physics beyond the SM.
%This deficit is an indication of physics beyond the SM, and there are many other experimental
%phenomena which point to the same conclusion.
For example, the total mass of luminous matter observable constitutes less than $5\,\%$ of the
total mass of the Universe.
Neither is there a treatment of gravity in the SM.
\cite{Crivellin:2014zpa} VUB deiscrepancies

There are also theoretical reasons to suspect that the picture of particle physics outlined by the
SM is incomplete.
These reasons are often rather subjective, and revolve aroune the idea of natrualness.
Naturanless is a concept whereby a theory is deemed to be natural, or more likely, if it has few
free paramters, and has coupling strengths of order one.
The SM has a total of 18 free paramters, 13 of which are in the flavour sector, and there are four
orders of magnitude separating the mass of the up quark from that of the top.
Another problem is that the Higgs has been observed byt the
\atlas~\cite{Aad:2008zzm} and
\cms~\cite{Chatrchyan:2008aa}
collaborations to have a mass of $\sim125\gev$~\cite{Chatrchyan:2012ufa,Aad:2012tfa}, where quantum
loop corrections are of the order of $10^{19}$, the Planck scale.
This means that the cancelaltions required to acheive a mass comparable to the masses of the weak
vector bosons must be exact to 17 orders of magnitude.
This is known as fine tuning, and has no natural explaination.

A way to remove this occurance of fine tuning would be the addition of NP particles which
contribute to the loop corrections and reduce the magnitude of the cancellation.
Loops also appear in FCNC processes in the SM, and because particles within these loops can be
virtual, preciese measurements of FCNCs can be the ideal arena for NP searches.


The most general way to parameterize NP is with an effective Lagrangian describing generic
interactions at an energy scale $\mu$, in which long and shory distance effects are separated.
Long distance (equivilently low energy) effects are described by coefficients, $c$, which can be
calculated using peturbative methods.
Short distance (high energy) effects are parameterized in terms of operators, $\mathcal{O}$,
which must be calculated non-peturbativly because they contain QCD interactions.
The resulting effective Lagrangian includes a sum over all processes, $i$, which contribute at a
given dimension, $d$:
\begin{equation}
  \Lag{eff}
  =
  \Lag{SM} + \sum_d\frac1{\Lambda^{d-4}}
  + \sum_ic_i^{(d)}\mathcal{O}_i^{(d)}.
\end{equation}
Processes that cause FCNCs contribute in $d=6$, and can be wiritten as
\begin{equation}
  \Delta\mathcal{L}^\mathrm{FCNC}
  =
  \sum_{i\neq j}\frac{c_{ij}}{\Lambda^2}
  \left(\bar{\mathcal{O}}_{Li}\gamma^\mu\mathcal{O}_{Lj}\right)^2,
\end{equation}
where $c_{ij}$ are dimensionless FCNC couplings, where $i$ and $j$ are different quark generations.
Bounds set on the energy scale $\Lambda$ by the analysis in \Ref{Isidori:2010kg} are:
\begin{equation}
  \Lambda > \frac{|c_{ij}|^\frac12}{|\Vconj{ti}\V{tj}|}\cdot4.4\tev
  \sim
  \left\{
    \begin{array}{l}
      1.3\e{4}\tev\times |c_{sd}|^\frac12 \\
      5.1\e{2}\tev\times |c_{bd}|^\frac12 \\
      1.1\e{2}\tev\times |c_{bs}|^\frac12 \\
    \end{array}\right.
\end{equation}
These values are calculated under the assumption that NP has a natural flavour structure, where
$c_{ij}\sim\mathcal{O}(1)$.
So, either couplings are of order unity and NP begins to contribute at over $100\tev$; or couplings
strengths are $\mathrm{O}(10^{-5}$ and NP contributes at the $1\tev$ level;
it is expected for NP to appear at the $1\tev$ scale in order to solve the hierarcy problem.
Either way, there is a conflict between the most natural coupling and energy scale; this is known
as the flavour problem.
%So must conclude that NP has a highly non-generic flacour structure.

A solution to the flavour problem is the Minimal Flavour Violation (MFV) hypoethesis.
This is the, somewhat pessimistic, view that the flavour structure of NP mirrors that of the Yukawa
couplings; leading to NP flavour changing transitions to be hidden by those seen in the SM.



%These processes can be parameterised by an effective Lagrangian, which is applicable up to some
%energy scale, $\Lambda$:
%\begin{equation}
  %\Lag{eff} = \Lag{SM} + \sum\frac{c_i^{(d)}}{\Lambda^{d-4}}\mathcal{O}_i^{(d)}.
%\end{equation}
%Considering the specific instance of $\Delta F=2$ operators,
%Analysis of Ref\cite{Isidori:2010kg}
%\begin{equation}
  %\Delta\mathcal{L}^{\Delta F=2}
  %=
  %\sum_{i\neq j}\frac{c_{ij}}{\Lambda^2}
  %\left(\bar{\mathcal{O}}_{Li}\gamma^\mu\mathcal{O}_{Lj}\right)^2.
%\end{equation}
%
%(acting on some intial state i, and final state f.)
%\begin{equation}
  %\bra{f}\mathcal{H}_\mathrm{eff}\ket{i}
  %=
  %\right.
  %\sum_d\frac1{\Lambda^d}
  %\sum_ic_{d,i}\bra{f}\mathcal{O}_{d,i}\ket{i}
  %\left|_\Lambda.
%\end{equation}
%Here, the operator, $\mathcal{O}$, has dimension $d$; at which multiple operators can contribute.
%Coefficients c are dimensionless and independent of the energy scale.

%Strong bounds on $\Lambda$ for generic c, of order 1 (naturla ness)
%Want $\Lambda$ to be small (TeV) to solve the hierarchy problem
%Generic couplings mean Lambda $>$ around 100TeV.
%New particles with mass in TeV region coupling strengths must be of the less than $10^{-5}$.
%Either non-natural couplings, or no new phytsics until 100TeV or so



%Could mention the desert
%
%Minimal
%
%DeltaF=1 operators require Lambda>1.5TeV to 95\% CL



%These processes can be parameterized most generally by an effective Lagrangian where new physics
%contributes with dimension $d$, suppressed by powers of some effective scale $\Lambda$:
%\begin{equation}
  %%\Lag{eff} = \Lag{SM} + \sum\frac{c_i^{(d)}}{\Lambda^{d-4}}\mathcal{O}_i^{(d)}
  %\Lag{eff} = \Lag{SM} + \sum\frac{c_i^{(d)}}{\Lambda^{d-4}}\mathcal{O}_i^{(d)}
%\end{equation}



%The most natural couplings are of order one.



%%\subsection{FCNCs}
%Since the SM is such a good description of physics at energy scales that have been probed to date,
%it is resonable to assume that this diverges at some cut off energy scale, $\Lambda$.
%This scale can be set based on the solution of the hierarchy problem, which indicates that
%$\Lambda$ should be less than a few TeV.
%A bound can also be set on $\Lambda$ by considering processes which are absent from SM processes at
%tree level, such as flavour changing neutral currnents (FCNCs).
%This results in a value of $\Lambda$ which far exceeds the few TeV from solving the hierarchy
%problem.
%The conflict between these two determinations is named the \emph{Flavour Problem}.
%
%The most pessimistic solution to the flavour problem is \emph{Minimal Flavour Violation} (MFV)
%which simply assumes that beyond SM physics follows a Yukawa coupling like structure in the flavour
%sector, this would lead to no discernable new physics in the flavour sector.


Fine tuning also appears in \Lag{QCD}.
%The argument for the existence of axions is derived from the existence of the \emph{strong CP
%problem}, which is another
%example of questioning the existence of fine tuning in the SM.
The Lagrangian for QCD looks like:
\begin{align}
  \mathcal{L}_\mathrm{QCD} &=
  \mathcal{L}_\mathrm{QCD}^\mathrm{kin} +
  \mathcal{L}_\mathrm{QCD}^\mathrm{mass} +
  \mathcal{L}_\mathrm{QCD}^\theta \\
  \mathcal{L}_\mathrm{QCD}^\theta &=
  -\frac{n_fg^2\theta}{32\pi^2}F_{\mu\nu}\tilde F^{\mu\nu}.
  %&= -\frac14F_{\mu\nu}\tilde F^{\mu\nu}
  %&-&\frac{n_fg^2\theta}{32\pi^2}F_{\mu\nu}\tilde F^{\mu\nu}
  %&+&\bar\psi\left(i\gamma^\mu D_\mu-me^{i\theta^\prime\gamma_5}\right)\psi\\
  %&= -\frac14F_{\mu\nu}\tilde F^{\mu\nu}
  %&-&\mathcal{L}_\theta
  %&-&\mathcal{L}_\mathrm{mass}.
\end{align}
The term $\mathcal{L}_\theta$ conserves Charge symmetry, but violates Parity and Time conjugation
invariance~\cite{Peccei:2006as}.
Thus, $\mathcal{L}_\theta$ violates CP.
However, we know from the current experimental upper limit of the neutron electric dipole moment
\cite{Baker:2006ts},
$|d_n <2.9\e{-26}\,e\mathrm{cm}$ (at 90\% CL) that the value of $\theta$ is less than $10^{-9}$
where it can be in the range $[0,2\pi]$.
This fine tuning is referred to as the strong CP problem.












%A solution of the strong CP problem is to introduce a chiral $U(1)$ symmetry which is spontaneously
%broken and effectively replaces the CP violating angle $\theta$ with a CP conserving field.


%indicates that the SM is an incomplete picture of partilce physics.
%one of many reasons to believe that the SM is incomplete.
%But, there are other experimental phenomena which indicate
%unexplained by the SM, including gravity, dark matter and
%massive neutrinos.


%Furthermore

%Apart from the significant deficit in the amount of CPV supplied by the SM with respect to the
%amount required for a matter dominated Universe; there are a host of experimental reasons to
%suspect that the SM is incomplete.
%These include: the existence of gravity; the fact that the luminous matter totals less than $5\,\%$
%of the mass in the Universe; and that neutrinos have mass.

%Theoretical motivations for physics beyond the SM primarily revolve around the concept of
%naturalness.
%Naturalness is an entirely subjective concept, but generally a natural theory has coupling
%strengths of $\mathcal{O}(1)$ and few free parameters.
%The SM is deemed to be an unnarural theory because of the vast difference in energy that separates
%the weak from

%The SM is deemed to have probels with natural
%These reasons can be categorised as either experimental or theoretical.








%The \decay{b}{s} FCNC is forbidden at tree level in the SM and are only allowed in higher-order
%electroweak processes.
%New particles from extensions to the SM can enter these loops and significantly alter

%PAGE 9 PAtricks thesis






%Despite its countless successes,
%there are many experimental and theoretical arguments indicating that the SM is an incomplete
%picture of particle physics.
%Many theoretical problems arise from the idea of naturalness, that is that...
%
%%Experimental observations that are it is well known that the SM is incomplete; arguments for this
%%come from both the experimental and theoretical
%%leaves some experimentally observed phenomena unexplained.
%Experimentally, there are observed phenomena which are left unexplained by the SM.
%Neutrinos are treated as massless in the SM, but they are seen to oscillate in flavour space
%indicating that they must, in fact, have mass.
%Flat rotation curves of galaxies and gravitational lensing indicate the existence of Dark Matter,
%which is entriely unaccounted for by the SM.
%%Shortcomings include the
%% credence
%%For example the SM does not explain: gravity, dark matter, dark energy, and neutrino masses.
%
%Another problem is that the SM cannot reconcile the matter-antimatter asymmetry observable in the
%Universe today.
%The hypothesized process which caused this asymmetry is known as baryogenesis.
%%Baryogenesis is the term for a hypothesized process which resulted in the matter-dominated nature of
%%the Universe.
%Whatever this process may be, it must satisfy:
%\begin{itemize}
  %\item at least one baryon number (B) violating process,
  %\item Charge and Charge-Parity (CP) violation,
  %\item interactions out of thermal equilibrium.
%\end{itemize}
%These are the Sakharov conditions~\cite{1991SvPhU..34..392S}, and outline the minimum requirements
%of baryogenesis.
%The first of these criteria is an obvious one: at the time of the Big Bang $B=0$, whereas today
%$B\gg0$; hence $B$ must not be conserved in some process.
%If a process conserves charge then
%\begin{equation}
  %\Gamma(X\to Y+B)=\Gamma(\bar X \to \bar Y+\bar B),
%\end{equation}
%so $B$ will be conserved over time.
%However, this condition is insufficient.
%Consider a process $X\to q_Lq_L$ which has a CP-conjugate process $\bar X\to \bar q_R\bar q_R$;
%then
%\begin{equation}
  %\Gamma(X\to q_Lq_L) + \Gamma(X\to q_Rq_R)
  %=
  %\Gamma(\bar X\to \bar q_R\bar q_R) + \Gamma(\bar X\to \bar q_L\bar q_L)
%\end{equation}
%would still result in $B$ conservation even if C is violated.
%Thus, the process must be CP violating.
%The final criteria ensures that baryogenesis occurs at a higher rate than anti-baryogenesis.
%%Clearly the process must result in the violation of baryon number, and it must happen out of
%%thermal equilibrium otherwise the process would occur equally as often in each direction.
%%The third condition, CP violation (CPV) means that
%%$\Gamma(A+B\to C)\neq\Gamma(\bar A + \bar B \to \bar C)$, so the annihilation of the products of
%%the interaction cannot washout the asymmetry.
%
%As discussed, the flavour sector is the only source of CPV in the SM, but comes up short by around 10 orders of
%magnitude when explaining the matter dominated nature of the Universe~\cite{Cline:2006ts,Huet:1994jb}.
%%There is therefore a powerful reason to believe that NP enters the flavour sector.
%%The following chapter will elucidate as to how the flavour sector is the only source of CPV in the
%%SM.



\begin{itemize}
  \item Hyper CP
  \item P5primed
\end{itemize}


