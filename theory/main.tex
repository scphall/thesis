\chapter{The Standard Model and beyond}
\label{ch:theory}

%
This thesis contains the work undertaken in three analyses; each of which concerns a different area
of interest in high energy physics.
The following chapter aims to motivate each analysis in turn after introducing the Standard Model
of particle physics.

Firstly, the formulation of the SM will be outlined, with particular detail paid to the flavour
sector.
Various successes of the SM will next be discussed before going on to identify its shortcomings
using arguments from both experiment and theory.
These shortcomings will then be used to motivate the three analyses:
a search for \btodsphi, which contains \V{ub} (\Chap{ch:dsphi});
a search for \btokpipimumu and \btophikmumu, which are flavour changing neutral currents
(\Chap{ch:hhh});
and seaching for dark sector particles in \btokstmumu (\Chap{ch:db}).
Theory specifically relating to each of these decays will be detailed in the relevant chapter.

%The following chapter will elucidate as to how the flavour sector is the only source of CPV in the
%SM.
%It will then go on to motivate study of flavour physics at \lhcb by exploring some problems with
%the SM as a theory.

%\section{The Standard Model}
\label{sec:sm}
The behaviour of fundamental particles and forces are described by the \sm of
particle physics, which was concocted in the 1970s, when the Higgs
mechanism was incorporated into Glashow's electroweak theory by Salam and Weinberg.
The theory prescribes a treatment
as to how fundamental particles interact via three of the four
fundamental forces, namely: the strong, weak and electromagnetic forces.

Mathematically, the \sm is a locally gauge invariant quantum field theory.
It inhabits a space-time with a global Poincar\'e symmetry that obeys a local
$SU_C(3)\otimes SU_L(2)\otimes U_Y(1)$ symmetry\footnote{
The convention of natural units is used throughout.
Other conventions are that the indices $\mu$ and $\nu$ are used for four vectors, and generators
for the $SU_L(2)$ and $SU_C(2)$ are denoted by $\{i,j,k\}$ and $\{a,b,c\}$, respectively.}.
The $SU_L(2)\otimes U_Y(1)$ gauge group contains the electroweak formalism, and the $SU_C(3)$ group
contains that of the strong force.
%Each generator in this group corresponds to a spin-1 gauge-boson;
%the $SU_C(3)$ group is obeyed by the strong force in the theory of \QCD,
%while the electroweak sector obeys the $SU_L(2)\otimes U_Y(1)$ gauge group.
Generators for each group correspond to the vector bosons which mediate
interactions ---
the $3+1$ electroweak gauge bosons ($Z$, $W^\pm$ and the photon), and
the eight gluons of the strong force.
A summary of gauge bosons in the \sm is given in \Tab{tab:sm:gauge}.


\begin{table}
  \caption[Fundamental, force-mediating, gauge bosons]
  {
    Fundamental, force-mediating, gauge bosons in the \sm.
    All values are taken from Ref.~\protect\cite{PDG2014}.
  }
  \label{tab:sm:gauge}
  \begin{center}
    \begin{tabular}{ccrc}
      \toprule
      Force
      & Particle & \cellc{Mass}  & Charge\\
      \midrule
      Electromagnetic & $\gamma$ & $0\phantom{\gev}$ & $\phantom{-}0$ \\
      \multirow{2}{*}{Weak} & $W^\pm$ & $80.4\gev$ & $\pm1$ \\
      & $Z$ & $90.2\gev$ & $\phantom{-}0$ \\
      Strong & $G^a$ & $0\phantom{\gev}$ & $\phantom{-}0$ \\
      \bottomrule
    \end{tabular}
  \end{center}
\end{table}


Symmetries are fundamental to the dynamics of particle physics.
It was shown by Emmy Noether that for each symmetry in the action of a physical system there is a
conserved quantity~\cite{Noether}.
This is Noether's theorem.
For any sensible theory of physics, it is necessary that the laws remain the same independent of
time and space, these symmetries lead to the conservations of momentum and energy, respectively.
Discrete symmetries of important are \gls{T} and \gls{P}, so that physics occurs the same
regardless of the direction of time, and under reflections in space.

%Noether -> Symmetries
%C P T

Spin-$\tfrac12$ particles in the \sm are known as fermions, and
are described by spinor fields, $\psi$, and obey the Dirac equation:
\begin{equation}
  \big(i\gamma^\mu\partial_\mu - m\big)\psi = 0.
  \label{th:eq:dirac}
\end{equation}
The fermions of the \sm can be broadly categorized into those which couple to the strong force,
\emph{quarks}, and those which do not, \emph{leptons}.
There are six quarks: up, down, charm, strange, top, and bottom (\uquark, \dquark, \cquark,
\squark, \tquark and \bquark); and six leptons:
the electronic leptons ($\nu_e$, $e$),
the muonic leptons ($\nu_\mu$, $\mu$),
and the tauonic leptons ($\nu_\tau$, $\tau$).
Due to the properties of the strong force, quarks can only be observed as colour-neutral bound
states, usually these are
\emph{mesons} (quark-antiquark bound states) and \emph{baryons} (bound states of three quarks).
%the electron, muon, tau ($e$, $\mu$, $\tau$), and
%their corresponding neutrinos ($\nu_e$, $\nu_\mu$, $\nu_\tau$).
All fermions are organized into pairs forming three generations.
%The fermions of the \sm constitute six leptons (electron, electron neutrino, muon, muon neutrino,
%tau and tau neutrino) and six quarks ($u$p, $d$own, $c$harm, $s$trange, $t$op and $b$ottom), which
%are organized into pairs forming three generations.
For each fermion there is a corresponding antiparticle with the same mass and opposite charge ---
charge being the conserved quantity resulting from the global electroweak gauge symmetry by
Noether's theorem.
A summary of all fermions and some of their properties is given in \Tab{tab:sm:particles}.
There is also a single scalar field in the \sm: the Higgs boson.
%The only additional particle is the scalar Higgs boson.

\begin{table}
  \caption[Fundamental fermions of matter]{
    Fundamental fermions of matter.
    In the \sm each has a corresponding anti-particle of opposite charge.
    All values are taken from Ref.~\protect\cite{PDG2014}.
  }
  \label{tab:sm:particles}
  \begin{center}
    \begin{tabular}{ccrccrc}
      \toprule
      & \multicolumn{3}{c}{Leptons}
      & \multicolumn{3}{c}{Quarks} \\
      Generation
    & Particle & \cellc{Mass}  & Charge
    & Particle & \cellc{Mass}  & Charge\\
      \midrule
      \multirow{2}{*}{1} & \ep   & $0.511\mev$ & $-1$ & \uquark & $2.3\mev$ & $+^2/_3$ \\
      & \neue & $0\phantom{\mev}$  &  $\phantom{-}0$ & \dquark & $4.8\mev$ & $-^1/_3$ \\
      \multirow{2}{*}{2} & \mup   & $0.105\gev$ & $-1$ & \cquark & $95.0\mev$ & $+^2/_3$ \\
      & \neum & $0\phantom{\mev}$  &  $\phantom{-}0$ & \squark & $1.275\gev$ & $-^1/_3$ \\
      \multirow{2}{*}{3} & \taup   & $1.777\gev$ & $-1$ & \tquark & $173\gev$ & $+^2/_3$ \\
      & \neut & $0\phantom{\mev}$  &  $\phantom{-}0$ & \bquark & $4.18\gev$ & $-^1/_3$ \\
      \bottomrule
    \end{tabular}
  \end{center}
\end{table}



The \sm Lagrangian can be expressed as a sum of components:
\begin{equation}
  \Lag{SM} = \Lag{QCD} + \Lag{V} + \Lag{\ell} + \Lag{\it q} + \Lag{Higgs} + \Lag{Yuk}.
  \label{eq:th:lag}
\end{equation}
These Lagrangians describe the:
strong force interactions between colour carrying particles in the theory of \QCD
(\Lag{QCD});
weak vector self-interactions (\Lag{V});
electroweak behaviour of leptons (\Lag{\ell});
electroweak behaviour of quarks (\Lag{\it q});
Higgs interaction (\Lag{Higgs}); and
Yukawa couplings (\Lag{Yuk}).

As well as the discrete symmetries of \gls{P} and \gls{T}, there is also the \gls{C} symmetry.
The violation of these symmetries are of fundamental interest
to modern particle physics.
%It is expected that the combined $C\!PT$ symmetry is conserved, but other symmetries can be
%violated, measurements
%It is kk
%The combined \CP symmetry is
%The
%The latter three terms of \Eq{eq:th:lag} (\Lag{\it q}, (\Lag{Higgs}, and \Lag{Yuk}) are of
%fundamental importance as
%to how flavour, and \CP violation arise in the \sm, and will be discussed in detail.
%There are three fundamental discrete symmetries in the \sm: \gls{C}, \gls{P}, and \gls{T}.
Violation of the combined \CP symmetry, and flavour, arise in the \ckm matrix of the \sm, which emerges
after the Higgs mechanism breaks the local electroweak symmetry.
The important terms for this are \Lag{\it q}, (\Lag{Higgs}, and \Lag{Yuk} from \Eq{eq:th:lag}.

The Higgs doublet, $\Phi$, is defined to be
\begin{equation}
  \Phi = \frac{1}{\sqrt{2}}
  \begin{pmatrix}
    \phi_1 + i\phi_2 \\
    \phi_3 + i\phi_4 \\
  \end{pmatrix},
  \label{eq:th:phi}
\end{equation}
where each $\phi_i$ is a real field.
The Lagrangian of the Higgs field is:
\begin{align}
  \Lag{Higgs}
  &= \big(D_\mu\Phi\big)^\dagger\big(D^\mu\Phi\big) - V\big(\Phi\big) \nonumber\\
  &= \big(D_\mu\Phi\big)^\dagger\big(D^\mu\Phi\big) - \mu^2\big(\Phi^\dagger\Phi\big) +
  \lambda\big(\Phi^\dagger\Phi\big)^{2},
  \label{eq:th:laghiggs1}
\end{align}
where $\mu$ and $\lambda$ are constants, and $D_\mu$ is the covariant derivative.
Figure~\ref{fig:th:higgspot} shows that
taking $\mu^2<0$ and $\lambda>0$ shifts the ground state of the vacuum of $V(\Phi)$ away from zero.
When the system collapses in to the ground state a direction is chosen, this breaks the symmetry of
the system.
%as shown in \Fig{fig:th:higgspot},
%shifts the minimum of the potential $V(\Phi)$ away from zero by a
The amount by which the ground state shifts with respect to the origin is
\begin{equation}
  v = \sqrt{\frac{\mu^2}{\lambda}}.
\end{equation}
At this point the Higgs field gets a \VEV of $\langle\phi\rangle = v/\sqrt{2}$.
%$\braket{\phi} = \tfrac{1}{\sqrt{2}}v$.
The direction of the \VEV from the origin is arbitrary, but the choice of
\begin{align}
  \bra{0}\phi_1\ket{0} =
  \bra{0}\phi_2\ket{0} =
  \bra{0}\phi_4\ket{0} &= 0 \nonumber\\
  \bra{0}\phi_3\ket{0} &= v,
\end{align}
is convenient, and changes the Higgs doublet in \Eq{eq:th:phi} to
\begin{equation}
  \Phi = \frac{1}{\sqrt{2}}
  \begin{pmatrix}
    \eta_1 + i\eta_2 \\
    v + i\eta_4 \\
  \end{pmatrix}.
  \label{eq:th:eta}
\end{equation}
Here, $\eta_1$, $\eta_2$ and $\eta_4$, are Goldstone bosons which, by choosing an appropriate
gauge, become the longitudinal components of the weak vector bosons, and $\Phi$ simplifies to
\begin{equation}
  \Phi = \frac{1}{\sqrt{2}}
  \begin{pmatrix} 0 \\ v+H
  \end{pmatrix},
  \label{eq:th:phi2}
\end{equation}
where $H$ is the physical Higgs boson.
Inserting Eq.~\ref{eq:th:phi2} into Eq.~\ref{eq:th:laghiggs1} gives:
%\begin{multline}
  %\Lag{Higgs} =
  %\frac12\left(\partial_\mu H\right)\left(\partial^\mu H\right)
  %+\frac14g^2\left(v^2 + 2vH + H^2\right)W_\mu^+W^{-\mu}  \\
  %+\frac18\left(g^2 + g^{\prime2}\right)\left(v^2 + 2vH + H^2\right)Z_\mu Z^\mu
  %+ \mu^2H^2 + \frac14\lambda\left(H^4+4vH^3\right) + \cdots,
  %\label{eq:th:laghiggs2}
%\end{multline}
\begin{equation}
  \Lag{Higgs} =
  \frac{1}{2}\big(\partial_\mu H\big)\big(\partial^\mu H\big)
  %+\mu^2H^2
  +\frac{m_H^2}{2}H^2
  +\left(m_W^2W_\mu^+W^{-\mu} + \frac{m_Z^2}{2}Z_\mu Z^\mu\right)
  %\cdot
  \left(1 + \frac{H}{v}\right)^{\!2}.
  \label{eq:th:laghiggs2}
\end{equation}
%where $g$ and $g^\prime$ are coupling constants and other terms are three- and four-point
%interactions of the Higgs with itself and weak gauge bosons.
Thus, there is \SSB of the local $U_Y(1)$ gauge group.
A result of this is that weak gauge bosons become massive, while photons remain massless: as is
consistent with observations.

\begin{figure}
  \begin{center}
    \includegraphics[width=0.6\textwidth]{higgs_potential}
    \caption[Shape of the Higgs potential]
    {
      The shape of the Higgs potential, $V(\Phi)$, for the simple case of $\Phi=\phi_1+i\phi_2$ and
      $\mu^2<0$ and $\lambda>0$.
      Spontaneous symmetry breaking occurs when the vacuum settles in a minima, and this choice of
      direction breaks the symmetry of the gauge.
      A section of the potential is not shown, to convey the shape of the potential.
    }
    \label{fig:th:higgspot}
  \end{center}
\end{figure}

All fermions (excepting neutrinos) also acquire mass after \SSB.
The Dirac mass term for a chiral field has the form:
\begin{equation}
  \Lag{mass} = -m_\psi\big(\xbar\psi_R\psi_L + \xbar\psi_L\psi_R\big),
  \label{eq:diracmass}
\end{equation}
but the left- and right-handed fields
($\psi_L$ and $\psi_R$) have different $U_Y(1)$ charges and so transform differently.
Using \Eq{eq:diracmass} to give masses to fermions would therefore break gauge invariance.
%and so cannot be added to \Lag{SM} without destroying the gauge invariance, hence \Eq{eq:diracmass}
%cannot be used.
Instead, masses are generated through the Yukawa couplings, which
describe interactions between all fermionic fields and the Higgs doublet.
This can be written
\begin{equation}
  %\Lag{Yuk} = \sum_{\substack{\ell=\\e,\mu,\tau}}\big(\Lag{Yuk}^\ell\big) + \Lag{Yuk}^q,
  \Lag{Yuk} = \sum_{\ell}\big(\Lag{Yuk}^\ell\big) + \Lag{Yuk}^q,
  \label{eq:th:yukking}
\end{equation}
where terms encapsulate lepton and quark interactions, respectively.
%where $\ell$ and $q$ denote the lepton and quark interactions, respectively.
Each lepton term describes the interaction between the Higgs boson and the chiral fields
$\ell_R$ and the spinor
\begin{align}
  \chi_L = \begin{pmatrix}\nu_L \\ \ell_L \end{pmatrix},
\end{align}
via
\begin{equation}
  \Lag{Yuk}^\ell
  = - g_\ell\big(\xbar\chi_L\Phi \ell_R + \xbar \ell_R\Phi^\dagger\chi_L\big),
\end{equation}
where each $g_\ell$ is a coupling constant.
After \SSB the Lagrangian becomes
\begin{align}
  \Lag{Yuk}^\ell
  = - m_\ell \big(\xbar\ell_L\ell_R + \xbar\ell_R\ell_L\big)%\cdot
  \left(1 + \frac{H}{v}\right)
\end{align}
and the lepton masses
\begin{equation}
  m_\ell = \frac{v}{\sqrt{2}}g_\ell
  \label{eq:leptonmass}
\end{equation}
are dependent on the fundamental parameters $g_\ell$ and $v$.

Yukawa interactions for quarks involve the right-handed chiral operators of the up- and down-type
quarks, $q_R^i$ for $q\in\{u,d\}$, and the left-handed doublet
\begin{equation}
  Q_L^i = \begin{pmatrix}u^i_L\\d^i_L\end{pmatrix}.
\end{equation}
Before \SSB, the Yukawa Lagrangian is
\begin{equation}
  \Lag{Yuk}^q = - y_{ij}^u\Xbar{Q}_L^i\Phi u_R^j
  - y_{ij}^d\Xbar{Q}_L^i\widetilde\Phi d_R^j +\,\mathrm{h.c.}
  \label{eq:th:lagyukq}
\end{equation}
where $\widetilde\Phi_i = \varepsilon_{ij}\Phi_k$, there is an implicit sum over the generations
$i$ and $j$, and the shorthand h.c.~denotes hermitian conjugate.
The coupling constants, $y^{q}$, are $3\times3$ matrices characterizing Yukawa coupling strengths
between generations.
After \SSB, $\Lag{Yuk}^q$ becomes:
\begin{equation}
  \Lag{Yuk}^q =
  - \frac{v}{\sqrt{2}}
  \left(
  y_{ij}^\uquark\uquarkbar_L^i\uquark_R^j
  + y_{ij}^\dquark\dquarkbar_L^i\dquark_R^j
  +\,\mathrm{h.c.}
  \right)
  %\cdot
  \left(1 + \frac{H}{v}\right).
  \label{eq:th:lagyuk2}
\end{equation}
Similar to lepton masses, in \Eq{eq:leptonmass}, quark masses are defined as
\begin{align}
  m_{ij}^u &= \frac{v}{\sqrt{2}}y_{ij}^u \nonumber\\
  m_{ij}^d &= \frac{v}{\sqrt{2}}y_{ij}^d.
\end{align}
Thus far, the flavour basis has been used, but it is now more convenient to change to the mass
basis, in which the matrices $m^{u,d}$ are diagonal., it is more convenient to change to a basis in which the matrix $m^q$
is diagonal, using the rotation matrices $V_L$ and $V_R$, such that
\begin{align}
  {m_{il}^{u}}^\prime &=  \big(V_L^{u\dagger}\big)_{ij} m_{jk}^u\big(V_R^u\big)_{kl} \nonumber\\
  {m_{il}^{d}}^\prime &=  \big(V_L^{d\dagger}\big)_{ij} m_{jk}^d\big(V_R^d\big)_{kl}.
\end{align}
The addition of a prime distinguishes the mass basis from the flavour basis.
This transformation is exactly equivalent to transforming the up- and
down-type chiral quark fields according to:
\begin{align}
  q_L^\prime &= \big(V_L^q\big)q_{L}^{} \nonumber\\
  q_R^\prime &= \big(V_R^q\big)q_{R}^{}.
\end{align}
Applying these transformations to all parts of \Lag{SM} leaves the majority of it unchanged, since
$V_{L}^{q\dagger} V_{L}^{q} = V_{R}^{q\dagger} V_{R}^{q} = \mathds{1}$ by definition.
%$m_{ij}^\mathrm{diag} = V_{Lik}m_{kl}(V_R^\dagger)_{lj}$.
%This is exactly equivalent to transforming the chiral quark fields for up- and down-type quarks
%accordingly:
%\begin{align}
  %q_L^\alpha = \left(V_L^q\right)_{\alpha i}q_L &&
  %q_R^\alpha = \left(V_R^q\right)_{\alpha i}q_R,
%\end{align}
%where the index of the original basis is identified with $i$ and the mass basis uses $\alpha$.
%The rotations of the basis of the chiral quark fields leave much of \Lag{SM} unchanged since
%$V_{qL}^\dagger V_{qL} = V_{qR}^\dagger V_{qR} = \mathbb{1}$.
However, this is not the case for \Lag{\it q}.

The Lagrangian $\Lag{\it q}$ can be decomposed into
$\Lag{\it q}^\mathrm{NC}+\Lag{\it q}^\mathrm{CC}$, where the
superscripts denote \NC and charged current \CC components.
The \NC part of the Lagrangian characterizes interactions between quarks and the neutral electroweak
vector bosons, while the \CC part involves the interactions of quarks wiht the charged $W^\pm$
bosons.
After changing to the mass basis, \Lag{NC} remains unchanged, whereas \Lag{CC}
transforms as:
\begin{align}
  \Lag{\it q}^\mathrm{CC}
  &= i\frac{g}{2}\gamma^\mu
  \bigg[\uquarkbar_Ld_LW^+_\mu + \dquarkbar_Lu_LW^-_\mu
  \bigg]  \nonumber\\
  &= i\frac{g}{2}\gamma^\mu
  \bigg[
    \uquarkbar_L^\prime\left(V_{uL}^{\phantom{\dagger}}V_{dL}^\dagger\right)d_L^{\,\prime} W^+_\mu +
    \dquarkbar_L^{\,\prime}\left(V_{dL}^{\phantom{\dagger}}V_{uL}^\dagger\right)u_L^\prime W^-_\mu
  \bigg]  \nonumber\\
  &= i\frac{g}{2}\gamma^\mu
  \bigg[
    \VCKM\uquarkbar_L^\prime d_L^{\,\prime} W^+_\mu +
    \VCKMconj\dquarkbar_L^{\,\prime} u_L^\prime W^-_\mu
  \bigg].
\end{align}
In the final step the matrix
$\VCKM=V^{\phantom{\dagger}}_{uL}V_{dL}^\dagger$ is defined.
This is known as the \ckm
matrix, and parameterizes the couplings between up- and down-type quarks in charged weak currents.




%\subsection{The CKM matrix and Unitarity Triangle}
\label{sec:ckm}

The \ckm matrix is defined as:
\begin{equation}
  \VCKM = \left(V_{uL}^{\phantom{\dagger}} V_{dL}^\dagger\right) =
  \begin{pmatrix}
    \V{ud} & \V{us} & \V{ub} \\
    \V{cd} & \V{cs} & \V{cb} \\
    \V{td} & \V{ts} & \V{tb} \\
  \end{pmatrix},
\end{equation}
where each $|V_{ij}|$ parameterises the probability of an up-type quark, of generation $i$,
transitioning to down-type quark, of generation $j$, in a weak interaction.
In the \sm, it is assumed that the total charged current couplings of up- to down-type quarks is the
same as down- to up-type.
This assumption means that the \ckm matrix is unitary, $\VCKMconj\VCKM = \mathds{1}$, and therefore
it contains only four physical parameters: three angles ($\theta_{12}$, $\theta_{13}$ and
$\theta_{23}$) and one complex phase ($\delta$).
In fact, the observation of \CPV in kaon mixing~\cite{Christenson:1964fg}
led to the prediction of a third generation before
its discovery, precisely because a $3\times3$ matrix is the smallest necessary for a phase to enter
a unitary matrix.

%preferentially
%hierarchichal
%arbiatry

%The CKM matrix is the source of all flavour violation in the SM.
%In the SM, it is assumed that the total charged current couplings of up- to down-type quarks is the
%same as down- to up-type.
%This means that the CKM matrix is unitary, $V^\dagger V = \mathbb{1}$, and therefore it contains
%four physical parameters: three angles ($\theta_{12}$, $\theta_{13}$ and $\theta_{23}$) and one
%complex phase ($\delta$).

%\begin{equation}
  %\VCKM =
  %\begin{pmatrix}
    %\cxx{12}\cxx{13} & \sxx{12}\cxx{13} & \sxx{13}e^{-i\delta} \\
    %-\sxx{12}\cxx{23}-\cxx{12}\sxx{23}\sxx{13}e^{i\delta} &
    %\cxx{12}\cxx{23}-\sxx{12}\sxx{23}\sxx{13}e^{i\delta} & \sxx{23}\cxx{13} \\
    %\sxx{12}\sxx{23}-\cxx{12}\cxx{23}\sxx{13}e^{i\delta} &
    %-\cxx{12}\sxx{23}-\sxx{12}\cxx{23}\sxx{13}e^{i\delta} & \cxx{23}\cxx{13} \\
  %\end{pmatrix}
%\end{equation}

There are many ways of representing the \ckm matrix.
One way is as a product of three rotation
matrices, one of which contains the complex phase, this is known as the \emph{standard}
parameterisation:
\begin{equation}
  \VCKM =
  \begin{pmatrix}
    \cxx{12}\cxx{13} & \sxx{12}\cxx{13} & \sxx{13}e^{-i\delta} \\
    -\sxx{12}\cxx{23}-\cxx{12}\sxx{23}\sxx{13}e^{i\delta} &
    \cxx{12}\cxx{23}-\sxx{12}\sxx{23}\sxx{13}e^{i\delta} & \sxx{23}\cxx{13} \\
    \sxx{12}\sxx{23}-\cxx{12}\cxx{23}\sxx{13}e^{i\delta} &
    -\cxx{12}\sxx{23}-\sxx{12}\cxx{23}\sxx{13}e^{i\delta} & \cxx{23}\cxx{13} \\
  \end{pmatrix},
\end{equation}
where $\sxx{ij}$ and $\cxx{ij}$ denote $\sin\theta_{ij}$ and $\cos\theta_{ij}$, respectively.
A convenient simplification is the \emph{Wolfenstein} parameterisation, which is obtained by
defining
\begin{align}
  \sin\theta_{12}&=\lambda, \nonumber\\
  \sin\theta_{23}&=A\lambda^2, \nonumber\\
  \intertext{and}
  e^{-i\delta}\sin\theta_{13} &= A\lambda^3(\rho-i\eta),
\end{align}
which results in
\begin{align}
  \VCKM\simeq
  \begin{pmatrix}
      %1-\tfrac12\lambda & \lambda & A\lambda^3(\rho-i\eta+\tfrac{i}2\eta\lambda^2) \\
      %-\lambda & 1-\lambda^2-i\eta A^2\lambda^4 & A\lambda^2(1+i\eta\lambda^2) \\
      %A\lambda^3(1-\rho-i\eta) & -A\lambda^2 & 1 \\
    1-\tfrac12\lambda & \lambda & A\lambda^3\big(\rho-i\eta\big) \\
    -\lambda & 1-\lambda^2 & A\lambda^2 \\
    A\lambda^3\big(1-\rho-i\eta\big) & -A\lambda^2 & 1 \\
  \end{pmatrix}.
  \label{eq:th:wolfenstein}
\end{align}
The values of the Wolfenstein parameters $A$ and $\lambda$ are~\cite{PDG2014}:
\begin{spacing}{.8}
%{%
  %\setlength{\belowdisplayskip}{4pt}%
  %\setlength{\abovedisplayskip}{4pt}%
  \begin{align*}
  %\lambda &= 0.22537\pm0.00061, \nonumber\\\intertext{cat} A&=0.814\,^{+0.023}_{-0.024}.
    \lambda &= 0.22537\pm0.00061,
    \intertext{and}
    A&=0.814\,^{+0.023}_{-0.024}.
  \end{align*}
\end{spacing}
%}
So, \VCKM cannot be diagonal because $A\neq0$ and $\lambda\neq0$.
%it is clear that \VCKM is not diagonal, and
Therefore \glspl{FCNC}
%flavour-changing currents
are allowed in the \sm.
However, the diagonal elements are still the largest, meaning that
intra-generational interactions are preferred in the weak interaction and
the \ckm matrix exhibits a strongly hierarchic structure.
%However, the diagonal elements are close to unity and the CKM matrix exhibits a strong
%hierarchical structure, such that it is most probable that weak currents do not violate flavour.
%for which there is no explaination in the SM.


It has been asserted that the \ckm matrix is unitary, and therefore a
unitarity condition can be expressed as
$\Vconj{\alpha\beta}\V{\beta\gamma}=\delta_{\alpha\gamma}$.
%$V_{\alpha\beta}^{\phantom{\dagger}}V_{\beta\gamma}^* = \delta_{\alpha\gamma}$
%\begin{equation}
  %V_{\alpha\beta}^{\phantom{\dagger}}V_{\beta\gamma}^* = \delta_{\alpha\gamma},
  %\boldsymbol{V}_{ij}\boldsymbol{V}_{jk}^\dagger = \delta_{ik}.
  %\sum_{i=1}^3\left|V_{ij}\right|^2 = 1
  %\sum_{i=1}^3\left(V_{ij}^*V_{jk}\right) = \mathbb{1},
  %\label{eq:th:unitarity}
%\end{equation}
When $\delta_{\alpha\gamma}=0$, this condition gives six equations of the form:
\begin{align}
  %\sum_{i=1}^3V_{ij}^*V_{ki} &= 0 && \sum_{i=1}^3V_{ji}^*V_{ik} =0, & j&\neq k;
  \phantom{\beta\neq\gamma}
  &&\sum_{\beta=1}^3V_{\alpha\beta}^*V_{\beta\gamma}^{\phantom{*}} = 0,
  &&\sum_{\beta=1}^3V_{\alpha\beta}^{\phantom{*}}V_{\beta\gamma}^*=0,
  &&\alpha\neq\gamma;
  \label{eq:th:offdiag}
\end{align}
each mapping a closed triangle on the complex plane.
Two of these triangles have all sides of similar length
%One of these triangles, which has sides of similar length
$\big(\mathcal{O}(\lambda^3)\big)$;
one of these is
is known as \emph{the} \ut and is defined by
%Taking the equations of the triangles in Eq.~\ref{eq:th:unitarity} where all sides have length of
%$\mathcal{O}(\lambda^3)$ leaves two triangles, one of which is:
\begin{equation}
  %\V{ud}\Vconj{ub} + \V{cd}\Vconj{cb} + \V{td}\Vconj{tb} = 0.
  1 + \frac{\V{ud}\Vconj{ub}}{\V{cd}\Vconj{cb}} + \frac{\V{td}\Vconj{tb}}{\V{cd}\Vconj{cb}} = 0,
  \label{eq:th:ut}
\end{equation}
where the length of the base has been normalised to unity.
The apex of the \ut is at
\begin{align}
  %\bar\rho+i\bar\eta = (1-\tfrac12\lambda^2)(\rho+i\eta)
  \xbar\rho+i\xbar\eta &= \big(1-\tfrac12\lambda^2\big)\big(\rho+i\eta\big)
  \nonumber\\
  &=\frac{\V{ud}\Vconj{ub}}{\V{cd}\Vconj{cb}},
\end{align}
%If divided through by $\V{cd}\Vconj{cb}$, then Eq.~\ref{eq:th:ut} can be mapped onto the complex
%plane, where the apex is at $\bar\rho+i\bar\eta = (1-\tfrac12\lambda^2)(\rho+i\eta)$.
%and the angles are
and forms the angles
\begin{align}
  \alpha &= \arg\left(-\frac{\V{td}\Vconj{tb}}{\V{ud}\Vconj{ub}}\right), &
  \beta  &= \arg\left(-\frac{\V{cd}\Vconj{cb}}{\V{td}\Vconj{tb}}\right), &
  \gamma &= \arg\left(-\frac{\V{ud}\Vconj{ub}}{\V{cd}\Vconj{cb}}\right).
  %\alpha &=    \arg\left(-\frac{\V{td}\Vconj{tb}}{\V{ud}\Vconj{ub}}\right), \nonumber\\
  %\beta  &=\pi-\arg\left( \frac{\V{td}\Vconj{tb}}{\V{cd}\Vconj{cb}}\right), \nonumber\\
  %%& &\mathrm{and}
  %\gamma &=    \arg\left( \frac{\V{ud}\Vconj{ub}}{\V{cd}\Vconj{cb}}\right).
  \label{eq:ut:angles}
\end{align}
which define phase differences between edges.
Figure~\ref{fig:th:ut} depicts a schematic diagram of the \ut.
%are phases between CKM matrix elements.
%This triangle is depicted in \Fig{fig:th:ut}.
%, is simply a graphical representation of the CKM matrix.
%Measurements of CKM matrix elements constrain the angles, side lengths and apex of the UT, these
%constraints are also shown in Fig.~\ref{fig:th:ut}.

\begin{figure}
  \begin{center}
      \includegraphics[scale=1.1]{diagram_ut}
  \end{center}
  \caption[Schematic diagram of the Unitarity Triangle]
  {
    Schematic diagram of the UT given in Eq.~\protect\ref{eq:th:ut} on the complex
    plane, where the base has been normalised to unit length.
    The angles $\alpha$, $\beta$, and $\gamma$ are defined in Eq~\protect\ref{eq:ut:angles}.
  }
  \label{fig:th:ut}
\end{figure}


Each \ckm matrix element is a fundamental parameter in the \sm.
It is therefore important to
measure each of them; particularly because the \ckm matrix holds all
%Precise determination of the \ckm matrix elements is important, for  they are each fundamental
%parameters of the SM.
%They also contain all the information about flavour violation and \CPV that is allowed within the
the information about flavour violation and \CPV allowed within the
framework of the quark sector of the \sm.
All the measurements relating to the \ckm matrix can be shown in the \ut.
%Current measurements of angles and side lengths from \Ref{Charles:2015gya} are shown in
%\Fig{fig:th:ckmfitter}.








%\section{Physics beyond the Standard Model}
\label{sec:bsm}

For a long time the completion of the \sm was reliant on the discovery of the Higgs boson.
Finally, in 2012, the \cms~\cite{Chatrchyan:2008aa} and \atlas~\cite{Aad:2008zzm} detectors
observed a Higgs boson with a mass of $m_H\simeq125\gev$~\cite{Chatrchyan:2012ufa,Aad:2012tfa}.
This final piece of the picture has made the \sm a remarkably robust theory with no predictions
deviating significantly from experimental observations.
Indeed, the theory of \QED~--- which describes interactions between
photons and charged particles in the \sm --- is one of the most accurate theories yet constructed.
The coupling constant in \QED is the fine structure constant, $\alpha$, which has been measured
experimentally to be~\cite{PDG2012}
\begin{align}
  \alpha^{-1}_\mathrm{exp} &= 137.035\,999\,074\,(44), \nonumber\\
  \intertext{and predicted theoretically to be~\cite{Aoyama:2012wj}}
  \alpha^{-1}_\mathrm{th} &= 137.035\,999\,073\,(35). \nonumber
\end{align}
These measurements have precisions which are better than one part per billion.
%\begin{align}
  %\phantom{.} &&\alpha^{-1}_\mathrm{exp} &= 137.035\,999\,074\,(44), && \text{\cite{PDG2012}} \nonumber\\
  %\intertext{cats are the best}
  %\phantom{.} &&\alpha^{-1}_\mathrm{th} &= 137.035\,999\,073\,(35), && \text{\cite{Aoyama:2012wj}}
%\end{align}

Despite its countless successes, there are a plethora of indications --- both
experimental and theoretical --- that additional physics exists, \bsm.


\subsection{Failures and inconsistencies of the Standard Model}
\label{sec:bsm:fail}
%%%%%%%%%%%%%%%%%%%%%%%%%%%%%%%%%%%%%%%%%%%%%%%%%%%%%%%%%%%%%%%%%%%%%%%%%%%%%%%%%%%%%%%%%%%%%%%%%%%
% Experimental
%%%%%%%%%%%%%%%%%%%%%%%%%%%%%%%%%%%%%%%%%%%%%%%%%%%%%%%%%%%%%%%%%%%%%%%%%%%%%%%%%%%%%%%%%%%%%%%%%%%
There are some phenomena that have been observed experimentally which cannot be explained by the
\sm.
Oscillations of neutrinos in flavour space mean that they must have mass; this is not accounted for
the \sm framework.
Neither are the observations of the \BAU and \dm.

% CPV
The \sm cannot reconcile the matter-antimatter asymmetry observed in the
Universe today.
The hypothesized process which caused this asymmetry is known as baryogenesis.
Whatever this process may be it must satisfy the three Sakharov
conditions~\cite{1991SvPhU..34..392S}, which outline the minimum requirements for baryogenesis.
The first, most obvious, criteria is that baryogenesis must violate baryon number.
The second Sakharov condition is that both \gls{C} and \CP are violated.
Lastly, baryogenesis must occur out of thermal equilibrium.
While the \sm does contain \CPV, it is approximately ten orders of
magnitude~\cite{Cline:2006ts,Huet:1994jb} too small to explain the \BAU.
%Other pressing problems in the \sm are that it cannot reconcile massive neutrinos, gravity, or the
%amount of mass that is not accounted for in the Universe.
In \Chap{ch:dsphi} a measurement of the \CP-asymmetry in the decay \btodsphi is made in an effort
to explain the \BAU.


% DARK MATTER
It is well known that the vast majority of mass in the Universe is unaccounted for.
Luminous matter totals only \approx$4.9\pc$ of the Universe~\cite{Adam:2015rua,PDG2014}, and the rest
is known only as \dm (\approx$26.8\pc$) and dark energy (\approx$68.3\pc$).
Dark Matter is an old and well motivated concept with the first evidence found in 1939 by H.~W.~Babcock
in the form of flat galactic rotation curves~\cite{1970ApJ...159..379R,1980ApJ...238..471R}.
Since then, corroborating evidence from, for example, gravitational lensing around the Bullet
cluster~\cite{Markevitch:2003at}, and the Cosmic Microwave Background, have given further credence
to its existence.

The theory of \SUSY naturally supplies a \dm candidate in the shape of the lightest supersymmetric
particle, which is stable because of the imposed conservation of $R$-parity.
\SUSY is a theory which introduces an additional super-particle for each \sm fermion and
gauge boson, whose spin differs by a half integer.
The Higgs sector in \SUSY comprises four Higgs doublets; two are spin-0 and two are spin-$\tfrac12$,
and then there are two each for $Y=\pm\tfrac12$.
After \SUSY is broken there are five Higgs physical scalar particles, two are \CP-even ($h^0$,
$H^0$); one is a \CP-odd scalar ($A^0$) and two are charged ($H^\pm$).
%It also immediately solves the hierarchy problem because for every \sm particle that contributes to
%the Higgs mass, a \SUSY particle also contributes, but with the opposite sign.
Unfortunately, masses of the super-particles are unconstrained, and could be anywhere between a few
TeV and the Planck scale.

Observations of \dm can also be used to motivate \bsm models which include \emph{dark sectors}.
A dark sector is a name for a particle, or group of particles, which is gauged under a
different gauge group to the \sm particles and therefore cannot interact with them directly.
There are a plethora of such models, but generally dark particles can only interact with the \sm
via weakly interacting messenger particles, which could be either vector or scalar.
In generality, these are known as \emph{Dark Bosons}.

Some excitement was caused by a hint of a dark sector messenger particle from the Hyper-\CP
experiment~\cite{Burnstein:2004uk}, which observes three $\decay{\Sigma^+}{p\mumu}$ events which
survive a stringent selection.
These three events also peak in the invariant mass of the dimuon pair.
The narrowness of this peak is indicative of a two body decay, consistent with  $\decay{\Sigma^+}{pP^0}$
and the subsequent decay of the \np particle via $\decay{P^0}{\mumu}$, where
$m_{P^0}=214.3\pm0.5\mev$~\cite{Park:2005eka}.
The $P^0$ could be the supersymmetric Goldstino, or a dark boson from many other theories.

\glspl{FCNC} are heavily suppressed in the \sm.
Firstly, they are forbidden at tree level; secondly, loop level diagrams are suppressed by factors
coming from the \ckm matrix.
These low background signatures provide ideal environments in which to search for \bsm physics,
since new massive off-shell particles can contribute to the loops and cause significant deviations
from \sm expectations.

This thesis documents a search for a \np particle in the dimuon spectrum of \btokstrmumu in
\Chap{ch:db}.
Also detailed, in \Chap{db:hhh}, is an observation of a high statistics \fcnc decay,
which could be used for future \np searches.









%A hint at evidence for a \np particle comes from the Hyper-\CP experiment~\cite{Burnstein:2004uk}, which
%observes three $\decay{\Sigma^+}{p\mumu}$ events which survive a stringent selection.
%These three events also peak in the invariant mass of the dimuon pair.
%The narrowness of this peak is indicative of a two body decay, consistent with  $\decay{\Sigma^+}{pP^0}$
%and the subsequent decay of the \np particle via $\decay{P^0}{\mumu}$, where
%$m_{P^0}=214.3\pm0.5\mev$~\cite{Park:2005eka}.



%%%%%%%%%%%%%%%%%%%%%%%%%%%%%%%%%%%%%%%%%%%%%%%%%%%%%%%%%%%%%%%%%%%%%%%%%%%%%%%%%%%%%%%%%%%%%%%%%%%
% Theoretical
%%%%%%%%%%%%%%%%%%%%%%%%%%%%%%%%%%%%%%%%%%%%%%%%%%%%%%%%%%%%%%%%%%%%%%%%%%%%%%%%%%%%%%%%%%%%%%%%%%%
Theoretical shortcomings of the \sm include: its inability to incorporate gravity at the quantum
scale, the existence of dark energy.
However, theoretical reasons suggesting \bsm physics
are often rather subjective and revolve around the idea of \emph{naturalness}.
Naturalness is a concept whereby a theory is deemed to be natural, or more plausible, if it has few
free parameters, all of which have a magnitude $\mathcal{O}(1)$.
The \sm is not a natural theory.
For example: there are a total of 18 free parameters in the \sm, 13 of which reside in the flavour
sector.
The \ckm matrix is strongly hierarchic
%--- favouring flavour conserving weak interactions ---
and the quark masses vary by four orders of magnitude.

One of the fundamental parameters of the \sm  is \V{ub} and it is therefore important to accurately
measure it.
This parameter is particularly interesting because it is the source of the largest
tension in the \ut.
A determination of \V{ub} can be made using inclusive and exclusive measurements of semi-leptonic
$\decay{B}{X_u\ell\bar\nu_\ell}$ decays; where $X_u$ is some meson containing a \uquark quark.
Inclusive measurements are made difficult by large
$\decay{B}{X_c\ell\bar\nu_\ell}$ backgrounds, while exclusive semi-leptonic modes suffer from
theoretical uncertainties.
A value of $\left|\V{ub}\right|$ can also be obtained from the annihilation decay
$\decay{\Bp}{\taup\nu_\tau}$, but this suffers from low statistics.
Determinations of \V{ub} from these sources are:
\begin{align}
  &&\left|\V{ub}\right|_\mathrm{exc}
  &= \big(4.41\,^{+0.21}_{-0.23}\big)\e{-3}
  & \text{\cite{PDG2014}}& \nonumber\\
  &&\left|\V{ub}\right|_{\makebox[\widthof{$_\mathrm{exc}$}][l]{$_\mathrm{inc}$}}
  &= \big(3.28\pm{0.29}\big)\e{-3}
  & \text{\cite{PDG2014}}&\nonumber\\
  &&\left|\V{ub}\right|_{\makebox[\widthof{$_\mathrm{exc}$}][l]{$_{\tau\nu}$}}
  &= \big(4.22\pm{0.42}\big)\e{-3}  &
  \bam{Update} \text{\cite{PDG2012}}.&
  %%|\V{ub}| &= \big(4.22\pm{0.42}\big)\e{-3}  & \big(\decay{\Bp}{\taup\nu_\tau}\big)\text{\cite{{PhysRevD.88.031102}&\nonumber
  \label{eq:th:vub}
\end{align}
There are tensions between the inclusive and exclusive modes, which are expected to be identical,
and could therefore hint at a source of \np.
A more accurate value of \V{ub} from the decay $\decay{\Bp}{\taup\nu_\tau}$ might shed light on the
situation.
Current measurements of angles and side lengths of the \ut, from \Ref{Charles:2015gya}, are shown
in \Fig{fig:th:ckmfitter}.
This figure also shows
global \V{ub} measurements from the semi-leptonic and $\decay{\Bp}{\taup\nu_\tau}$
modes are shown alongside one another.
%measurements also shows current global fit of Limits on \ut measurements are shown in
%\Fig{fig:th:ckmfitter}, and

\begin{figure}
  \begin{center}
      \includegraphics[width=0.80\textwidth]{rhoeta_small_Vub}
  \end{center}
  \caption[Unitarity triangle and current constraints]
  {
    Diagram of the \ut with coloured bands indicating various constraints on
    side lengths, angles and position of the apex, which is taken from the CKMfitter group in
    Ref.~\protect\cite{Charles:2015gya}.
    The constraints on \V{ub} from the combination of inclusive and exclusive modes
    ($\left|\V{ub}\right|_\mathrm{SL}$) is given separately to a value obtained using
    $\BF\left(\decay{\Bp}{\taup\nu_\tau}\right)$, ($\left|\V{ub}\right|_{\tau\nu}$).
  }
  \label{fig:th:ckmfitter}
\end{figure}

Unnatural \np models with parameters that differ wildly in magnitude tend to
lead to parameters or processes that must cancel to absurdly
high precision in order to agree with experimental observations.
These precise cancellations are known as \emph{fine tuning}.
In the \sm, quantum loop corrections to the Higgs mass are of the order $10^{19}$
for $m_H\simeq125\gev$~\cite{Chatrchyan:2012ufa,Aad:2012tfa}.
This means that the cancellations required to result in a Higgs mass comparable to the masses of
the weak vector bosons must be exact to 17 orders of magnitude.
This instance of fine tuning is known as the \emph{hierarchy problem}.
A solution for the hierarchy problem would be to introduce \np particles, whose contributions to
loop level processes reduce the magnitude of fine tuning required to a level that might be deemed
acceptable.
The theory of \SUSY immediately solves the hierarchy problem because for every \sm particle that
contributes to the Higgs mass, a \SUSY particle also contributes, but with the opposite sign.

Fine tuning also appears in \QCD.
A gauge invariant term that can be added to \Lag{QCD} is
\begin{equation}
  \Lag{QCD}^\theta = \theta\frac{g^2}{32\pi^2}
  G_{\mu\nu}^\alpha\widetilde G^{\mu\nu}_\alpha,
  \label{eq:strongcp}
\end{equation}
where $\theta$ and $g$ are constants, and $\alpha$ indicates a sum over colours.
The operator $G_{\mu\nu}$ is the gluon field strength tensor, and
\begin{equation}
  \widetilde G^{\mu\nu}_\alpha = \frac12\varepsilon_{\mu\nu\rho\sigma}G^{\rho\sigma}_\alpha.
\end{equation}
Interactions in $\Lag{QCD}^\theta$ would conserve \gls{C} symmetry, but violate both \gls{P} and
\gls{T} conjugation~\cite{Peccei:2006as}.
Such symmetry violations contradict the observed properties of the strong
force.
Bounds placed on the value of the neutron dipole moment, $|d_n| <2.9\e{-26}\,\mathrm{ecm}$
(at 90\% CL)~\cite{Baker:2006ts} require $\theta$ to be very small,
$\theta<10^{-19}$~\cite{Crewther:PQref9}, when \emph{a priori} it could be in the range
$0<\theta<2\pi$.
This occurrence of fine tuning is referred to as the \emph{strong \CP problem}.

Despite the evidence for \bsm physics and the list of problems that must be solved, its precise
manifestation is unknown.
There are numerous theories concerning NP scenarios which seek to solve various problems.

%Some models have a \emph{dark} or \emph{hidden} sector which, apart from gravity, only
%communicates with the visible sector feebly via messenger particles.
%These messenger particles could potentially be observed after they decay into \sm particles after
%mixing with a $H$, $Z$, $\gamma$ or $\nu$.
A solution to the strong \CP problem is to introduce an additional chiral symmetry, such that
$\theta$, in \Eq{eq:strongcp}, becomes a field: the quanta of which are called \emph{axions}.
These axions could be the messenger particle between a dark and visible
sector~\cite{Peccei:2006as}.

%%%%%%%%%%%%%%%%%%%%%%%%%%%%%%%%%%%%%%%%%%%%%%%%%%%%%%%%%%%%%%%%%%%%%%%%%%%%%%%%%%%%%%%%%%%%%%%%%%
%%%%%%%%%%%%%%%%%%%%%%%%%%%%%%%%%%%%%%%%%%%%%%%%%%%%%%%%%%%%%%%%%%%%%%%%%%%%%%%%%%%%%%%%%%%%%%%%%%
%%%%%%%%%%%%%%%%%%%%%%%%%%%%%%%%%%%%%%%%%%%%%%%%%%%%%%%%%%%%%%%%%%%%%%%%%%%%%%%%%%%%%%%%%%%%%%%%%%




%%%%%%%%%%%%%%%%%%%%%%%%%%%%%%%%%%%%%%%%%%%%%%%%%%%%%%%%%%%%%%%%%%%%%%%%%%%%%%%%%%%%%%%%%%%%%%%%%%
%%%%%%%%%%%%%%%%%%%%%%%%%%%%%%%%%%%%%%%%%%%%%%%%%%%%%%%%%%%%%%%%%%%%%%%%%%%%%%%%%%%%%%%%%%%%%%%%%%
%%%%%%%%%%%%%%%%%%%%%%%%%%%%%%%%%%%%%%%%%%%%%%%%%%%%%%%%%%%%%%%%%%%%%%%%%%%%%%%%%%%%%%%%%%%%%%%%%%






Some searches look directly for evidence of NP, this is the case for the analysis detailed in
\Chap{ch:db}, where a new particle, \db, is searched for in the dimuon invariant mass spectrum of
\decay{\Bd}{\Kstarent\mumu} consistent with \decay{\db}{\mumu}.
This is sensitive to a range of models which predict a light particle with a mass in the range
$2m_\mu\lesssim m_\db\lesssim4000\mev$, such as the axion model.
It is also sensitive to the $P^0$ that was hinted at by the Hyper-\CP experiment.

Instead of counting on NP to behave in an expected way, it is possible to search in a model
independent manner by exploring general physics couplings.
To do this it is useful to introduce the \OPE~\cite{PhysRev.179.1499}.



%%\subsection{Uncertainties in theoretical predictions}
\section{Dealing with QCD}

For the branching fraction measurements discussed in this thesis, theoretical uncertainties from \QCD make
predictions difficult\footnote{
  The following section is based on Ref.~\cite{Pich:1998xt}.
}.
\QCD describes the interactions of colour charged particles (quarks and
gluons),
and exhibits two peculiarities: confinement, and asymptotic freedom.
%interactions between quarks via the strong force which is mediated by gluons.
Confinement means that over long distances (\approx$1\fm$)
the interaction strength of the strong force does not weaken --- unlike all other known
forces.
This means that as a quark is separated from others, there is enough energy in the gluon field to
create new quark-antiquark pairs, where the resulting bound states always have net zero colour
charge.
Free quarks cannot be seen over macroscopic distances,
and are instead observed as mesons, baryons, tetra-quarks~\cite{LHCb-PAPER-2014-014} or
even penta-quarks~\cite{LHCb-PAPER-2015-029}.
%but are seen to behave as free
%particles in deep inelastic scattering.
Asymptotic freedom means that forces between quarks become asymptotically weaker as the energy of
the system increases, and the distance decreases.

Predictions of $b$-hadron processes involving \QCD can also be made using an \EFT.
Despite the large mass of the \bquark quark with respect to $\Lambda_\mathrm{QCD}\simeq200\mev$,
the system can be treated perturbatively since $\alpha_\mathrm{QCD}(m_\bquark)$ is sufficiently
small.
This is known as a \HQET.
In contrast to an \EFT where the weak fields have been integrated out, in a \HQET
it is not possible to remove heavy quark contributions entirely because the \bquark quark
cannot decay without violating flavour number.
Essentially the $b$-hadron system is treated akin to a hydrogen atom, where the \bquark quark takes
the place of the nucleus, allowing for a highly simplified theoretical treatment, with corrections
of order $m_\bquark^{-1}$.

Despite the use of \gls{HQET}, the fact is that hadrons are inherently non-perturbative objects,
and so it is useful to make further assumptions.
An important supposition is that of \emph{factorisation}, which assumes that the short-distance,
process dependent, \QCD effects are separable from hadronization, the long distance effects.
Hadronization is very difficult to calculate with \QCD; for this reason \emph{form factors} are
used to empirically encapsulate the process.
Form factors must be measured experimentally and are the dominant source of uncertainty in hadronic
$B$ decays.
%calculation of $B$ mesons decaying into final states containing hadrons.

%The mass of the \bquark quark is sufficiently high that QCD calculations in $B$ decays can be made
%using peturbation theory.
%Furthermore, initial conditions can modelled using Heavy Quark Effective Theory (HQET), which
%essentially models a bound state of a heavy and light quark like a hydrogen atom, with the \bquark
%taking the role of the nucleus.
%However, this latter approximation breaks down at low energies where the hadron has an energy
%comparable to the mass of the \bquark quark.

%Another important assumption for QCD predictions is that of factorizability.
%A decay is factorizable if one can separate the initial, partonic, state from the hadronization of
%the final state quarks.
%Hadronization is very difficult to model, and therefore empirical models, encoded into
%form-factors, are used.
%It is these form-factors which are the dominant source of theoretical uncertainty.


%These oddities mean that QCD must be dealt with in different ways depending on the energy regime of
%interest.
%For high momentum interations the coupling strength, $\alpha_\mathrm{QCD}$ is small and the system
%can be dealt with using peturbation theory.
%But, for low momentum interactions $\alpha_\mathrm{QCD}$ increases because of the
%\emph{running} of the coupling.
%In the latter regime the system cannot be modelled with peturbation theory because it is not
%infrared safe and rather Lattice QCD must be used.
%Another inadequacy of peturbation theory is that it considers asymptotic states of quarks and
%gluons as free states, where in actuallity the physical states which are observed are hadrons.
%
%Difficulties with calculating QCD interactions leads to the necessity of form factors, which are
%empirical functions with parameters measured experimentally.
%These are
%
%
%\begin{itemize}
  %\item Heavy quark effective field theory simpolify calculations
    %interactions of the heacy wuark are soft cmopared tot the large mass of the $B$, and partonic
    %process is expanded in terms of $\Lambda_{QCD}/m_B$
  %\item QCD factorization setarates partonic process from the hadronisation of the $sq$ pair
  %\item Hadronisation is encoded into hadronic form-factors which is the dominant source of
    %uncertainty for theoretical predictions
%\end{itemize}









This thesis contains the work undertaken in three analyses; each of which concerns a different area
of interest in high energy physics.
The following chapter aims to motivate each analysis in turn after introducing the Standard Model
of particle physics.

Firstly, the formulation of the SM will be outlined, with particular detail paid to the flavour
sector.
Various successes of the SM will next be discussed before going on to identify its shortcomings
using arguments from both experiment and theory.
These shortcomings will then be used to motivate the three analyses:
a search for \btodsphi, which contains \V{ub} (\Chap{ch:dsphi});
a search for \btokpipimumu and \btophikmumu, which are flavour changing neutral currents
(\Chap{ch:hhh});
and seaching for dark sector particles in \btokstmumu (\Chap{ch:db}).
Theory specifically relating to each of these decays will be detailed in the relevant chapter.

%The following chapter will elucidate as to how the flavour sector is the only source of CPV in the
%SM.
%It will then go on to motivate study of flavour physics at \lhcb by exploring some problems with
%the SM as a theory.

\section{The Standard Model}
\label{sec:sm}
The behaviour of fundamental particles and forces are described by the \sm of
particle physics, which was concocted in the 1970s, when the Higgs
mechanism was incorporated into Glashow's electroweak theory by Salam and Weinberg.
The theory prescribes a treatment
as to how fundamental particles interact via three of the four
fundamental forces, namely: the strong, weak and electromagnetic forces.

Mathematically, the \sm is a locally gauge invariant quantum field theory.
It inhabits a space-time with a global Poincar\'e symmetry that obeys a local
$SU_C(3)\otimes SU_L(2)\otimes U_Y(1)$ symmetry\footnote{
The convention of natural units is used throughout.
Other conventions are that the indices $\mu$ and $\nu$ are used for four vectors, and generators
for the $SU_L(2)$ and $SU_C(2)$ are denoted by $\{i,j,k\}$ and $\{a,b,c\}$, respectively.}.
The $SU_L(2)\otimes U_Y(1)$ gauge group contains the electroweak formalism, and the $SU_C(3)$ group
contains that of the strong force.
%Each generator in this group corresponds to a spin-1 gauge-boson;
%the $SU_C(3)$ group is obeyed by the strong force in the theory of \QCD,
%while the electroweak sector obeys the $SU_L(2)\otimes U_Y(1)$ gauge group.
Generators for each group correspond to the vector bosons which mediate
interactions ---
the $3+1$ electroweak gauge bosons ($Z$, $W^\pm$ and the photon), and
the eight gluons of the strong force.
A summary of gauge bosons in the \sm is given in \Tab{tab:sm:gauge}.


\begin{table}
  \caption[Fundamental, force-mediating, gauge bosons]
  {
    Fundamental, force-mediating, gauge bosons in the \sm.
    All values are taken from Ref.~\protect\cite{PDG2014}.
  }
  \label{tab:sm:gauge}
  \begin{center}
    \begin{tabular}{ccrc}
      \toprule
      Force
      & Particle & \cellc{Mass}  & Charge\\
      \midrule
      Electromagnetic & $\gamma$ & $0\phantom{\gev}$ & $\phantom{-}0$ \\
      \multirow{2}{*}{Weak} & $W^\pm$ & $80.4\gev$ & $\pm1$ \\
      & $Z$ & $90.2\gev$ & $\phantom{-}0$ \\
      Strong & $G^a$ & $0\phantom{\gev}$ & $\phantom{-}0$ \\
      \bottomrule
    \end{tabular}
  \end{center}
\end{table}


Symmetries are fundamental to the dynamics of particle physics.
It was shown by Emmy Noether that for each symmetry in the action of a physical system there is a
conserved quantity~\cite{Noether}.
This is Noether's theorem.
For any sensible theory of physics, it is necessary that the laws remain the same independent of
time and space, these symmetries lead to the conservations of momentum and energy, respectively.
Discrete symmetries of important are \gls{T} and \gls{P}, so that physics occurs the same
regardless of the direction of time, and under reflections in space.

%Noether -> Symmetries
%C P T

Spin-$\tfrac12$ particles in the \sm are known as fermions, and
are described by spinor fields, $\psi$, and obey the Dirac equation:
\begin{equation}
  \big(i\gamma^\mu\partial_\mu - m\big)\psi = 0.
  \label{th:eq:dirac}
\end{equation}
The fermions of the \sm can be broadly categorized into those which couple to the strong force,
\emph{quarks}, and those which do not, \emph{leptons}.
There are six quarks: up, down, charm, strange, top, and bottom (\uquark, \dquark, \cquark,
\squark, \tquark and \bquark); and six leptons:
the electronic leptons ($\nu_e$, $e$),
the muonic leptons ($\nu_\mu$, $\mu$),
and the tauonic leptons ($\nu_\tau$, $\tau$).
Due to the properties of the strong force, quarks can only be observed as colour-neutral bound
states, usually these are
\emph{mesons} (quark-antiquark bound states) and \emph{baryons} (bound states of three quarks).
%the electron, muon, tau ($e$, $\mu$, $\tau$), and
%their corresponding neutrinos ($\nu_e$, $\nu_\mu$, $\nu_\tau$).
All fermions are organized into pairs forming three generations.
%The fermions of the \sm constitute six leptons (electron, electron neutrino, muon, muon neutrino,
%tau and tau neutrino) and six quarks ($u$p, $d$own, $c$harm, $s$trange, $t$op and $b$ottom), which
%are organized into pairs forming three generations.
For each fermion there is a corresponding antiparticle with the same mass and opposite charge ---
charge being the conserved quantity resulting from the global electroweak gauge symmetry by
Noether's theorem.
A summary of all fermions and some of their properties is given in \Tab{tab:sm:particles}.
There is also a single scalar field in the \sm: the Higgs boson.
%The only additional particle is the scalar Higgs boson.

\begin{table}
  \caption[Fundamental fermions of matter]{
    Fundamental fermions of matter.
    In the \sm each has a corresponding anti-particle of opposite charge.
    All values are taken from Ref.~\protect\cite{PDG2014}.
  }
  \label{tab:sm:particles}
  \begin{center}
    \begin{tabular}{ccrccrc}
      \toprule
      & \multicolumn{3}{c}{Leptons}
      & \multicolumn{3}{c}{Quarks} \\
      Generation
    & Particle & \cellc{Mass}  & Charge
    & Particle & \cellc{Mass}  & Charge\\
      \midrule
      \multirow{2}{*}{1} & \ep   & $0.511\mev$ & $-1$ & \uquark & $2.3\mev$ & $+^2/_3$ \\
      & \neue & $0\phantom{\mev}$  &  $\phantom{-}0$ & \dquark & $4.8\mev$ & $-^1/_3$ \\
      \multirow{2}{*}{2} & \mup   & $0.105\gev$ & $-1$ & \cquark & $95.0\mev$ & $+^2/_3$ \\
      & \neum & $0\phantom{\mev}$  &  $\phantom{-}0$ & \squark & $1.275\gev$ & $-^1/_3$ \\
      \multirow{2}{*}{3} & \taup   & $1.777\gev$ & $-1$ & \tquark & $173\gev$ & $+^2/_3$ \\
      & \neut & $0\phantom{\mev}$  &  $\phantom{-}0$ & \bquark & $4.18\gev$ & $-^1/_3$ \\
      \bottomrule
    \end{tabular}
  \end{center}
\end{table}



The \sm Lagrangian can be expressed as a sum of components:
\begin{equation}
  \Lag{SM} = \Lag{QCD} + \Lag{V} + \Lag{\ell} + \Lag{\it q} + \Lag{Higgs} + \Lag{Yuk}.
  \label{eq:th:lag}
\end{equation}
These Lagrangians describe the:
strong force interactions between colour carrying particles in the theory of \QCD
(\Lag{QCD});
weak vector self-interactions (\Lag{V});
electroweak behaviour of leptons (\Lag{\ell});
electroweak behaviour of quarks (\Lag{\it q});
Higgs interaction (\Lag{Higgs}); and
Yukawa couplings (\Lag{Yuk}).

As well as the discrete symmetries of \gls{P} and \gls{T}, there is also the \gls{C} symmetry.
The violation of these symmetries are of fundamental interest
to modern particle physics.
%It is expected that the combined $C\!PT$ symmetry is conserved, but other symmetries can be
%violated, measurements
%It is kk
%The combined \CP symmetry is
%The
%The latter three terms of \Eq{eq:th:lag} (\Lag{\it q}, (\Lag{Higgs}, and \Lag{Yuk}) are of
%fundamental importance as
%to how flavour, and \CP violation arise in the \sm, and will be discussed in detail.
%There are three fundamental discrete symmetries in the \sm: \gls{C}, \gls{P}, and \gls{T}.
Violation of the combined \CP symmetry, and flavour, arise in the \ckm matrix of the \sm, which emerges
after the Higgs mechanism breaks the local electroweak symmetry.
The important terms for this are \Lag{\it q}, (\Lag{Higgs}, and \Lag{Yuk} from \Eq{eq:th:lag}.

The Higgs doublet, $\Phi$, is defined to be
\begin{equation}
  \Phi = \frac{1}{\sqrt{2}}
  \begin{pmatrix}
    \phi_1 + i\phi_2 \\
    \phi_3 + i\phi_4 \\
  \end{pmatrix},
  \label{eq:th:phi}
\end{equation}
where each $\phi_i$ is a real field.
The Lagrangian of the Higgs field is:
\begin{align}
  \Lag{Higgs}
  &= \big(D_\mu\Phi\big)^\dagger\big(D^\mu\Phi\big) - V\big(\Phi\big) \nonumber\\
  &= \big(D_\mu\Phi\big)^\dagger\big(D^\mu\Phi\big) - \mu^2\big(\Phi^\dagger\Phi\big) +
  \lambda\big(\Phi^\dagger\Phi\big)^{2},
  \label{eq:th:laghiggs1}
\end{align}
where $\mu$ and $\lambda$ are constants, and $D_\mu$ is the covariant derivative.
Figure~\ref{fig:th:higgspot} shows that
taking $\mu^2<0$ and $\lambda>0$ shifts the ground state of the vacuum of $V(\Phi)$ away from zero.
When the system collapses in to the ground state a direction is chosen, this breaks the symmetry of
the system.
%as shown in \Fig{fig:th:higgspot},
%shifts the minimum of the potential $V(\Phi)$ away from zero by a
The amount by which the ground state shifts with respect to the origin is
\begin{equation}
  v = \sqrt{\frac{\mu^2}{\lambda}}.
\end{equation}
At this point the Higgs field gets a \VEV of $\langle\phi\rangle = v/\sqrt{2}$.
%$\braket{\phi} = \tfrac{1}{\sqrt{2}}v$.
The direction of the \VEV from the origin is arbitrary, but the choice of
\begin{align}
  \bra{0}\phi_1\ket{0} =
  \bra{0}\phi_2\ket{0} =
  \bra{0}\phi_4\ket{0} &= 0 \nonumber\\
  \bra{0}\phi_3\ket{0} &= v,
\end{align}
is convenient, and changes the Higgs doublet in \Eq{eq:th:phi} to
\begin{equation}
  \Phi = \frac{1}{\sqrt{2}}
  \begin{pmatrix}
    \eta_1 + i\eta_2 \\
    v + i\eta_4 \\
  \end{pmatrix}.
  \label{eq:th:eta}
\end{equation}
Here, $\eta_1$, $\eta_2$ and $\eta_4$, are Goldstone bosons which, by choosing an appropriate
gauge, become the longitudinal components of the weak vector bosons, and $\Phi$ simplifies to
\begin{equation}
  \Phi = \frac{1}{\sqrt{2}}
  \begin{pmatrix} 0 \\ v+H
  \end{pmatrix},
  \label{eq:th:phi2}
\end{equation}
where $H$ is the physical Higgs boson.
Inserting Eq.~\ref{eq:th:phi2} into Eq.~\ref{eq:th:laghiggs1} gives:
%\begin{multline}
  %\Lag{Higgs} =
  %\frac12\left(\partial_\mu H\right)\left(\partial^\mu H\right)
  %+\frac14g^2\left(v^2 + 2vH + H^2\right)W_\mu^+W^{-\mu}  \\
  %+\frac18\left(g^2 + g^{\prime2}\right)\left(v^2 + 2vH + H^2\right)Z_\mu Z^\mu
  %+ \mu^2H^2 + \frac14\lambda\left(H^4+4vH^3\right) + \cdots,
  %\label{eq:th:laghiggs2}
%\end{multline}
\begin{equation}
  \Lag{Higgs} =
  \frac{1}{2}\big(\partial_\mu H\big)\big(\partial^\mu H\big)
  %+\mu^2H^2
  +\frac{m_H^2}{2}H^2
  +\left(m_W^2W_\mu^+W^{-\mu} + \frac{m_Z^2}{2}Z_\mu Z^\mu\right)
  %\cdot
  \left(1 + \frac{H}{v}\right)^{\!2}.
  \label{eq:th:laghiggs2}
\end{equation}
%where $g$ and $g^\prime$ are coupling constants and other terms are three- and four-point
%interactions of the Higgs with itself and weak gauge bosons.
Thus, there is \SSB of the local $U_Y(1)$ gauge group.
A result of this is that weak gauge bosons become massive, while photons remain massless: as is
consistent with observations.

\begin{figure}
  \begin{center}
    \includegraphics[width=0.6\textwidth]{higgs_potential}
    \caption[Shape of the Higgs potential]
    {
      The shape of the Higgs potential, $V(\Phi)$, for the simple case of $\Phi=\phi_1+i\phi_2$ and
      $\mu^2<0$ and $\lambda>0$.
      Spontaneous symmetry breaking occurs when the vacuum settles in a minima, and this choice of
      direction breaks the symmetry of the gauge.
      A section of the potential is not shown, to convey the shape of the potential.
    }
    \label{fig:th:higgspot}
  \end{center}
\end{figure}

All fermions (excepting neutrinos) also acquire mass after \SSB.
The Dirac mass term for a chiral field has the form:
\begin{equation}
  \Lag{mass} = -m_\psi\big(\xbar\psi_R\psi_L + \xbar\psi_L\psi_R\big),
  \label{eq:diracmass}
\end{equation}
but the left- and right-handed fields
($\psi_L$ and $\psi_R$) have different $U_Y(1)$ charges and so transform differently.
Using \Eq{eq:diracmass} to give masses to fermions would therefore break gauge invariance.
%and so cannot be added to \Lag{SM} without destroying the gauge invariance, hence \Eq{eq:diracmass}
%cannot be used.
Instead, masses are generated through the Yukawa couplings, which
describe interactions between all fermionic fields and the Higgs doublet.
This can be written
\begin{equation}
  %\Lag{Yuk} = \sum_{\substack{\ell=\\e,\mu,\tau}}\big(\Lag{Yuk}^\ell\big) + \Lag{Yuk}^q,
  \Lag{Yuk} = \sum_{\ell}\big(\Lag{Yuk}^\ell\big) + \Lag{Yuk}^q,
  \label{eq:th:yukking}
\end{equation}
where terms encapsulate lepton and quark interactions, respectively.
%where $\ell$ and $q$ denote the lepton and quark interactions, respectively.
Each lepton term describes the interaction between the Higgs boson and the chiral fields
$\ell_R$ and the spinor
\begin{align}
  \chi_L = \begin{pmatrix}\nu_L \\ \ell_L \end{pmatrix},
\end{align}
via
\begin{equation}
  \Lag{Yuk}^\ell
  = - g_\ell\big(\xbar\chi_L\Phi \ell_R + \xbar \ell_R\Phi^\dagger\chi_L\big),
\end{equation}
where each $g_\ell$ is a coupling constant.
After \SSB the Lagrangian becomes
\begin{align}
  \Lag{Yuk}^\ell
  = - m_\ell \big(\xbar\ell_L\ell_R + \xbar\ell_R\ell_L\big)%\cdot
  \left(1 + \frac{H}{v}\right)
\end{align}
and the lepton masses
\begin{equation}
  m_\ell = \frac{v}{\sqrt{2}}g_\ell
  \label{eq:leptonmass}
\end{equation}
are dependent on the fundamental parameters $g_\ell$ and $v$.

Yukawa interactions for quarks involve the right-handed chiral operators of the up- and down-type
quarks, $q_R^i$ for $q\in\{u,d\}$, and the left-handed doublet
\begin{equation}
  Q_L^i = \begin{pmatrix}u^i_L\\d^i_L\end{pmatrix}.
\end{equation}
Before \SSB, the Yukawa Lagrangian is
\begin{equation}
  \Lag{Yuk}^q = - y_{ij}^u\Xbar{Q}_L^i\Phi u_R^j
  - y_{ij}^d\Xbar{Q}_L^i\widetilde\Phi d_R^j +\,\mathrm{h.c.}
  \label{eq:th:lagyukq}
\end{equation}
where $\widetilde\Phi_i = \varepsilon_{ij}\Phi_k$, there is an implicit sum over the generations
$i$ and $j$, and the shorthand h.c.~denotes hermitian conjugate.
The coupling constants, $y^{q}$, are $3\times3$ matrices characterizing Yukawa coupling strengths
between generations.
After \SSB, $\Lag{Yuk}^q$ becomes:
\begin{equation}
  \Lag{Yuk}^q =
  - \frac{v}{\sqrt{2}}
  \left(
  y_{ij}^\uquark\uquarkbar_L^i\uquark_R^j
  + y_{ij}^\dquark\dquarkbar_L^i\dquark_R^j
  +\,\mathrm{h.c.}
  \right)
  %\cdot
  \left(1 + \frac{H}{v}\right).
  \label{eq:th:lagyuk2}
\end{equation}
Similar to lepton masses, in \Eq{eq:leptonmass}, quark masses are defined as
\begin{align}
  m_{ij}^u &= \frac{v}{\sqrt{2}}y_{ij}^u \nonumber\\
  m_{ij}^d &= \frac{v}{\sqrt{2}}y_{ij}^d.
\end{align}
Thus far, the flavour basis has been used, but it is now more convenient to change to the mass
basis, in which the matrices $m^{u,d}$ are diagonal., it is more convenient to change to a basis in which the matrix $m^q$
is diagonal, using the rotation matrices $V_L$ and $V_R$, such that
\begin{align}
  {m_{il}^{u}}^\prime &=  \big(V_L^{u\dagger}\big)_{ij} m_{jk}^u\big(V_R^u\big)_{kl} \nonumber\\
  {m_{il}^{d}}^\prime &=  \big(V_L^{d\dagger}\big)_{ij} m_{jk}^d\big(V_R^d\big)_{kl}.
\end{align}
The addition of a prime distinguishes the mass basis from the flavour basis.
This transformation is exactly equivalent to transforming the up- and
down-type chiral quark fields according to:
\begin{align}
  q_L^\prime &= \big(V_L^q\big)q_{L}^{} \nonumber\\
  q_R^\prime &= \big(V_R^q\big)q_{R}^{}.
\end{align}
Applying these transformations to all parts of \Lag{SM} leaves the majority of it unchanged, since
$V_{L}^{q\dagger} V_{L}^{q} = V_{R}^{q\dagger} V_{R}^{q} = \mathds{1}$ by definition.
%$m_{ij}^\mathrm{diag} = V_{Lik}m_{kl}(V_R^\dagger)_{lj}$.
%This is exactly equivalent to transforming the chiral quark fields for up- and down-type quarks
%accordingly:
%\begin{align}
  %q_L^\alpha = \left(V_L^q\right)_{\alpha i}q_L &&
  %q_R^\alpha = \left(V_R^q\right)_{\alpha i}q_R,
%\end{align}
%where the index of the original basis is identified with $i$ and the mass basis uses $\alpha$.
%The rotations of the basis of the chiral quark fields leave much of \Lag{SM} unchanged since
%$V_{qL}^\dagger V_{qL} = V_{qR}^\dagger V_{qR} = \mathbb{1}$.
However, this is not the case for \Lag{\it q}.

The Lagrangian $\Lag{\it q}$ can be decomposed into
$\Lag{\it q}^\mathrm{NC}+\Lag{\it q}^\mathrm{CC}$, where the
superscripts denote \NC and charged current \CC components.
The \NC part of the Lagrangian characterizes interactions between quarks and the neutral electroweak
vector bosons, while the \CC part involves the interactions of quarks wiht the charged $W^\pm$
bosons.
After changing to the mass basis, \Lag{NC} remains unchanged, whereas \Lag{CC}
transforms as:
\begin{align}
  \Lag{\it q}^\mathrm{CC}
  &= i\frac{g}{2}\gamma^\mu
  \bigg[\uquarkbar_Ld_LW^+_\mu + \dquarkbar_Lu_LW^-_\mu
  \bigg]  \nonumber\\
  &= i\frac{g}{2}\gamma^\mu
  \bigg[
    \uquarkbar_L^\prime\left(V_{uL}^{\phantom{\dagger}}V_{dL}^\dagger\right)d_L^{\,\prime} W^+_\mu +
    \dquarkbar_L^{\,\prime}\left(V_{dL}^{\phantom{\dagger}}V_{uL}^\dagger\right)u_L^\prime W^-_\mu
  \bigg]  \nonumber\\
  &= i\frac{g}{2}\gamma^\mu
  \bigg[
    \VCKM\uquarkbar_L^\prime d_L^{\,\prime} W^+_\mu +
    \VCKMconj\dquarkbar_L^{\,\prime} u_L^\prime W^-_\mu
  \bigg].
\end{align}
In the final step the matrix
$\VCKM=V^{\phantom{\dagger}}_{uL}V_{dL}^\dagger$ is defined.
This is known as the \ckm
matrix, and parameterizes the couplings between up- and down-type quarks in charged weak currents.




\subsection{The CKM matrix and Unitarity Triangle}
\label{sec:ckm}

The \ckm matrix is defined as:
\begin{equation}
  \VCKM = \left(V_{uL}^{\phantom{\dagger}} V_{dL}^\dagger\right) =
  \begin{pmatrix}
    \V{ud} & \V{us} & \V{ub} \\
    \V{cd} & \V{cs} & \V{cb} \\
    \V{td} & \V{ts} & \V{tb} \\
  \end{pmatrix},
\end{equation}
where each $|V_{ij}|$ parameterises the probability of an up-type quark, of generation $i$,
transitioning to down-type quark, of generation $j$, in a weak interaction.
In the \sm, it is assumed that the total charged current couplings of up- to down-type quarks is the
same as down- to up-type.
This assumption means that the \ckm matrix is unitary, $\VCKMconj\VCKM = \mathds{1}$, and therefore
it contains only four physical parameters: three angles ($\theta_{12}$, $\theta_{13}$ and
$\theta_{23}$) and one complex phase ($\delta$).
In fact, the observation of \CPV in kaon mixing~\cite{Christenson:1964fg}
led to the prediction of a third generation before
its discovery, precisely because a $3\times3$ matrix is the smallest necessary for a phase to enter
a unitary matrix.

%preferentially
%hierarchichal
%arbiatry

%The CKM matrix is the source of all flavour violation in the SM.
%In the SM, it is assumed that the total charged current couplings of up- to down-type quarks is the
%same as down- to up-type.
%This means that the CKM matrix is unitary, $V^\dagger V = \mathbb{1}$, and therefore it contains
%four physical parameters: three angles ($\theta_{12}$, $\theta_{13}$ and $\theta_{23}$) and one
%complex phase ($\delta$).

%\begin{equation}
  %\VCKM =
  %\begin{pmatrix}
    %\cxx{12}\cxx{13} & \sxx{12}\cxx{13} & \sxx{13}e^{-i\delta} \\
    %-\sxx{12}\cxx{23}-\cxx{12}\sxx{23}\sxx{13}e^{i\delta} &
    %\cxx{12}\cxx{23}-\sxx{12}\sxx{23}\sxx{13}e^{i\delta} & \sxx{23}\cxx{13} \\
    %\sxx{12}\sxx{23}-\cxx{12}\cxx{23}\sxx{13}e^{i\delta} &
    %-\cxx{12}\sxx{23}-\sxx{12}\cxx{23}\sxx{13}e^{i\delta} & \cxx{23}\cxx{13} \\
  %\end{pmatrix}
%\end{equation}

There are many ways of representing the \ckm matrix.
One way is as a product of three rotation
matrices, one of which contains the complex phase, this is known as the \emph{standard}
parameterisation:
\begin{equation}
  \VCKM =
  \begin{pmatrix}
    \cxx{12}\cxx{13} & \sxx{12}\cxx{13} & \sxx{13}e^{-i\delta} \\
    -\sxx{12}\cxx{23}-\cxx{12}\sxx{23}\sxx{13}e^{i\delta} &
    \cxx{12}\cxx{23}-\sxx{12}\sxx{23}\sxx{13}e^{i\delta} & \sxx{23}\cxx{13} \\
    \sxx{12}\sxx{23}-\cxx{12}\cxx{23}\sxx{13}e^{i\delta} &
    -\cxx{12}\sxx{23}-\sxx{12}\cxx{23}\sxx{13}e^{i\delta} & \cxx{23}\cxx{13} \\
  \end{pmatrix},
\end{equation}
where $\sxx{ij}$ and $\cxx{ij}$ denote $\sin\theta_{ij}$ and $\cos\theta_{ij}$, respectively.
A convenient simplification is the \emph{Wolfenstein} parameterisation, which is obtained by
defining
\begin{align}
  \sin\theta_{12}&=\lambda, \nonumber\\
  \sin\theta_{23}&=A\lambda^2, \nonumber\\
  \intertext{and}
  e^{-i\delta}\sin\theta_{13} &= A\lambda^3(\rho-i\eta),
\end{align}
which results in
\begin{align}
  \VCKM\simeq
  \begin{pmatrix}
      %1-\tfrac12\lambda & \lambda & A\lambda^3(\rho-i\eta+\tfrac{i}2\eta\lambda^2) \\
      %-\lambda & 1-\lambda^2-i\eta A^2\lambda^4 & A\lambda^2(1+i\eta\lambda^2) \\
      %A\lambda^3(1-\rho-i\eta) & -A\lambda^2 & 1 \\
    1-\tfrac12\lambda & \lambda & A\lambda^3\big(\rho-i\eta\big) \\
    -\lambda & 1-\lambda^2 & A\lambda^2 \\
    A\lambda^3\big(1-\rho-i\eta\big) & -A\lambda^2 & 1 \\
  \end{pmatrix}.
  \label{eq:th:wolfenstein}
\end{align}
The values of the Wolfenstein parameters $A$ and $\lambda$ are~\cite{PDG2014}:
\begin{spacing}{.8}
%{%
  %\setlength{\belowdisplayskip}{4pt}%
  %\setlength{\abovedisplayskip}{4pt}%
  \begin{align*}
  %\lambda &= 0.22537\pm0.00061, \nonumber\\\intertext{cat} A&=0.814\,^{+0.023}_{-0.024}.
    \lambda &= 0.22537\pm0.00061,
    \intertext{and}
    A&=0.814\,^{+0.023}_{-0.024}.
  \end{align*}
\end{spacing}
%}
So, \VCKM cannot be diagonal because $A\neq0$ and $\lambda\neq0$.
%it is clear that \VCKM is not diagonal, and
Therefore \glspl{FCNC}
%flavour-changing currents
are allowed in the \sm.
However, the diagonal elements are still the largest, meaning that
intra-generational interactions are preferred in the weak interaction and
the \ckm matrix exhibits a strongly hierarchic structure.
%However, the diagonal elements are close to unity and the CKM matrix exhibits a strong
%hierarchical structure, such that it is most probable that weak currents do not violate flavour.
%for which there is no explaination in the SM.


It has been asserted that the \ckm matrix is unitary, and therefore a
unitarity condition can be expressed as
$\Vconj{\alpha\beta}\V{\beta\gamma}=\delta_{\alpha\gamma}$.
%$V_{\alpha\beta}^{\phantom{\dagger}}V_{\beta\gamma}^* = \delta_{\alpha\gamma}$
%\begin{equation}
  %V_{\alpha\beta}^{\phantom{\dagger}}V_{\beta\gamma}^* = \delta_{\alpha\gamma},
  %\boldsymbol{V}_{ij}\boldsymbol{V}_{jk}^\dagger = \delta_{ik}.
  %\sum_{i=1}^3\left|V_{ij}\right|^2 = 1
  %\sum_{i=1}^3\left(V_{ij}^*V_{jk}\right) = \mathbb{1},
  %\label{eq:th:unitarity}
%\end{equation}
When $\delta_{\alpha\gamma}=0$, this condition gives six equations of the form:
\begin{align}
  %\sum_{i=1}^3V_{ij}^*V_{ki} &= 0 && \sum_{i=1}^3V_{ji}^*V_{ik} =0, & j&\neq k;
  \phantom{\beta\neq\gamma}
  &&\sum_{\beta=1}^3V_{\alpha\beta}^*V_{\beta\gamma}^{\phantom{*}} = 0,
  &&\sum_{\beta=1}^3V_{\alpha\beta}^{\phantom{*}}V_{\beta\gamma}^*=0,
  &&\alpha\neq\gamma;
  \label{eq:th:offdiag}
\end{align}
each mapping a closed triangle on the complex plane.
Two of these triangles have all sides of similar length
%One of these triangles, which has sides of similar length
$\big(\mathcal{O}(\lambda^3)\big)$;
one of these is
is known as \emph{the} \ut and is defined by
%Taking the equations of the triangles in Eq.~\ref{eq:th:unitarity} where all sides have length of
%$\mathcal{O}(\lambda^3)$ leaves two triangles, one of which is:
\begin{equation}
  %\V{ud}\Vconj{ub} + \V{cd}\Vconj{cb} + \V{td}\Vconj{tb} = 0.
  1 + \frac{\V{ud}\Vconj{ub}}{\V{cd}\Vconj{cb}} + \frac{\V{td}\Vconj{tb}}{\V{cd}\Vconj{cb}} = 0,
  \label{eq:th:ut}
\end{equation}
where the length of the base has been normalised to unity.
The apex of the \ut is at
\begin{align}
  %\bar\rho+i\bar\eta = (1-\tfrac12\lambda^2)(\rho+i\eta)
  \xbar\rho+i\xbar\eta &= \big(1-\tfrac12\lambda^2\big)\big(\rho+i\eta\big)
  \nonumber\\
  &=\frac{\V{ud}\Vconj{ub}}{\V{cd}\Vconj{cb}},
\end{align}
%If divided through by $\V{cd}\Vconj{cb}$, then Eq.~\ref{eq:th:ut} can be mapped onto the complex
%plane, where the apex is at $\bar\rho+i\bar\eta = (1-\tfrac12\lambda^2)(\rho+i\eta)$.
%and the angles are
and forms the angles
\begin{align}
  \alpha &= \arg\left(-\frac{\V{td}\Vconj{tb}}{\V{ud}\Vconj{ub}}\right), &
  \beta  &= \arg\left(-\frac{\V{cd}\Vconj{cb}}{\V{td}\Vconj{tb}}\right), &
  \gamma &= \arg\left(-\frac{\V{ud}\Vconj{ub}}{\V{cd}\Vconj{cb}}\right).
  %\alpha &=    \arg\left(-\frac{\V{td}\Vconj{tb}}{\V{ud}\Vconj{ub}}\right), \nonumber\\
  %\beta  &=\pi-\arg\left( \frac{\V{td}\Vconj{tb}}{\V{cd}\Vconj{cb}}\right), \nonumber\\
  %%& &\mathrm{and}
  %\gamma &=    \arg\left( \frac{\V{ud}\Vconj{ub}}{\V{cd}\Vconj{cb}}\right).
  \label{eq:ut:angles}
\end{align}
which define phase differences between edges.
Figure~\ref{fig:th:ut} depicts a schematic diagram of the \ut.
%are phases between CKM matrix elements.
%This triangle is depicted in \Fig{fig:th:ut}.
%, is simply a graphical representation of the CKM matrix.
%Measurements of CKM matrix elements constrain the angles, side lengths and apex of the UT, these
%constraints are also shown in Fig.~\ref{fig:th:ut}.

\begin{figure}
  \begin{center}
      \includegraphics[scale=1.1]{diagram_ut}
  \end{center}
  \caption[Schematic diagram of the Unitarity Triangle]
  {
    Schematic diagram of the UT given in Eq.~\protect\ref{eq:th:ut} on the complex
    plane, where the base has been normalised to unit length.
    The angles $\alpha$, $\beta$, and $\gamma$ are defined in Eq~\protect\ref{eq:ut:angles}.
  }
  \label{fig:th:ut}
\end{figure}


Each \ckm matrix element is a fundamental parameter in the \sm.
It is therefore important to
measure each of them; particularly because the \ckm matrix holds all
%Precise determination of the \ckm matrix elements is important, for  they are each fundamental
%parameters of the SM.
%They also contain all the information about flavour violation and \CPV that is allowed within the
the information about flavour violation and \CPV allowed within the
framework of the quark sector of the \sm.
All the measurements relating to the \ckm matrix can be shown in the \ut.
%Current measurements of angles and side lengths from \Ref{Charles:2015gya} are shown in
%\Fig{fig:th:ckmfitter}.








\section{Physics beyond the Standard Model}
\label{sec:bsm}

For a long time the completion of the \sm was reliant on the discovery of the Higgs boson.
Finally, in 2012, the \cms~\cite{Chatrchyan:2008aa} and \atlas~\cite{Aad:2008zzm} detectors
observed a Higgs boson with a mass of $m_H\simeq125\gev$~\cite{Chatrchyan:2012ufa,Aad:2012tfa}.
This final piece of the picture has made the \sm a remarkably robust theory with no predictions
deviating significantly from experimental observations.
Indeed, the theory of \QED~--- which describes interactions between
photons and charged particles in the \sm --- is one of the most accurate theories yet constructed.
The coupling constant in \QED is the fine structure constant, $\alpha$, which has been measured
experimentally to be~\cite{PDG2012}
\begin{align}
  \alpha^{-1}_\mathrm{exp} &= 137.035\,999\,074\,(44), \nonumber\\
  \intertext{and predicted theoretically to be~\cite{Aoyama:2012wj}}
  \alpha^{-1}_\mathrm{th} &= 137.035\,999\,073\,(35). \nonumber
\end{align}
These measurements have precisions which are better than one part per billion.
%\begin{align}
  %\phantom{.} &&\alpha^{-1}_\mathrm{exp} &= 137.035\,999\,074\,(44), && \text{\cite{PDG2012}} \nonumber\\
  %\intertext{cats are the best}
  %\phantom{.} &&\alpha^{-1}_\mathrm{th} &= 137.035\,999\,073\,(35), && \text{\cite{Aoyama:2012wj}}
%\end{align}

Despite its countless successes, there are a plethora of indications --- both
experimental and theoretical --- that additional physics exists, \bsm.


\subsection{Failures and inconsistencies of the Standard Model}
\label{sec:bsm:fail}
%%%%%%%%%%%%%%%%%%%%%%%%%%%%%%%%%%%%%%%%%%%%%%%%%%%%%%%%%%%%%%%%%%%%%%%%%%%%%%%%%%%%%%%%%%%%%%%%%%%
% Experimental
%%%%%%%%%%%%%%%%%%%%%%%%%%%%%%%%%%%%%%%%%%%%%%%%%%%%%%%%%%%%%%%%%%%%%%%%%%%%%%%%%%%%%%%%%%%%%%%%%%%
There are some phenomena that have been observed experimentally which cannot be explained by the
\sm.
Oscillations of neutrinos in flavour space mean that they must have mass; this is not accounted for
the \sm framework.
Neither are the observations of the \BAU and \dm.

% CPV
The \sm cannot reconcile the matter-antimatter asymmetry observed in the
Universe today.
The hypothesized process which caused this asymmetry is known as baryogenesis.
Whatever this process may be it must satisfy the three Sakharov
conditions~\cite{1991SvPhU..34..392S}, which outline the minimum requirements for baryogenesis.
The first, most obvious, criteria is that baryogenesis must violate baryon number.
The second Sakharov condition is that both \gls{C} and \CP are violated.
Lastly, baryogenesis must occur out of thermal equilibrium.
While the \sm does contain \CPV, it is approximately ten orders of
magnitude~\cite{Cline:2006ts,Huet:1994jb} too small to explain the \BAU.
%Other pressing problems in the \sm are that it cannot reconcile massive neutrinos, gravity, or the
%amount of mass that is not accounted for in the Universe.
In \Chap{ch:dsphi} a measurement of the \CP-asymmetry in the decay \btodsphi is made in an effort
to explain the \BAU.


% DARK MATTER
It is well known that the vast majority of mass in the Universe is unaccounted for.
Luminous matter totals only \approx$4.9\pc$ of the Universe~\cite{Adam:2015rua,PDG2014}, and the rest
is known only as \dm (\approx$26.8\pc$) and dark energy (\approx$68.3\pc$).
Dark Matter is an old and well motivated concept with the first evidence found in 1939 by H.~W.~Babcock
in the form of flat galactic rotation curves~\cite{1970ApJ...159..379R,1980ApJ...238..471R}.
Since then, corroborating evidence from, for example, gravitational lensing around the Bullet
cluster~\cite{Markevitch:2003at}, and the Cosmic Microwave Background, have given further credence
to its existence.

The theory of \SUSY naturally supplies a \dm candidate in the shape of the lightest supersymmetric
particle, which is stable because of the imposed conservation of $R$-parity.
\SUSY is a theory which introduces an additional super-particle for each \sm fermion and
gauge boson, whose spin differs by a half integer.
The Higgs sector in \SUSY comprises four Higgs doublets; two are spin-0 and two are spin-$\tfrac12$,
and then there are two each for $Y=\pm\tfrac12$.
After \SUSY is broken there are five Higgs physical scalar particles, two are \CP-even ($h^0$,
$H^0$); one is a \CP-odd scalar ($A^0$) and two are charged ($H^\pm$).
%It also immediately solves the hierarchy problem because for every \sm particle that contributes to
%the Higgs mass, a \SUSY particle also contributes, but with the opposite sign.
Unfortunately, masses of the super-particles are unconstrained, and could be anywhere between a few
TeV and the Planck scale.

Observations of \dm can also be used to motivate \bsm models which include \emph{dark sectors}.
A dark sector is a name for a particle, or group of particles, which is gauged under a
different gauge group to the \sm particles and therefore cannot interact with them directly.
There are a plethora of such models, but generally dark particles can only interact with the \sm
via weakly interacting messenger particles, which could be either vector or scalar.
In generality, these are known as \emph{Dark Bosons}.

Some excitement was caused by a hint of a dark sector messenger particle from the Hyper-\CP
experiment~\cite{Burnstein:2004uk}, which observes three $\decay{\Sigma^+}{p\mumu}$ events which
survive a stringent selection.
These three events also peak in the invariant mass of the dimuon pair.
The narrowness of this peak is indicative of a two body decay, consistent with  $\decay{\Sigma^+}{pP^0}$
and the subsequent decay of the \np particle via $\decay{P^0}{\mumu}$, where
$m_{P^0}=214.3\pm0.5\mev$~\cite{Park:2005eka}.
The $P^0$ could be the supersymmetric Goldstino, or a dark boson from many other theories.

\glspl{FCNC} are heavily suppressed in the \sm.
Firstly, they are forbidden at tree level; secondly, loop level diagrams are suppressed by factors
coming from the \ckm matrix.
These low background signatures provide ideal environments in which to search for \bsm physics,
since new massive off-shell particles can contribute to the loops and cause significant deviations
from \sm expectations.

This thesis documents a search for a \np particle in the dimuon spectrum of \btokstrmumu in
\Chap{ch:db}.
Also detailed, in \Chap{db:hhh}, is an observation of a high statistics \fcnc decay,
which could be used for future \np searches.









%A hint at evidence for a \np particle comes from the Hyper-\CP experiment~\cite{Burnstein:2004uk}, which
%observes three $\decay{\Sigma^+}{p\mumu}$ events which survive a stringent selection.
%These three events also peak in the invariant mass of the dimuon pair.
%The narrowness of this peak is indicative of a two body decay, consistent with  $\decay{\Sigma^+}{pP^0}$
%and the subsequent decay of the \np particle via $\decay{P^0}{\mumu}$, where
%$m_{P^0}=214.3\pm0.5\mev$~\cite{Park:2005eka}.



%%%%%%%%%%%%%%%%%%%%%%%%%%%%%%%%%%%%%%%%%%%%%%%%%%%%%%%%%%%%%%%%%%%%%%%%%%%%%%%%%%%%%%%%%%%%%%%%%%%
% Theoretical
%%%%%%%%%%%%%%%%%%%%%%%%%%%%%%%%%%%%%%%%%%%%%%%%%%%%%%%%%%%%%%%%%%%%%%%%%%%%%%%%%%%%%%%%%%%%%%%%%%%
Theoretical shortcomings of the \sm include: its inability to incorporate gravity at the quantum
scale, the existence of dark energy.
However, theoretical reasons suggesting \bsm physics
are often rather subjective and revolve around the idea of \emph{naturalness}.
Naturalness is a concept whereby a theory is deemed to be natural, or more plausible, if it has few
free parameters, all of which have a magnitude $\mathcal{O}(1)$.
The \sm is not a natural theory.
For example: there are a total of 18 free parameters in the \sm, 13 of which reside in the flavour
sector.
The \ckm matrix is strongly hierarchic
%--- favouring flavour conserving weak interactions ---
and the quark masses vary by four orders of magnitude.

One of the fundamental parameters of the \sm  is \V{ub} and it is therefore important to accurately
measure it.
This parameter is particularly interesting because it is the source of the largest
tension in the \ut.
A determination of \V{ub} can be made using inclusive and exclusive measurements of semi-leptonic
$\decay{B}{X_u\ell\bar\nu_\ell}$ decays; where $X_u$ is some meson containing a \uquark quark.
Inclusive measurements are made difficult by large
$\decay{B}{X_c\ell\bar\nu_\ell}$ backgrounds, while exclusive semi-leptonic modes suffer from
theoretical uncertainties.
A value of $\left|\V{ub}\right|$ can also be obtained from the annihilation decay
$\decay{\Bp}{\taup\nu_\tau}$, but this suffers from low statistics.
Determinations of \V{ub} from these sources are:
\begin{align}
  &&\left|\V{ub}\right|_\mathrm{exc}
  &= \big(4.41\,^{+0.21}_{-0.23}\big)\e{-3}
  & \text{\cite{PDG2014}}& \nonumber\\
  &&\left|\V{ub}\right|_{\makebox[\widthof{$_\mathrm{exc}$}][l]{$_\mathrm{inc}$}}
  &= \big(3.28\pm{0.29}\big)\e{-3}
  & \text{\cite{PDG2014}}&\nonumber\\
  &&\left|\V{ub}\right|_{\makebox[\widthof{$_\mathrm{exc}$}][l]{$_{\tau\nu}$}}
  &= \big(4.22\pm{0.42}\big)\e{-3}  &
  \bam{Update} \text{\cite{PDG2012}}.&
  %%|\V{ub}| &= \big(4.22\pm{0.42}\big)\e{-3}  & \big(\decay{\Bp}{\taup\nu_\tau}\big)\text{\cite{{PhysRevD.88.031102}&\nonumber
  \label{eq:th:vub}
\end{align}
There are tensions between the inclusive and exclusive modes, which are expected to be identical,
and could therefore hint at a source of \np.
A more accurate value of \V{ub} from the decay $\decay{\Bp}{\taup\nu_\tau}$ might shed light on the
situation.
Current measurements of angles and side lengths of the \ut, from \Ref{Charles:2015gya}, are shown
in \Fig{fig:th:ckmfitter}.
This figure also shows
global \V{ub} measurements from the semi-leptonic and $\decay{\Bp}{\taup\nu_\tau}$
modes are shown alongside one another.
%measurements also shows current global fit of Limits on \ut measurements are shown in
%\Fig{fig:th:ckmfitter}, and

\begin{figure}
  \begin{center}
      \includegraphics[width=0.80\textwidth]{rhoeta_small_Vub}
  \end{center}
  \caption[Unitarity triangle and current constraints]
  {
    Diagram of the \ut with coloured bands indicating various constraints on
    side lengths, angles and position of the apex, which is taken from the CKMfitter group in
    Ref.~\protect\cite{Charles:2015gya}.
    The constraints on \V{ub} from the combination of inclusive and exclusive modes
    ($\left|\V{ub}\right|_\mathrm{SL}$) is given separately to a value obtained using
    $\BF\left(\decay{\Bp}{\taup\nu_\tau}\right)$, ($\left|\V{ub}\right|_{\tau\nu}$).
  }
  \label{fig:th:ckmfitter}
\end{figure}

Unnatural \np models with parameters that differ wildly in magnitude tend to
lead to parameters or processes that must cancel to absurdly
high precision in order to agree with experimental observations.
These precise cancellations are known as \emph{fine tuning}.
In the \sm, quantum loop corrections to the Higgs mass are of the order $10^{19}$
for $m_H\simeq125\gev$~\cite{Chatrchyan:2012ufa,Aad:2012tfa}.
This means that the cancellations required to result in a Higgs mass comparable to the masses of
the weak vector bosons must be exact to 17 orders of magnitude.
This instance of fine tuning is known as the \emph{hierarchy problem}.
A solution for the hierarchy problem would be to introduce \np particles, whose contributions to
loop level processes reduce the magnitude of fine tuning required to a level that might be deemed
acceptable.
The theory of \SUSY immediately solves the hierarchy problem because for every \sm particle that
contributes to the Higgs mass, a \SUSY particle also contributes, but with the opposite sign.

Fine tuning also appears in \QCD.
A gauge invariant term that can be added to \Lag{QCD} is
\begin{equation}
  \Lag{QCD}^\theta = \theta\frac{g^2}{32\pi^2}
  G_{\mu\nu}^\alpha\widetilde G^{\mu\nu}_\alpha,
  \label{eq:strongcp}
\end{equation}
where $\theta$ and $g$ are constants, and $\alpha$ indicates a sum over colours.
The operator $G_{\mu\nu}$ is the gluon field strength tensor, and
\begin{equation}
  \widetilde G^{\mu\nu}_\alpha = \frac12\varepsilon_{\mu\nu\rho\sigma}G^{\rho\sigma}_\alpha.
\end{equation}
Interactions in $\Lag{QCD}^\theta$ would conserve \gls{C} symmetry, but violate both \gls{P} and
\gls{T} conjugation~\cite{Peccei:2006as}.
Such symmetry violations contradict the observed properties of the strong
force.
Bounds placed on the value of the neutron dipole moment, $|d_n| <2.9\e{-26}\,\mathrm{ecm}$
(at 90\% CL)~\cite{Baker:2006ts} require $\theta$ to be very small,
$\theta<10^{-19}$~\cite{Crewther:PQref9}, when \emph{a priori} it could be in the range
$0<\theta<2\pi$.
This occurrence of fine tuning is referred to as the \emph{strong \CP problem}.

Despite the evidence for \bsm physics and the list of problems that must be solved, its precise
manifestation is unknown.
There are numerous theories concerning NP scenarios which seek to solve various problems.

%Some models have a \emph{dark} or \emph{hidden} sector which, apart from gravity, only
%communicates with the visible sector feebly via messenger particles.
%These messenger particles could potentially be observed after they decay into \sm particles after
%mixing with a $H$, $Z$, $\gamma$ or $\nu$.
A solution to the strong \CP problem is to introduce an additional chiral symmetry, such that
$\theta$, in \Eq{eq:strongcp}, becomes a field: the quanta of which are called \emph{axions}.
These axions could be the messenger particle between a dark and visible
sector~\cite{Peccei:2006as}.

%%%%%%%%%%%%%%%%%%%%%%%%%%%%%%%%%%%%%%%%%%%%%%%%%%%%%%%%%%%%%%%%%%%%%%%%%%%%%%%%%%%%%%%%%%%%%%%%%%
%%%%%%%%%%%%%%%%%%%%%%%%%%%%%%%%%%%%%%%%%%%%%%%%%%%%%%%%%%%%%%%%%%%%%%%%%%%%%%%%%%%%%%%%%%%%%%%%%%
%%%%%%%%%%%%%%%%%%%%%%%%%%%%%%%%%%%%%%%%%%%%%%%%%%%%%%%%%%%%%%%%%%%%%%%%%%%%%%%%%%%%%%%%%%%%%%%%%%




%%%%%%%%%%%%%%%%%%%%%%%%%%%%%%%%%%%%%%%%%%%%%%%%%%%%%%%%%%%%%%%%%%%%%%%%%%%%%%%%%%%%%%%%%%%%%%%%%%
%%%%%%%%%%%%%%%%%%%%%%%%%%%%%%%%%%%%%%%%%%%%%%%%%%%%%%%%%%%%%%%%%%%%%%%%%%%%%%%%%%%%%%%%%%%%%%%%%%
%%%%%%%%%%%%%%%%%%%%%%%%%%%%%%%%%%%%%%%%%%%%%%%%%%%%%%%%%%%%%%%%%%%%%%%%%%%%%%%%%%%%%%%%%%%%%%%%%%






Some searches look directly for evidence of NP, this is the case for the analysis detailed in
\Chap{ch:db}, where a new particle, \db, is searched for in the dimuon invariant mass spectrum of
\decay{\Bd}{\Kstarent\mumu} consistent with \decay{\db}{\mumu}.
This is sensitive to a range of models which predict a light particle with a mass in the range
$2m_\mu\lesssim m_\db\lesssim4000\mev$, such as the axion model.
It is also sensitive to the $P^0$ that was hinted at by the Hyper-\CP experiment.

Instead of counting on NP to behave in an expected way, it is possible to search in a model
independent manner by exploring general physics couplings.
To do this it is useful to introduce the \OPE~\cite{PhysRev.179.1499}.



\subsection{Operator Product Expansion}

%The most pessimistic solution to the flavour problem is Minimal Flavour Violation (MFV)
%which simply assumes that beyond SM physics follows a Yukawa coupling like structure in the flavour
%sector, this would lead to no discernible new physics in the flavour sector.

Effective field theories are based on the premise that they processes are defined by the typical
energy scale, $\mu$, at which they operate.
Contributions from particles with very high mass, much greater than $\mu$ are suppressed.
An equally valid interpretation is that massive, unstable particles have short lifetimes and can
only interact over very small distances.
Processes operating at different energy scales are separated both spatially and temporally; and are
therefore decoupled.

Effective field theories allow processes to be modelled at a scale relevant to the particles
involved.
This is often applied to weak interactions of the \bquark quark.
Instead of calculating \bquark decays using the full range of processes outlined by the SM, the
particles who contribute at scales above $m_\bquark$ are integrated out, above a scale $\Lambda$,
leaving the physics that is most relevant at the scale $\mu\simeq m_\bquark$.
Here, processes are dominated by QCD, but higher scale effects still contribute.
A canonical example of this is Fermi's effective theory of weak decays which predates electroweak
theory; in it, a process (such as $\beta$-decay) is collapsed into a four point interaction.


The most general way to parameterize the full effective theory is with an effective Hamiltonian
describing generic
interactions at an energy scale $\mu$, in which long and short distance effects are separated.
Long distance (equivalently low energy) effects are described by coefficients, $c$, which can be
calculated using perturbative methods.
Short distance (or high energy) effects are characterized by terms of operators, $\mathcal{O}$,
which must be calculated non-peturbatively because they contain QCD interactions.
The resulting effective Hamiltonian includes a sum over all processes, $i$, which contribute at a
given dimension, $d$:
\begin{equation}
  \bra{f}\Ham{full}\ket{i} =
  \left.\sum_{d}\frac{1}{\Lambda^{d-4}}
  \sum_i c_i^{(d)}\bra{f}\Op{i}^{(d)}\ket{i}\right|_{\Lambda}.
\end{equation}




%The effective Hamiltonian for the FCNC $\decay{b}{s\ell^+\ell^-}$ is given by:
%\begin{equation}
  %\Ham{eff} = -4\frac{G_F}{\sqrt{2}}\Vconj{ts}\V{tb}\frac{e^2}{16\pi^2}
  %\sum_{i}\big(c_i(\mu)\mathcal{O}_i(\mu)+c_i(\mu)^\prime\mathcal{O}_i^\prime(\mu)\big)
%\end{equation}
%similar to \Eq{eq:th:lageff}, where the coefficients are known as Wilson coefficients, which
%correspond to the Wilson operators $\mathcal{O}_{1-10}$.
%The operators $\mathcal{O}_{1-6}$ are sensitive to long distance contributions such as \ccbar
%loops.
%Operators which are particularly sensitive to NP contributions in \decay{b}{s\mumu} transitions are
%\begin{align}
  %\Op{7\pz} &= \frac{m_b}{e}\big(\bar s \sigma_{\mu\nu}P_Rb\big)F^{\mu\nu}
  %&\Op{7\pz}^\prime &= \frac{m_b}{e}\big(\bar s \sigma_{\mu\nu}P_Lb\big)F^{\mu\nu}
  %\nonumber\\
  %%\mathcal{O}_8 &= g\frac{m_b}{e^2}\big(\bar s \sigma_{\mu\nu}T^aP_Rb\big)G^{\mu\nu a}
  %%&\mathcal{O}_8^\prime &= g\frac{m_b}{e^2}\big(\bar s \sigma_{\mu\nu}T^aP_Rb\big)G^{\mu\nu a}
  %%\\
  %\Op{9\pz} &= \big(\bar s\gamma_\mu P_Lb\big)\big(\bar\ell\gamma^\mu\ell\big)
  %&\Op{9\pz}^\prime &= \big(\bar s\gamma_\mu P_Rb\big)\big(\bar\ell\gamma^\mu\ell\big)
  %\nonumber\\
  %\Op{10} &= \big(\bar s\gamma_\mu P_Lb\big)\big(\bar\ell\gamma^\mu\gamma_5\ell\big)
  %&\Op{10}^\prime &= \big(\bar s\gamma_\mu P_Rb\big)\big(\bar\ell\gamma^\mu\gamma_5\ell\big)
  %\phantom{\frac{1}{1}}
  %%\\
  %%\mathcal{O}_{S} &= \frac{m_b}{m_{B_s}}\big(\bar s\gamma_\mu P_Rb\big)\big(\bar\ell\ell\big)
  %%\\
  %%\mathcal{O}_{P} &= \frac{m_b}{m_{B_s}}\big(\bar s\gamma_\mu P_Rb\big)\big(\bar\ell\gamma_5\ell\big)
%\end{align}
%where $P_{L,R}$ are the left and right projection operators.
%The operators \Op{7} and \Op{9} describe the emission of a photon or $Z$ from a penguin loop,
%and \Op{10} corresponds to a box type diagram with \Wp; these are shown in \Fig{fig:hhh:loops}.
%Primed operators are the suppressed helicity, whose contributions are vanishingly small in the SM.

%\begin{figure}
  %\begin{center}
    %\includegraphics[scale=1]{feynman_btosmumu_penguin}
    %\includegraphics[scale=1]{feynman_btosmumu_box}
    %\caption[Schematic Feynman diagrams for loop and box diagrams]
    %{\small
      %Schematic Feynman diagrams for the
      %(left) penguin loop diagrams corresponding to the operators \Op{7} and \Op{9} depending on
      %whether a $\gamma$ or $Z$ is emitted from the loop;
      %(right) \Op{10} box diagram mediated by \Wp bosons.
    %}
    %\label{fig:hhh:loops}
  %\end{center}
%\end{figure}

%As well as in loops, virtual particles can also contribute in some tree level diagrams.
%The small value of $|\V{ub}|$ means that annihilation decays of \Bp mesons are heavily
%suppressed in the SM.
%%They are mediated by
%%Tree level diagrams in the SM are typically high statistics modes, however annihilation type decays
%%are heavily suppressed.
%These rare modes are propagated by a \Wp in the SM; but this could be exchanged for any charged
%boson, such as an $H^+$ from SUSY, this could alter the branching fraction or
%introduce significant CPV.
%%The decay \btodsphi is an annihilation decay of the \Bp.

%New physics models must be able to accommodate Dark Matter.
%Some models have a \emph{dark} or \emph{hidden} sector which, apart from gravity, only
%communicates with the visible sector feebly via messenger particles.
%These messenger particles could potentially be observed after they decay into SM particles after
%mixing with a $H$ or $Z$.
%The axion could be such a particle, as could the inflaton or dark $Z$; models including these
%particles are further detailed in \Sec{sec:db:intro}.
%
%%Dark Matter is the lightest supersymmetric particle which is stable and messenger particle is super
%%goldstino
%Supersymmetry (SUSY) is a theory which introduces an additional super-particle for each SM fermion and
%gauge boson, whose spin is different by a half integer.
%The Higgs sector in SUSY comprises four Higgs doublets; two are spin-0 and two are spin-$\tfrac12$,
%and then there are two each for $Y=\pm\tfrac12$.
%After SUSY is broken there are five Higgs physical scalar particles, two are CP-even ($h^0$,
%$H^0$) one is CP-odd scalar ($A^0$) and two charged are charged ($H^\pm$).
%Supersymmetry supplies a Dark Matter candidate in the shape of the lightest supersymmetric
%particle, which could communicate with the visible sector via a super-golstino.
%It also immediately solves the hierarchy problem.
%The masses of the super-particles are unconstrained, and could be anywhere between a few TeV and
%the Planck scale.
%%The theory of SUSY immediately solves the hierarcy problem and provide a candidate for Dark Matter
%%in the shape of the lightest supersymmetric particle.
%%Super-goldstino
%%The messenger particles are super goldtinos that arise from the breaking of the symmetry,
%%However, the scale of the masses of the new particles are undefined, meaning
%%they could appear anywhere from a few TeV up to the Planck scale.
%
%
%The most pessimistic solution to the flavour problem is Minimal Flavour Violation (MFV)
%which simply assumes that beyond SM physics follows a Yukawa coupling like structure in the flavour
%sector, this would lead to no discernible new physics in the flavour sector.
%Assuming that nature has not chosen MFV, then contradictions from flavour problem indicate that NP
%searches should be made for both: particles
%with high mass, and particles which have small coupling strengths.
%The \lhcb experiment can probe the mass scale, since precision measurements of tree and loop
%diagrams are sensitive to virtual particles contributing at all orders, whose on-shell mass could
%be many TeV.
%The relatively high luminosity of interactions supplied by the \lhc mean that \lhcb is also
%sensitive to low coupling strengths, such as messenger particles from the dark sector.
%The following chapters explore a variety of different searches for beyond Standard Model physics.
%



%\subsection{Uncertainties in theoretical predictions}
\section{Dealing with QCD}

For the branching fraction measurements discussed in this thesis, theoretical uncertainties from \QCD make
predictions difficult\footnote{
  The following section is based on Ref.~\cite{Pich:1998xt}.
}.
\QCD describes the interactions of colour charged particles (quarks and
gluons),
and exhibits two peculiarities: confinement, and asymptotic freedom.
%interactions between quarks via the strong force which is mediated by gluons.
Confinement means that over long distances (\approx$1\fm$)
the interaction strength of the strong force does not weaken --- unlike all other known
forces.
This means that as a quark is separated from others, there is enough energy in the gluon field to
create new quark-antiquark pairs, where the resulting bound states always have net zero colour
charge.
Free quarks cannot be seen over macroscopic distances,
and are instead observed as mesons, baryons, tetra-quarks~\cite{LHCb-PAPER-2014-014} or
even penta-quarks~\cite{LHCb-PAPER-2015-029}.
%but are seen to behave as free
%particles in deep inelastic scattering.
Asymptotic freedom means that forces between quarks become asymptotically weaker as the energy of
the system increases, and the distance decreases.

Predictions of $b$-hadron processes involving \QCD can also be made using an \EFT.
Despite the large mass of the \bquark quark with respect to $\Lambda_\mathrm{QCD}\simeq200\mev$,
the system can be treated perturbatively since $\alpha_\mathrm{QCD}(m_\bquark)$ is sufficiently
small.
This is known as a \HQET.
In contrast to an \EFT where the weak fields have been integrated out, in a \HQET
it is not possible to remove heavy quark contributions entirely because the \bquark quark
cannot decay without violating flavour number.
Essentially the $b$-hadron system is treated akin to a hydrogen atom, where the \bquark quark takes
the place of the nucleus, allowing for a highly simplified theoretical treatment, with corrections
of order $m_\bquark^{-1}$.

Despite the use of \gls{HQET}, the fact is that hadrons are inherently non-perturbative objects,
and so it is useful to make further assumptions.
An important supposition is that of \emph{factorisation}, which assumes that the short-distance,
process dependent, \QCD effects are separable from hadronization, the long distance effects.
Hadronization is very difficult to calculate with \QCD; for this reason \emph{form factors} are
used to empirically encapsulate the process.
Form factors must be measured experimentally and are the dominant source of uncertainty in hadronic
$B$ decays.
%calculation of $B$ mesons decaying into final states containing hadrons.

%The mass of the \bquark quark is sufficiently high that QCD calculations in $B$ decays can be made
%using peturbation theory.
%Furthermore, initial conditions can modelled using Heavy Quark Effective Theory (HQET), which
%essentially models a bound state of a heavy and light quark like a hydrogen atom, with the \bquark
%taking the role of the nucleus.
%However, this latter approximation breaks down at low energies where the hadron has an energy
%comparable to the mass of the \bquark quark.

%Another important assumption for QCD predictions is that of factorizability.
%A decay is factorizable if one can separate the initial, partonic, state from the hadronization of
%the final state quarks.
%Hadronization is very difficult to model, and therefore empirical models, encoded into
%form-factors, are used.
%It is these form-factors which are the dominant source of theoretical uncertainty.


%These oddities mean that QCD must be dealt with in different ways depending on the energy regime of
%interest.
%For high momentum interations the coupling strength, $\alpha_\mathrm{QCD}$ is small and the system
%can be dealt with using peturbation theory.
%But, for low momentum interactions $\alpha_\mathrm{QCD}$ increases because of the
%\emph{running} of the coupling.
%In the latter regime the system cannot be modelled with peturbation theory because it is not
%infrared safe and rather Lattice QCD must be used.
%Another inadequacy of peturbation theory is that it considers asymptotic states of quarks and
%gluons as free states, where in actuallity the physical states which are observed are hadrons.
%
%Difficulties with calculating QCD interactions leads to the necessity of form factors, which are
%empirical functions with parameters measured experimentally.
%These are
%
%
%\begin{itemize}
  %\item Heavy quark effective field theory simpolify calculations
    %interactions of the heacy wuark are soft cmopared tot the large mass of the $B$, and partonic
    %process is expanded in terms of $\Lambda_{QCD}/m_B$
  %\item QCD factorization setarates partonic process from the hadronisation of the $sq$ pair
  %\item Hadronisation is encoded into hadronic form-factors which is the dominant source of
    %uncertainty for theoretical predictions
%\end{itemize}







\section{The Flavour Problem and Minimal Flavour Violation}

The most general way to parameterize NP is with an effective Lagrangian describing generic
interactions at an energy scale $\mu$, in which long and shory distance effects are separated.
Long distance (equivalently low energy) effects are described by coefficients, $c$, which can be
calculated using perturbative methods.
Short distance (high energy) effects are parameterized in terms of operators, $\mathcal{O}$,
which must be calculated non-peturbativly because they contain QCD interactions.
The resulting effective Lagrangian includes a sum over all processes, $i$, which contribute at a
given dimension, $d$:
\begin{equation}
  \Lag{eff}
  =
  \Lag{SM} + \sum_d\frac1{\Lambda^{d-4}}
  + \sum_ic_i^{(d)}\mathcal{O}_i^{(d)}.
\end{equation}
Processes that cause FCNCs contribute in $d=6$, and can be wiritten as
\begin{equation}
  \Delta\mathcal{L}^\mathrm{FCNC}
  =
  \sum_{i\neq j}\frac{c_{ij}}{\Lambda^2}
  \left(\Xbar{\mathcal{O}}_{Li}\gamma^\mu\mathcal{O}_{Lj}\right)^2,
\end{equation}
where $c_{ij}$ are dimensionless FCNC couplings, where $i$ and $j$ are different quark generations.
Bounds set on the energy scale $\Lambda$ by the analysis in \Ref{Isidori:2010kg} are:
\begin{equation}
  \Lambda > \frac{|c_{ij}|^\frac12}{|\Vconj{ti}\V{tj}|}\times4.4\tev
  \sim
  \left\{
    \begin{array}{l}
      %1.3\e{4}\tev\times |c_{sd}|^\frac12 \\
      %5.1\e{2}\tev\times |c_{bd}|^\frac12 \\
      %1.1\e{2}\tev\times |c_{bs}|^\frac12 \\
      |c_{sd}|^\frac12\times 1.3\e{4}\tev \\
      |c_{bd}|^\frac12\times 5.1\e{2}\tev \\
      |c_{bs}|^\frac12\times 1.1\e{2}\tev \\
    \end{array}\right.
\end{equation}
These values are calculated under the assumption that NP has a natural flavour structure, where
$c_{ij}\approx\mathcal{O}(1)$.
So, either couplings are of order unity and NP begins to contribute at over $100\tev$; or couplings
strengths are $\mathrm{O}(10^{-5})$ and NP contributes at the $1\tev$ level;
it is expected for NP to appear at the $1\tev$ scale in order to solve the hierarcy problem.
Either way, there is a conflict between the most natural coupling and energy scale; this is known
as the flavour problem.
%So must conclude that NP has a highly non-generic flacour structure.

A solution to the flavour problem is the Minimal Flavour Violation (MFV) hypoethesis.
This is the, somewhat pessimistic, view that the flavour structure of NP mirrors that of the Yukawa
couplings; leading to NP flavour changing transitions to be hidden by those seen in the SM.




\section{Searching for New Physics}


\clearpage



%Another indication of physics beyond the SM is that the Higgs mass ($\sim125\gev$) is close to the
%mass of the $W$ and $Z$ bosons.
%This means that loop level corrections.
%New physics contributing to these loops with the opposite sign to SM processes decrease the amount
%of fine tuning required.
%Particles entering into Higgs mass correction loops will enter in the same way into other loop
%processes.

%Particles from NP models con contribute

%As previously stated, there are no vertices exhibiting flavour changing neutral currents (FCNCs) in the SM.
%However, FCNCs can occur in loops diagrams.
%Because particles in these loops can be virtual, such processes have access to much higher mass
%scales than the energy supplied by the collision (by the Pauli exclusion principle).
%Therefore contributions from very massive particles can contribute...
%These particles will contribute to the amplitude of the overall interaction, such that loop level
%interactions...

%\section{The Flavour problem}
%It is natural to assume that the SM is only valid up to some energy scale $\Lambda$, where the
%value of the cut-off is undetermined.
%Using the argument of the hierarchy problem suggests that $\Lambda$ should be less than a few \tev.
%\bam{Motivates FCNCs}
%Beyond SM physics can be characterized by a general effective Lagrangian:
%\begin{equation}
  %\Lag{eff} = \sum_{n>2} \frac1{\Lambda^n}C_i\matho_i.
%\end{equation}
%Here, $\matho_i$ is an effective operator parameterizing the genera

%Flavour changing neutral currents are forbidden, GIM...

%It is unknown how NP will manifest itself...

%This figure shows two, slightly conflicting, constraints for the side
%$|\V{ud}\Vconj{ub}|/|\V{cd}\Vconj{cb}|$ coming from different measurements of \V{ub}.
%One of these is from inclusive measurements of \decay{B}{X_u\ell\nu_\ell}, and the other is from
%This figure shows that the angle $\beta$ and
%\begin{figure}
  %\begin{center}
    %\includegraphics[width=0.5\textwidth]{rhoeta_small_Vub}
  %\end{center}
  %\caption[Unitary triangle measurements]{\small
    %Constraints on the UT in the complex plane. The $|\V{ub}|$ constraint is split into two
    %contributions: inclusive and exclusive semileptonic decays (plain dark green)
    %and $|\V{ub}|$ from \decay{\Bp}{\tau\nu_\tau} (hashed green).
    %The red hashed region of the global combination corresponds to 68\% CL.
  %}
  %\label{fig:th:ut2}
%\end{figure}

%\section{Testing the flavour sector of the SM}
%There are convincing/powerful/... reasons that

%If new physics were to enter as a CPV phase in the flavour sector of the SM, this would be seen in
%measurements of the angles and sides of the UT.
%The total particle physics Lagrangian can be written as:
%\begin{equation}
  %\Lag{Tot} = \Lag{SM} + \Lag{NP}.
%\end{equation}
%Additional CP-violating phases in \Lag{NP} would cause the triangle not to close, such that
%$\alpha+\beta+\gamma\neq\pi$.
%Therefore it is vital to measure the CKM matrix elements to as high a precision as possible in as
%many ways as possible.
%The current constraints on the UT are shown in Fig.~\ref{fig:th:ut2}.

%Diagonal elements of the CKM matrix are all known to high fractional precision...
