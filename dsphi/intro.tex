%%spell
The decay \btodsphi proceeds via the annihilation of a the constituent quarks in the \Bp meson into
a virtual \Wp boson, as shown in \Fig{fig:dsphi:feyn}.
These types of decays are rare in the SM due to the magnitude of $|\V{ub}|$ (see \Eq{eq:th:vub});
in fact, no fully hadronic decays proceeding via annihilation have yet been observed.

Branching fraction predictions for \btodsphi are in the range (1--7)$\e{-7}$ in the SM
\cite{Zou:2009zza,Mohanta:2002wf,PhysRevD.76.057701,Lu:2001yz},

As discussed in \Sec{sec:th:bsm:man}, there is much interest in t

Previous experimental limit by \babar \cite{Aubert:2005gd}
\begin{equation}
  \BF\big(\btodsphi\big) < 1.8\e{-6} \qquad \text{90\,\% C.L.}
\end{equation}


In the SM the decay is mediated by a \Wp gauge boson.

\begin{figure}
  \begin{center}
    %\includegraphics[width=0.48\textwidth]{feynman_dsphi_sm}
    %\includegraphics[width=0.48\textwidth]{feynman_dsphi_susy}
    \includegraphics[scale=1]{feynman_dsphi_sm}
    \includegraphics[scale=1]{feynman_dsphi_susy}
    \caption[Feynman diagram for the decay \btodsphi]
    {\small
      A Feynman diagram for the decay \btodsphi being mediated by a
      (left) \Wp in the SM, and
      (right) $H^+$ in SUSY.
    }
    \label{fig:dsphi:feyn}
  \end{center}
\end{figure}



Effective hamiltonian
\begin{align}
  \Ham{eff} &= -4\frac{G_F}{\sqrt{2}}\Vconj{ub}\V{cs}
  \big(
  c_1(\mu)\Op{1}(\mu) + c_2(\mu)\Op{2}(\mu)
  \big)
\end{align}
where the operators are:
\begin{align}
  \Op{1} &= \big(\bar b\gamma_\mu P_Ls\big) \big(\bar c\gamma_\mu P_Lu\big) \nonumber\\
  \Op{2} &= \big(\bar b\gamma_\mu P_Lu\big) \big(\bar c\gamma_\mu P_Ls\big)
\end{align}
\cite{Buchalla:1995vs}
for generic \decay{\bar b}{scu} transitions.

























In theory say that there is no longer a discrepancy (1407.1320).
\cite{PDG2012}
