The decay \btodsphi proceeds via the annihilation of the constituent quarks in a \Bp meson into
a virtual \Wp boson in the SM\footnote{
  This analysis was published in \Ref{Aaij:2012zh}.}.
To achieve the final state, the \Wp decays into a $\cquark\bar\squark$ pair and an additional
$\squark\bar\squark$ pair must be \emph{popped} from the QCD field.
This is the only diagram that can perpetuate such a decay at tree level because the initial state
quarks are all different to those in the final state.
Annihilation decays of \Bp mesons are rare in the SM due to the magnitude of $|\V{ub}|$ (see
\Eq{eq:th:vub}); in fact, no fully hadronic decays proceeding via annihilation have yet been
observed.


Predictions for the brancing fraction $\BF\big(\btodsphi\big)$ are calculated using the OPE defined
by the effective Hamiltonian~\cite{Zou:2009zza,Mohanta:2002wf,PhysRevD.76.057701,Lu:2001yz}:
\begin{equation}
  \Ham{eff}=
  -\frac{G_F}{\sqrt{2}} \V{ub}\Vconj{cs}
  \big[
    C_1(\Lambda)\Op{1}+C_2(\Lambda)\Op{2}
    \big]
\end{equation}
where
\begin{align}
  \Op{1} &= \big(\bar b\gamma_\mu P_Lu\big) \big(\bar c\gamma_\mu P_Ls\big) \nonumber\\
  \Op{2} &= \big(\bar b\gamma_\mu P_Ls\big) \big(\bar c\gamma_\mu P_Lu\big).
\end{align}
and $C_{1,2}$ are Wilson coefficients defined at scale $\Lambda=m_b$.
The short distance operators \Op{1} and \Op{2} both describe the transition $\bar b\!\to\bar scu$.
Despite the relatively simple effective Hamiltonian, QCD effects make $\BF\big(\btodsphi\big)$ very
difficult to predict.
Figure~\ref{fig:dsphi:feyn} shows hows Feynman diagrams for a first order decay of \btodsphi in the
SM and SUSY, where the $\squark\bar\squark$ pair are produced by a decaying gluon which could have
originated from any of the four quarks.
If the gluon were to be from one of the initial state quarks then the assumption of factorization
does not hold.


\begin{figure}
  \begin{center}
    %\includegraphics[width=0.48\textwidth]{feynman_dsphi_sm}
    %\includegraphics[width=0.48\textwidth]{feynman_dsphi_susy}
    \includegraphics[scale=1]{feynman_dsphi_sm}
    \includegraphics[scale=1]{feynman_dsphi_susy}
    \caption[Feynman diagram for the decay \btodsphi]
    {\small
      A Feynman diagram for the decay \btodsphi being mediated by a
      (left) \Wp in the SM, and
      (right) $H^+$ in SUSY.
    }
    \label{fig:dsphi:feyn}
  \end{center}
\end{figure}


Predictions for the branching fraction $\BF(\btodsphi)$ in the SM are in
the range (1--7)$\e{-7}$~\cite{Zou:2009zza,Mohanta:2002wf,PhysRevD.76.057701,Lu:2001yz}.
However, significant enhancements could be observed were the decay to be mediated by additional BSM
particles.
Figure~\ref{fig:dsphi:feyn} shows Feynman diagrams for a first order decay of \btodsphi in the SM
and SUSY.
One theoretical prediction would expect that a NP model with two Higgs doublets would have a
branching fraction to be enhanced over the SM by a factor of about ten~\cite{Mohanta:2002wf}:
\begin{align}
  %\BF\big(\btodsphi\big)_\mathrm{SM}&=0.67\e{-6}, \nonumber\\
  %\BF\big(\btodsphi\big)_\mathrm{2HDM}&=\pz8.0\e{-6}, \nonumber\\
  %\BF\big(\btodsphi\big)_\mathrm{RPV}&=3.06\e{-4}.
  \BF\big(\btodsphi\big)|{\makebox[\widthof{$_\mathrm{2HDM}$}][l]{$_\mathrm{SM}$}}
  &=0.67\e{-6}, \nonumber\\
  \BF\big(\btodsphi\big)|_\mathrm{2HDM}
  &=\pz8.0\e{-6}, \nonumber\\
  \BF\big(\btodsphi\big)|{\makebox[\widthof{$_\mathrm{2HDM}$}][l]{$_\mathrm{SM}$}}
  &=3.06\e{-4}.
\end{align}
%\begin{itemize}
  %\item
%\end{itemize}

The large range in branching fraction predictions is due to the difficulties in modelling the
hadronization of the final state quarks; the hadronic form-factors lead to large uncertainties.
Calculating the decay dynamics is further complicated because the decay \btodsphi is not fully
factorizable, since the \ssbar pair can come from the final state quarks (factorizable) or the
initial quarks, in which case they contribute to the short range process.

The CP-backward asymmetry, \acp, of the decay is also interesting.
In the SM $\acp=0$, because the Feynman diagram only contains one phase (in \V{ub}), but
interference from BSM physics diagrams can alter this significantly.
Predictions from \Ref{Mohanta:2002wf} are:
\begin{align}
  \acp\big(\btodsphi\big)|_\mathrm{2HDM}
  &\leq 59\,\%, \nonumber\\
  \acp\big(\btodsphi\big)|{\makebox[\widthof{$_\mathrm{2HDM}$}][l]{$_\mathrm{RPV}$}}
  &\leq 14\,\%.
\end{align}

There was additional interest in the branching fraction of the decay \btodsphi because of the
historical conflict between the measurements of \V{ub} from inclusive semi-leptonic decays and
from the decay $\decay{\Bp}{\taup\nu_\tau}$, see \Sec{sec:bsm:faisec:bsm:fail}.
The schematic Feynman diagram for \btodsphi is identical to $\decay{\Bp}{\taup\nu_\tau}$, except
that an \ssbar pair must be produced in the case of the former.
Since the publication of this analysis, the analysis in \Ref{Crivellin:2014zpa} concluded that the
discrepancies cannot explained by new physics, and rather is due to underestimated uncertainties in
either theory or experiment.

In addition to searching for the decay \btodsphi it is a trivial extension to also look for the
decay \bctodsphi, which is also an annihilation type decay and has the same schematic Feynman
diagram.
This will be much more suppressed due to the difference in production cross-section of the \Bp and
\Bc mesons.

%The analysis was done on $1\invfb$ of data collected by the \lhcb detector in 2011 from $pp$
%collisions with a centre-of-mass energy of $7\tev$.




%Previous experimental limit by \babar \cite{Aubert:2005gd}
%\begin{equation}
  %\BF\big(\btodsphi\big) < 1.8\e{-6} \qquad \text{90\,\% C.L.}
%\end{equation}






























