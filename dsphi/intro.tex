\section{Introduction}

\begin{figure}[bh]
  \begin{center}
    \includegraphics[scale=1]{feynman_dsphi_sm}
    \includegraphics[scale=1]{feynman_dsphi_susy}
    \caption[Feynman diagram for the decay \btodsphi]
    {
      A Feynman diagram for the decay \btodsphi being mediated by a
      (left) \Wp in the SM, and
      (right) $H^+$ in SUSY.
      The \ssbar pairs shown here are formed from a gluon that can come from any quark.
      The arrangement of quarks forming the final state mesons shown is the colour favoured decay
    }
    \label{fig:dsphi:feyn}
  \end{center}
\end{figure}

In the \sm, the decay \btodsphi proceeds via the annihilation of the constituent \bquark and \uquark
quarks of a \Bp meson into a virtual \Wp boson from the \gls{CC} interaction.
This transition is suppressed \ckm
matrix element \V{ub}\footnote{
  All mentions of the \phii meson refer to the $\phi(1020)$.
}.
To achieve the final state, the \Wp decays into a $\cquark\squarkbar$ pair and an additional
\ssbar pair must be created from the \QCD field.
This is the only diagram that can perpetuate such a decay at tree-level because the initial state
quarks are all different to those in the final state.
A Feynman diagram of the decay \btodsphi is shown in \Fig{fig:dsphi:feyn}, where
the final state mesons can be formed in the way indicated, or the \ssbar pair from the \QCD field
can form the \phii, although this is colour-suppressed.
Also, the gluon that forms the \ssbar pair can originate from any of the initial or final state
quarks.
This analysis was published in \Ref{LHCb-PAPER-2012-025}.

Annihilation decays of \Bp mesons are rare in the SM due to the magnitude of
$|\V{ub}|\sim4\e{-3}$.
In fact, no fully hadronic decays proceeding via annihilation-type diagrams have yet been
observed.




%and SUSY, where the \ssbar pair are produced by a decaying gluon which could have
%originated from any of the four quarks.
%If the gluon were to be from one of the initial state quarks then the assumption of factorisation
%does not hold.

Predictions for the branching fraction $\BF\big(\btodsphi\big)$ are calculated using the OPE defined
by the effective Hamiltonian~\cite{Zou:2009zza,Mohanta:2002wf,PhysRevD.76.057701,Lu:2001yz}:
\begin{equation}
  \Ham{eff}=
  -4\frac{G_F}{\sqrt{2}} \V{ub}\Vconj{cs}
  \big[
    C_1(\Lambda)\Op{1}+C_2(\Lambda)\Op{2}
    \big]
\end{equation}
where
\begin{align}
  \Op{1} &= \big(\bquarkbar\gamma_\mu P_Lu\big) \big(\cquarkbar\gamma_\mu P_Ls\big) \nonumber\\
  \Op{2} &= \big(\bquarkbar\gamma_\mu P_Ls\big) \big(\cquarkbar\gamma_\mu P_Lu\big).
\end{align}
The Wilson coefficients $C_1$ and $C_2$ are defined at the scale $\Lambda=m_b$,
and the projection operators are defined as $P_L=\tfrac12(1-\gamma_5)$ and
$P_R=\tfrac12(1+\gamma_5)$.
The short distance operators \Op{1} and \Op{2} both describe the transition $b\!\to scu$.
Therefore, not only is there uncertainty in the branching fraction introduced by \V{ub}, but the
number of quarks in the decay make \QCD calculations very difficult.
Calculating the amplitude of the decay \btodsphi is made particularly complicated because the decay
is inherently non-factorizable, since the \ssbar pair can come from any of the initial or final
state quarks at leading order.
There are also inherent uncertainties in the form-factors that describe the hadronization
process of the final state quarks.
Reference~\cite{Mohanta:2002wf} predicts that
\begin{equation}
  \BF\big(\btodsphi\big)|{\makebox[\widthof{$_\mathrm{SM}$}][l]{$_\mathrm{SM}$}}
  =1.88\e{-6}, \nonumber\\
\end{equation}
by na\"ively assuming factorizability holds, and by using an improved
technique~\cite{Beneke:2000ry}, whereby perturbative \QCD corrections are applied to the
factorisation method, a value of
\begin{equation}
  \BF\big(\btodsphi\big)|{\makebox[\widthof{$_\mathrm{SM}$}][l]{$_\mathrm{SM}$}}
  =0.67\e{-6}, \nonumber\\
\end{equation}
is calculated.
The \QCD corrections lead to a new branching fraction prediction which differs by a factor of two
to the uncorrected result --- this is arguably testament to the difficulties of accounting for \QCD
in such calculations.
Other \sm predictions tend to lie between
\approx$1\e{-7}$ and
\approx$7\e{-7}$~\cite{Zou:2009zza,Mohanta:2002wf,PhysRevD.76.057701,Lu:2001yz}.

Despite the theoretical uncertainties, the value of
in $\BF(\btodsphi)$ could be significantly enhanced if
the decay to be mediated by additional \bsm
particles, particularly other charged bosons.
For example, a \twoHDM~--- such as \SUSY~---~the decay \btodsphi
would be mediated by a charged Higgs $H^+$; this is shown in \Fig{fig:dsphi:feyn}.
More particles mean more Feynman diagrams that could add to the total amplitude.
Reference~\cite{Mohanta:2002wf} also makes predictions for the branching fraction of the decay
\btodsphi in a \twoHDM and a model with \rpv:
\begin{align}
  %\BF\big(\btodsphi\big)|{\makebox[\widthof{$_\mathrm{2HDM}$}][l]{$_\mathrm{SM}$}}
  %&=0.67\e{-6}, \nonumber\\
  \BF\big(\btodsphi\big)|_\mathrm{2HDM}
  &=8.0\pz\e{-6}, \nonumber\\
  \BF\big(\btodsphi\big)|{\makebox[\widthof{$_\mathrm{2HDM}$}][l]{$_\mathrm{RPV}$}}
  &=3.06\e{-4}. \nonumber
\end{align}
%The above number for the \sm branching fraction was calculated using the \QCD improved
%factorization method~\cite{Beneke:2000ry}, but to illustrate the difficulties in \QCD calculations,
%the same paper quotes $1.88\e{-6}$ using the factorization approximation.
These numbers indicate that, while the exact \sm value of $\BF\big(\btodsphi\big)$ is not well
known, the value for models with additional mediating particles could be enhanced by a factor of
over 100.

The \CP asymmetry, \acp, of a process is defined in terms of decay rates of $B$ hadrons:
\begin{equation}
  \acp = \frac{\Gamma(\Bbar\!\to\xbar{f}) - \Gamma(B\!\to f)}
  {\Gamma(\Bbar\!\to\xbar{f}) + \Gamma(B\!\to f)}
\end{equation}
for some final state $f$.
A positive value of \acp would indicate a preference of the antimatter process, above the matter
process.
In the \sm $\acp(\btodsphi)=0$, because
at leading order there is only one phase, in \V{ub}, but
interference from \bsm physics diagrams could alter this significantly.
Predictions from \Ref{Mohanta:2002wf} are:
\begin{align}
  \acp\big(\btodsphi\big)|_\mathrm{2HDM}
  &\leq 59\,\%, \nonumber\\
  \acp\big(\btodsphi\big)|{\makebox[\widthof{$_\mathrm{2HDM}$}][l]{$_\mathrm{RPV}$}}
  &\leq 14\,\%.
  \label{eq:dsphi:acpdef}
\end{align}
So, both measurements of $\BF\big(\btodsphi\big)$ and $\acp\big(\btodsphi\big)$ could lead to
evidence for \np.


\subsection{Other annihilation-type hadronic decays}
The annihilation of the \Bp meson can perpetuate numerous decays resulting in fully hadronic
states, including a charmed meson.
The decay \decay{\Bp}{\Dp\Kstarz} proceeds in the same way as
\btodsphi, but the former needs a \ddbar pair to be created from the \QCD vacuum, rather than an
\ssbar pair.
Similarly, the decay \decay{\Bp}{\Ds\Kstarzb} is identical to the \btodsphi excepting that instead
of \decay{\Wp}{\cquark\squarkbar} the \Wp decays into an $\cquark\dquarkbar$ pair.
The decays \decay{\Bp}{\Dp\Kstarzb} and \decay{\Bp}{\Ds\Kstarz} are non-trivial diagrams in the
\sm, and heavily suppressed, but have similar final states.
The same final states can also come from the annihilation of the constituent
quarks of the \Bc meson.
While the following chapter only discusses the search for the decay \btodsphi, these other
interesting decay modes are searched for in
\Ref{LHCb-PAPER-2012-025}.






