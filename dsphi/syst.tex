\section{Systematic uncertainties}
\label{sec:dsphi:syst}

Sources of systematic uncertainty appear at all levels of the selection and modelling of the decay.
The following section lists each contribution and defends the choice of the value of the
uncertainty assigned.
It should be noted that it is only the relative efficiency between the signal and normalization
channels that contributes to the systematics, rather than their absolute values.

The selection of candidate \btodsphi decays is the source of some small uncertainty.
It is known that the geometry of the detector and the reconstruction process is well described in
simulation.
As descrbed abocve, the efficiency of the BDT is evaluated using data, so there is a negligible
difference in the efficiency ratio.
In total, a $1\pc$ uncertainty is assigned for the selection.

There are some differences between the performance of the trigger in data and in the simulation.
The HLT lines are observed to be described well in simulation, as should be expected, since they
are software level triggers.
Also, it was observed that the \lone hardware triggers is reliable above $\pt>4\gev$, but slightly
mismodelled at low \pt.
Using only the events above $4\gev$, it was observed that the trigger efficiency alters by about
$4\pc$.
This is the value of the systematic uncertainty that is applied.

Both the signal and normalization channels have a \Ds, and therefore efficiencies from the \Dp and
\Lc vetoes cancel to a large extent.
Given that these cuts are very efficient, the difference is negligible, and the efficiency ratio is
assumed to be unity for the calculation.
The only differences between the two modes are due the decay kinematics, since the \phii is lighter
than the \Dz.
There is a difference of $1.5\pc$ between \eff{veto} for the decays \decay{\Bsb}{\Dsp\pim} and
\btodsd, and
since the mass of the \phii is nearer the \Dz mass than the pion mass, a systematic uncertainty of
$1\pc$ is assigned.

%The window around the \Ds was evaluated by comparing simulated events with data for the high
%statistics decay \bstodspi.
%The BDT selection is assigned a systematic uncertainty of $3\pc$ using comparisons with data.
%Mass windows
%\Ds mass has same eff for sig and norm
%difference is between D0 and \phii

There are other systematic uncertainties that affected the selection.
The BDT cut was assigned an uncertainty of $3\pc$ which was due to the sizes of the
validation samples used to calculate the efficiencies.
Mass windows around the \Ds and \phii lead to a $3\pc$ systematic uncertainty.
Also, the low statistics of the simulation samples used to deduce efficiencies led to a $3\pc$
uncertainty.

The mass fits introduced systematic uncertainties from each component.
If the parameters in the signal shape are allowed to float the fit results in a yield which is
$5\%$ higher than the nominal fit.
This is assigned as a systematic uncertainty.


The total uncertainty from the background shape is $5\pc$; and is estimated by making changes to
the background model.
By removing either the \btodsstrphi or \bstodsstrkstrk components, the yield changes by only $1\pc$.
Changing the constraints on \rA/\rB and \rC/\rD for \bstodskstrk fby a factor of 2 results in a
$1\pc$ change in signal yield.
The combinatorial background was estimated by allowing the slope to float free, this led to an
approximate $3.5\pc$ systematic uncertainty.
%Total is 5\%, none of these alterations lead to a significance of lower than 3\stdev

%SWave contamination
%could be some small Swave
%other channels percent level contaminations are seen, but when considered next to the systematic
%from normalization, this is negligible
%Statistics too limited to do precise extraction

Contributions from all sources of systematic uncertainties are summarized in \Tab{tab:dsphi:syst}.
The dominant systematic uncertainty --- discounting the uncertainties on the branching fraction of
the normalization channel --- was from the mass fits, which is unsurprising considering the
complexity of the fit, and the treatment of the backgrounds.
Regardless of the constraints that are --- or are not --- included in the fit, the lowest
significance obtained is still greater than $3\stdev$.
Therefore the significance was quoted as greater than $3\stdev$.

\begin{table}
  \caption[Systematic uncertainties]
  {\small
    Summary of the systematic uncertainties contributing to the branching fraction of the decay
    \btodsphi.
  }
  \label{tab:dsphi:syst}
  \begin{center}
    \begin{tabular}{lc}
      \toprule
      \cellc{Source of systematic} & Uncertainty (\%) \\
      \midrule
      Selection & $\pz1$ \\
      Trigger & $\pz4$ \\
      BDT & $\pz3$ \\
      \Dp and \Lc vetoes & $\pz1$ \\
      Mass windows & $\pz3$ \\
      Simulation statistics & $\pz3$ \\
      %Total Selection & 7 \\
      Mass shape & $\pz5$ \\
      Background shapes & $\pz5$ \\
      %Total Mass Fit & 7 \\
      \littlerule
      Total & 10 \\
      \bottomrule
    \end{tabular}
  \end{center}
\end{table}


