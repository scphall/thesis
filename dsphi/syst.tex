\section{Systematic uncertainties}
\label{sec:dsphi:syst}

Sources of systematic uncertainty appear at all levels of the selection and modelling of the decay.
The following section lists each contribution and defends the choice of the cvalue of the
uncertainty assigned.
It should be noted that it is only the relative efficiency between the signal and normalization
channels that contributes to the systematics, rather than their absolute values.

The selection of candidate \btodsphi decays is the source of some small uncertainty.
It is known that the geometry of the detector and the reconstruction process is well described in
simulation.
As descrbed abocve, the efficiency of the BDT is evaluated using data, so there is a negligible
difference in the efficiency ratio.
In total, a $1\pc$ uncertainty is assigned for the selection.

There are some differences be


%Trig
%\lone is reliable above 4\gev in \pt,
%using only these candidates, the trigger eff alters by 4\%, this is the systematic we apply
%statistically limited
%
%BDT
%BtoDD \cite{LHCb-PAPER-2012-001}
%has 1\% syst
%additional phi BDT has 2\% syst
%
%Veto
%Both channels has \Ds, so effs mainly cancel
%only difference is from decay kinematics
%given the eff of these cuts, difference must be ever so small
%Above, we assume the ratio is 1.0
%In data, the diffenece between B to Dspi an B to DsD is 1.5\%
%The \phii is much nearer in mass than \Dz than pi, so the difference should be small
%1\%
%
%Mass windows
%\Ds mass has same eff for sig and norm
%difference is between D0 and \phii
%
%Mass fits
%\begin{itemize}
  %\item signal shape:
    %run with shape free, 5\% higher yield, we assign ths at the syst
  %\item \btodsstphi:
    %remove completely leads to 1\% change in yiled
  %\item \bstodskstrk:
    %ratio \rA/\rB and \rC/\rD constrained from sim
    %instead we fix this 100 times lower yield dhanges by 3\%
    %100 times higher is 2\%.
    %The ration of \rA/\rC  and \rB/\rD varied by large amounts leading to 5\% change n yield
    %we don't know ratios precisely, we do know to within a factor of 2, within this range 1\% syst
  %\item \bstodsstrkstrk:
    %remove entirely changes yiled by 1\%
  %\item comb:
    %leave slope fere, tield changes by 5\%, but this is a clear overestimate, 3.5\%
%\end{itemize}
%
%Total is 5\%, none of these alterations lead to a significance of lower than 3\stdev
%
%SWave contamination
%could be some small Swave
%other channels percent level contaminations are seen, but when considered next to the systematic
%from normalization, this is negligible
%Statistics too limited to do precise extraction
%
%Limited MC
%3\% on ratio of effs

Contributions from all sources of systematic uncertainties are summarized in \Tab{tab:dsphi:syst}.
The dominant systematic uncertainty --- discounting the uncertainties on the branchinng fraction of
the normalization channel --- was from the mass fits, which is unsurprising considering the
complexity of the fit, and the treatment of the backgrounds.
Regardless of the constraints that are --- or are not --- included in the fit, the lowest
significance obtained is still greater than $3\stdev$.
Therefore the significance was quoted as greater than $3\stdev$.

\begin{table}
  \caption{\small
    Summary of the systematic uncertainties contributing to the branching fraction of the decay
    \btodsphi.
  }
  \label{tab:dsphi:syst}
  \begin{center}
    \begin{tabular}{cc}
      \toprule
      Source & Uncertainty (\%) \\
      \midrule
      Selection & $\pz1$ \\
      Trigger & $\pz4$ \\
      BDT & $\pz3$ \\
      \Dp and \Lc vetoes & $\pz1$ \\
      Mass Windows & $\pz3$ \\
      Simulation statistics & $\pz3$ \\
      %Total Selection & 7 \\
      Mass shape & $\pz5$ \\
      Background shapes & $\pz5$ \\
      %Total Mass Fit & 7 \\
      Total & 10 \\
      \bottomrule
    \end{tabular}
  \end{center}
\end{table}


