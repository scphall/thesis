\subsection{Efficiency calculations}
The calculation of the branching fraction $\BF(\btodsphi)$ requires the total efficiencies for the
signal and normalization channels.
These are calculated as the product of all efficiencies that contribute to the final selection.
For the majority of stages in the selection process, the efficiency can be calculated using
simulated events of \btodsphi and \btodsd.
However, the efficiency for the BDT cut cannot be estimated with simulation because the \pid
variables, upon which the BDT output depends, are poorly described by simulation.

%Efficiencies relevant to the calculation of the
%branching fraction, given in \Eq{eq:dsphi:bf}, were
%calculated using simulated events of the decays \btodsphi and \btodsd.

%The majority of the efficiencies listed in \Tab{tab:dsphi:eff} are calculated using simulated
%\btodsphi events.
%However, calculating the efficiency of the BDT variable using simulated events is not reliable,
%because of the number of PID --- and other --- variables that are poorly described by simulation.
%Therefore, the value of \eff{BDT} is determined with a data driven method.

To obtain the efficiency of $\bdt_{X}$, for $X\in\{\Ds,\phi\}$ a data driven method is used.
First, the validation sample is
binned in three dimensions: $\pt(X)$, $\chisqfd(X)$, and $\bdt_{X}$.
The bin widths are chosen such that each two dimensional bin in $(\pt,\chisqfd)$ have approximately
equal statistics.
Here, \chisqfd is the flight distance of $X$ in units of \chisq.
The variables \pt and \chisqfd are used because they are two of the most discriminating variables
in the \bdt, and well described by simulation.
The large statistics of the validation sample mean that in each bin defined by \pt and \chisqfd
there is a BDT distribution.
Then, each individual simulated event is assigned an efficiency based on the BDT distribution in
the bin defined by the \pt and \chisqfd of the $X$.
These individual efficiencies can then be amalgamated into an overall efficiency.
This method is also used in \Ref{LHCb-PAPER-2012-001}.

A summary of all efficiencies can be found in \Tab{tab:dsphi:eff}.


\begin{table}
  \caption[Efficiencies for calculating $\BF\big(\btodsphi\big)$]
  {
    Efficiencies, in \%, for the signal decay \btodsphi and the normalization channel \btodsd.
    The veto efficiency of \btodsphi is assumed to be the same as for \btodsd.
    All efficiencies were calculated using simulated events, with the exception of the \bdt, which
    is calculated using a data driven method, as described in text.
  }
  \label{tab:dsphi:eff}
  \begin{center}
    \begin{tabular}{lcccc}\toprule
      \cellc{Source of efficiency}&\btodsphi&\btodsd\\
      \midrule
      Geometry of the \lhcb detector
      & $14.62\pm0.05$ & $12.75\pm0.05$ \\
      Reconstruction and stripping
      & $\pz1.53\pm0.04$ & $\pz1.98\pm0.04$ \\
      Trigger
      & $95.8\pm0.3$ & $94.4\pm0.3$ \\
      Preselection
      & $86.0\pm0.9$ & $75.0\pm0.6$ \\
      BDT
      & $51.4\pm0.2$ & $99.2\pm0.2$ \\
      \Dp and \Lc vetoes
      & $95.0$ & $95.0\pm0.2$ \\
      \littlerule
      Total
      & $0.091\pm0.003$ & $0.166\pm0.003$ \\
      \bottomrule
    \end{tabular}
  \end{center}
\end{table}



