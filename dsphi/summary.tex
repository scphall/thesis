\section{Summary}
\label{sec:dsphi:conc}

The analysis outlined in the preceding chapter shows first evidence for the decay \btodsphi, with
greater than $3\stdev$ significance.
While the measured branching fraction
\begin{equation*}
  \BF\big(\btodsphi\big) =
  \big(1.87\,^{+1.25}_{-0.73}\stat\pm0.19\syst\pm0.32\normerr\big)\e{-6}.
  \label{eq:dsphi:result}
\end{equation*}
is somewhat higher than SM predictions, they are not incompatible when considering large
theoretical uncertainties.
%The above result has a branching fraction which is higher than the SM predictions, although there
%are large uncertainties associated with both the theoretical and experimental values.
Another measurement was the \CP asymmetry, which could deviate significantly from zero were NP to
be present at leading order.
The value measured was
\begin{equation*}
  \acp\big(\btodsphi\big)=
  -\big(0.01\pm0.41\stat\pm0.03\syst\big),
\end{equation*}
which is consistent with zero \CPV.

%An updated analysis would take advantage of the additional data collected by the \lhcb experiment
%in 2012, and potentially in Run-2 of the \lhc.
%Using additional \Ds decay channels should also be considered, particularly \decay{\Ds}{\pip\pipi}
%which has a branching fraction of $(1.09\pm0.05)\e{-2}$: a factor of about five less than
%$\BF\big(\decay{\Ds}{\kkpi}\big)$.
%However, this would introduce larger and more complex backgrounds.






