\subsection{Mass fit}
\label{sec:dsphi:fit}
The signal yield of the decay \btodsphi is determined by performing an unbinned maximum likelihood
fit to the invariant mass spectrum of the candidate \Bp mesons.
Invariant mass distributions of
selected candidate \Ds and \phii mesons are shown in \Fig{fig:dsphi:mesons}.

\begin{figure}
  \begin{center}
    \includegraphics[width=0.48\textwidth]{blank}
    \includegraphics[width=0.48\textwidth]{blank}
    \caption{\small
      Invariant mass distributions of the
      (left) \Ds, and
      (right) \phii candidates.
    }
    \label{fig:dsphi:mesons}
  \end{center}
\end{figure}

%The fit is done simultaneously to the four regions indicated in \Tab{tab:dsphi:hel}, this enables
%additional constraints to be placed on the background distribution.

In the fit, there are several components: the signal \btodsphi; combinatorial background; and
specific backgrounds that peak below the \Bp mass.
The final state particles in this analysis of the decay \btodsphi are $\kkpi\kk$, given the mass
cuts on the \Ds and \phii candidates, there are no sources of background which peak at the \Bp
mass.
However, there are backgrounds from genuine $B$-hadron decays in which a particle --- or multiple
particles --- are not reconstructed.
These backgrounds form below the mass of the \Bp.
All these specific backgrounds are irreducible and must be accounted for in the fit.

After the selection requirements
the most significant of these are the decays\footnote{
  For these decays, the \Kstarz refers to the $K^*(892)^0$ meson.
}:
%\begin{align}
  %%&\decay{\Bp}{(\Dss\!\to(\Ds\to\kkpi)\gamma)(\phi\to\kk)}\\
  %%&\decay{\Bp}{\Dss\phi} \to (\Ds\gamma)\phi\to(\kkpi\gamma)(\kk)
  %&\Bp \!\to \Dss\phi \!\to (\Ds\gamma)\phi \!\to (\kkpi\gamma)(\kk) \\
  %&\Bsb \!\to \Ds\Kstarz\Kp \!\to (\kkpi)(\kpi)\Kp \\
%\end{align}
%\begin{align}
  %\Bp &\!\to \Dss\phi &\!\to& (\Ds\gamma)\phi &\!\to& (\kkpi\gamma)(\kk) \\
  %\Bsb &\!\to \Ds\Kstarz\Kp &\!\to& (\kkpi)(\kpi)\Kp \\
  %\Bsb &\!\to \Dss\Kstarz\Kp &\!\to& (\Ds\gamma)\Kstarz\Kp &\!\to& (\kkpi\gamma)(\kpi)\Kp \\
%\end{align}
  %\btodsstrphi
  %\bstodskstrk
  %\bstodsstrkstrk.
%\btodsstrphi, \bstodskstrk and \bstodsstrkstrk.
%which all contain at least one final state particle that is not reconstructed.
\begin{center}
  \begin{tabular}{llll}
    $\Bp\to$ & \Dss & \phii \\
    & \;\nlto $\Ds\gamma$ & \;\nlto $\boldsymbol{\kk}$ \\
    & \phantom{\Ds}\nlto $\boldsymbol{\kkpi}$ \\\rule{0pt}{4ex}
    $\Bsb\to$ & \Ds & \Kstarz & $\boldsymbol{\Km}$ \\
    & \;\nlto $\boldsymbol{\kkpi}$ & \;\nlto $\boldsymbol{\Kp}\pim$ \\\rule{0pt}{4ex}
    $\Bsb\to$ & \Dss & \Kstarz & $\boldsymbol{\Km}$ \\
    & \;\nlto$\Ds\gamma$ & \;\nlto$\boldsymbol{\Kp}\pim$ \\
    & \phantom{\Ds}\nlto $\boldsymbol{\kkpi}$ \\
  \end{tabular}
\end{center}
where an emboldened particle indicates a reconstructed track.
The decay \bstodskstrk has never been observed, but given that the branching fraction of the decay
$\decay{\Bd}{\Dm\Kstarzb\Kp}$ is $(8.8\pm1.9)\e{-4}$~\cite{PDG2012}, it should be in the
\btodsphi selection.
There is no contribution in the mass range of interest from the \decay{\Bd}{\Dm\Kstarzb\Kp} mode,
because of the mass difference between the \Bs and \Bd mesons.
As well as \bstodskstrk, the decay \bstodsstrkstrk was also expected to be present in the sample.


%There are no backgrounds which contribute to a peaking component directly below the \Bp mass for
%the decay \btodsphi.
%However, on top of the combinatorial background, the low-mass side has a number of components which
%result from excited \Dssp or \Kstarz.


Both the \Bp and \Ds mesons have quantum numbers $J^P=0^-$, and the \phii is $1^-$ .
Therefore the decay \btodsphi is a transition of a pseudoscalar to a pseudoscalar and a vector
meson.
In order for angular momentum to be conserved, the vector particle must be produced in the $j=0$
state, where the spin is orthogonal to the particle's momentum.
Therefore, the vector \phii in the decay \btodsphi must be lonitudinally polarized in the final
state, and its daughter kaons have an angular distribution proportional to $\cos^2\thetahel$, as
shown in \Fig{fig:dsphi:hel}.
The angle \thetahel is the helicity angle of the \phii, defined as the angle between the \Kp and
the \Bp in the rest frame of the \phii.
This is used to further separate the signal decay \btodsphi from other backgrounds.

\begin{figure}
  \begin{center}
    \includegraphics[width=0.48\textwidth]{thetahel}
    %\includegraphics[width=0.48\textwidth]{thetahel2}
    \caption{\small
      Distribution of $\cos\thetahel$, where \thetahel is the angle between the \Bp and \Kp in the
      rest frame of the \phii from simulated events.
      The shaded region indicates $|\cos\thetahel|>0.4$, which defines the signal region.
      For the decay \btodsphi, where the \Bp and \Ds mesons are both pseudoscalars, and the
      \phii is a vector, the \phii is forced into the $j=0$ state.
      Therefore, the angular distrubution of \thetahel is proportional to $\cos^2\thetahel$.
    }
    \label{fig:dsphi:hel}
  \end{center}
\end{figure}

It is possible to remove about $93\pc$ of background by requiring that $|\cos\thetahel|>0.4$,
this is the same cut value as used in \Ref{LHCb-PAPER-2011-008}.
Therefore, two regions are defined in terms of $\cos\thetahel$, and both are used in a simultaneous
fit.
This also allows the combinatorial and peaking backgrounds to be separated from the signal because
these do not necessarily have a longitudilnally polarized \phii.

%The signal region for the \phii candidate is defined to be within $20\mev$ of $m_\phii^\pdg$.
%A sideband region is defined to be $20<m_

%These peaking backgournds, and the combinatorial background,
%do not have to have a $\phi$ meson in a longitudilnally polarized.
%Therefore, the $\cos\thetahel$ distribution should be flat for all backgrounds.

%can be used to separate the signal decay from backgrounds.

%In order to maximize the signal yield, the full selected dataset was separated into four regions
%defined by the helicity angle and the invariant mass of the $\phi$ candidate, these regions are
%shown in \Table{tab:dsphi:hel}.

Fit regions are further split according to the invariant mass of the \phii candidate.
A signal region is defined for \phitokk candidates with a mass within $20\mev$ of the nominal \phii
mass, and a sideband region is defined for candidates with a mass in the range
\mbox{$20<\left|m_\phi^\pdg-m_\kk\right|<40\mev$}.

The resulting fit regions --- defined by \thetahel and $m_{\kk}$ --- have a signal region, \rA,
containing most of the signal, and a purely background region, \rD.
By performing a simultaneous fit to the four regions, which are defined explicitly in
\Tab{tab:dsphi:hel}, there is a better grasp on all the backgrounds.

\begin{table}
  \caption[Fit regions]
  {\small
    Fit regions used to search for the decay \btodsphi.
    Approximately $89\,\%$ was expected to be in region \rA.
  }
  \label{tab:dsphi:hel}
  \begin{center}
    \begin{tabular}{cccc}
      \toprule
      &&\multicolumn{2}{c}{$|m_{\kk}-m_\phi^\pdg|$ (MeV)}\\
      &&$\in[0,20]$&$\in[20,40]$ \\
      \midrule
      \multirow{2}{*}{$|\cos\thetahel|$}
      &$>0.4$ & \rA & \rB \\
      &$<0.4$ & \rC & \rD \\
      \bottomrule
    \end{tabular}
  \end{center}
\end{table}



%\subsection{Contributions to the mass fit}
The mass fit constitutes a simultaneous fit to four regions, in each of which there are four
background distributions as well as a signal component (a total of at least 32 free parameters).
Clearly this is very challenging, especially considering that there are low statistics and complex
background models.
It is therefore necessary to use relationships --- derived from simulation or data --- to fix as
many parameters as possible.
The following section will outline how the shape of each distribution is derived and how different
parameter are constrained.

The signal shape is described by a A Gaussian function, with a mean $\mu$ and standard deviation
$\sigma$.
Figure~\ref{fig:dsphi:sigshape} shows the Gaussian distribution that has been fitted to simulated
events of the decay \btodsphi.
The value of $\sigma$ from this fit is $11\mev$, which is then scaled up by $20\pc$ to account for
differences in resolution between simulation and data.
The mean, $\mu$, is fixed to be $5283\mev$, which is the mean mass observed in
\decay{\Bp}{\Dz\pip} (and also observed in the \Bs mode).
It is also determined from simulation that the total signal is distributed between the regions:
\rA 89\pc, \rB 4\pc, \rC 7\pc, and in \rD there is negligible expected signal contribution.
Therefore, in the following, \rA and \rD may be referred to as the signal and background regions,
respectively.

The shape of \Bp candidates originating from the \btodsstrphi background is taken from simulated
events.
The $\phi$ from the decay \btodsstrphi does not need to be longitudinally polarized because the
\Dssp is a vector meson ($J^P=1^-$).
Therefore the background from the decay \btodsstrphi contributes in all fit regions.
Since the shapes of these reconstructed candidates are non-trivial, a kernel density estimation
technique~\cite{Cranmer:2000du} is used to describe the shape.
Figure~\ref{fig:dsphi:sigshape} shows the kernelized distribution for the whole set of simulated
events, as well as each individual helicity region.
It was assumed that \Dssp was unpolarized, as observed in many other cases where a
$B$ decays into a vector-vector final state.
Despite this, different \phii polarizations caused the shape of the distribution to differ
between the two helicity regions.

\begin{figure}
  \begin{center}
    \includegraphics[width=0.48\textwidth]{blank}
    \includegraphics[width=0.48\textwidth]{blank}\\
    \includegraphics[width=0.48\textwidth]{blank}
    \includegraphics[width=0.48\textwidth]{blank}
    \caption[Shape contributions from signal \btodsphi, and \btodsstrphi]
    {
      Signal shape from simulated events
      Shapes of peaking background contributions for
      decays of the form $\decay{\Bs}{D_s^{(*)+}\Kstarz\Kp}$.
      The contributions from \btodsstrphi and \bstodskstrk are kernelized distributions from
      simulated events.
      The \bstodsstrkstrk shape is obtained by convolving the
      effect of missing a photon or pion with the \bstodskstrk spectrum.
    }
    \label{fig:dsphi:sigshape}
  \end{center}
\end{figure}

It was expected that $\sim7$ events from the decay \btodsstrphi contribute to the background of the
four fit regions, spread over $\sim300\mev$.
At this level, the difference in the true distributions and those shown in \Fig{fig:dsphi:sigshape}
--- especially considering rising shape is the same for all polarizations of the \phii ---
leads to a negligible difference in yield.
For this reason, the longitudinally polarized \btodsstrphi component was used in all regions of the
fit.
The largest uncertainty here was from the yield of this component.
%Far larger, is the systematic effect from the yield.

Just as was done with the signal component, the ratios between yields of each fit region was fixed
using simulation.
Approximately $95\pc$ of the contribution from the decay \btodsstrphi is expected to be in the
signal region.


\begin{figure}
  \begin{center}
    \includegraphics[width=0.48\textwidth]{blank}
    \caption[Shapes of background contributions of \bstodskstrk and \bstodskstrk]
    {\small
      Shapes of peaking background contributions for
      decays of the form $\decay{\Bs}{D_s^{(*)+}\Kstarz\Kp}$.
      The contribution from \bstodskstrk is a kernelized distribution from
      simulated events.
      The \bstodsstrkstrk shape is obtained by convolving the
      effect of missing a photon or pion with the \bstodskstrk spectrum.
    }
    \label{fig:dsphi:bkgshape}
  \end{center}
\end{figure}


%Therefore the value of $\cos\thetahel$ is uniformly
%distributed between zero and one.
%The photon from the decaying \Dssp is not reconstructed, and therefore the background from the
%decay \btodsstrphi peaks below the nominal \Bp mass.


%The decay \btodsstrkstrk has never been observed


Other sources of background that peak below the mass of the \Bp meson are from the decay modes
$\decay{\Bsb}{D_s^{(*)+}\Kstarz\Km}$.
These arise when the pion from the \decay{\Kstarz}{\Kp\pim} is not reconstructed.
Since there is missing energy from the pion (and photon in the case of the \Dssp decay), these peak
significantly below the \Bp mass.
However, their background shapes are non-trivial, extending up to just below the signal window,
and therefore a good understanding of the shapes are required.
Once again, kernel density estimation techniques~\cite{Cranmer:2000du} were used; these shapes are
show in \Fig{fig:dsphi:bkgshape}.

As mentioned previously, the decays \bstodskstrk and \bstodsstrkstrk have never been observed, but
are expected to contribute.
The shape for \bstodskstrk is taken directly from kernelized simulated events.
There were no simulated events available with which to understand the shape of the \bstodsstrkstrk
background.
Instead the shape taken from the \bstodskstrk background is convolved with a distribution
parameterizing the loss of a photon (or pion) as seen between the decays \btodsstrphi and
\bstodskstrk.


Yields from these decays can, clearly, not be estimated.
Especially since the yield from the decay \bstodskstrk is highly sensitive to the width of the
$a_1(1260)$ --- because it decays via $a_1^+(1260)\Kstarz\Kp$ --- which is poorly known~\cite{PDG2012}.
However, the ratios of yields for the \bstodskstrk decay can be determined using simulated events.
The ratios between the yields from \rA/\rB and \rC/\rD was found to be $0.5\pm0.24$, and between the
regions \rA/\rC and \rB/\rD was determined to be $1.50\pm.034$.
These values are used as Gaussian constraints in the fit.
The ratio of branching fractions
\begin{equation}
  \frac{\BF\big(\decay{\Bdb}{\Dp\Kstarz\Km}\big)}
  {\BF\big(\decay{\Bdb}{\Dstarp\Kstarz\Km}\big)}
  \sim 1.5,
\end{equation}
and it is reasonable to expect the same to be true for the branching fraction ratio for the
\bstodskstrk and \bstodsstrkstrk modes.
Therefore, the ration of yields for each region is fixed to $1.5$.

%The background shapes for \btodsstrphi and \bstodskstrk are taken from simulated events,
%reconstructed as \btodsphi candidates.
%Since the shapes of these reconstructed candidates are non-trivial, a

%The shapes of peaking background contributions are taken from kernel density
%estimates of simulated events, as described.
%Yields of background components are also fixed from...
%Therefore, the only floating parameters in the simultaneous fit is the total signal yield and the
%combinatorial background yields and shape.
%However the combinatorial background between regions is fixed.

The last remaining background component to be constrained is that of the combinatorial background,
which is modelled with a decaying exponential function.
Since the distribution of $\cos\thetahel$ is flat for combinations of random tracks, the yields
between the regions \rA/\rC and \rB/\rD  are fixed to $1.5$,
The value of the slope is Gaussian constrained to a fit across a wider range of mass in data.

A summary of constraints applied to the fit is given in \Table{fig:tab:constraints}.

\begin{table}
  \caption[Constraints applied to the fit to \btodsphi data]
  {\small
    Fit parameters used in in the fit to determine the yield of the decay \btodsphi.
    A label of $f$, means that the value is fixed in the fit; and labels of $s$ and $d$ mean
    constrained using simulated events and data over a wider mass range, respectively.
    The yield of the background from \bstodsstrkstrk is also fixed to be $67\pc$ of
    $N\big(\bstodskstrk\big)$
  }
  \label{fig:tab:constraints}
  \begin{center}
    \begin{tabular}{ccccc}
      \toprule
      Fit component & Parameter & Value & Parameter & Value \\
      \midrule
      \btodsphi
      & yield \rA & $6.00\pm2.70$
      & \rC/\rA   & $0.075$ $f$ \\
      & $\mu$     & $5283\mev$ $f$
      & \rB/\rA   & $0.044$ $f$ \\
      & $\sigma$     & $13.2\mev$ $f$
      & \rD/\rA   & $0.003\pc$ $f$ \\
      \midrule
      \btodsstrphi
      & yield \rA & $8.67\pm7.36$
      & \rB/\rA   & $0.044$ $f$ \\
      & \rC/\rA   & $0.00\pm0.12$
      & \rD/\rC   & $0.044$ $f$ \\
      \midrule
      \bstodskstrk
      & yield \rA & $4.94\pm1.29$
      & \rA/\rB, \rC/\rD & $0.50\pm0.24$ $s$ \\
      & \rA/\rC, \rB/\rD & $1.50\pm0.34$ $s$ \\
      %\midrule
      %\bstodsstrkstrk
      %& yield/$N\big(\bstodskstrk\big)$ & 1.5 $f$ \\
      \midrule
      Combinatorial
      & yield \rA & $24.0\pm6.7$
      & yield \rB & $16.5\pm6.0$ \\
      & \rA/\rC & 1.5 $f$
      & \rB/\rD & 1.5 $f$ \\
      & exponent & $-(1.8\pm0.2)\e{-3}$ $d$ \\
      \bottomrule
    \end{tabular}
  \end{center}
\end{table}
