\section{Selection}

%LIST
%
%Rely on HLT2 tolo lines
%cuts designed to mimic HLT1 and HL2 topo triggers:
%B
%sum pt $>$ 5\gev
%VHI2DOF $<$ 10
%BPVIPCHI2 $<$25
%BPVLTIME$>$0.2\ps
%BIRA$>$0.999
%
%D
%sum PT $>$1.8GeV
%MAXDOCA$<$0.5mm
%VCHI2DOF $<$10
%BPVVDCHI2$>$36
%
%D tracks
%TRCHI2<4
%PT>100MeV
%P>1Gev
%MIPCHI2>4
%DLL<20()-10)
%
%phi
%sumPT>1
%VCHI2DOF<16
%BPVVDCHI2>16
%
%phi tracks
%TRCHI2<4
%PT>100MeV
%P>2GEv
%DLL<20()-10)
%
%Min one track \& two...


The data sample used in this analysis is the
$1\invfb$ collected by the \lhcb detector in 2011 from $pp$
collisions with a centre-of-mass energy of $7\tev$.
Events are selected if they fire {\tt L0Hadron TOS} or {\tt L0Global TIS}.
Further trigger requirements are applied at the \hlttwo level, where events are required to pass at
least one of the topological triggers (see \Sec{sec:lhcb:trig} for more details) with {\tt TOS}.

Candidate \Ds and \phii mesons are reconstructed only in the decays \decay{\Ds}{\kkpi} and
\phitokk, where the \Ds decay is the Cabibbo favoured mode.
Each daughter track has a $\chisqtrk/\ndof<4$, a \pt of at least $100\mev$, and a minimum momentum
of $1$ or $2\gev$ for tracks originating from the \Ds and \phii, respectively.
Each track in the event must also be detached from the PV, and are only accepted if the
$\chisqvtx/\ndof$ increases by more than four when the track is used in the vertex fit, compared to
when it is omitted.
Very loose PID requirements are also placed on the tracks, this is more to reduce the rate of the
stripping line that to identify pions and kaons, this is done later in the selection process.

The \Ds and \phii are also subject to reqiurements on the \chisq of the vertex separation,
\chisqvs, which is tighter for the \Ds becuase of its finite lifetime.
Full stripping requirements are given in \Tab{tab:dsphi:sel}.

\begin{table}
  \caption{\small
    Stripping selection applied to the \btodsphi candidates.
  }
  \label{tab:dsphi:sel}
  \begin{center}
    \begin{tabular}{cccc}
      \toprule
      Particle & \multicolumn{3}{c}{Cut} \\
      \midrule
      \Bp
      & $\sum\pt^\mathrm{tracks}$ &$>$& $5\gev$ \\
      & \chisqvtx/\ndof &$<$& 10 \\
      & \chisqip &$<$& 25 \\
      & $\tau$ &$>$& $0.2\ps$ \\
      & $\cos\thetadir$ &$>$& $0.999$ \\
      %& $\bdt_\mathrm{strip}$ &$>$& 0.05 \\
      \midrule
      \Ds
      & $\sum\pt^\mathrm{tracks}$ &$>$& $1.8\gev$ \\
      & \chisqvtx/\ndof &$<$& 10 \\
      & DOCA &$<$& $0.5\mm$ \\
      \midrule
      Tracks from \Ds
      & \pt &$>$& $100\mev$ \\
      & $p$ &$>$& $1\gev$ \\
      & $\chisqtrk/\ndof$ &$<$& $4$ \\
      & $\min\chisqip$ &$>$& $4$ \\
      & \chisqvs &$>$& 36 \\
      & $\dllkpi(K)$ &$>$& $-10$ \\
      & $\dllkpi(\pi)$ &$<$& $20$ \\
      \midrule
      \phii
      & $\sum\pt^\mathrm{tracks}$ &$>$& $1\gev$ \\
      & \chisqvtx/\ndof &$<$& 16 \\
      \midrule
      Tracks from $\phi$
      & \pt &$>$& $100\mev$ \\
      & $p$ &$>$& $2\gev$ \\
      & $\chisqtrk/\ndof$ &$<$& $4$ \\
      & $\min\chisqip$ &$>$& $4$ \\
      & \chisqvs &$>$& 16 \\
      %\midrule
      %STUFF FOR 1 AND 2 TRACKS
      \bottomrule
    \end{tabular}
  \end{center}
\end{table}


After the stripping selection, further cuts are applied.
The invariant mass of the \Ds candidates are required to be within $25\mev$j
its nominal value cited in \Ref{PDG2012}.
Cuts are also applied to reconstructed \phii mass, this is described later, in \Sec{sec:dsphi:hel}.
The decay vertex of the \Ds is required to be downstream of the decay vertex of the \Bp, and the
$p$-value formed from the sum of the \chisqip and \chisqvtx of the \Bp candidate is also required
to be greater than $0.1\pc$.
Charmless backgrounds are suppressed by requreing that the \chisqfd from the \Bp vertex was greater
than two.

There is potential for cross-feed between a \Dp meson decaying into a $\Km\pip\pip$ final stateand
the signal \Ds final state if the \pip is misidentified as a \Kp.
The resulting invariant \kkpi combination --- with one kaon being a misidentified pion --- may fall
within $25\mev$ of the nominal \Ds mass.
%This is possible because the \Ds is approximately $100\mev$ heavier than the \Dp.
There is also contamination from the decay \decay{\Lc}{p\Km\pip}.
These two modes of cross-feeds are suppressed by a series of conditions.
Firstly, if the mass of the \kk pair from the \Ds decay is within $10\mev$ of the \phii mass, then
the candidate is accepted as coming from the decay \decay{\Ds}{\kkpi}.
Otherwise, a track is assigned the mass of a pion or proton, so as the new object is consistent
with a \kkpi or $p\Km\pip$ respectively.
If the invariant mass of this newly reconstructed object falls within the nominal mass of the \Dp
or \Lc, then the ambiguous (swapped) track is subject to the stringent PID requirements of
$\dllkpi>10$ or $\dllkp>0$, respectively.


There are no backgrounds which contribute to a peaking component directly below the \Bp mass for
the decay \btodsphi.
However, on top of the combinatorial background, the low-mass side has a number of compoents which
result from excited \Dssp or \Kstarz.
The most significant of these are the decays
\footnote{
  For these decays the \Kstarz refers to the $K^*(892)^0$ meson.
}
\btodsstrphi, \bstodskstrk and \bstodsstrkstrk, which all contain at least one final state particle
that is not reconstructed.

The decay \bstodskstrk has never been observed, but given the branching fraction of the decay
$\decay{\Bd}{\Dm\Kstarzb\Kp}$ is $(8.8\pm1.9)\e{-4}$~\cite{PDG2012}, it should be in the
\btodsphi selection.
There is no contribution from the $\decay{\Bd}{\Dm\Kstarzb\Kp}$ mode, because the loss in mass from
the pion means that it peaks too low.
As well as this, the decay \bstodsstrkstrk was also expected to be present in the sample, though at
lower mass.




%These peaking backgrounds do not necessarily result in a longitudilnally polarized \phii.
%They are however, irriducible, and must be accounted for in the fit, this is described later in
%\Sec{sec:dsphi:fit}.

%In order to remove candidates from \Dp cross-feed,
%if the new $\Km\pip\pip$ mass falls within $25\mev$ of $m_{\Dp}^\pdg$,
%the kaon pair must either form an invariant mass within
%$10\mev$ of the known \phii mass, or the ambiguous track must pass stringent PID requirements
%($\dllkpi(\Kp)>10$).
%If the mass of the $p\Km\pip$ object falls within $25\mev$ of the \Lc mass,
%The candidate decay chain is then subject to the stripping requirements, which are outlined in
%\Tab{tab:dsphi:sel}.
%Requirements in the stripping are that the \Ds vertex is downstream of the \Bp and the vertices are
%well defined.
%One variable listed in \Tab{tab:dsphi:sel} is $\bdt_\mathrm{strip}$, which is the output of a BBDT
%in the stripping, whose input variables are a few kinematic and geometric variables of the \Bp and
%\Ds.
%The cut on this BBDT is very loose, and is approximately 100\pc efficient
%The \decay{\Ds}{\kkpi} and \decay{\phi}{\kk} candidates are required to have invariant masses
%within 25 and $20\mev$ of their known masses~\cite{PDG2012} respectively.



\subsection{Boosted Decision Tree}
Combinatorial background is reduced using a BDT.
Boosted Decision Trees were trained using the bagging method, as described in \Sec{sec:bdt:bag}, to
identify the decays \decay{\Ds}{\kkpi} and \decay{\phi}{\kk}.
The signal and background data used to train these BDTs came entirely from data.
%Signal samples came from the decays $\decay{\Bbar^0_{s}{\Ds\pim}$ and $\decay{\Bs}{\jpsi\phi}$
%data, which was background subtracted by sWeighting~\cite{splot}.
%Background samples were taken from the sideband distributions of the same data.
The BDT technique used in this analysis, also used in \Ref{LHCb-CONF-2012-009}, is different to
other analyses in this thesis in that a BDT is trained for each the \Ds and $\phi$.
The boosting technique used here was bagging, which --- as described in \Sec{sec:bdt:bag} --- gives
a response in the range $0<\mathrm{BDT}<1$.
Cutting on the product of the two BDT responses improves the performance, and therefore the cut is
$\bdt_{\Ds}\times\bdt_\phi>X$, as opposed to $\bdt_{\Ds}>X_1$ and $\bdt_\phi>X_2$.

The BDT used to separate background from the signal \decay{\Ds}{\kkpi} decays was trained using
a signal sample of \decay{\Bs}{\Dsm\pip} sWeighted~\cite{splot} data, and background from the \Dsm
sidebands.
This BDT uses an unusually large array of training variables, given in Table~\ref{tab:dsphi:vars},
which include PID, and track quality variables.
In total, five variables are input for the parent \Ds and $\phi$, and 23 properties for each
daughter track.
Similarly for the \decay{\phi}{\kk} BDT, the signal sample was \decay{\Bs}{\jpsi\phi}.


\subsection{Optimization}
The cut for the BDT wang the metric $S/\sqrt{S+B}$,
In this case, the number of signal events, $S$, was estimated from the yield from the decay
\decay{\Bs}{\Dsm\pip}, according to:
\begin{equation}
  S = \frac{ \BF\big(\btodsphi\big) }{ \BF\big(\bstodspi\big) }
  \frac{ \eff{geo}\big(\btodsphi\big) }{ \eff{geo}\big(\bstodspi\big) }
  \frac{f_d}{f_s}.
  N\big(\bstodspi\big)
\end{equation}
Here, \eff{geo} is the geometric efficiency caused by the detector, and $f_s/f_d$ quantifies
fraction of \Bs mesons produced relative to \Bd mesons.
The background yield is estimated as:
\begin{equation}
  B = c\cdot N_\mathrm{c}\big(\bstodspi\big)\cdot N_\mathrm{c}\big(\bstojpsiphi\big),
\end{equation}
where $N_\mathrm{c}$ indicates the yield of combinatoric background for a given decay, and
$c$ is a constant scaled such that
$N_\mathrm{c}\big(\bstodspi\big)\cdot
N_\mathrm{c}\big(\bstojpsiphi\big)=N_\mathrm{c}\big(\btodsphi\big)$ with no BDT cut.
The optimizatoin proceedure results in the optimal cut as $\bdt_{\Ds}\times\bdt_\phi>0.57$.



\subsection{Fit regions}
\label{sec:dsphi:hel}
A \Bp meson has quantum numbers $J^P=0^-$, and the \Ds and $\phi$ mesons have
quantum numbers $0^-$ and $1^-$ respectively.
Therefore the decay \btodsphi is a transistion of a pseudoscalar to a pseudoscalar and a vector
meson.
In order for angular momentum to be conserved, the vector particle must be produced in the $j=0$
state, where the spin is orthogonal to the particle's momentum.
The $\phi$ is the vector meson in the decay \btodsphi, in its final state it is lonitudinally
polarized, and its daughter kaons have an angular distribution proportional to $\cos^2\thetahel$;
where \thetahel is the helicity angle, defined as the angle between the \Kp and the \Bp in the rest frame of the \Ds.
This fact is used to further separate the signal decay \btodsphi from other backgrounds.


It is possible to remove about $93\pc$ of background by requiring that $|\cos\thetahel|>0.4$,
this is the same cut value as used in \Ref{LHCb-PAPER-2011-008}.
Therefore, two regions are defined in terms of $\cos\thetahel$, and both are used in a simultaneous
fit.
This also allows the combinatorial and peaking bacgrounds to be separated from the signal becuase
these do not necessarily have a longitudilnally polarized \phii.

%The signal region for the \phii candidate is defined to be within $20\mev$ of $m_\phii^\pdg$.
%A sideband region is defined to be $20<m_

%These peaking backgournds, and the combinatorial background,
%do not have to have a $\phi$ meson in a longitudilnally polarized.
%Therefore, the $\cos\thetahel$ distribution should be flat for all backgrounds.

%can be used to separate the signal decay from backgrounds.

%In order to maximize the signal yield, the full selected dataset was separated into four regions
%defined by the helicity angle and the invariant mass of the $\phi$ candidate, these regions are
%shown in \Table{tab:dsphi:hel}.

Fit regions are further split according to the invariant mass of the \phii candididate.
A signal region is defined for \phitokk candidates with a mass within $20\mev$ of the nominal \phii
mass, and a sideband region is defined for candidates with a mass in the range
$20<m_\phi^\pdg<40\mev$.

The resulting fit regions --- defined by \thetahel and $m_{\kk}$ --- have a signal region, \rA,
containing most of the signal, and a purely background region, \rD.
By performing a simultaneous fit to the four regions, which are defined explicitly in
\Tab{tab:dsphi:hel}, there is a better grasp on all the backgrounds.

\begin{table}
  \caption[Fit regions]
  {\small
    Fit regions used to search for the decay \btodsphi.
    Approximately $89\,\%$ was expected to be in region \rA.
  }
  \label{tab:dsphi:hel}
  \begin{center}
    \begin{tabular}{cccc}
      \toprule
      &&\multicolumn{2}{c}{$|m_{KK}-m_\phi^\pdg|$ (MeV)}\\
      &&$\in[0,20]$&$\in[20,40]$ \\
      \midrule
      \multirow{2}{*}{$|\cos\thetahel|$}
      &$>0.4$ & \rA & \rB \\
      &$<0.4$ & \rC & \rD \\
      \bottomrule
    \end{tabular}
  \end{center}
\end{table}





