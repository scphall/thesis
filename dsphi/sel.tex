\section{Selection}

%LIST
%
%Rely on HLT2 tolo lines
%cuts designed to mimic HLT1 and HL2 topo triggers:
%B
%sum pt $>$ 5\gev
%VHI2DOF $<$ 10
%BPVIPCHI2 $<$25
%BPVLTIME$>$0.2\ps
%BIRA$>$0.999
%
%D
%sum PT $>$1.8GeV
%MAXDOCA$<$0.5mm
%VCHI2DOF $<$10
%BPVVDCHI2$>$36
%
%D tracks
%TRCHI2<4
%PT>100MeV
%P>1Gev
%MIPCHI2>4
%DLL<20()-10)
%
%phi
%sumPT>1
%VCHI2DOF<16
%BPVVDCHI2>16
%
%phi tracks
%TRCHI2<4
%PT>100MeV
%P>2GEv
%DLL<20()-10)
%
%Min one track \& two...






The \decay{\Ds}{\kkpi} and \decay{\phi}{\kk} candidates are required to have invariant masses
within 25 and $20\mev$ of their known masses~\cite{PDG2012} respectively.

B mass is found by constraining the Ds to its nominal mass

D and Lc vetoes
D to KKpi



\subsection{Boosted Decision Tree}
Combinatorial background is reduced using a BDT.
Boosted Decision Trees were trained using the bagging method, as described in \Sec{sec:bdt:bag}, to
identify the decays \decay{\Ds}{\kkpi} and \decay{\phi}{\kk}.
The signal and background data used to train these BDTs came entirely from data.
%Signal samples came from the decays $\decay{\Bbar^0_{s}{\Ds\pim}$ and $\decay{\Bs}{\jpsi\phi}$
%data, which was background subtracted by sWeighting~\cite{splot}.
%Background samples were taken from the sideband distributions of the same data.
The BDT technique used in this analysis, also used in \Ref{LHCb-CONF-2012-009}, is different to
other analyses in this thesis in that a BDT is trained for each the \Ds and $\phi$.
The boosting technique used here was bagging, which --- as described in \Sec{sec:bdt:bag} --- gives
a response in the range $0<\mathrm{BDT}<1$.
Cutting on the product of the two BDT responses improves the performance, and therefore the cut is
$\bdt_{\Ds}\times\bdt_\phi>X$, as opposed to $\bdt_{\Ds}>X_1$ and $\bdt_\phi>X_2$.

The BDT used to separate background from the signal \decay{\Ds}{\kkpi} decays was trained using
a signal sample of \decay{\Bs}{\Dsm\pip} sWeighted~\cite{splot} data, and background from the \Dsm
sidebands.
This BDT uses an unusually large array of training variables, given in Table~\ref{tab:dsphi:vars},
which include PID, and track quality variables.
In total, five variables are input for the parent \Ds and $\phi$, and 23 properties for each
daughter track.
Similarly for the \decay{\phi}{\kk} BDT, the signal sample was \decay{\Bs}{\jpsi\phi}.


\subsection{Optimization}
The cut for the BDT wang the metric $S/\sqrt{S+B}$,
In this case, the number of signal events, $S$, was estimated from the yield from the decay
\decay{\Bs}{\Dsm\pip}, according to:
\begin{equation}
  S = \frac{ \BF\big(\btodsphi\big) }{ \BF\big(\bstodspi\big) }
  \frac{ \eff{geo}\big(\btodsphi\big) }{ \eff{geo}\big(\bstodspi\big) }
  \frac{f_d}{f_s}.
  \num{\bstodspi}
\end{equation}
Here, \eff{geo} is the geometric efficiency caused by the detector, and $f_s/f_d$ quantifies
fraction of \Bs mesons produced relative to \Bd mesons.
The background yield is estimated as:
\begin{equation}
  B = c\cdot N^\mathrm{comb}\big(\bstodspi\big)\cdot N^\mathrm{comb}\big(\bstojpsiphi\big),
\end{equation}
where $N^\mathrm{comb}$ indicates the yield of combinatoric background for a given decay, and
$c$ is a constant scaled such that
$N^\mathrm{comb}\big(\bstodspi\big)\cdot
N^\mathrm{comb}\big(\bstojpsiphi\big)=N^\mathrm{comb}\big(\btodsphi\big)$ with no BDT cut.
The optimizatoin proceedure results in the optimal cut as $\bdt_{\Ds}\times\bdt_\phi>0.57$.



\subsection{Helicity angle}
There are no backgrounds which contribute to a peaking component below the \Bp mass for the decay
\btodsphi.
However, on top of the combinatorial background, the low-mass side has a number of compoents which
result from excited \Dssp or \Kstarz.
The most significant of these are the decays
\footnote{
  For these decays the \Kstarz refers to the $K^*(892)^0$ meson.
}
\btodsstrphi, \bstodskstrk and \bstodsstrkstrk.

A \Bp meson has quantum numbers $J^P=0^-$, and the \Ds and $\phi$ mesons have
quantum numbers $0^-$ and $1^-$ respectively.
Therefore the decay \btodsphi is a transistion of a pseudoscalar to a pseudoscalar and a vector
meson.
In order for angular momentum to be conserved, the vector particle must be produced in the $j=0$
state, where the spin is orthogonal to the particle's momentum.
The $\phi$ is the vector meson in the decay \btodsphi, in its final state it is lonitudinally
polarized, and its daughter kaons have an angular distribution proportional to $\cos^2\thetahel$;
where \thetahel is the helicity angle, defined as the angle between the \Kp and the \Bp in the rest frame of the \Ds.
This fact is used to further separate the signal decay \btodsphi from other backgrounds.

These backgrounds do not have to have a $\phi$ meson in a longitudilnally polarized, and therefore
the $\cos\thetahel$ distribution can be used to separate the signal decay from backgrounds.
In order to maximize the signal yield, the full selected dataset was separated into four regions
defined by the helicity angle and the invariant mass of the $\phi$ candidate, these regions are
shown in \Table{tab:dsphi:hel}.

\begin{table}
  \caption{\small
    Fit regions used to search for the decay \btodsphi.
    Approximately $89\,\%$ was expected to be in region \rA.
  }
  \label{tab:dsphi:hel}
  \begin{center}
    \begin{tabular}{cccc}
      \toprule
      &&\multicolumn{2}{c}{$|m_{KK}-m_\phi^\pdg|$ (MeV)}\\
      &&$\in[0,20]$&$\in[20,40]$ \\
      \midrule
      \multirow{2}{*}{$|\cos\thetahel|$}
      &$>0.4$ & \rA & \rB \\
      &$<0.4$ & \rC & \rD \\
      \bottomrule
    \end{tabular}
  \end{center}
\end{table}









