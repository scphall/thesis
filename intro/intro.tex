\chapter{Introduction}
\label{ch:intro}

Scientific endeavour has long been a quest to understand the Universe in which we live.
From the vastness of space, to the incomprehensible world of quantum mechanics,
increased understanding comes in leaps and bounds in a feedback loop of scientific discovery.
Still the most ancient questions remain unanswered, including: why does the Universe the way it is?
The subject of high energy particle physics sits at the forefront of modern physics and seeks to
answer, arguably, this most basic question.
%While we appear to have grasped an understanding of particle physics at some energies, a knowledge
%gap extends up to higher energies, which is
%as the technological limit of particle acceleration is reached.

Our current understanding of the interactions governing the behaviour of fundamental particles is
encapsulated in the \sm.
This is a remarkably successful theory which results in no significant deviations between theory
and experimental measurements.
That being said, there are a plethora of phenomena which remain completely unexplained by the \sm.
%These include the absence of gravity in the \sm framework, and its inability to explain the matter
%dominated nature of today's Universe.
The \sm is introduced fully in \Chap{ch:theory}.

The \lhc is currently the world's most energetic particle accelerator.
It is host to a number of
experiments probing high energy proton-proton ($pp$) collisions for evidence of \np
which might go towards resolving the inadequacies of the \sm.
Located on the \lhc are four main particle detectors.
One of these is the \lhcb detector, which
specialises in precision measurements of beauty flavoured hadrons in the hunt for \np, this is
introduced in \Chap{ch:lhcb}.

\gls{HEP} experiments can broadly be split into two camps: ones that search for \np
\emph{directly} and those that search \emph{indirectly}.
These names are somewhat misleading, because both can be clear indicators of the existence of
\np.
Direct searches rely on observing a particle that is not encompassed by the \sm,
and seeing its immediate manifestation, such as a peak in an invariant mass spectrum.
%its decay products forming a peak in their invariant mass spectrum.
Indirect searches
use precisely measured observables and compare them with theoretical predictions; discrepancies
between the two will indicate either some flaw in the theoretical understanding, or new and unknown
physics processes.
%look for deviations in theoretical predictions for processes mediated by loop
%diagrams which might indicate \np.
This can be an advantageous search method, because virtual particles can contribute to processes
and therefore indirect searches have access to much higher mass scales than direct ones.

This thesis describes three analyses: in \Chap{ch:dsphi}, \Chap{ch:hhh}, and \Chap{ch:db},
performed on data collected by the \lhcb detector.
The first two of these analyses search for indications of \np indirectly, and the third is a
direct search for a new particle.


\clearpage
