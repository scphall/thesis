\chapter{Introduction}
\label{ch:intro}


Scientific endeavour has long been a quest to understand the Universe in which we live.
From the vastness of space, to the incomprehensible world of quantum mechanics,
increased understanding comes in leaps and bounds as the feedback loop of scientific discovery as
technology advances.
Still the most ancient questions remain unanswered, including: Why does the Universe the way it is?
The subject of high energy particle physics sits at the forefront of modern physics and seeks to
answer, arguably, this most basic question.
While we appear to have a grasp of particle physics at some energies, there is a knowledge gap at
higher energies, as the technological limit of particle acceleration is reached.

Our current understanding of the interactions governing the behaviour of fundamental particles is
encapsulated in the \sm.
This is a remarkably successful theory which results in no significant deviations between theory
and experimental measurements.
That being said, there are a plethora of phenomena which remain completely unexplained by the \sm.
%These include the absence of gravity in the \sm framework, and its inability to explain the matter
%dominated nature of today's Universe.
The \sm is introduced fully in \Chap{ch:theory}.

The \lhc is currently the world most energetic particle accelerator, and is host to a number of
experiments probing high energy collisions for evidence of contributions from \np which might go
towards explaining, or resolving, the inadequacies of the \sm.
Located on the \lhc are four main particle detectors, one of these is the \lhcb experiment, which
specializes in precision measurements of beauty
flavoured hadrons in the hunt for \np, this is introduced in \Chap{ch:lhcb}.

In high energy physics experiments can broadly be split into two camps: ones that search for \np
\emph{directly} and those that search \emph{indirectly}.
These names are somewhat misleading, because both can be clear indicators of the existence of
physics beyond the \sm.
Direct searches rely on the potential observation of a particle that is not encompassed by the \sm,
and seeing its immediate manifestation, such as a peak in an invariant mass spectrum.
%its decay products forming a peak in their invariant mass spectrum.
Indirect searches look for deviations in theoretical predictions for processes mediated by loop
diagrams which might indicate \np.
This can be an advantageous search method, because off-shell particles can contribute to processes
and be observed as significant deviations from theory.

This thesis describes three analyses: in \Chap{ch:dsphi}, \Chap{ch:hhh}, and \Chap{ch:db};
performed on data collected by the \lhcb detector.
The first two of these analyses are searches indications of \np indirectly, and the third is a
search for a new particle.




\clearpage





%This thesis...
%$B$ decays
%measurements test the theoretcial framework of the \sm, itroduced in chap.
%sensitive to \np because particles are off-shell
%
%results based on data from \lhcb detector, as described in chap.
%
%analysis chapters.
%
%\begin{itemize}
  %\item Direct and indirect...
%%%%\end{itemize}
