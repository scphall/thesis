%&spell
\section[Branching fraction of the decay \btophikmumu]{Branching fraction of the decay
  $\boldsymbol{\btophikmumu}$}

The branching fraction of the signal decay \btophikmumu is determined relative to the normalization
channel \btojpsiphik:
\begin{equation}
  \BF\big(\btokphimumu\big)=
  \frac{ N^\prime\big(\btophikmumu\big) }{ N\big(\btophikmumu\big) }
  \cdot\BF\big(\btojpsiphik\big)
  \cdot\BF\big(\jpsitomumu\big).
\end{equation}
Here, $N^\prime\big(\btophikmumu\big)$ denotes the number of signal events extracted from an
unbinned fit of \btophikmumu candidates, with each event weighted by the relative efficiency
$\varepsilon\big(\btojpsiphik\big) / \varepsilon_\qsq\big(\btophikmumu\big)$.
This is done because the efficiency of the decay \btophikmumu was shown to vary significantly over
the full \qsq range, the weights are determined in bins of \qsq.
The branching fraction measurements used were
$\BF\big(\btojpsiphik\big)=\big(5.2\pm1.7\big)\e{-5}$~\cite{PDG2012},
and $\big(\jpsitomumu\big)=5.93\pm0.06$~\cite{PDG2012}.

Yeilds for both the signal and normalization channels are extracted from unbinned maximum
likelihood fits of the invariant mass of the \Bp candidates.
The signal component for the normalization channel is the sum of two Gaussian functions, with a
power-law tail on the low mass side; the same function is used for fitting the weighted signal
distribution where all parameters are fixed from a fit to the high statistics normalization mode.
Combinatorial background is modelled as a second order Chebychev polynomail.
These fits give the values
$N^\prime\big(\btokphimumu\big)=25.2\,^{+6.0}_{-5.3}$ and
$N\big(\btojpsiphik\big)=1908\pm63$.

The above values lead to a measured branching fraction of
\begin{equation}
  \BF\big(\btophikmumu\big)=
  \big(0.81\,^{+0.18}_{-0.16}\stat\pm0.03\syst\pm0.27\norm\big).
\end{equation}
However, the charmonium vetoes remove $\big(2\,^{+10}_{-\pz2}\big)\,\%$ of signal events, as
calculated using simulated events.
Taking this into account results in a value of
\begin{equation}
  \BF\big(\btophikmumu\big)=
  \big(0.82\,^{+0.19}_{-0.17}\stat\,^{+0.10}_{-0.04}\syst\pm0.27\norm\big).
\end{equation}
The ratio of branching fractions of the signal and normalization channels is

\begin{equation}
  \frac{ \BF\big(\btophikmumu\big) }{ \BF\big(\btojpsiphik\big) }
  =\big(1.58\,^{+0.36}_{-0.32}\stat\,^{+0.19}_{-0.07}\syst\big),
\end{equation}
which is quoted because there are large relative uncertainties associated with the branching
fraction of the normalization channel ($\sim33\,\%$).





%The normalization channel has a large relative uncertaity on its measured branching fraction
%$\BF\big(\btojpsiphik\big)=\big(5.2\pm1.7\big)\e{-5}$~\cite{PDG2012}.
%With






