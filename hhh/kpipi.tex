\section[Differential branching fraction measurements of the decay \btokpipimumu]{
  Differential branching fraction measurements of the decay \tmath{\btokpipimumu}
}

Given the statistics available for this channel, the differential branching fraction,
$\dBF(\btokpipimumu)\dqsq$ was calculated in bins of \qsq.
The differential branching fraction for a bin of width $\Delta\qsq$ is
%\begin{multline}
  %\frac{d}{\dqsq}\BF\left(\btokpipimumu\right)
  %=
  %\frac{1}{\Delta\qsq} \cdot
  %\frac{\num{sig}}{\num{norm}} \cdot
  %\frac{\eff{norm}}{\eff{sig}} \\
  %\times\BF\left(\btopsitwosk\right) \cdot
  %\BF\left(\psitwostojpsipipi\right) \cdot
  %\BF\left(\jpsitomumu\right),
%\end{multline}
\begin{equation}
  \frac{d}{\dqsq}\BF\left(\btokpipimumu\right)
  =
  \frac{1}{\Delta\qsq} \cdot
  \frac{\num{sig}}{\num{norm}} \cdot
  \frac{\eff{norm}}{\eff{sig}} \cdot
  \BF_\mathrm{tot}\left(\btopsitwosk\right)
\end{equation}
where,
\begin{equation}
  \BF_\mathrm{tot}\left(\btopsitwosk\right)
  =
  \BF\left(\btopsitwosk\right) \cdot
  \BF\left(\psitwostojpsipipi\right) \cdot
  \BF\left(\jpsitomumu\right),
\end{equation}
\num{sig} is the yield of the signal decay \btokpipimumu in the given \qsq bin and \num{norm}
is the yield of the normalization channel.
Total efficiencies for reconstruction and selection are denoted by \eff{sig} and \eff{norm} for the
signal and normalization channels respectively.

The normalization channel was chosen to be \btopsitwosk, where \psitwostojpsipipi and \jpsitomumu,
which has a total branching fraction of $(1.264 \pm 0.0052)\e{-5}$~\cite{PDG2012}.
This was chosen for the normalization channel over the decay \btojpsikpipi because it has a relative
uncertainty of $4\,\%$ rather than $16\,\%$ for $\BF(\btojpsikpipi) = (8.1\pm1.3)\e{-4}$.


{\renewcommand{\arraystretch}{1.2}
\begin{table}
  \begin{center}
    \caption{\small
      %Signal yields for the decay $\btokpipimumu$
      %and resulting differential branching fractions in bins of \qsq.
      %The first contribution to the uncertainty is statistical, the second systematic,
      %where the uncertainty due to the branching fraction of the normalisation channel is included.
      %The $q^2$ binning used corresponds to the binning used in previous analyses of
      %$b\to s\mup\mun$
      %decays~\cite{LHCb-PAPER-2013-017,LHCb-PAPER-2013-037,LHCb-PAPER-2013-019}.
      %Results are also presented for the $q^2$ range from $1$ to $6\gevgevcccc$, where theory
      %predictions are expected to be most reliable.
    }
    \label{tab:hhh:diffbf}
    \begin{tabular}{ccc}\toprule
      \qsq bin $[\gevgevcccc]$  & $N_\sig$ & $\tfrac{\dBF}{\dqsq}\;[\e{-8}\pergevgevcccc]$
      \\\midrule
      $[\pz0.10,\pz2.00]$ & $134.1\,^{+12.9}_{-12.3}$     & $7.01\,^{+0.69}_{-0.65} \pm 0.47$ \\
      $[\pz2.00,\pz4.30]$ & $\pz56.5\,^{+\pz9.7}_{-\pz9.1}$ & $2.34\,^{+0.41}_{-0.38} \pm 0.15$ \\
      $[\pz4.30,\pz8.68]$ & $119.9\,^{+14.6}_{-13.7}$     & $2.30\,^{+0.28}_{-0.26} \pm 0.20$ \\
      $[10.09,12.86]$     & $\pz54.0\,^{+10.1}_{-\pz9.4}$   & $1.83\,^{+0.34}_{-0.32} \pm 0.17$ \\
      $[14.18,19.00]$     & $\pzz3.3\,^{+\pz2.8}_{-\pz2.1}$ & $0.10\,^{+0.08}_{-0.06} \pm 0.01$ \\
      %\midrule
      %$[\pz1.00,\pz6.00]$ & $144.8\,^{+14.9}_{-14.3}$     & $2.75\,^{+0.29}_{-0.28} \pm 0.16$ \\
      \bottomrule
    \end{tabular}
  \end{center}
\end{table}
}

%Due to the high statistics of the decay \btokpipimumu, its braching fraction was also measured
%differentially, in bins of the invariant mass squared of the dimuon system (\qsq).

%Measurements pertaining to the decay \btokpipimumu were made with respect to the normalization
%channel \btopsitwosk, where \psitwostojpsipipi and \jpsitomumu.
%\begin{equation}
  %\BF\left(\b\right)
%\end{equation}





%The decay \btojpsikpipi was used for cross-checks
%These measurements were made with respect to the normalization channels




\begin{figure}
  \begin{center}
    \includegraphics[width=0.48\textwidth]{b2kpipimumu}
    \includegraphics[width=0.48\textwidth]{kpipimumu_q2_r2}
    \includegraphics[width=0.48\textwidth]{kpipimumu_q2_r3}
    \includegraphics[width=0.48\textwidth]{kpipimumu_q2_r4}
    \includegraphics[width=0.48\textwidth]{kpipimumu_q2_r5}
    \includegraphics[width=0.48\textwidth]{kpipimumu_q2_r6}
    \caption{\small
      Invariant mass distributions of \btokpipimumu candidates in bins of \qsq with projections of
      fits overlaid, the upper left plot is a separate fit to the full \qsq range.
      The signal signal component (light blue) is modelled by the sum of two Gaussian
      distributions, and the background component (dark blue) is an exponential function.
      In the three \qsq bins $4.30<\qsq<8.68\gevgev$, $10.09<\qsq<12.86\gevgev$, and
      $14.18<\qsq<19.00\gevgev$ scaling factors in the background components are used to account
      the removal of the radiative tails of charmonium vetoes.
      The lower right
    }
    \label{}
  \end{center}
\end{figure}



\begin{figure}
  \begin{center}
    \includegraphics[width=0.48\textwidth]{b2kpipijpsi}
    \includegraphics[width=0.48\textwidth]{b2psi2sk}
    \caption{\small
      Distributions of the invariant mass of the
      (a) cross-check channel \btojpsikpipi and
      (b) normalization mode \btopsitwosk.
      Projections from the fit are overlayed, where the light blue is the yield of the indicated
      decay, and the dark blue is the background component.
    }
    \label{fig:hhh:norm}
  \end{center}
\end{figure}



\begin{figure}
  \begin{center}
    \includegraphics[width=0.48\textwidth]{kpipi_fromjpsi}
    \includegraphics[width=0.48\textwidth]{kpipi_frommumu}
    \caption{\small
      Invariant mass distributions of the \kpipi object, from the
      (a) signal decay \btokpipimumu and the
      (b) control channel \btojpsikpipi which have been background subtracted
      using the \emph{sPlot}~\cite{splot} technique.
      The vertical lines indicate the central masses of the \kone{1270} and \kone{1400}
      resonances.
    }
    \label{fig:hhh:kpipi}
  \end{center}
\end{figure}



\begin{figure}
  \begin{center}
    \includegraphics[width=0.48\textwidth]{diffbf}
    \caption{\small
      Differential branching fraction $\tfrac{d}{\dqsq}\BF\left(\btokpipimumu\right)$
      (as given in Table~\protect\ref{tab:hhh:diffbf}), where the
      errors shown include both statistical and systematic uncertainties.
      Shaded areas indicate the vetoed \jpsi and \psitwos resonances.
    }
    \label{fig}
  \end{center}
\end{figure}





