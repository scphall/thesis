\section{Summary}
\label{sec:hhh:conc}

The two \fcnc decays \btokpipimumu and \btophikmumu were both observed for the first time.
Their branching fractions were measured to be:
\begin{align*}
  \BF\big( \btokpipimumu \big)&=
  \big(4.36\,^{+0.29}_{-0.27}\stat\pm0.21\syst\pm0.18\normerr\big)\e{-7}, \\
  \BF\big(\btophikmumu\big)&=
  \big(0.82\,^{+0.19}_{-0.17}\stat\,^{+0.10}_{-0.04}\syst\pm0.27\normerr\big)\e{-7}.
\end{align*}
The ratio of these branching fractions is \approx$5.3$, which is consistent with the ratio of
branching fractions for the decays \decay{\Bp}{\kpipi} and \decay{\Bp}{\phik}.
Additionally, the differential decay rate of the decay \btokpipimumu was calculated with respect to
\qsq; these results are given in \Table{tab:kpipi:diffbf} and shown in \Fig{fig:kpipi:diffbf}.

Also shown were the resonances of the \kpipi and \phik systems for the signal decays, and for the
same final states where the muons come from a \jpsi.


%Improvements could trivially be made after another measurement of
%$\BF\big(\btojpsiphik\big)$ with reduced uncertainties.
%For this reason, \Ref{LHCb-PAPER-2014-030} quotes the ratio of branching fractions.
%
%Future measurements of these decays would benefit from additional statistics.
%Particularly in order to understand the complex resonance structure observed in the hadronic
%systems.
%If one could begin to understand the structure of the \kpipi system, not only would make for
%interesting spectroscopy measurements, but also it would give a handle on to the \kpipi angular
%distributions.
%This, in turn, would make \btokpipimumu sensitive to the same operators to \btokstrmumu.
%
%Increased data would also allow measurements other three-hadron, two-muon final states.
%Measurements of the branching fractions of the decays \decay{\Bp}{\pip\pipi\mumu} and
%\decay{\Bp}{\kk\pip\mumu} would give access to the ratio of \ckm matrix elements $\V{td}/\V{ts}$.
%
%The decay \btokpipimumu has a relatively large branching fraction, and could be used in future
%searches for \np particles in the dimuon distribution: in a similar way to the analysis described
%in \Chap{ch:db}.


