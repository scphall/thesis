\section{Summary}
\label{sec:hhh:conc}

The two \fcnc decays \btokpipimumu and \btophikmumu were both observed for the first time.
Their branching fractions were measured to be:
\begin{align*}
  \BF\big( \btokpipimumu \big)&=
  \big(4.36\,^{+0.29}_{-0.27}\stat\pm0.21\syst\pm0.18\normerr\big)\e{-7}, \\
  \BF\big(\btophikmumu\big)&=
  \big(0.82\,^{+0.19}_{-0.17}\stat\,^{+0.10}_{-0.04}\syst\pm0.27\normerr\big)\e{-7}.
\end{align*}
Additionally, the differential decay rate of the decay \btokpipimumu was calculated with respect to
\qsq; these results are given in \Table{tab:kpipi:diffbf} and shown in \Fig{fig:kpipi:diffbf}.

\begin{itemize}
  \item Could in future normalize to \decay{\Bp}{\jpsi\Kstar}
  \item Good to do resonant analysis, like in \decay{\Bp}{\kpipi\gamma}
  \item Improve with different/better normalization channel
  \item Conclude iwth statement that, given to relavtively large BF of \btokpipimumu, could in
    future use to search for inflaton like particles!
  \item Angular observables in this final state are not clean because of complex resonance
    structure of the \kpipi system.
\end{itemize}


